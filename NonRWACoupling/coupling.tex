\levelstay{Coupling}

\quickfig{\columnwidth}{coupled_circuits.pdf}{Two resonators coupled capacitively and inductively.}{fig:coupledOscillatorsLAndC:diagram}

Consider two coupled LC oscillators as shown in Figure \ref{fig:coupledOscillatorsLAndC:diagram}.
Kirchhoff's laws for this circuit give us four equations
\begin{align}
  I_{C_a} + I_{L_a} + I_g &= 0 \nonumber \\
  I_{C_b} + I_{L_b} - I_g &= 0 \nonumber \\
  L_a \dot I_{L_a} + L_g \dot I_{L_b} &= V_a \nonumber \\
  L_b \dot I_{L_b} + L_g \dot I_{L_a} &= V_b
  \, .
\end{align}
Going through the usual process of finding the Lagrangian and then converting to the Hamiltonian, we find the Hamiltonian
\begin{align}
  H_g / \hbar
  &= \omega_a' a^* a + \omega_b' b^* b \nonumber \\
  &- g_+ (ab + a^* b^*) + g_- (ab^* + a^* b)
\end{align}
where we defined
\begin{align}
  g_c \equiv \frac{1}{2} \frac{1}{C_g' \sqrt{Z_a' Z_b'}} &\qquad
  g_l \equiv \frac{1}{2} \frac{\sqrt{Z_a' Z_b'}}{L_g'} \nonumber \\
  \text{and} \qquad \qquad
  g_+ \equiv g_c + g_l &\qquad g_- \equiv g_c - g_l
\end{align}
where $\omega'$, $Z'$, $C'$, and $L'$ are normalized frequency, impedance, capacitance, and inductance of the coupled resonators.
Details are given in Appendix \ref{appendix:coupling}.
Hamilton's equations of motion can now be expressed in matrix form
\begin{equation*}
  \frac{d}{dt}
  \left( \begin{array}{c} a \\ b \\ a^* \\ b^* \end{array} \right)
  = -i \underbrace{\left( \begin{array}{cccc}
    \omega_a' & g_- & 0 & -g_+ \\
    g_- & \omega_b' & -g_+ & 0 \\
    0 & g_+ & -\omega_a' & -g _- \\
    g_+ & 0 & -g_- & -\omega_b'
  \end{array} \right)}_M
  \left( \begin{array}{c} a \\ b \\ a^* \\ b^* \end{array} \right) \, .
\end{equation*}
Here we can see that there are two \emph{aspects} to the coupling.
The coupling $g_-$ connects $a$ to $b$ and $a^*$ to $b^*$, i.e. it acts within the upper and lower 2x2 sub-blocks of $M$.
On the other hand, the coupling $g_+$ lies entirely on the anti-diagonal coupling $a$ to $b^*$ and $b$ to $a^*$.
The structure of $M$ and the roles of the two aspects of coupling can be clearly seen by writing $M$ in algebraic form as
\begin{equation}
  M = -i \left[
    \sigma_z \otimes
      \left(
        g_- \sigma_x + \frac{\Delta}{2} \sigma_z + \frac{S}{2} \mathbb{I}
      \right)
    -i g_+ (\sigma_y \otimes \sigma_x)
  \right]
\end{equation}
where $S \equiv \omega_a' + \omega_b'$ and $\Delta \equiv \omega_a' - \omega_b'$.

\leveldown{Rotating wave approximation}

The RWA drops the $ab$ and $a^* b^*$ terms of the coupling Hamiltonian, i.e. it drops the terms proportional to $g_+$.
Therefore, if we engineer the coupling such that $g_+$ is identically zero, then the RWA will be exact!
The condition that $g_+ = 0$ means that $g_c = -g_l$, i.e. the capacitive and inductive coupling must be of equal magnitude but opposite sign.

Dropping those terms is equivalent to dropping the antidiagonal of $M$ and therefore the $\sigma_y \otimes \sigma_x$ term in $M$'s algebraic representation.
In the matrix, we can see that dropping these terms is precisely equivalent to decoupling the clockwise and counterclockwise rotating modes.
In other words, the RWA drops the part of the dynamics coupling the clockwise-rotating modes to the anti-clockwise-rotating modes.


\levelstay{Eigenvalues}

The rotating wave approximation provides a convenient approximation for the eigenvalues of $M$.
In the rotating wave aproximation, the algebraic representation of $M$ reduces to
\begin{equation}
  M_\text{RWA} = -i \sigma_z \otimes \left(
    g_- \sigma_x + \frac{\Delta}{2} \sigma_z + \frac{S}{2} \mathbb{I}
  \right)
  \, .
  \label{eq:matrix_algebra_rwa}
\end{equation}
Equation (\ref{eq:matrix_algebra_rwa}) makes finding the eigenvalues particularly simple because the eigenvalues of a tensor product are just the products of the eigenvalues of the individual factors.
The eigenvalues of the quantity in parentheses, and therefore the normal mode frequencies, are
\begin{equation}
  \pm \omega_\pm
  = \pm \left(
    \frac{\omega_a' + \omega_b'}{2} \pm \sqrt{g_-^2 + (\Delta / 2)^2 }
  \right)
  \, .
\end{equation}
These eigenvalues are shown in Figure \ref{Figure:coupled_resonators_L_and_C:avoided_crossing} as a function of both $g_-$ and $\omega_b'$.
There we can see the famous avoided level crossing of coupled resonators.
\quickfig{0.8\columnwidth}{avoided_crossing.pdf}{Normal frequencies $\omega_+$ (red) and $\omega_-$ (blue) as a function of $\omega_b'$ for $\omega_a' = 10$. Dashed lines are for $g_-=0.02$ and solid lines are for $g_-=0.1$.}{Figure:coupled_resonators_L_and_C:avoided_crossing}
Note that the rotating wave approximation for the eigenvalues works in the limit $g_+ \ll \omega_a', \omega_b'$.
Note also that the symmetry of Figure \ref{Figure:coupled_resonators_L_and_C:avoided_crossing} exists because we plot the normal frequencies against the frequency $\omega_b'$ rather than the uncoupled frequency $\omega_b$.

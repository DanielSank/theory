\section{Coupling} \label{appendix:coupling}

\leveldown{Kirchhoff's laws}

In order to get useful equations of motion from Kirchhoff's laws, we need to work with all voltages or all currents.
We'll use voltages.
The capacitor currents are easily related to voltage via the constitutive equation for a capacitor: $C_i \dot V_i = I_i$.
Note that for the coupling capacitor, the constitutive relation gives $I_g = C_g (\dot V_a - \dot V_b)$.
Relating the inductor currents to voltages requires more work.
The bottom two of the equations from Kirchhoff's laws can be written in matrix form as
\begin{equation*}
  \begin{pmatrix} V_a \\ V_b \end{pmatrix}
  = \underbrace{ \begin{pmatrix}
    L_a & L_g \\
    L_g & L_b
  \end{pmatrix}}_{T_L}
  \begin{pmatrix} \dot I_a \\ \dot I_b \end{pmatrix}
  \, .
\end{equation*}
The inverse of $T_L$ is
\begin{align*}
  T_L^{-1}
  &= \frac{1}{L_a L_b - L_g^2} \begin{pmatrix}
    L_b & -L_g \\ -L_g & L_a
  \end{pmatrix} \\
  & \equiv \begin{pmatrix}
    1 / L_a' & -1 / L_g' \\ -1 / L_g' & 1 / L_b'
  \end{pmatrix}
\end{align*}
so that
\begin{align*}
  \dot I_{L_a} &= \frac{V_a}{L_a'} - \frac{V_b}{L_g'} \\
  \dot I_{L_b} &= \frac{V_b}{L_b'} - \frac{V_a}{L_g'} \, .
\end{align*}
Note that
\begin{align*}
  L_a' & \equiv L_a - \frac{L_g^2}{L_b} \\
  \text{and} \quad
  L_b' & \equiv L_b - \frac{L_g^2}{L_a}
\end{align*}
are the inductances to ground for each resonator, including the inductance through the mutual.
Now we can rewrite all of the currents in the first two of Kirchhoff's laws entirely in terms of $V_a$ and $V_b$:
\begin{align*}
  C_a \ddot V_a + \frac{V_a}{L_a'} - \frac{V_b}{L_g'} + C_g (\ddot V_a - \ddot V_b) &= 0 \\
  C_b \ddot V_b + \frac{V_b}{L_b'} - \frac{V_a}{L_g'} + C_g (\ddot V_b - \ddot V_a) &= 0 \, .
\end{align*}
Traditional analysis of circuits in the physics literature uses flux and charge instead of current and voltage, so defining $\Phi = \int V \, dt$, we can write our equations of motion as
\begin{align}
  \ddot \Phi_a (C_a + C_g) - C_g \ddot \Phi_b + \frac{\Phi_a}{L_a'} - \frac{\Phi_b}{L_g'} &= 0 \nonumber \\
  \ddot \Phi_b (C_b + C_g) - C_g \ddot \Phi_a + \frac{\Phi_b}{L_b'} - \frac{\Phi_a}{L_g'} &= 0
  \label{eq:coupled_equations_of_motion}
  \, .
\end{align}

\levelstay{Hamiltonian form}

By inspection and a bit of fiddling around, one can check that equations (\ref{eq:coupled_equations_of_motion}) are generated by the Lagrangian
\begin{align}
  \mathcal{L}
  =& \underbrace{\frac{C_g}{2} \left(\dot \Phi_a - \dot \Phi_b \right)^2
   + \frac{C_a}{2} \dot \Phi_a^2 + \frac{C_b}{2} \dot \Phi_b^2}_\text{kinetic} \nonumber \\
  & \underbrace{- \frac{\Phi_a^2}{2 L_a'} - \frac{\Phi_b^2}{2 L_b'} + \frac{\Phi_a \Phi_b}{L_g'}}_\text{potential} \, .
\end{align}
The canonical momenta conjugate to $\Phi_a$ and $\Phi_b$ are
\begin{align*}
  Q_a & \equiv \frac{\partial \mathcal{L}}{\partial \dot \Phi_a} = (C_a + C_g) \dot \Phi_a - C_g \dot \Phi_b \\
  Q_b & \equiv \frac{\partial \mathcal{L}}{\partial \dot \Phi_b} = (C_b + C_g) \dot \Phi_b - C_g \dot \Phi_a
  \, .
\end{align*}
or in matrix form
\begin{equation*}
  \begin{pmatrix} Q_a \\ Q_b \end{pmatrix}
  =
  \underbrace{
    \begin{pmatrix}
       (C_a + C_g) & -C_g \\
       -C_g & (C_b + C_g)
    \end{pmatrix}
  }_{T_C}
  \begin{pmatrix} \dot \Phi_a \\ \dot \Phi_b \end{pmatrix}
  \, .
\end{equation*}
The matrix $T_C$ is just the capacitance matrix of the circuit.
Its inverse is
\begin{align*}
  T_C^{-1} =& \frac{1}{C_a C_b + C_g (C_a + C_b)}
  \begin{pmatrix}
    (C_b + C_g) & C_g \\
    C_g & (C_a + C_g)
  \end{pmatrix} \\
  \equiv& \begin{pmatrix}
    1 / C_a' & 1 / C_g' \\
    1 / C_g' & 1 / C_b'
  \end{pmatrix}
\end{align*}
Note that $C_a'$ and $C_b'$ are simply the total capacitances to ground for each resonator!

The Hamiltonian function $H$ for the system is defined formally by the equation
\begin{equation*}
  H \equiv \left( \sum_{i = \{a, b\}} Q_i \dot \Phi_i \right) - \mathcal{L}
\end{equation*}
where $\Phi_i$ are the coordinates and $Q_i$ are the conjugate momenta, but where we have to replace the $\dot \Phi$'s in both the $Q \dot \Phi$ terms and in $\mathcal{L}$ with $\Phi$'s and $Q$'s.
To do this, first note that the kinetic term in the Lagrangian can be expressed as\footnote{Because matrix transposition and inversion commute, and because the matrix $T_C$ is symmetric, we can bring $T_C^{-1}$ from the bra onto the ket for free.}
\begin{align*}
  \mathcal{L}_\text{kinetic}
  =& \frac{1}{2} \bbraket{\dot \Phi}{T_C}{\dot \Phi} \\
  =& \frac{1}{2} \bbraket{T_C^{-1} Q}{T_C}{T_C^{-1} Q} \\
  =& \frac{1}{2} \bbraket{Q}{T_C^{-1}}{Q}
  \, .
\end{align*}
Note also that $\sum_i \dot \Phi_i Q_i = \bbraket{Q}{T_C^{-1}}{Q}$, so we can write the Hamiltonian as
\begin{align}
  H
  =& \left( \sum_{i = \{a, b\}} Q_i \dot \Phi_i \right) - \mathcal{L} \nonumber \\
  =& \bbraket{Q}{T_C^{-1}}{Q} - \left( \mathcal{L}_\text{kinetic} + \mathcal{L}_\text{potential} \right) \nonumber \\
  =& \bbraket{Q}{T_C^{-1}}{Q} - \left( \frac{1}{2} \bbraket{Q}{T_C^{-1}}{Q} + \mathcal{L}_\text{potential} \right) \nonumber \\
  =& \frac{Q_a^2}{2 C_a'} + \frac{Q_b^2}{2 C_b'} + \frac{\Phi_a^2}{2 L_a'} + \frac{\Phi_b^2}{2 L_b'} \nonumber \\
  &+ \underbrace{\frac{Q_a Q_b}{C_g'} - \frac{\Phi_a \Phi_b}{L_g'}}_{\text{coupling Hamiltonian }H_g}
  \, .
\end{align}

\levelstay{Dimensionless variables}

We will now simplify this Hamiltonian so that we can easily find its normal modes.
First, we define
\begin{align*}
  X_i &\equiv \frac{1}{\sqrt{2 \hbar}} \frac{1}{\sqrt{Z_i'}} \Phi_i \\
  Y_i &\equiv \frac{1}{\sqrt{2 \hbar}} \sqrt{Z_i'} Q_i
\end{align*}
where $Z_i' \equiv \sqrt{L_i' / C_i'}$, and we've added the constant $\hbar$ with dimensions of action to make $X$ and $Y$ dimensionless.
This entire analysis has been classical, and in the classical case $\hbar$ can be thought of as \emph{anything} with dimensions of action.
Of course, in the quantum case, we should simply think of $\Phi$, $Q$, $X$, and $Y$ as operators and $\hbar$ as Planck's constant.

In the new coordinates, the Hamiltonian is
\begin{align}
  H / \hbar
  =& \omega_a' \left(X_a^2 + Y_a^2 \right)
  +  \omega_b' \left(X_b^2 + Y_b^2 \right) \nonumber \\
  &+ 2 \frac{1}{C_g' \sqrt{Z_a' Z_b'}} Y_a Y_b
   - 2 \frac{\sqrt{Z_a Z_b}}{L_g'} X_a X_b
\end{align}
where $\omega_i' \equiv \sqrt{L_i' / C_i'}$ are called \textbf{partial frequencies} and play an important role in the analysis of the system, particularly when making approximations.

\levelstay{Rotating modes}

Finally we define
\begin{align}
  a &\equiv X_a + i Y_a \nonumber \\
  b &\equiv X_b + i Y_b
\end{align}
to arrive at
\begin{align}
  H / \hbar
  &= \omega_a' a^* a + \omega_b' b^* b \nonumber \\
  & - \frac{1}{2} \frac{1}{C_g' \sqrt{Z_a' Z_b'}} (ab + a^* b^* - a b^* - a^* b) \nonumber \\
  &- \frac{1}{2} \frac{\sqrt{Z_a' Z_b'}}{L_g'} (ab + a^* b^* + a^* b + a b^*)
  \, .
\end{align}
The stars indictate Hermitian conjugation, which in the classical case reduces to complex conjugation.
The coupling term can be reorganized in a very useful form:
\begin{align}
  H_g / \hbar =
  &- \left( a b + a^* b^* \right)
    \frac{1}{2} \left(
      \frac{1}{C_g' \sqrt{Z_a' Z_b'}} + \frac{\sqrt{Z_a' Z_b'}}{L_g'}
    \right) \nonumber \\
  &+ \left( a b^* + a^* b \right)
    \frac{1}{2} \left(
      \frac{1}{C_g' \sqrt{Z_a' Z_b'}} - \frac{\sqrt{Z_a' Z_b'}}{L_g'}
    \right) \nonumber \\
  &= -g_+ (ab + a^* b^*) + g_- (ab^* + a^* b)
\end{align}
where we defined
\begin{align}
  g_c \equiv \frac{1}{2} \frac{1}{C_g' \sqrt{Z_a' Z_b'}} &\qquad
  g_l \equiv \frac{1}{2} \frac{\sqrt{Z_a' Z_b'}}{L_g'} \nonumber \\
  \text{and} \qquad \qquad
  g_+ \equiv g_c + g_l &\qquad g_- \equiv g_c - g_l
  \, ,
\end{align}
which is the starting point in the main text.

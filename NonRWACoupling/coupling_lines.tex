\levelstay{Coupled transmission lines}

\quickfig{\columnwidth}{coupled_lines.pdf}
{a) A directional coupler.
A wave incident on the top left port (blue) is mostly transmitted to the top right port  (large blue), while a small fraction is coupled into the bottom left port (small blue).
Similarly, a wave injected into the bottom right port (red) is mostly transmitted to the bottom left port (large red) while a small fraction is coupled into the top right port (small red).
b) Capacitively and inductively coupled transmission lines.
Each line is modelled as an infinite ladder of lumped capacitors and inductors. Coupling capacitance per length $C_g$ and coupling inductance per length $L_g$ coupling the two lines.
}
{fig:directional_coupler}

Our investigation into electromagnetic resonators with simultaneous electric and magnetic coupling was motivated by an attempt to understand how a directional coupler works.
A directional coupler is a four port microwave device that directs signal flow as shown in Fig.~\ref{fig:directional_coupler}\,a.
A wave injected into the top left port is mostly transmitted through to the top right port, while a small fraction is coupled to the lower left port and none is coupled to the bottom right port.
The key element in a directional coupler is a pair of coupled transmission lines which we model as an infinite ladder of lumped capacitances and inductances as illustrated in Fig.~\ref{fig:directional_coupler}\,a.
The coupled lines act as a directional coupler if the right-moving wave in the top line, represented by $a_+$ and analogous to the blue wave in Fig.~\ref{fig:directional_coupler}, couples only to the left moving wave in the bottom line, represented by $b_+$.
Our original goal was to understand how to realize the desired couplings between the travelling wave amplitudes by proper choice of the coupling capacitance $C_g$ and coupling inductance $L_g$.
Expecting that analysis to be difficult, we instead first studied the case of coupled lumped element resonators first, as a warm-up problem.
We found that the travelling wave amplitudes in the case of coupled lines are analogous to the rotating mode amplitudes in the case of coupled resonators; making the RWA exact and eliminate the coupling between $a$ and $b^*$ in the case of coupled resonators corresponds to eliminating the coupling between $a_+$ and $b_-$ in the case of coupled transmission lines.
In other words, the same electri-magnetic balance required to make the RWA exact in the case of coupled resonators is what is required to make a pair of coupled transmission lines act as a directional coupler.

We now analyze the coupled lines in detail.
Line $a$ has inductance and capacitance \emph{per length} $L_a$ and $C_a$, and similarly for line $b$.
The lines have characteristic impedances $Z_a$ and $Z_b$.
The lines are coupled by a coupling inductance and capacitance per length $L_g$ and $C_g$.
We define amplitudes $a$ and $b$ by
\begin{align}
  a_\pm \equiv & \frac{V_a}{\sqrt{Z_a}} \pm \sqrt{Z_a} I_a \nonumber \\
  b_\pm \equiv & \frac{V_b}{\sqrt{Z_b}} \mp \sqrt{Z_b} I_b
  \, .
\end{align}
Note the similarity to the definitions of $a$ and $b$ in the case of coupled resonators.
Assuming a sinusoidal time dependence with frequency $\omega$, Kirchhoff's laws for the coupled line circuit can be expressed in matrix form
\begin{equation}
  \frac{d}{dx}
  \begin{pmatrix}
    a_+ \\ b_+ \\ a_- \\ b_-
  \end{pmatrix}
  = i
  \begin{pmatrix}
    \beta_a & -\chi & 0 & \kappa \\
    \chi & -\beta_b & -\kappa & 0 \\
    0 & - \kappa & -\beta_a & \chi \\
    \kappa & 0 & -\chi & \beta_b
  \end{pmatrix}
  \begin{pmatrix}
    a_+ \\ b_+ \\ a_- \\ b_-
  \end{pmatrix}
\end{equation}
where $\beta = \omega / v$ is the wave vector and
\begin{align}
  \kappa =& \frac{\omega}{2}
    \left( \frac{L_g}{\sqrt{Z_a Z_b}} - C_g \sqrt{Z_a Z_b} \right) \nonumber \\
  \chi =& \frac{\omega}{2}
    \left( \frac{L_g}{\sqrt{Z_a Z_b}} + C_g \sqrt{Z_a Z_b} \right)
  \, .
\end{align}
See Appendix \ref{appendix:coupled_lines} for details.
Note the resemblance between the matrix here and the matrix $M$ that arose in the analysis of coupled resonators: Each one has approximately block-diagonal form where the anti-diagonal contains only a single paramter, in this case $\kappa$.
Waves $a_+$ and $b_-$ decouple and the lines behave as a directional coupler if $\kappa=0$, which ocurs when
\begin{equation}
  Z_g \equiv \sqrt{\frac{L_g}{C_g}} = \sqrt{Z_a Z_b}
\end{equation}
i.e. when the coupling impedance is equal to the geometric mean of characteristic impedances of the twoi lines.

\levelstay{Introduction}

Analysis of driven and coupled resonators is pervasive throughout physics and engineering, and the case of electrical resonators has attracted particularly strong attention due to the success of superconducting circuits in quantum computing \cite{Krantz:guide2019, Blais:review2020}.
Theoretical treatments almost universally use the so-called ``rotating wave approximation'' (RWA) wherein certain ``fast'' terms in the system's equations of motion are neglected.
The RWA is quite useful in that it leads to simple equations of motion for driven and coupled resonators, and it greatly simplifies expressions for the normal mode frequencies of coupled resonators.
Nevertheless, the validity of the RWA seems to be poorly understood.
For example, the RWA has been found to fail spectacularly in the context of superconducting qubit readout where a weakly anharmonic resonator (transmon qubit) is coupled to a harmonic resonator.
In that case, transitions between quantum levels in the coupled system were found to be an important factor in the degradation of the readout process even though those transitions are absent under the RWA \cite{Sank:rotating_wave:2016}.
If breaking the assumptions of the RWA can lead to degraded performance in practical applications, a natural question is whether we can engineer the driving or coupling of electromagnetic resonators such that the RWA is exact, i.e. such that the terms typically dropped under the RWA are actually exactly zero.

In this paper, we show that by balancing the electric and magnetic aspects of driven and coupled circuits, we can design them so that the RWA is exact.
We first review the meaning of the RWA with an example where the RWA has an intuitive geometric interpretation: a magnetic moment subjected to time varying magnetic field.
Next, in the first main section of the paper, we provide mathematical analysis of a driven resonator, showing how proper balance beweteen the amplitudes and phases of simulateous capacitive (electric) and inductive (magnetic) makes the RWA exact.
Finally, in the second main section of the paper, we analyze two coupled resonators, showing that the RWA can be made exact if the resonators are coupled both capactively and inductively.

\levelstay{Power Spectrum}

We now finally come to the computation of physical power spectra. The question is, if we sample a real time signal and then compute the DFT, how do we turn this into a physical power spectrum? To find the answer we just compute the DFT of a real sinusoid and compare it to the known power of a sinusoid. Consider the signal $x(t) = A \cos (2 \pi f t)$ where $f$ is in the baseband and corresponds to the Fourier frequency $q$. Such a signal has total power $P=A^2/2$. If the signal has a finite linewidth $B$ then the power spectral density would be $S=A^2/2B$. This is the known reference against which we compare the result of the DFT. We already computed the DFT of a cosine with the result
\begin{displaymath}
X(k) = \frac{AN}{2} \left[ \delta_{k,q} + \delta_{k,N-q} \right]
\end{displaymath}
Since the power spectrum doesn't depend on phase, and should be proportional to $A^2$ it's clear that we should take the modulus square,
\begin{displaymath}
|X(k)|^2 = \frac{A^2N^2}{4} \left[ \delta_{k,q} + \delta_{k,N-q} \right]
\end{displaymath}
Throw out the aliased component since it doesn't have any extra information. Then, in order to get rid of the $N$ dependence and to get the correct numerical factor, multiply by $2/N^2$,
\begin{displaymath}
\frac{2}{N^2}|X(k)|^2 = \frac{A^2}{2} \delta_{k,q}
\end{displaymath}
A spectral density $S$ should have the property that $S$ multiplied by a span in frequency space gives the correct total power. The real frequency span between each point in DFT frequency space is $1/T$, so the total power ought to be $P=\frac{1}{T}\sum_k S(k)$. Applying this formula to our expression so far yields
\begin{displaymath}
\frac{1}{T}\frac{2}{N^2}\sum_k |X(k)|^2 = \frac{1}{T}\frac{A^2}{2}
\end{displaymath}
This differs from the correct power by a factor of $1/T$. Therefore, the correct power spectral density is
\begin{equation}
S(f) = \frac{2T}{N^2}|X(k=fT)|^2 \label{eq:powerSpectralDensity}
\end{equation}
Equation (\ref{eq:powerSpectralDensity}) is our most important result.
It tells you how to convert from a DFT computed from a measured time series, to a physical power spectral density.
Note that the units are exactly what they should be, $A^2/\textrm{Hz}$.
This formula is only exactly correct if you can be sure that there is no aliasing contaminating the baseband.
If this is not the case, then the spectral density you compute at a given frequency will be different from what was actually at that frequency.
If the measured signals are coherent then the measured spectral density may be too low or too high.
If the signal is noise, then the computed spectral density will always be greater than what actually existed in the measured signal.

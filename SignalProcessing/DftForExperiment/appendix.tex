\levelstay{Sums for Fourier Series power analysis}

The sum we want to compute is\begin{eqnarray*}
  S
  & = & \frac{1}{4}\sum_{k}\left[\textrm{sinc}(\xi-k)+\textrm{sinc}(\xi+k)\right.\\
  & + & \left.2\cos(2\pi\xi)\textrm{sinc}(\xi-k)\textrm{sinc}(\xi+k)\right]
  \, .
\end{eqnarray*}
We write this as
\begin{equation*}
  S=\frac{1}{4}\left[S_{1}+S_{1}+2\cos(2\pi\xi)S_{2}\right]
  \, .
\end{equation*}
The first sum we need to do is
\begin{equation*}
  S_1 = \sum_{k=-\infty}^{\infty}\textrm{sinc}(\xi-k)^{2}
  \, .
\end{equation*}
This is easy if we use the Poisson summation formula which states
that
\begin{equation*}
  \sum_{n}f(n-\xi)=\sum_{m}\tilde{f}(m)e^{-i2\pi\xi m}
  \, .
\end{equation*}
Using this formula we get
\begin{eqnarray*}
  S_{1} & = & \sum_{m}\widetilde{\textrm{sinc}^{2}\left(m\right)}e^{-i2\pi\xi m}\\
  S_{1} & = & \sum_{m}\textrm{tri}(m)e^{-i2\pi\xi m}
\end{eqnarray*}
where $\textrm{tri}(x)$ is a triangle function; ie, $\textrm{tri}(x)=x+1$ for $-1<x<0$ and $\textrm{tri}(x)=-x+1$ for $0<x<1$, and $\textrm{tri}(x)=0$ elsewhere.
Since $\textrm{tri}(m)$ is only nonzero for $m=0$ the sum is simply
\begin{equation*}
  S_1 = 1
  \, .
\end{equation*}
The second sum we need to do is
\begin{equation*}
  S_2 = \sum_{k=-\infty}^{\infty}\textrm{sinc}(\xi-k)\textrm{sinc}(\xi+k)
  \, .
\end{equation*}
For now we just write $f$ instead of $\textrm{sinc}$ to tidy up notation.
Since $\textrm{sinc}$ is an even function we can stick a minus sign in the first function like so,
\begin{equation*}
  S_2 = \sum_{k}f(k-\xi)f(k+\xi)
  \, .
\end{equation*}
From here we just crank,\begin{eqnarray*}
  S_2
  & = & \sum_{k}\int_{q}\int_{q'}\tilde{f}(q)e^{i2\pi q(k-\xi)}\tilde{f}(q')e^{i2\pi q'(k+\xi)}\\
  & = & \int_{q}\int_{q'}\tilde{f}(q)\tilde{f}(q')e^{-i2\pi\xi(q-q')}\sum_{k}e^{i2\pi k(q+q')}\\
  & = & \int_{q}\int_{q'}\tilde{f}(q)\tilde{f}(q')e^{-i2\pi\xi(q-q')}\sum_{n}\delta(n-q-q')\\
  & = & \sum_{n}\int_{q}\tilde{f}(q)\tilde{f}(n-q)e^{-i2\pi\xi(2q-n)}\\
  & = & \sum_{n}e^{i2\pi\xi n}\int_{q}\tilde{f}(q)\tilde{f}(n-q)e^{i2\pi q(-2\xi)}
  \, .
\end{eqnarray*}
Now we have to pay attention to what $\tilde{f}$ actually is.
It turns out that the Fourier transform of the sinc function is simply a constant function from -1/2 to +1/2 with unit amplitude.
This means that $\tilde{f}(q)\tilde{f}(n-q)$ is only nonzero for $n=0$, in which case it is equal to $\tilde{f}(q)$.
Therefore we get
\begin{eqnarray*}
  S_2
  & = & \int_{q}\tilde{f}(q)e^{i2\pi q(-2\xi)}\\
  & = & f(-2\xi)\\
  & = & \textrm{sinc}(-2\xi)\\
  & = & \textrm{sinc}(2\xi)
  \, .
\end{eqnarray*}
Plugging into the original expression gives
\begin{eqnarray*}
  S
  & = & \frac{1}{4}\left[1+1+2\cos(2\pi\xi)\textrm{sinc}(2\xi)\right]\\
  & = & \frac{1}{2}\left[1+\cos(2\pi\xi)\textrm{sinc}(2\xi)\right]\\
  & = & \frac{1}{2}\left[1+\textrm{sinc}(4\xi)\right]
\end{eqnarray*}
as stated in the main text.

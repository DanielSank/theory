%% LyX 1.6.6 created this file.  For more info, see http://www.lyx.org/.
%% Do not edit unless you really know what you are doing.
\documentclass[twocolumn,english,aps,prl]{revtex4}
\usepackage[T1]{fontenc}
\usepackage[latin9]{inputenc}
\usepackage{amstext}
\usepackage{graphicx}
\usepackage{esint}

\makeatletter
%%%%%%%%%%%%%%%%%%%%%%%%%%%%%% Textclass specific LaTeX commands.
\@ifundefined{textcolor}{}
{%
 \definecolor{BLACK}{gray}{0}
 \definecolor{WHITE}{gray}{1}
 \definecolor{RED}{rgb}{1,0,0}
 \definecolor{GREEN}{rgb}{0,1,0}
 \definecolor{BLUE}{rgb}{0,0,1}
 \definecolor{CYAN}{cmyk}{1,0,0,0}
 \definecolor{MAGENTA}{cmyk}{0,1,0,0}
 \definecolor{YELLOW}{cmyk}{0,0,1,0}
 }

%%%%%%%%%%%%%%%%%%%%%%%%%%%%%% User specified LaTeX commands.
\makeatother

\makeatother

\usepackage{babel}

\makeatother

\usepackage{babel}

\begin{document}

\title{Discrete Fourier Transform for Experimental Data}


\author{Daniel Sank}


\affiliation{University of California, Santa Barbara}


\date{22 October, 2009}
\begin{abstract}
Spectral analysis of data requires the use of the discrete Fourier
transform (DFT). In this note I will discuss the most basic and important
formulas needed to go from a time series of measured data values to
power spectrum with correct physical units. 
\end{abstract}
\maketitle

\section{The Four Fourier Transforms}

There are four different types of Fourier transform, corresponding
to the four different ways in which data may be sampled in time: data
may be known at discrete or continuous times, and over a finite or
infinite extent of time. The four transforms are 
\begin{itemize}
\item Fourier transform (FT) - continuous time, infinite extent 
\item Fourier series (FS) - continuous time, finite extent 
\item Discrete time Fourier transform (DTFT) - discrete time, infinite extent 
\item Discrete Fourier transform (DFT) - discrete time, finite extent 
\end{itemize}
Each of these transforms is useful in making calculations and in developing
theoretical understanding, but in the lab where we have discretely
sampled data over finite durations of time, only the DFT is applicable.


\section{DFT basics}

From the strictly mathematical point of view a series of experimental data is characterized by only one parameter, the number of points in the time series, which we denote by $N$. This means that any purely mathematical formulas we develop should involve only this one parameter.

\subsection{Definition}

Consider a time series of data written as $x(n)$ where $n$ lables the discrete time axis and runs from $0$ to $N-1$. The DFT, like all Fourier transforms, is based on the fact that $x(n)$ can be written as a sum of exponentials,

\begin{equation}
x(n)=\frac{1}{N}\sum_{k=0}^{N-1}X(k)\, e^{i2\pi nk/N} \label{eq:dftInverseDef}
\end{equation}

where the complex weights $X(k)$ are given by

\begin{equation}
X(k)=\sum_{n=0}^{N-1}x(n)\, e^{-i2\pi nk/N} \label{eq:dftDef}
\end{equation}

The indices $n$ and $k$ run from $0$ to $N-1$ which we call the \textbf{baseband}. This is a convention, and others are possible. The factor of $1/N$ in the first equation is also a convention (which both Matlab and numpy use). What matters is that the product of the prefactors in front of each sum is $1/N$. The various programming languages use different conventions, so make sure to always check this before using any particular DFT. Formulas given below assume the DFT specified by equations (\ref{eq:dftInverseDef}) and (\ref{eq:dftDef}), but the appropriate modifications of these for various DFT conventions should be clear once you understand the present case.



\subsection{General Properties}

Here we list the most important properties of the DFT. 
\begin{enumerate}
\item Any signal $x(n)$ is completely determined by knowledge of $X(k)$ for $k$ in the baseband $[0..N-1]$. However, if we have a series $x(n)$ in hand and we regard (2) as a formula that spits out $X(k)$ for $\emph{any}$ value of $k$, we can compute $X(k)$ for $k$ outside the baseband. It turns out that these new $X(k)$s are not independent of the ones in the baseband. To see this, first note that any integer $k$ can be written as $k=p+mN$ where $p \in [0..N-1]$ and $m$ is an integer. We then find that
\begin{eqnarray*}
e^{-i2\pi nk/N} &=& e^{-i2\pi n(p+mN)/N}\\
&=& e^{-i2\pi np/N}e^{-i2\pi mn}\\
&=& e^{-i2\pi np/N}
\end{eqnarray*}
since $n$ is an integer, and therefore
\begin{eqnarray*}
X(k) = \sum_{n=0}^{N-1} x(n)e^{-i2\pi nk/N} &=& \cdots \\
\cdots = \sum_{n=0}^{N-1} x(n)e^{-i2\pi np/N} &=& X(p)
\end{eqnarray*}
\textbf{In summary}, given a series $x(n)$, the weights $X(k)$ are only uniquely determined for $k \in [0..N-1]$. Fourier coefficients can be computed for $k$ outside this range, but they are all related to the ones in the baseband by translation in $k$ space:
\begin{equation}
X(k) = X(k+mN) \quad \textrm{for all integers }m \label{eq:translationalSymmetry}
\end{equation}
Therefore, it is sensible to ignore all DFT coefficients outside of the baseband.

\item From equation (\ref{eq:dftInverseDef}) we see that our signal $x(n)$ is built up of exponentials $\exp(i2\pi nk/N)$. We call the frequency $k/N$ the $k^{th}\textbf{ Fourier Frequency}$. The minimum Fourier frequency is 0 and the greatest is $(N-1)/N\approx1$. Note that these frequencies are to be understood in {}``index units,'' meaning that the exponential $\exp(i2\pi nq/N)$ goes through $q/N$ oscillations per step of the index $n$.

\item The DFT of an exponential is of fundamental importance. Given a signal $x(n)=\exp(i2\pi nq/N)$ the DFT is
\begin{eqnarray*}
X(k) &=& \sum_{n=0}^{N-1} e^{i2\pi nq/N}e^{-i2\pi nk/N} \\
X(k) &=& N \sum_{m=-\infty}^{\infty} \delta_{k,q+mN}
\end{eqnarray*}
We have a series of delta peaks separated from each nearest neighbor by $N$ in frequency space. This is in agreement with our previous discussion which said that all DFT coefficients separated by $N$ must be equal. As discussed, it makes sense to ignore the coefficients for $k$ outside the baseband, in which case the DFT of the exponential becomes
\begin{equation}
X(k) = \delta_{k,p} \label{eq:dftExponential}
\end{equation}
where $p$ is the \emph{unique} integer, inside the baseband, which is related to $q$ by translation by an integer multiple of $N$, ie. $p=q-mN$.


\item $\textbf{Aliasing - }$What happens if we have an exponential signal $x(n)=e^{i2\pi nq/N}$ when $q$ is outside the baseband? According to (\ref{eq:dftExponential}) we get a delta peak $\delta_{k,p}$ with $p$ in the baseband and $p = q-mN$ for some integer $m$. The value of $m$ doesn't matter, the point is that an exponential at a frequency $q$ outside the baseband has the same DFT as an exponential with Fourier frequency $p$ inside the baseband. A consequence is that if you measure a signal $A\exp\left[i2\pi nq/N\right]+B\exp\left[i2\pi n(q+mN)/N\right]$ the DFT will look exactly the same as if you had measured $(A+B)\exp\left[i2\pi nq/N\right]$. The indistinguishability of these signals is called $\textbf{aliasing}$, because the actual signal at higher frequency $q+mN$ \emph{looks} like the lower frequency $q$ in frequency space. What's going on here is that signals with Fourier frequency outside the baseband oscillate more than once per data point, but since we only sample once per data point these oscillations are hidden and the DFT sees the signal at a lower frequency.

In calculations it can be annoying to replace $q$ values outside the baseband by $p+mN$ explicitly. Instead, you can just do the replacement once the computation is complete. To get this right use the following \textbf{rules for computing DFTs}
\begin{equation}
\left[ \textrm{DFT}\left( e^{i2\pi nq/N} \right)\right](k) = \delta_{k,q}
\end{equation}
Then, if $q$ is outside the baseband, at the end of the calculation find the corresponding $p$ inside the baseband and make the replacement
\begin{equation}
\delta_{k,q} \rightarrow \delta_{k,p} \quad q=p+mN \label{eq:aliasReplacement}
\end{equation}

\end{enumerate}

\section{Physical Frequencies}

Our mathematical time series was parametrized by only the number of points $N$. Physically, however, we have another parameter associated to the time series, the total time spanned by the experiment $T$, or equivalently either the time step between points $\Delta t$, or its inverse, the sampling rate $f_{\textrm{s}}$. These quantities are related by
\begin{displaymath}
T = N\Delta t = \frac{N}{f_{\textrm{s}}}
\end{displaymath}
So that only $N$ and one of the three equivalent time scales are independent.

A data point at time $t$ is given by $x(n=t/\Delta t)$. We use this formula to go between the purely mathematical formulas involving $N$ to those involving physical quantities. Using these relations we can re-express an exponential at Fourier frequency $k$ as
\begin{displaymath}
\exp \left(i2\pi\frac{nk}{N}\right) = \exp\left(i 2 \pi \frac{tk}{\Delta t N} \right) = \exp \left( i 2\pi t\frac{k}{T} \right)
\end{displaymath}
It is now clear that the physical Fourier frequencies are $k/T$, and since $k$ goes in integer steps the frequency resolution is $1/T$. This means that the frequency resolution of our transform is determined by the total measurement time. Longer measurement time gives better frequency resolution. This also means that the lowest frequency we can measure above DC is $1/T$.

To summarize, the physical frequencies that result in a DFT performed on a time series with $N$ points taken over a total time $T$ are
\begin{displaymath}
\frac{1}{T}\left[0,1,\ldots,N-1\right]\quad\textrm{or}\quad\frac{f_{s}}{N}\left[0,1,\ldots,N-1\right]
\end{displaymath}

\section{Example}

We now give an example to tie everything together. Consider a signal
\begin{displaymath}
s(t) = \exp (i 2\pi f t)
\end{displaymath}
for $f=120\textrm{Hz}$, and imagine we measure it for one second with one hundred sample points,
\begin{displaymath}
N=100 \qquad T=1\mathrm{s}
\end{displaymath}
The sampled function is
\begin{displaymath}
s(n) = \exp(i2\pi fnT/N) = \exp(i2\pi 120n/N)
\end{displaymath}
The DFT is simply $X(k)=\delta_{k,120}$, but since the baseband only runs from 0 to 99, the result is aliased and we pick up the signal at $k=20$, or in frequency units $f=20/T=20\mathrm{Hz}$. Note that if we somehow knew that the signal only had frequency components in the range say from 100Hz to 150Hz, then we would be able to interpret the DFT peak at 20Hz as the alias of the real signal at 120Hz. That's how aliasing works: if you have knowledge about where your signal might be, then you can figure out where the DFT weight is coming from, but otherwise you only know that there's signal at one of the equivalent frequencies from the various Brillouin zones. Signal analysis equipment generally deals with this problem by filtering the input so that it's guaranteed that the incoming signal resides only in the baseband. Of course, in real life you don't have complex signals (unless you're using an IQ mixer!) so the situation is a little different. This explained in the next section.

\section{The Nyquist frequency}

Next we consider the maximum measureable frequency. The maximum value of $k$ is $N-1\approx N$. This suggests that the maximum measured frequency in index units is
\begin{displaymath}
f_{max}=\frac{k_{max}}{N}=\frac{N}{N}=1
\end{displaymath}
ie, the maximum measured frequency corresponds to an exponential that goes through 1 cycle per index. In physical units this frequency is given by
\begin{displaymath}
f_{max}=\frac{\text{1 cycle}}{\text{index}}\rightarrow\frac{\text{1 cycle}}{\Delta t}=f_{s}
\end{displaymath}
which says that the maximum measured frequency is just the sampling frequency. This is wrong. There are several different ways of understanding why, but it is essentially due to the fact that all of the signals we measure in the lab are made up of real numbers. Real sines and cosines have $\emph{negative}$ frequency components which are outside the baseband, and this leads to aliasing.

It turns out that the maximum measurable frequency is actually $f_s/2$, or $N/2T$ in index units. This frequency, located halfway up the baseband, is called the \textbf{Nyquist frequency}. To understand this we compute the DFT of a cosine signal. Take $x(n)=A\cos (2\pi nq/N+\phi)$ with $q$ in the baseband and compute the DFT coefficients,
\begin{eqnarray*}
x(n) & = & A\cos\left(2\pi nq/N+\phi \right)\\
x(n) & = & \frac{A}{2}\left[e^{i\phi}e^{i2\pi nq/N}+e^{-i\phi}e^{-i2\pi nq/N}\right]\\
\text{so}\qquad X(k) & = & \frac{AN}{2}\left[e^{i\phi}\delta_{k,q}+e^{-i\phi}\delta_{k,-q}\right]\\
X(k) & = & \frac{AN}{2}\left[e^{i\phi}\delta_{k,q}+e^{-i\phi}\delta_{k,N-q}\right]\end{eqnarray*}
We have weight at $q$ and, because of the negative frequency, also at $N-q$. The negative frequency component of the sinusoid has been aliased! Now compare this to the DFT of a sinusoid at $N-q$, $y(n)=\cos(2\pi n[N-q]/N)$,
\begin{eqnarray*}
y(n) & = & A\cos\left(2\pi n[N-q]/N+\phi\right)\\
y(n) & = & \frac{A}{2}\left[e^{i\phi}e^{-i2\pi nq/N}+e^{-i\phi}e^{i2\pi nq/N}\right]\\
\text{so}\qquad Y(k) & = & \frac{AN}{2}\left[e^{i\phi}\delta_{k,q}+e^{-i\phi}\delta_{k,-q}\right]\\
Y(k) & = & \frac{AN}{2}\left[e^{i\phi}\delta_{k,q}+e^{-i\phi}\delta_{k,N-q}\right]
\end{eqnarray*}
Because of the aliased terms, $X(k)$ and $Y(k)$ are identical, even though they come from cosines at different frequencies which are \emph{both in the baseband}. Each cosine shows up at the ``correct'' frequency and also at an aliased frequency which is the reflection of $q$ about the Nyquist frequency. The upshot is that when you're dealing with real time signals, their complex form always has negative frequency components which are aliased by the DFT. Therefore, the lower and upper halves of the baseband are dependent on one another. The exact dependence is a complex conjugate symmetry. You can easily prove for any \textbf{real time series} $x(n)$,
\begin{equation}
X(k)^* = X(N-k) \label{eq:conjugateSymmetry}
\end{equation}
Therefore, only half of the baseband contains unique information, as manifest in our examples $x(n)$ and $y(n)$.

The situation is now that only the first half of the baseband is uniquely determined. Therefore, in order to ensure that a DFT is a faithful representation of the measured time series, one must be sure that the measured signal has frequency components no higher than the Nyquist frequency. If this condition is satisfied, then the DFT coefficients found in the lower half of the baseband exactly match those present in the original signal \footnote{Actually, in this case it's possible to reconstruct the \emph{continuous} time signal just from the DFT coefficients. This fact is sometimes referred to as the Shannon Sampling Theorem. A nice proof is given in the optics book by J. Goodman.}. Of course if you know that the signal's frequency components are contained entirely in just the upper half of the baseband, then the coefficients in the lower half are still a faithful representation if you take into account the reflection and complex conjugate symmetry.

In practice, signal processing equipment usually uses a lowpass filter on the input to kill off everything above the Nyquist frequency. This is called an \textbf{anti-aliasing filter}. In that case, the lower half of the baseband can be read off without worrying about aliasing.


\section{Power Spectrum}

We now finally come to the computation of physical power spectra. The question is, if we sample a real time signal and then compute the DFT, how do we turn this into a physical power spectrum? To find the answer we just compute the DFT of a real sinusoid and compare it to the known power of a sinusoid. Consider the signal $x(t)=A\cos (w\pi f t)$ where $f$ is in the baseband and corresponds to the Fourier frequency $q$. Such a signal has total power $P=A^2/2$. If the signal has a finite linewidth $B$ then the power spectral density would be $S=A^2/2B$. This is the known reference against which we compare the result of the DFT. We already computed the DFT of a cosine with the result
\begin{displaymath}
X(k) = \frac{AN}{2} \left[ \delta_{k,q} + \delta_{k,N-q} \right]
\end{displaymath}
Since the power spectrum doesn't depend on phase, and should be proportional to $A^2$ it's clear that we should take the modulus square,
\begin{displaymath}
|X(k)|^2 = \frac{A^2N^2}{4} \left[ \delta_{k,q} + \delta_{k,N-q} \right]
\end{displaymath}
Throw out the aliased component since it doesn't have any extra information. Then, in order to get rid of the $N$ dependence and to get the correct numerical factor, multiply by $2/N^2$,
\begin{displaymath}
\frac{2}{N^2}|X(k)|^2 = \frac{A^2}{2} \delta_{k,q}
\end{displaymath}
A spectral density $S$ should have the property that $S$ multiplied by a span in frequency space gives the correct total power. The real frequency span between each point in DFT frequency space is $1/T$, so the total power ought to be $P=\frac{1}{T}\sum_k S(k)$. Applying this formula to our expression so far yields
\begin{displaymath}
\frac{1}{T}\frac{2}{N^2}\sum_k |X(k)|^2 = \frac{1}{T}\frac{A^2}{2}
\end{displaymath}
This differs from the correct power by a factor of $1/T$. Therefore, the correct power spectral density is
\begin{equation}
S(f) = \frac{2T}{N^2}|X(k=fT)|^2 \label{eq:powerSpectralDensity}
\end{equation}
Equation (\ref{eq:powerSpectralDensity}) is our most important result. It tells you how to convert from a DFT computed from a measured time series, to a physical power spectral density. Note that the units are exactly what they should be, $A^2/\textrm{Hz}$. This formula is only exactly correct if you can be sure that there is no aliasing contaminating the baseband. If this is not the case, then the spectral density you compute at a given frequency will be different from what was actually at that frequency. If the measured signals are coherent then the measured spectral density may be too low or too high. If the signal is noise, then the computed spectral density will always be greater than what actually existed in the measured signal.

\end{document}
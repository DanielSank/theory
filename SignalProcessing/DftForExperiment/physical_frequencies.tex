\levelstay{Physical Frequencies}

Our mathematical time series was parametrized by only the number of points $N$. Physically, however, we have another parameter associated to the time series, the total time spanned by the experiment $T$, or equivalently either the time step between points $\Delta t$, or its inverse, the sampling rate $f_{\textrm{s}}$. These quantities are related by
\begin{displaymath}
  T = N\Delta t = \frac{N}{f_{\textrm{s}}}
\end{displaymath}
so that only $N$ and one of the three equivalent time scales are independent.

A data point at time $t$ is given by $x(n=t/\Delta t)$. We use this formula to go between the purely mathematical formulas involving $N$ to those involving physical quantities. Using these relations we can re-express an exponential at Fourier frequency $k$ as
\begin{displaymath}
  \exp \left(i2\pi\frac{nk}{N}\right)
  = \exp\left(i 2 \pi \frac{tk}{\Delta t N} \right)
  = \exp \left( i 2\pi t\frac{k}{T} \right)
  \, .
\end{displaymath}
It is now clear that the physical Fourier frequencies are $k/T$, and since $k$ goes in integer steps the frequency resolution is $1/T$.
This means that the frequency resolution of our transform is determined by the total measurement time.
Longer measurement time gives better frequency resolution.
This also means that the lowest frequency we can measure above DC is $1/T$.

To summarize, the physical frequencies that result in a DFT performed on a time series with $N$ points taken over a total time $T$ are
\begin{displaymath}
  \frac{1}{T}\left[0,1,\ldots,N-1\right]\quad\textrm{or}\quad\frac{f_{s}}{N}\left[0,1,\ldots,N-1\right]
  \, .
\end{displaymath}

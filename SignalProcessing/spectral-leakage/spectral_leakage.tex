\documentclass[twocolumn]{article}

%Packages
\usepackage{amsmath}
\usepackage{amstext}
\usepackage{amssymb}
\usepackage{appendix}
\usepackage{coseoul}
\usepackage{enumerate}
\usepackage{graphicx}
\usepackage{import}
\usepackage{lscape}
\usepackage{modular}

\usepackage[pdfpagemode=UseNone,pdfstartview=FitH,colorlinks=true,linkcolor=blue,citecolor=blue,urlcolor=blue]{hyperref}
\usepackage[all]{hypcap}


% General physics constructs
\newcommand{\bra}[1]{\langle #1 |}
\newcommand{\ket}[1]{| #1 \rangle }
\newcommand{\braket}[2]{\langle #1|#2\rangle}
\newcommand{\bbraket}[3]{ \langle #1 | #2 | #3 \rangle }
\newcommand{\norm}[1]{\| #1\|}
\newcommand{\avg}[1]{\left \langle #1 \right \rangle}
\newcommand{\angavg}[1]{\left \langle #1 \right \rangle}
\newcommand{\abs}[1]{\left \lvert #1 \right \rvert}
\newcommand{\VS}{\textit{\textbf{V}}}
\newcommand{\Tr}{\textrm{Tr}}
\renewcommand{\Re}{\textrm{Re}}
\renewcommand{\Im}{\textrm{Im}}
\newcommand{\basis}[1]{\{\ket{#1}\}}

\newcommand{\omegaqubit}{\omega_{10}}

% Figures. Example usage:
% \quickfig{\columnwidth}{my_image}{This is the caption}{fig:my_fig}
\DeclareRobustCommand{\quickfig}[4]{
\begin{figure}
\begin{centering}
\includegraphics[width=#1]{#2}
\par\end{centering}
\caption{#3}
\label{#4}
\end{figure}
}

\DeclareRobustCommand{\quickwidefig}[4]{
\begin{figure*}[h]
\begin{centering}
\includegraphics[width=#1]{#2}
\par\end{centering}
\caption{#3}
\label{#4}
\end{figure*}
}


\title{Spectral Leakage}
\author{Daniel Sank \\ \small{Google Quantum AI}}
\date{Date: 2016}

\begin{document}
\maketitle


\section{Fourier series}

Consider a continuous signal $x(t)$ measured over a time interval $[-T/2, T/2]$.
This signal can be represented as a \textbf{Fourier series}
\begin{equation}
x(t) = \sum_{k=-\infty}^\infty c_k e^{i 2 \pi k t / T}
\end{equation}
where
\begin{equation}
c_k \equiv \frac{1}{T} \int_{-T/2}^{T/2} dt \, x(t) \, e^{-i 2 \pi k t / T} \, .
\end{equation}
The index $k$ gives the frequency of each component in units of cycles of the signal over the measured interval.
In other words, the term with $k=1$ has frequency $k/T$.
This is important to remember, the frequency resolution of a Fourier series is precisely the inverse of the measurement time $T$.
We call the frequencies $\{k/T\}$ the \textbf{Fourier frequencies}.

A complex sinusoid with frequency equal to one of the Fourier frequencies has a delta function Fourier series.
For example, the signal $s(t) = \exp \left[ i 2 \pi l t / T \right]$ has Fourier series $c_k = \delta_{kl}$.

\subsection{Shift in frequency}

Consider a signal $s(t)$ with Fourier series coefficients $s_k$.
Multiplying $s(t)$ by a complex exponential shifts the Fourier coefficients.
If we construct a new signal $u(t) \equiv s(t) \exp[i2\pi l t / T]$, then the Fourier series is
\begin{align}
u_k
&= \int_{-T/2}^{T/2} s(t) e^{i 2 \pi l t / T} e^{-i 2 \pi k t/ T} \nonumber \\
&= \int_{-T/2}^{T/2} s(t) e^{-i 2 \pi (k-l) t/ T} \nonumber \\
&= c_{k-l} \, .
\end{align}


\section{Non-commensurate frequency}

Consider a signal $x(t) = \exp[i 2 \pi \xi t / T]$ where $\xi$ is an arbitrary real number.
The Fourier series coefficients of this signal are
\begin{align}
c_k
&= \frac{1}{T} \int_{-T/2}^{T/2} dt \, e^{i 2 \pi \xi t / T} e^{-i 2 \pi k t / T} \nonumber \\
&= \frac{1}{T} \left. \left( \frac{\exp\left[i 2 \pi (\xi - k) t / T \right]}{i 2 \pi (\xi - k) / T} \right) \right|_{-T/2}^{T/2} \nonumber \\
&= \frac{1}{T} \left( \frac{\exp\left[i \pi (\xi - k) \right] - \exp\left[-i \pi (\xi - k) \right]}{i 2 \pi (\xi - k) / T} \right) \nonumber \\
&= \frac{\sin \left(\pi (\xi - k) \right)}{\pi (\xi - k)} \, .
\end{align}
This function has several important properties.
First, if $\xi$ is an integer, then all $c_k = \delta_{\xi k}$.
This is not surprising because if $\xi$ is an integer then $x(t)$ precisely matches one of the component functions $\exp\left(i 2 \pi k t / T \right)$.
This situation is illustrated by the black curve in Figure \ref{fig:leakage} where we have put $\xi=0$.
The underlying $\sin(\pi x)/(\pi x)$ function goes to zero at every $k$ except for $k=0$.
If $\xi$ is not an integer then the $\sin(\pi x) / (\pi x)$ function shifts such that $c_k \neq 0$ for \emph{all} $k$.
This is illustrated by the blue curve in Figure \ref{fig:leakage} where we have used $\xi = 0.2$.
Note that most of the amplitude of the signal sits near the real frequency $\xi$, but has leaked into neighboring frequency bins.
This phenomenon, wherein a signal at a precise frequency $\xi$ shows up with amplitude at other bins nearby is called \emph{spectral leakage}.
In summary, given a complex sinusoid at a frequency $\xi$ not commensurate with the measurement window, the Fourier series leaks into the bins near $\xi$ according to

\quickfig{0.8\columnwidth}{spectral_leakage.pdf}
{Fourier series of a complex sinusoid $\exp\left(i 2 \pi \xi t / T\right)$ for integer and non-integer frequencies. Lines show the $\sin(\pi x) / (\pi x)$ function for $\xi=0$ (black) and $\xi=0.2$ (blue). The dots indicate the values of the curves at the Fourier frequencies.}
{fig:leakage}


\section{Widows}



\end{document}
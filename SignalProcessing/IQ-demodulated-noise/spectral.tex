\section{Power spectra of IF signals}

Consider a noise process $\xi(t)$.
The mixer produces two IF signals $I(t)$ and $Q(t)$
\begin{align}
\xi_I(t) &= \xi(t) \cos(\Omega t) \\
\xi_Q(t) &= -\xi(t) \sin(\Omega t)
\end{align}
where $\Omega$ is the LO frequency.

\subsection{Spectra}

Let's look at the auto and cross correlations of $\xi_I$ and $\xi_Q$.
First consider the autocorrelation of $\xi_I$:
\begin{align}
\stataverage{\xi_I(t) \xi_I(t+\tau)}
&= \stataverage{ \xi(t) \xi(t+\tau) \cos(\Omega t) \cos(\Omega (t + \tau))} \\
&= \frac{1}{2} \stataverage{ \xi(t) \xi(t+\tau) \left( \cos(\Omega (2t + \tau)) + \cos(\Omega \tau) \right) } \, .
\end{align}
If we average over a time $\delta t$ such that $\delta t \gg 1/\Omega$, then the $\cos(\Omega (2t + \tau))$ term averages away.
If we additionally assume that $\xi$ is wide sense stationary, then
\begin{equation}
\stataverage{\xi(t) \xi(t+\tau)} = \stataverage{\xi(0)\xi(\tau)} \equiv K_{\xi}(\tau)
\end{equation}
and we get
\begin{equation}
K_{\xi_I}(\tau) \equiv \stataverage{\xi_I(0) \xi_I(\tau)} = \frac{1}{2} \stataverage{\xi(0)\xi(\tau)} \cos(\Omega \tau) \, .
\end{equation}
The the single sided spectral density (which we call the ``engineer's'' spectral density with a superscript $^e$) is
\begin{align}
S_{\xi_I}^e(\omega)
&= 2 \int_{-\infty}^\infty dt \, \cos(\omega \tau) K_{\xi_I}(\tau) \\
&= \int_{-\infty}^\infty dt \, \cos(\omega \tau) \cos(\Omega \tau) K_{\xi}(\tau) \\
&= \frac{1}{2} \int_{-\infty}^\infty dt \, \left( \cos((\Omega + \omega) \tau) + \cos((\Omega - \omega) \tau) \right) K_{\xi}(\tau) \\
&= \frac{1}{4} \left( S_{\xi}^e(\Omega - \omega) + S_{\xi}^e(\Omega + \omega) \right) \approx \frac{S_{\xi}^e(\Omega)}{2}
\end{align}
where in the last line we assumed $\omega \ll \Omega$ and that $S_{\xi}$ changes little over the range $\left[ \Omega - \omega , \Omega + \omega \right]$.

\subsection{Cross spectrum}

Using similar arguments we find that the cross spectrum of $I$ and $Q$ is approximately zero.
Intuitively this happens because when $\Omega \tau$ is an integer multiple of $2\pi$ $\xi_I$ has maximum mean square but $\xi_Q$ is identically zero.
Averaging over a time $\delta t$ such that $\delta t \gg 1 / \Omega$ leads to most of the spectrum of $\xi_I$ coming from times where $\xi_Q$ is zero and the result is that $\xi_I$ and $\xi_Q$ are uncorrelated.

\subsection{Summary}

The low frequency spectra of the IF signals $I(t)$ and $Q(t)$ is the same as the spectral density of $\xi$ near the LO.
The IF signals $I$ and $Q$ are approximately uncorrelated.

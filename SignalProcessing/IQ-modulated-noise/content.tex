\section{Reference}

The derivation here was taken from \cite{MattL:modulated_noise:2015}.

\section{Conventions}

There are two possibly confusing conventions in this document
\begin{itemize}
\item We define the correlation function as
\begin{equation}
R_{fg}(t) = \avg{f(t) g(t+\tau)^*}
\end{equation}
\item We define the spectral density as
\begin{equation}
S_{ff}(\omega) = \int_{-\infty}^\infty R_{ff}(t) e^{i \omega t}\,dt
\end{equation}
\end{itemize}

If the processes are wide sense stationary the dependence of $t$ vanishes and we write e.g. $R_{ff}(\tau)$ without the $t$.

\section{Definitions}

We define the correlation function of two processes $f$ and $g$ by
\begin{equation}
R_{fg}(t, \tau) \equiv  \langle f(t) g(t+\tau)^* \rangle \, .
\end{equation}

\section{Calculation}

The output of the RF port of the IQ mixer is
\begin{equation}
V(t) = I(t) \cos(\Omega t) - Q(t) \sin(\Omega t) \, .
\end{equation}
To understand this signal we must construct the process $Z(t)$ defined by
\begin{equation}
Z(t) = I(t) + i Q(t) \, .
\end{equation}
The process $V(t)$ can then be written as
\begin{align}
V(t)
&= \Re \left[ Z(t) e^{i \Omega t} \right] \\
&= \frac{1}{2} \left[ Z(t) e^{i \Omega t} + Z(t)^* e^{-i \Omega t} \right] \, .
\end{align}
We can now write the correlation function of $V$ as ($V$ is real valued and we assume the $Z$ is wide sense stationary because $I$ and $Q$ are)
\begin{align}
R_{VV}
&= \langle V(t) V(t+\tau) \rangle \\
&= \frac{1}{4} \langle
Z(t)Z(t+\tau) e^{i\Omega (2t + \tau)}
+ Z(t)^* Z(t+\tau)^* e^{-i\Omega (2t + \tau)} \\
& + Z(t) Z(t+\tau)^* e^{-i \Omega \tau}
+ Z(t)^* Z(t+\tau) e^{i \Omega \tau}
\rangle \\
&= \frac{1}{2} \Re \left[ R_{ZZ}(\tau) e^{-i \Omega \tau}
+ R_{ZZ^*}(\tau) e^{i \Omega (2t + \tau)} \right]
\end{align}
Because $R_{VV}$ depends on $t$, it is impossible in general to construct a spectral density for $V$.
However, we can see that if $R_{ZZ^*}$ were to vanish, then the $t$ dependence would be gone and we would be able to define a spectral density.
It turns out that this happens in a common practical case, as we now show.

The cross-correlation of $Z$ and $Z^*$ is\footnote{We neglect the $*$'s on the subscripts here because $I$ and $Q$ are real.}
\begin{align}
R_{ZZ^*}\
&= \langle (I(t) + i Q(t))(I(t+\tau) + i Q(t+\tau)) \rangle \\
&= R_{II}(\tau) - R_{QQ}(\tau) + i \left( R_{IQ}(\tau) + R_{IQ}(-\tau) \right)
\end{align}
where we've used the fact that $R_{QI}(\tau) = R_{IQ}(-\tau)$.
We now see that if $R_{II} = R_{QQ}$ and $R_{IQ}(\tau) = - R_{IQ}(-\tau)$, then $R_{ZZ^*} = 0$.
In the case that the noise injected in the the $I$ and $Q$ ports have the same spectral density (e.g. they come from the same model DAC chip) and are uncorrelated (e.g. the noise coming out of the two DAC chips is uncorrelated) these conditions are satisfied!

Now let us write out $R_{ZZ}$,
\begin{align}
R_{ZZ}(\tau)
&= \langle Z(t) Z(t+\tau)^* \rangle \\
&= \langle (I(t) + i Q(t))(I(t+\tau) - i Q(t+\tau)) \rangle \\
&= R_{II}(\tau) + R_{QQ}(\tau) - i \left( R_{IQ}(\tau) - R_{IQ}(-\tau) \right) \, .
\end{align}
With the condition above satisfied this becomes (using $R_{QQ} = R_{II}$)
\begin{equation}
R_{ZZ}(\tau) = 2 \left( R_{II}(\tau) - i R_{IQ}(\tau) \right) \, .
\end{equation}
Using this form we can write down the autocorrelation of $V$
\begin{align}
R_{VV}(\tau)
&= \frac{1}{2} \left( R_{ZZ}(\tau) e^{-i \Omega \tau} \right) \\
&= R_{II}(\tau) \cos(\Omega \tau) - R_{IQ}(\tau) \sin(\Omega \tau) \, . 
\end{align}
Finally, Fourier transforming gives
\begin{align}
S_{VV}(\omega) &=
\frac{1}{2} \left[ S_{II}(\omega - \Omega) + S_{II}(\omega + \Omega) \right]
+ \frac{1}{2i} \left[ S_{IQ}(\omega - \Omega) - S_{IQ}(\omega + \Omega) \right] \\
&= \frac{1}{2}S_{II}(\omega-\Omega) \, .
\end{align}
where in the last line we assumed that $I$ and $Q$ are uncorrelated and that $S_{II}(\omega + \Omega) = 0$, which is true when $\Omega$ is a large RF carrier frequency and $I$ is a lower frequency drive line.

\section{Interpretation for upconverting IQ mixer}

Suppose we have a spectral density $S_{I,Q}(\omega)$ on the $I$ and $Q$ lines driving an IQ mixer, and an LO frequency of $\Omega$.
Model the mixer as
\begin{equation}
\text{RF}(t) = I(t)\cos(\Omega t) - Q(t)\sin(\Omega t) \, .
\end{equation}
Then the spectral density of the RF output is $S_\text{RF}(\omega) = (1/2) S_{I,Q}(\omega - \Omega)$.
Of course, in a real application you have to take the conversion loss of the mixer into account!


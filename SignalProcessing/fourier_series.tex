%% LyX 1.6.6 created this file.  For more info, see http://www.lyx.org/.
%% Do not edit unless you really know what you are doing.
\documentclass[english]{article}
\usepackage[T1]{fontenc}
\usepackage[latin9]{inputenc}
\usepackage[letterpaper]{geometry}
\geometry{verbose,tmargin=2cm,bmargin=2cm,lmargin=2cm,rmargin=2cm}
\usepackage{amstext}
\usepackage{esint}
\usepackage{babel}

\begin{document}

\title{Signal Analysis}


\author{Daniel Sank}

\maketitle

\section{Introduction}

Most of us know how to deal with Fourier transforms (FT) and Fourier
series (FS). The Fourier transform is the correct way to describe
a continuous signal defined over the entire real line, and the Fourier
series is the correct way to describe a continuous signal defined
over an interval. In the lab, however, data are measured at discrete
times and for finite total duration. Therefore, you can't really use
the FT or the FS to describe real data. The correct description for
discrete data over a finite interval is the $\textbf{discrete Fourier transform}$
(DFT). The purpose of this document is to explain the DFT by describing
it as a modification of the FS. I'm taking that approach simply so
that I can start from something you already understand.

I think the proper order to understand these transforms is FT, FS,
(DTFT optional), DFT. I'm assuming that the reader has basic knowledge
of the FT and FS, but has never looked at the DFT or the DTFT before.
If you need a refresher on the FT you can take a look at the Wikipedia
article, or for a ground up approach take a look at $\textit{Linear Algebra and the Fourier Transform}$.


\section{Fourier Series}

We begin with a complete study of the Fourier series, the transform
appropriate for continuous signals over a finite interval. Once you
understand the subtleties of the Fourier series the discrete Fourier
transform will be a relatively simple step.


\subsection{Fundamentals}


\subsubsection{Definition}

Given a function $F(t)$ defined over an interval $t\in[0,T]$ the
$\textbf{Fourier coefficients}$ of that function are defined as\[
\tilde{f}(n)=\frac{1}{T}\int_{t=0}^{T}f(t)e^{-i2\pi nt/T}\, dt\]
The original function can be expressed as a $\textbf{Fourier series}$
involving exponentials and the Fourier coefficients,\[
f(t)=\sum_{n=-\infty}^{\infty}\tilde{f}(n)e^{i2\pi nt/T}\]
This pair of equations hold true for any function defined on the interval
$[0,T]$ as long as it has a finite number of discontinuities. Notice
that the sum involves oscillating functions $e^{i2\pi nt/T}$ which
have frequencies $n/T$. We refer to these frequencies as the $\textbf{Fourier frequencies}$.
Note that for real signals we have $\tilde{f}(-n)=\tilde{f}(n)^{*}$.
Using this equality we have\begin{eqnarray*}
f(t) & = & \tilde{f}(0)+\sum_{n=1}^{\infty}\tilde{f}(n)e^{i2\pi nt/T}+\tilde{f}(n)^{*}e^{-i2\pi nt/T}\\
 & = & \tilde{f}(0)+2\sum_{n=1}^{\infty}\Re\left[\tilde{f}(n)\right]\cos\left(2\pi nt/T\right)-\Im\left[\tilde{f}(n)\right]\sin\left(2\pi nt/T\right)\end{eqnarray*}


Before we move on to real things, there's a change of variables and
notation that we should do to make our lives simpler. We've defined
everything over a time interval of length $T$. It's a lot better
to work in the dimensionless variable $t/T$. If we make this change
but $\emph{retain the variable name t}$, the transform pair looks
like this,\begin{eqnarray*}
\tilde{f}(n) & = & \int_{t=0}^{1}f(t)e^{-i2\pi nt}\, dt\\
f(t) & = & \sum_{n=-\infty}^{\infty}\tilde{f}(n)e^{i2\pi nt}\end{eqnarray*}
This is an abuse of notation, both because we're using $t$ for a
dimensionless quantity, and because this function $f$, which is a
function of a dimenisonless variable, isn't really the same function
as before. We'll use this notation anyway because it's handy. You
just have to remember that $t$ is in units of $T$ and that our frequencies
are therefore in units of $1/T$.


\subsubsection{Power}

The total power in our signal is defined as\[
\text{Power}=\int_{0}^{1}|f(t)|^{2}\, dt\]
Using the series representation we have\begin{eqnarray*}
\text{Power} & = & \int\left[\sum_{n}\tilde{f}(n)e^{i2\pi nt}\right]\left[\sum_{m}\tilde{f}(m)^{*}e^{-i2\pi t}\right]\, dt\\
 & = & \sum_{n,m}\tilde{f}(n)\tilde{f}(m)^{*}\int_{0}^{1}e^{i2\pi(n-m)t}\, dt\\
 & = & \sum_{n,m}\tilde{f}(n)\tilde{f}(m)^{*}\delta_{n,m}\\
\text{Power}=\int_{0}^{1}|f(n)|^{2}\, dt & = & \sum_{n}|f(n)|^{2}\end{eqnarray*}
which is called Parseval's theorem. It just says that you can find
the power in your signal by summing the mod squares of the Fourier
coefficients. This motivates the definition of the $\textbf{physicist's power spectrum}$
as\[
S_{p}(n)=|\tilde{f}(n)|^{2}\]
which gives the power at each positive and negative frequency.

Let's look at a particular case. Consider a sinusoidal signal at the
$m^{\textrm{th}}$ Fourier frequency: $f(t)=A\cos\left(2\pi mt\right)$.
The power of this signal, found by integrating its square over one
period, is\begin{eqnarray*}
\textrm{Power} & = & \int_{t=0}^{1}\left[A\cos\left(2\pi mt\right)\right]^{2}\, dt\\
\textrm{Power} & = & A^{2}/2\end{eqnarray*}
This power can also be found from the Fourier coefficients. Compute
the Fourier components for this signal, assuming $m\neq0$,\begin{eqnarray*}
\tilde{f}(n) & = & \int_{0}^{1}A\cos\left(2\pi nt\right)e^{-i2\pi nt}\, dt\\
 & = & \frac{A}{2}\int\left[e^{i2\pi mt}+e^{-i2\pi mt}\right]e^{-i2\pi nt}\\
 & = & \frac{A}{2}\left[\delta_{n,m}+\delta_{n,-m}\right]\end{eqnarray*}
Then $S_{p}=\left(\frac{A}{2}\right)^{2}\left[\delta_{n,m}+\delta_{n,-m}\right]$.
You can see that the physicist's power spectrum has components at
$m$ and $-m$. If you sum these components you get $A^{2}/2$, the
total power in the sinusoid. This shows that the power of a sinusoid
at frequency $m$ shows up in the physicist's spectrum as $m$ and
$-m$. If you don't want to think about negative frequencies you can
invent the $\textbf{engineer's power spectral density}$ defined as\[
S_{e}(n)=2S_{p}(n)\]
for positive $n$ only, and then say that the power of the sinusoid
at frequency $n$ is just $S_{e}(n)$. You can get away with this
when you have a real (no imaginary part) signal because $\tilde{f}(-n)=\tilde{f}(n)$
which means that $S_{p}(-n)=S_{p}(n)$, and therefore,\[
\textrm{Power at frequency }n=S_{p}(n)+S_{p}(-n)=2S_{p}(n)=S_{e}(n)\]



\subsection{Noncommensurate Frequencies}

We saw that the Fourier coefficients for a cosine wave with amplitude
$A$ at frequency $m$ are $\tilde{f}(n)=\frac{A}{2}\left(\delta_{n,m}+\delta_{n,-m}\right)$.
This is nice because it means that if a signal has a frequency component
at the $m^{\textrm{th}}$ frequency, then that signal component shows
up only exactly in the $m^{\textrm{th}}$ Fourier coefficient. In
short, that's the whole point of a Fourier decomposition. But what
happens if a signal has a component at a frequency that's $\emph{not}$
one of the Fourier frequencies? Let's compute the coefficients of
the signal\[
f(t)=A\cos\left(2\pi\nu t\right)\]
where $\nu$ is a number that could be an integer (Fourier frequency)
or not.\begin{eqnarray*}
\tilde{f}(n) & = & \int_{t=0}^{1}A\cos\left(2\pi\nu t\right)e^{-i2\pi nt}\, dt\\
\tilde{f}(n) & = & \frac{A}{2}\int\left[e^{i2\pi(\nu-n)t}+e^{-i2\pi(\nu+n)}\right]\, dt\\
\tilde{f}(n) & = & \frac{A}{2}\left[e^{i2\pi(\nu-n)/2}\textrm{sinc}\left(\nu-n\right)+e^{-i2\pi(\nu+n)/2}\textrm{sinc}\left(\nu+n\right)\right]\end{eqnarray*}
where $\textrm{sinc}(x)\equiv\sin(\pi x)/\pi x$. It looks like the
Fourier coefficients of the sinusoid are nonzero everywhere instead
of at just $\pm\nu$. But note that if $\nu$ is an integer then $\textrm{sinc}\left(\nu-n\right)=\delta_{\nu-n}$
because the sinc function vanishes at all integers other than zero.
In this case the coefficients $\tilde{f}(n)$ are exactly the same
as what we found previously. You can think of it as if the sinc function
were there the whole time, but because it was centered at an integer,
its existence was invisible to our Fourier coefficients.


\subsubsection{Leakage}

So, what happens if $\nu$ is not an integer? Then $\textrm{sinc}(\nu-n)$
is nonzero for $\emph{all}$ integers $n$ and the Fourier coefficients
of our sinusoid are nonzero for every frequency! For example, if we
had a signal $\sin(2\pi*2.5*t)$, then $\tilde{f}(2)=0.63$. This
is called $\textbf{spectral leakage}$ because energy at frequency
2.5 is leaking into the nearby frequency of 2. Since the sinc function
decays for large argument, you can see that the energy of noncommensurate
frequencies leaks only into other nearby Fourier frequencies. How
much power is actually added into our acquired signal by this noncommensurate
frequency component? We can find out by simply doing the integral
in the time domain,\begin{eqnarray*}
\textrm{Power} & = & \int_{t=0}^{1}A^{2}\cos\left(2\pi\nu t\right)^{2}\, dt\\
\textrm{Power} & = & \frac{A^{2}}{2}\left(1+\textrm{sinc}\left(4\nu\right)\right)\end{eqnarray*}
where, again, $\textrm{sinc}(x)\equiv\sin(\pi x)/\pi x$. Here again
we see that if $\nu$ is an integer then the sinc function vanishes
and the power is what we expect for a sinusoid, namely $A^{2}/2$
(notice that for $\nu=0$ the power is $A^{2}$, which is correct
for the DC component). If $\nu$ is not an integer then the power
that shows up in our measured signal differs from the signal's actual
power. The error is less severe for frequencies farther away from
DC, due to the decay of the sinc function.%
\footnote{It's a fun, if challenging, exercise to show that you can recover
this expression for the power of a noncommensurate sinusoid by explicitly
summing the squares of the Fourier coefficients computed above. If
you want to do it you may want to look up the {}``Poisson summation
formula.''%
}


\subsubsection{Windowing}

We can gain some insight into what's going on with these noncommensurate
frequencies by connecting our Fourier series to the Fourier transform.
The integral defining the Fourier coefficients of a function $f$
is exactly the same as the integral you would do to get the Fourier
$\emph{transform}$, at frequency $n$, of $f$ multiplied by a square
function: \[
\tilde{f}(n)=\int_{t=0}^{1}f(t)e^{-i2\pi nt}\, dt=\int_{t=-\infty}^{\infty}f(t)*\textrm{rect}_{[0,1]}e^{-i2\pi nt}\, dt\]
Since the FT of a square function is a sinc, $\tilde{f}(n)$ will
wind up being the convolution of the Fourier transform of $f$ with
a sinc function. If the signal is periodic with the period of the
tophat, then it's Fourier transform has a bunch of sinc functions
sitting at equally spaced frequencies in such a way that they don't
interfere with one another because they sit at each others zero points.
If the signal has a component that doesn't have this periodicity,
then the sinc function from that sinusoid will corrupt nearby frequencies
in frequency space. Some figures here would be really nice.

The sinc function is pretty undesirable as a smearing function because
it decays rather slowly. It would be nice the function involved in
our spectral leakage decayed faster. Now that we understand the existence
of the sinc as being due to the fact that we 'windowed' our original
signal with a square function it is clear that if we want a different
smearing function we have to use a different window. This is what
is actually done with real data: you multiply your finite duration
time trace by a $\textbf{windowing function}$ that goes to zero at
the endpoints of the time interval. This eliminates the discontinuities
at the endpoints and improves life in the frequency domain. Of course,
as you expect, multiplying your data by a window has other effects
that need to be understood. It's all fairly simple now that we went
through the case with the square window. We know that in the frequency
domain, the noncommensurate frequencies will get smeared into neighboring
frequencies with weight given by the FS of the window function.

Let's look at all of this quantitatively. Let's assume we're measuring
a finite duration segment of a continuous signal. That signal can
be written as a Fourier integral\[
f(t)=\int\tilde{f}(\nu)e^{i2\pi\nu t}\, d\nu\]
We can now write down exactly what our Fourier coefficients will look
like, adding in a window function which we leave unspecified for now\begin{eqnarray*}
\tilde{f}(n) & = & \int_{t=0}^{1}f(t)w(t)e^{-i2\pi nt}\, dt\\
 & = & \int_{t=0}^{1}\left[\int\tilde{f}(\nu)e^{i2\pi\nu t}\right]w(t)e^{-i2\pi nt}\, dt\\
 & = & \int d\nu\tilde{f}(\nu)\int_{t=0}^{1}e^{-i2\pi(n-\nu)t}w(t)\, dt\\
\tilde{f}(n) & = & \int d\nu\tilde{f}(\nu)\tilde{w}(n-\nu)\end{eqnarray*}

\end{document}

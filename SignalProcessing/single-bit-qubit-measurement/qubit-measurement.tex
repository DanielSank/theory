\documentclass[twocolumn]{article}

%Packages
\usepackage{amsmath}
\usepackage{amstext}
\usepackage{amssymb}
\usepackage{appendix}
\usepackage{coseoul}
\usepackage{graphicx}
\usepackage{import}
\usepackage{lscape}
\usepackage{modular}

\usepackage[pdfpagemode=UseNone,pdfstartview=FitH,colorlinks=true,linkcolor=blue,citecolor=blue,urlcolor=blue]{hyperref}
\usepackage[all]{hypcap}


% General physics constructs
\newcommand{\bra}[1]{\langle #1 |}
\newcommand{\ket}[1]{| #1 \rangle }
\newcommand{\braket}[2]{\langle #1|#2\rangle}
\newcommand{\bbraket}[3]{ \langle #1 | #2 | #3 \rangle }
\newcommand{\boltzmann}{k_b}

% Common math
\newcommand{\norm}[1]{\left \lvert #1 \right \rvert}
\newcommand{\abs}[1]{\left \lvert #1 \right \rvert}  % These two are redundant. Consider removing one.

\newcommand{\avg}[1]{\left \langle #1 \right \rangle}  % Should get rid of this, as "average" isn't specific.
\newcommand{\angavg}[1]{\left \langle #1 \right \rangle}

\newcommand{\VS}{\textit{\textbf{V}}}
\newcommand{\Tr}{\textrm{Tr}}
\renewcommand{\Re}{\textrm{Re}}
\renewcommand{\Im}{\textrm{Im}}
\newcommand{\basis}[1]{\{\ket{#1}\}}

% Quantum
\newcommand{\nboseeinstein}{n_\text{BE}}
\newcommand{\gammaup}{\Gamma_\uparrow}
\newcommand{\gammadown}{\Gamma_\downarrow}
\newcommand{\gammaupdown}{\Gamma_{\uparrow \downarrow}}
\newcommand{\gammaemission}{\Gamma_\text{loss}}
\newcommand{\qualityfactoremission}{Q_{d,\text{loss}}}

% Qubits
\newcommand{\omegaqubit}{\omega_{10}}

% Circuits
\newcommand{\impedance}{Z_0}
\newcommand{\resistorsource}{R_s}
\newcommand{\vsource}{V_s}
\newcommand{\vsourcerms}{V_{s,\text{rms}}}
\newcommand{\vloadrms}{V_{l,\text{rms}}}

% Signals and noise
\newcommand{\psdsingle}{S_\text{ss}}
\newcommand{\psddouble}{S_\text{ds}}
\newcommand{\noiseavailable}{S_{p,a}^e}
\newcommand{\spectralengineer}{S^e}
\newcommand{\spectralsymmetric}{S^\text{symm}}
\newcommand{\spectralattenuator}{\spectralengineer_{\poweravailable, \text{att.}}}

% Microwaves
\newcommand{\vright}{V_+}
\newcommand{\vleft}{V_-}
\newcommand{\iright}{I_+}
\newcommand{\ileft}{I_-}
\newcommand{\poweravailable}{P_a}

% Figures. Example usage:
% \quickfig{\columnwidth}{my_image}{This is the caption}{fig:my_fig}
\DeclareRobustCommand{\quickfig}[4]{
\begin{figure}
\begin{centering}
\includegraphics[width=#1]{#2}
\par\end{centering}
\caption{#3}
\label{#4}
\end{figure}
}

\DeclareRobustCommand{\quickwidefig}[4]{
\begin{figure*}[h]
\begin{centering}
\includegraphics[width=#1]{#2}
\par\end{centering}
\caption{#3}
\label{#4}
\end{figure*}
}

\DeclareRobustCommand{\quickfigcentered}[4]{
  \begin{figure}
  \makebox[\textwidth][c]{\includegraphics[width=#1]{#2}}
  \caption{#3}
  \label{#4}
  \end{figure}
}


\author{Daniel Sank \\ \small{Google Quantum AI}}

\title{Spectral density of frequency fluctuations using a qubit as a 1-bit detector}
\date{March 2016}

\begin{document}

\maketitle

\section{Background}

We would like to measure flux noise in a qubit at relatively low frequency, but still with as high bandwidth as possible.
We use the Ramsey fringe but not in the usual way.


To understand the strategy, we first investigate a single shot of a Ramsey experiment.
We start the experiment at time $t$.
Starting in state $\ket{0}$, we do a with a $\pi / 2$ pulse about the $Y$ axis, yielding the state
\begin{displaymath}
\frac{1}{\sqrt{2}} \left( \ket{0} + \ket{1} \right) \, .
\end{displaymath}
We then wait a time $\tau$; during this time the qubit acquires a noisy phase $\delta \phi$ given by
\begin{equation}
\delta \phi = \int_t^{t+\tau} \delta \omega(t') dt'
\end{equation}
where $\delta \omega(t)$ is the time dependent noisy frequency of the qubit.
If the frequency fluctuations come from flux noise then we have
\begin{equation}
\delta \omega(t) = \frac{d\omega}{d \Phi} \delta \Phi (t)
\end{equation}
but we just work in terms of $\delta \omega$.
After the qubit acquires this noisy phase $\delta \phi$, we do a second $\pi / 2$ pulse, but this time we do it about the $X$ axis.
If $\delta \phi = 0$, then the rotation about the $X$ axis does nothing and the final state is
\begin{equation}
\frac{1}{\sqrt{2}} \left( \ket{0} + \ket{1} \right) \, .
\end{equation}
On the other hand, if $\delta \phi = +\pi/2$ then after the second $\pi/2$ pulse we wind up in $\ket{0}$.
If $\delta \phi = -\pi/2$ then after the second $pi/2$ pulse we would wind up in $\ket{1}$.
In this way, the measurement of the qubit state, yielding either $\ket{0}$ or $\ket{1}$ gives some kind of information about $\delta \phi$.
This is kind of interesting: the qubit is a 1-bit detector for the continuous angle $\delta \phi$.
What happens for values of $\delta \phi$ between $-\pi / 2$ and $\pi / 2$?
Using basics of the Bloch sphere one finds that the probability $P$ to find the qubit in $\ket{0}$ at the end of the procedure is
\begin{equation}
P = \frac{1}{2} \left( 1 + \sin (\delta \phi) \right) \, .
\end{equation}
We set up the construction specifically to get this $\sin$ function because it makes $P$ maximally sensitive to $\delta \phi$.

If we were to run the argument again but this time starting the qubit in $\ket{1}$, we would find that $P$ is the probability to find the qubit in $\ket{1}$ again in the end.
So really we should think of $P$ as the probability to find the qubit in the \emph{same} state that it started in.

We repeat this measurement over and over we build a sequence of measured states $\{ \Psi_n \}$.
We then construct a new sequence $\{ x_n \}$, where $x_n=1$ if $\Psi_n = \Psi_{n-1}$ and $x_n = -1$ if $\Psi_n \neq \Psi_{n-1}$.
In other words, $x_n=1$ if there is no state switch, and $x_n = -1$ if there is a state switch.
The basic idea is that the power spectral density of the sequence $\{ x_n \}$ tells us \emph{something} about the power spectral density of $\delta \phi$, and therefore it tells us something about the power spectral density of $\delta \omega$.
Obviously, if we know the power spectral density of $\delta \omega$ then we know the power spectral density of $\Phi$, which is the goal.

However, we must figure out exactly \emph{what} the power spectral density of $\{ x_n \}$ tells us.
The probability distribution for $x_n$ is
\begin{align}
P_{x_n}(x) = &
\delta (x - 1) \frac{1}{2}( 1 + \sin(\delta \phi_n)) \nonumber \\
+& \delta (x + 1) \frac{1}{2}( 1 - \sin(\delta \phi_n) \label{eq:P_x_n}
\end{align}
where
\begin{equation}
\delta \phi_n = \int_{t_n}^{t_n + \tau} \delta \omega(t') dt' \, .
\end{equation}
Our strategy is to compute the spectral density of $\{ x_n \}$ and see how it depends on the spectral density of $\delta \omega$.
Note that if we had no noise at all, i.e. $\delta \phi = 0$, then there is always a probability of $1/2$ to get a state switch and the sequence $\{ x_n \}$ would have exactly the statistics of a sequence of uncorrelated coin flips, i.e. it would be white noise.
Fluctuations in the qubit frequency coming from flux noise should add additional noise on top of this white background.
Our job is to calculate the relation between the spectral density of qubit frequency fluctuations and the measured spectral density of $\{ x_n \}$.

\section{Calculation}

The discrete Fourier transform of $\{ x_n \}$, is
\begin{equation}
\tilde{x}_k \equiv \sum_{n=0}^{N-1} x_n e^{-i 2 \pi n k / N} \, . \label{eq:DFT}
\end{equation}
Each term in the sum is a random variable drawn from the distribution (\ref{eq:P_x_n}).
The value $X_k$ is therefore also a random variable and we should compute its statistics.
This is actually quite difficult so I offer two possible approaches.

\subsection{Mean square}

Instead of computing the full statistics of $X_k$ we compute its mean square,
\begin{equation}
\langle | \tilde{x}_k |^2 \rangle = \sum_{n,m = 0}^{N-1} \langle x_n x_m \rangle e^{-i 2 \pi (n-m) k / N} \, .
\end{equation}
The average over $x_n x_m$ can be done by hand because each factor can only take two possible values $\pm 1$.
Writing out every combination we get
\begin{align}
\langle x_n x_m \rangle = &\frac{1}{4} \times \left( \right. \nonumber \\
  &(1 + \sin(\delta \phi_n))(1 + \sin(\delta \phi_m)) \nonumber \\
+ &(1 - \sin(\delta \phi_n))(1 - \sin(\delta \phi_m)) \nonumber \\
- &(1 + \sin(\delta \phi_n))(1 - \sin(\delta \phi_m)) \nonumber \\
- &(1 - \sin(\delta \phi_n))(1 + \sin(\delta \phi_m)) \left. \right) \nonumber \\
&= \sin(\delta \phi_n)\sin(\delta \phi_m) \, .
\end{align}
This calculation is not quite right because in the case where $n=m$ the average $\langle x_n x_m \rangle$ must always be equal to 1.
Therefore, the correct expression is
\begin{equation}
\langle x_n x_m \rangle = \sin(\delta \phi_n) \sin(\delta \phi_m) + \cos(\phi_n)^2 \delta_{n,m} \, .
\end{equation}
Therefore \begin{align}
\langle |\tilde{x}_k|^2 \rangle = & \sum_{n,m = 0}^{N-1} \nonumber \\
& \left( \sin(\delta \phi_n) \sin(\delta \phi_m) + \cos(\delta \phi_n)^2 \delta_{n,m} \right) \nonumber \\
& \times e^{-i 2 \pi (n-m) k / N} \, .
\end{align}
Let suppose that $\delta \phi_n \ll 1$ so that we can expand the $\sin$ functions to first order.
Doing this we get
\begin{align}
\langle |\tilde{x}_k|^2 \rangle
&= \sum_{n,m=0}^{N-1} \left( \delta \phi_n \delta \phi_m + \delta_{n,m} \right) e^{-i 2 \pi (n-m) k / N} \nonumber \\
&= \left| \sum_{n=0}^{N-1} \delta \phi_n e^{-i 2 \pi n k / N} \right|^2 + N \nonumber \\
&= \left| \tilde{\delta \phi}_k \right|^2 + N
\end{align}
where $\tilde{\delta \phi}_k$ means the $k^\text{th}$ Fourier coefficient of $\{ \delta \phi_n \}$.
This result makes sense.
The first part is just the spectral density of the underlying angle noise.
The second part is a frequency independent (i.e. white) noise coming from the fact that for the case $\delta \phi = 0$ we have an \emph{uncorrelated} $1/2$ probability that each measurement shot yields a different qubit state than the previous shot.
Note that using the full $\sin$ functions would make this result considerably more complicated.

The physical, single sided power spectral density $S$ relates to the discrete Fourier transform via
\begin{equation}
S_{\delta \phi}(f) = \frac{2T}{N^2} \left| \tilde{\delta \phi}_{k=fT} \right|^2
\end{equation}
where $T$ is the length of time over which the data sequence is measured.
Therefore we find
\begin{equation}
S_{\delta \phi}(f) = \frac{2T}{N^2} \langle \left| \tilde{x}_{k=fT} \right|^2 \rangle - \frac{2T}{N} \, .
\end{equation}
Defining the sampling interval as $\delta t$ and using $T = N \delta t$ we can rewrite this result as
\begin{equation}
S_{\delta \phi}(f) = \frac{2 \delta t^2}{T} \langle \left| \tilde{x}_{k=fT} \right|^2 \rangle - 2 \delta t \, .
\end{equation}
This says that the white noise floor coming from the quantum ``shot noise'' is reduced by taking data faster.
From this formula we now know how to convert the discrete Fourier transform of the measured $\pm 1$ values to the underlying power spectral density of the qubit angle fluctuations.

% \subsection{Full distribution}

Now we attempt to compute the full distribution of $\tilde{x}_k$.
In fact, we try to compute the distribution of the real part
\begin{equation}
\Re \tilde{x}_k = \sum_{n=0}^{N-1} x_n \cos(2\pi n k / N) = \sum_{n=0}^{N-1} \xi_n
\end{equation}
where we have defined $\xi_n \equiv x_n \cos(2\pi n k / N)$.
Using the scaling laws for probability distributions we find
\begin{align}
P_{\xi_n}(\xi) &= P_{x_n}(\xi / \cos(2\pi n k / N)) \nonumber \\
= \frac{1}{2} \bigg[ & \delta \left( \frac{\xi}{\cos(2\pi n k / N)} - 1 \right) (1 + \sin(\delta \phi_n)) \nonumber \\
+ & \delta \left( \frac{\xi}{\cos(2 \pi n k / N)} + 1 \right) (1 - \sin(\delta \phi_n)) \bigg]
\end{align}
The distrubution of a sum of random variables is equal to the convolution of the distributions of the things being summed.
Therefore,
\begin{equation}
P_{\Re\tilde{x}_k} = \bigotimes_{n=0}^{N-1} P_{\xi_n}
\end{equation}
where $\otimes$ indicates convolution.
Convolution is difficult, so instead we work with the Fourier transform.\footnote{We use the Fourier transform convention $\mathcal{FT}(f)(\nu) = \int dx \, f(x) e^{-i 2 \pi x \nu}$.}
The Fourier transform of a convolution is the product of the Fourier transforms, so
\begin{equation}
\mathcal{FT}(P_{\Re \tilde{x}_k}) = \prod_{n=0}^{N-1} \mathcal{FT}(P_{\xi_n}) \, .
\end{equation}
Fortunately the Fourier transform of $P_{\xi_n}$ is quite simple
\begin{align}
\mathcal{FT}(P_{\xi_n})(\nu) = \frac{1}{2}
\bigg[
& e^{-i 2 \pi \nu \cos(2\pi n k / N)}(1 + \sin(\delta \phi_n)) \nonumber \\
+ & e^{ i 2 \pi \nu \cos(2\pi n k / N)}(1 - \sin(\delta \phi_n))
\bigg] \nonumber \\
= & \cos(2\pi \nu \cos(2\pi n k / N)) \nonumber \\
- i & \sin(2\pi \nu \cos(2 \pi n k / N))\sin(\delta \phi_n) \, .
\end{align}
Finally we have
\begin{align}
\mathcal{FT}(P_{\Re \tilde{x}_k})(\nu) = & \prod_{n=0}^{N-1} \nonumber \\
& \cos(2 \pi \nu \cos(2 \pi n k / N)) \nonumber \\
& - i \sin(2 \pi \nu \cos(2 \pi n k / N))\sin(\delta \phi_n) \, .
\end{align}
At this point I am stuck, but since we have the mean square calculation we don't really need this anyway.


\section{Data processing}

Because the white noise exceeds the level of the noise we're trying to measure, we need to somehow remove it.
Fortunately, because this noise is completely uncorrelated, it is removed if we split the time sequence into two interleaved sequences and compute their cross-spectrum.
This is similar to the idea of sampling a signal with two detectors and computing the cross spectrum to remove the detectors' input noise.

Split the measured bit sequence into two subsequences:
\begin{align}
A &= \{ a_0, a_1, \ldots a_{N/2} \} = \{z_0, z_2, \ldots, z_{N-2} \} \nonumber \\
B &= \{ b_0, b_1, \ldots b_{N/2} \} = \{z_1, z_3, \ldots, z_{N-1} \} \, .
\end{align}
We compute the DFT of each of these sequences, $\tilde{A}$ and $\tilde{B}$, multiply them, average over data sets or frequency bins, and then take the modulus.
We'll see how this affects the signal and noise in the following analysis.

\subsection{Interleaving effect on noise}

Consider a data sequence made entirely of the sampling noise.
Each coefficient in the normalized DFT of a white noise has Gaussian distributed real and imaginary parts, each with standard deviation $\sigma / \sqrt{2N}$, where $\sigma$ is the standard deviation in the time domain.
When we multiply corresponding coefficients from the two interleaved sequences, the magnitudes multiply and the phases add.
The phases are uniformly distributed and independent, so the final phase is also uniformly distributed.

The magnitude requires more analysis.
Denote the DFTs as $\tilde{A}$ and $\tilde{B}$, and denote the product as $\tilde{C}_k = \tilde{A}_k \tilde{B}_k^*$.
The modulus of $\tilde{A}_k$ is distributed as
\begin{equation}
P_r(r>0) = \left( \frac{2}{\sigma^2 / (N/2)} \right) r \exp \left( - \frac{r^2}{\sigma^2 / (N/2)} \right) \, .
\end{equation}
The distribution of a product $Z = XY$ of two random variables is given by
\begin{equation}
P_Z(z) = \int_{-\infty}^\infty P_X(x) P_Y(z/x) \frac{dx}{|x|} \, ,
\end{equation}
so the distribution of the modulus of the product of the DFT's of the two interleaved sequences is (let $q \equiv \sigma^2 / (N/2)$)
\begin{align}
P_{\left\lvert \tilde{C}_k \right \rvert}(z)
&= \int_{-\infty}^\infty \frac{dx}{|x|} P_r(x) P_r(z/x) \nonumber \\
&= \left( \frac{2}{q} \right)^2
z \int_0^\infty \frac{dx}{x} \,
\exp \left( - \frac{x^2 + (z/x)^2}{q} \right) \nonumber \\
&= \left( \frac{2}{\sigma^2 / (N/2)} \right)^2
z K_0 \left( \frac{2z}{\sigma^2 / (N/2)} \right)
\end{align}
where $K_0$ is the $0^\text{th}$ modified Bessel function of the second kind.
Note that for our binary noise signal with values $\pm 1$ the variance is $\sigma^2 = 1$.
The mean of $\left \lvert \tilde{C}_k \right \rvert$ is therefore
\begin{equation}
\langle P_{\left \lvert \tilde{C}_k \right\rvert}(z) \rangle
= \int_0^\infty dz \, z \, P_{\left \lvert \tilde{C}_k \right \rvert}(z)
= \frac{\pi}{4} \sigma^2 = \frac{\pi}{4}\, .
\end{equation}
Therefore, the mean cross spectrum for the sampling noise is (multiply by $2 \delta t$)
\begin{equation}
\text{Mean noise cross spectrum} = \frac{\pi}{2} \delta t \, \sigma^2 = \frac{\pi}{2} \delta t \, .
\end{equation}

If we now average over neighboring bins or over data sets, this noise cross spectral power is multiplied by $1/\sqrt{M}$ where $M$ is the number of points averaged.
As broadband noise is often viewed on a log-log plot, when averaging bins together we like to increase the number of averaged bins proportional with the frequency, as this leads to a uniform density of points in the plot.
This is illustrated in Figure \ref{fig:interleaved_noise}.

\quickfig{\columnwidth}{interleaved_noise.png}
{Binary white noise with interleaved processing procedure.
We arbitrarily chose $\delta t = 0.3$.
Note the $1/\sqrt{f}$ dependence of the noise after averaging; this is results of $\sqrt{N}$ statistics of averaging the noise, combined with the choice of the averaging bandwidth linearly proportional to frequency.}
{fig:interleaved_noise}

\subsection{Interleaving effect on signal}

The interleaving procedure is designed to remove white noise and preserve the signal we're trying to measure, which in many cases is $1/f$ noise.



\end{document}

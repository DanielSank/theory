\documentclass{article}

%Packages
\usepackage{amsmath}
\usepackage{amstext}
\usepackage{amssymb}
\usepackage{appendix}
\usepackage{coseoul}
\usepackage{graphicx}
\usepackage{import}
\usepackage{lscape}
\usepackage{modular}

\usepackage[pdfpagemode=UseNone,pdfstartview=FitH,colorlinks=true,linkcolor=blue,citecolor=blue,urlcolor=blue]{hyperref}
\usepackage[all]{hypcap}


% General physics constructs
\newcommand{\bra}[1]{\langle #1 |}
\newcommand{\ket}[1]{| #1 \rangle }
\newcommand{\braket}[2]{\langle #1|#2\rangle}
\newcommand{\bbraket}[3]{ \langle #1 | #2 | #3 \rangle }
\newcommand{\boltzmann}{k_b}

% Common math
\newcommand{\norm}[1]{\left \lvert #1 \right \rvert}
\newcommand{\abs}[1]{\left \lvert #1 \right \rvert}  % These two are redundant. Consider removing one.

\newcommand{\avg}[1]{\left \langle #1 \right \rangle}  % Should get rid of this, as "average" isn't specific.
\newcommand{\angavg}[1]{\left \langle #1 \right \rangle}

\newcommand{\VS}{\textit{\textbf{V}}}
\newcommand{\Tr}{\textrm{Tr}}
\renewcommand{\Re}{\textrm{Re}}
\renewcommand{\Im}{\textrm{Im}}
\newcommand{\basis}[1]{\{\ket{#1}\}}

% Quantum
\newcommand{\nboseeinstein}{n_\text{BE}}
\newcommand{\gammaup}{\Gamma_\uparrow}
\newcommand{\gammadown}{\Gamma_\downarrow}
\newcommand{\gammaupdown}{\Gamma_{\uparrow \downarrow}}
\newcommand{\gammaemission}{\Gamma_\text{loss}}
\newcommand{\qualityfactoremission}{Q_{d,\text{loss}}}

% Qubits
\newcommand{\omegaqubit}{\omega_{10}}

% Circuits
\newcommand{\impedance}{Z_0}
\newcommand{\resistorsource}{R_s}
\newcommand{\vsource}{V_s}
\newcommand{\vsourcerms}{V_{s,\text{rms}}}
\newcommand{\vloadrms}{V_{l,\text{rms}}}

% Signals and noise
\newcommand{\psdsingle}{S_\text{ss}}
\newcommand{\psddouble}{S_\text{ds}}
\newcommand{\noiseavailable}{S_{p,a}^e}
\newcommand{\spectralengineer}{S^e}
\newcommand{\spectralsymmetric}{S^\text{symm}}
\newcommand{\spectralattenuator}{\spectralengineer_{\poweravailable, \text{att.}}}

% Microwaves
\newcommand{\vright}{V_+}
\newcommand{\vleft}{V_-}
\newcommand{\iright}{I_+}
\newcommand{\ileft}{I_-}
\newcommand{\poweravailable}{P_a}

% Figures. Example usage:
% \quickfig{\columnwidth}{my_image}{This is the caption}{fig:my_fig}
\DeclareRobustCommand{\quickfig}[4]{
\begin{figure}
\begin{centering}
\includegraphics[width=#1]{#2}
\par\end{centering}
\caption{#3}
\label{#4}
\end{figure}
}

\DeclareRobustCommand{\quickwidefig}[4]{
\begin{figure*}[h]
\begin{centering}
\includegraphics[width=#1]{#2}
\par\end{centering}
\caption{#3}
\label{#4}
\end{figure*}
}

\DeclareRobustCommand{\quickfigcentered}[4]{
  \begin{figure}
  \makebox[\textwidth][c]{\includegraphics[width=#1]{#2}}
  \caption{#3}
  \label{#4}
  \end{figure}
}


\title{What does the Fourier transform tell us about spectral density?}
\author{Daniel Sank \\ \small{Google Quantum AI}}

\begin{document}

\maketitle

\levelstay{Basics}

Consider a noise process $X$.
The mean power over a time interval $\{-T/2,T/2\}$ is defined as
\begin{equation}
P = \frac{1}{T} \avg{\int_{-T/2}^{T/2} \abs{x(t)}^2 \, dt}
\end{equation}
where the average is taken over realizations $x(t)$ of the noise.
A given realization can be written as a Fourier transform,
\begin{equation}
x(t) \equiv
\int \tilde{x}(\omega) e^{i \omega t} \, \frac{d \omega}{2\pi} \, .
\end{equation}
Let's use this to rewrite the mean power:
\begin{align}
P
&= \frac{1}{T}\avg{
\int_{-T/2}^{T/2} dt
\int \frac{d\omega}{2\pi}
\int \frac{d\omega'}{2\pi}
\tilde{x}(\omega) \tilde{x}(\omega')^*
e^{i(\omega - \omega')t}} \nonumber \\
&= \int \int \frac{d\omega}{2\pi} \frac{d\omega'}{2\pi}
\avg{\tilde{x}(\omega)\tilde{x}(\omega')^*}
\frac{\sin(T(\omega - \omega')/2)}{T (\omega - \omega')/2} \, .
\end{align}
Suppose we filter the signal so that only frequencies near $\pm \Omega$ remain, which we represent by restricting the frequency integrals to a range, say, of $\{\Omega-\Delta \omega, \Omega+\Delta \omega\}$ and $\{-\Omega - \Delta \omega, -\Omega + \Delta \omega\}$.
This gives four little squares in the $\omega/\omega'$ plane over which we ned to do the integral.
However the $\sin(x)/x$ function is strongly peaked at $x=0$, so we actually only need to integrate over the patches where $\omega$ and $\omega'$ are both positive or both negative.
Focusing on the positive patch and defining $\omega = \Omega + p$ and $\omega' = \Omega + q$, we have
\begin{align}
P_+ &=
  \int_{-\Delta \omega/2}^{\Delta\omega/2}
  \int_{-\Delta \omega/2}^{\Delta\omega/2}
  \frac{dp}{2\pi} \frac{dq}{2\pi}
  \avg{\tilde{x}(\Omega+p)\tilde{x}(\Omega+q)^*} \frac{\sin(T(p-q)/2)}{T(p-q)/2} \nonumber \\
  (\text{assume }\Delta \omega \ll T) \quad &=
  \frac{1}{T} \int_{-\Delta \omega/2}^{\Delta \omega / 2}
  \frac{dp}{2\pi} \avg{\tilde{x}(\Omega + p) \tilde{x}(\Omega + p)^*} \nonumber \\
  &= \frac{\avg{\abs{\tilde{x}(\Omega)}^2}}{T} \frac{\Delta \omega}{2 \pi}
\end{align}
where in the last line we assumed that $\tilde{x}$ is relatively constant over frequency width $\Delta \omega$.
Now we repeat the computation for the $-\Omega$.
Assuming $x(t) \in \mathbb{R}$, then $\tilde{x}(-\omega) = \tilde{x}(\omega)^*$, so we get
\begin{align*}
  P_- &=
  \int_{-\Delta \omega/2}^{\Delta\omega/2}
  \int_{-\Delta \omega/2}^{\Delta\omega/2}
  \frac{dp}{2\pi} \frac{dq}{2\pi}
  \avg{\tilde{x}(-\Omega+p)\tilde{x}(-\Omega+q)^*} \frac{\sin(T(p-q)/2)}{T(p-q)/2} \\
  &= \int_{-\Delta \omega/2}^{\Delta\omega/2}
  \int_{-\Delta \omega/2}^{\Delta\omega/2}
  \frac{dp}{2\pi} \frac{dq}{2\pi}
  \avg{\tilde{x}(\Omega-p)^* \tilde{x}(\Omega-q)} \frac{\sin(T(p-q)/2)}{T(p-q)/2} \\
  (p \leftrightarrow -q) &= \int_{-\Delta \omega/2}^{\Delta\omega/2}
  \int_{-\Delta \omega/2}^{\Delta\omega/2}
  \frac{dp}{2\pi} \frac{dq}{2\pi}
  \avg{\tilde{x}(\Omega+q)^* \tilde{x}(\Omega+p)} \frac{\sin(T(p-q)/2)}{T(p-q)/2} \\
  &= P_+ \, .
\end{align*}
Therefore, the power is
\begin{equation}
P = P_- + P_+ = 2 \frac{\avg{\abs{\tilde{x}(\Omega)}^2}}{T} \frac{\Delta \omega}{2\pi} \, .
\end{equation}
Therefore, we can \emph{estimate} the power in a certain frequency band as twice the mod square of the Fourier transform of the acquired signal, divided by the acquisition time.
The factor of two comes in because we're used a double sided Fourier transform.

If we define a spectral density $S$ as the power per bandwidth of the signal at frequency $\Omega$, then we've shown that
\begin{equation}
2 \frac{\abs{\tilde{x}(\Omega)}^2}{T} \text{ is an estimator for } S(\Omega) \, .
\end{equation}
Note that when we imagined filtering a small frequency band of our signal, we included both positive and negative frequencies at the same time.
Therefore, The symbol $S$ we just defined is the \textbf{single sided spectral density} defined only for $0 < \Omega < \infty$.

Note we implicitly assumed that $\Omega \gg 1/T$ when we said that the four frequency regions could be considered separately.
Our analysis fails at frequencies lower than $1/T$.
This is a recurring theme in signal analysis: the acquisition time sets the frequency resolution.
We'll see more about this later one when we study spectral leakage.

\levelstay{Weiner Kinchein}

Now that we've defined the power spectral density $S$, let's relate it to the time domain properties of $f(t)$.
The Fourier transform of $S$ turns out to be interesting:
\begin{align*}
  \int_{-\infty}^\infty \frac{d\Omega}{2\pi} S(\Omega) e^{i \Omega \tau}
  =& \int_{-\infty}^\infty \frac{d\Omega}{2\pi} \frac{2}{T} e^{i \Omega \tau} \left\lvert \tilde{x}(\Omega) \right \rvert^2 \\
  =& \int_{-\infty}^\infty \frac{d\Omega}{2\pi} \frac{2}{T} e^{i \Omega \tau}
    \left( \int_{-T/2}^{T/2} dt f(t) e^{-i \Omega t} \right)
    \left( \int_{-T/2}^{T/2} dt' f(t')^* e^{i \Omega t'} \right) \\
  =& \frac{2}{T} \int_{-T/2}^{T/2} dt \int_{-T/2}^{T/2} dt' f(t) f(t')^* \delta(\tau + t' - t) \\
  T \to \infty \qquad =& \frac{2}{T} \int_{-T/2}^{T/2} dt f(t) f(t - \tau)^* \\
  =& 2 \avg{f(0)f(-\tau)^*} \\
  =& 2 \avg{f(\tau) f(0)^*}
\end{align*}
For $f(t) \in \mathbb{R}$, we have $S(-\omega) = S(\omega)$, we can rewrite this as
\begin{align*}
  \int_{-\infty}^\infty \frac{d\Omega}{2\pi} S(\Omega) e^{i \Omega \tau}
  =& \int_0^\infty \frac{d\Omega}{2\pi} S(\Omega) e^{i \Omega \tau} + \int_{-\infty}^0 \frac{d\Omega}{2\pi} S(\Omega) e^{i \Omega \tau} \\
  =& \int_0^\infty \frac{d\Omega}{2\pi} S(\Omega) e^{i \Omega \tau} + \int_{\infty}^0 - \frac{d\Omega}{2\pi} S(-\Omega) e^{-i \Omega \tau} \\
  =& \int_0^\infty \frac{d\Omega}{2\pi} S(\Omega) e^{i \Omega \tau} + \int_0^\infty \frac{d\Omega}{2\pi} S(\Omega) e^{-i \Omega \tau} \\
  =& \int_0^\infty \frac{d\Omega}{2\pi} S(\Omega) 2 \cos(\Omega \tau) \\
\end{align*}
so (Weiner-Kinchein for real signals)
\begin{equation*}
\int_0^\infty \frac{d\Omega}{2\pi} S(\Omega) \cos(\Omega \tau) = \avg{f(0) f(\tau)}
\end{equation*}
where $S$ is the single sided spectral density.
\end{document}

\levelstay{IQ demodulation system}

\quickwidefig{\columnwidth}
{signal_processing_chain_complete.pdf}
{Complete signal processing chain.
Microwave frequency signal (and noise) come into the IQ mixer via the RF port.
The RF signal and noise are mixed with a local oscillator (LO) to convert to two lower intermediate frequency (IF) signals $I(t)$ and $Q(t)$.
The IF signals are then filtered to remove the remaining high frequency component (see text) and digitally sampled.
The signal is then digitally processed by multiplication by a window function $W_n$ as described in the main text to produce final results $I$ and $Q$.}
{fig:IQ_mixer}

An IQ demodulation system is shown in Figure \ref{fig:IQ_mixer}.
An IQ mixer takes signals at high frequency and brings them to lower frequency so they can be more easily measured.
The local oscillator (LO) signal $\cos(\Omega t)$ is used to mix the RF signal down to a lower frequency.
The down-mixed signals are low-pass filtered so that the following digitization step is not subjected to aliasing effects.
The signal is then sampled in an analog to digital converter (ADC) and further processed using digital signal processing (DSP) techniques.

We model the mixer as producing
\begin{equation}
  I_\text{mixer}(t) = A \cdot \text{RF}(t) \cos(\Omega t) \qquad
  Q_\text{mixer}(t) = -A \cdot \text{RF}(t) \sin(\Omega t) \, .
\end{equation}
The factor of $A$ is included to model conversion loss in the mixer.
Precisely how $A$ is related to the spec'd conversion loss found in the mixer's datasheet is not described here, simply because the meaning of ``conversion loss'' is usually left vague in the datasheets and I don't want to record something incorrect.
Note also that the sin/cos convention between the two IF ports in a real mixer may be different from what's written here.
The analog signal is then amplified by $G$.

\leveldown{Coherent signal - dc case}

Suppose in the incoming RF signal were $\text{RF}(t) = V_s \cos(\Omega t + \phi)$.
Then the $I$ and $Q$ signals coming out of the mixer's IF ports would be
\begin{align*}
  I_\text{mixer}(t) &= \frac{A V_s}{2} \left( \cos(\phi) + \cos(2\Omega t + \phi) \right) \\
  Q_\text{mixer}(t) &= \frac{A V_s}{2} \left( \sin(\phi) - \sin(2\Omega t + \phi) \right) \, .
\end{align*}
If the low-pass filters remove the $2 \Omega$ terms then we're left with
\begin{align}
  I(t) &= \frac{A G V_s}{2} \cos(\phi) \nonumber \\
  Q(t) &= \frac{A G V_s}{2} \sin(\phi) \, ,
\end{align}
so the dc $I(t)$ and $Q(t)$ voltages can be thought of as the Cartesian coordinates giving the amplitude and phase of the original signal, up to scaling factors.
Therefore, the mixer has done its job of allowing us to find the amplitude and phase of a high frequency signal by making only low frequency measurements.

\levelstay{Coherent signal - ac case}

In practice we use the IQ mixer to detect amplitude and phase of several simultaneous sinusoidal signals each with different frequencies.
For example, we might have
\begin{equation}
  RF(t) = V_{s,1} \cos([\Omega + \omega_1] t + \phi_1 ) + V_{s,2} \cos([\Omega + \omega_2] t + \phi_2) \, .
\end{equation}
The IF $I$ and $Q$ wave forms in this case are
\begin{align}
  I(t) &= \frac{A G}{2} \left( V_{s,1} \cos(\omega_1 t + \phi_1) + V_{s,2} \cos(\omega_2 t + \phi_2) \right) \\
  Q(t) &= \frac{A G}{2} \left( V_{s,1} \sin(\omega_1 t + \phi_1) + V_{s,2} \sin(\omega_2 t + \phi_2) \right) \, .
\end{align}
Thinking of $I(t)$ and $Q(t)$ as real and imaginary parts of complex signals, we can think of our signals as a superposition of two rotations, one at frequency $\omega_1$ and another at frequency $\omega_2$, each with its own amplitude and phase.
In order to find both amplitudes and both phases, we essentially do Fourier integrals.
To do this, we digitize the wave forms, yielding
\begin{align}
  I_n &= \left( \frac{A G D}{2 V_\text{FS}} \right) \left( V_{s,1} \cos(\omega_1 n \delta t + \phi_1) + V_{s,2} \cos(\omega_2 n \delta t + \phi_2) \right) \\
  Q_n &= \left( \frac{A G D}{2 V_\text{FS}} \right) \left( V_{s,1} \sin(\omega_1 n \delta t + \phi_1) + V_{s,2} \sin(\omega_2 n \delta t + \phi_2) \right)
\end{align}
where $\delta t$ is the digital sampling interval and the factor $D / V_\text{FS}$ makes a signal at the full-scale voltage of the ADC have digital value of $D$.
Then, in digital logic we construct the complex series $z_n$ defined by $z_n \equiv I_n + i Q_n$.
For the signals written above this is
\begin{equation}
z_n =
  \left( \frac{A G D}{2 V_\text{FS}} \right)
  \left(
    V_{s,1} \exp \left( i \left[ \omega_1 n \delta t + \phi_1 \right] \right)
  + V_{s,2} \exp \left( i \left[ \omega_2 n \delta t + \phi_2 \right] \right)
  \right) \, .
\end{equation}
If we now, in digital logic, compute the sum
\begin{equation}
  Z(\omega_k) = \frac{1}{N} \sum_{n=0}^{N-1} z_n \underbrace{e^{-i \omega_k n \delta t}}_{W_n^*}
\end{equation}
we recover the amplitude and phase for the component at frequency $\omega_k$.
For example, we'd get
\begin{equation*}
  Z(\omega_1)
  = \frac{A G D V_{s,1}}{2 V_\text{FS}} \exp(i \phi_1)
  = C_\text{RF} \frac{V_s}{2} \exp(i \phi_1)
\end{equation*}
where $C_\text{RF} \equiv AGD/V_\text{FS}$ is total voltage scale factor going from RF to digital.

The factor $W_n$, called the ``window function'', was designed to pull out the amplitude and phase of a specific frequency component.
We can re-express the sum over the complex $z_n$ and $W_n^*$ in terms of two real sums by writing $W_n \equiv W_{I,n} + i W_{Q,n}$ and separating the real and imaginary parts:
\begin{align*}
  Z =& \frac{1}{N} \sum_{n=0}^{N-1} z_n W_n^* \\
  \Re Z \equiv I =& \frac{1}{N} \sum_{n=0}^{N-1} I_n W_{I,n} + Q_n W_{Q,n} \\
  \Im Z \equiv Q =& \frac{1}{N} \sum_{n=0}^{N-1} Q_n W_{I,n} - I_n W_{Q,n} \, .
\end{align*}
These two equations for $I$ and $Q$ explain why the DSP unit computes the two sums shown in Figure \ref{fig:IQ_mixer}.

\levelstay{Summary}

Given an input RF signal $\text{RF}(t) = V_s \cos((\Omega + \omega) t + \phi)$, the complete signal processing chain is:
\begin{itemize}

  \item \textbf{Mixing, filtering, and amplifying} produces
    \begin{align*}
      I(t) =& \frac{A G V_s}{2} \abs{h(\omega)} \cos(\omega t + \phi) \\
      Q(t) =& \frac{A G V_s}{2} \abs{h(\omega)} \sin(\omega t + \phi)
    \end{align*}
    which can be thought of as
    \begin{equation*}
      z(t) = \frac{A G V_s}{2} \abs{h(\omega)} \exp(i (\omega t + \phi))
    \end{equation*}

  \item \textbf{Digitizing} produces
    \begin{align*}
      I_n =& \frac{A G D V_s}{2 V_\text{FS}} \abs{h(\omega)} \cos(\omega n \delta t + \phi) \\
      Q_n =& \frac{A G D V_s}{2 V_\text{FS}} \abs{h(\omega)} \sin(\omega n \delta t + \phi)
    \end{align*}
    which can be though of as
    \begin{equation*}
      z_n = \frac{A G D V_s}{2 V_\text{FS}} \abs{h(\omega)} \exp(i(\omega n \delta t + \phi))
    \end{equation*}

  \item \textbf{DSP} produces
    \begin{align*}
      I =& \frac{1}{N} \sum_{n=0}^{N-1} I_n W_{I,n} + Q_n W_{Q,n} \\
      Q =& \frac{1}{N} \sum_{n=0}^{N-1} Q_n W_{I,n} - I_n W_{Q,n}
    \end{align*}
    which can be thought of as the real and imaginary parts of
    \begin{align*}
      Z
      =& \frac{1}{N} \sum_{n=0}^{N-1} z_n W_n^* \\
      =& \frac{1}{N} \sum_{n=0}^{N-1} z_n \exp (-i \omega t) \\
      =& C_\text{RF} \frac{V_s}{2} \abs{h(\omega)} \exp(i \phi)
    \end{align*}
\end{itemize}

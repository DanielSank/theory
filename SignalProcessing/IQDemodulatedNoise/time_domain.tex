\levelstay{SNR in the time domain - transient}

Now consider the case where, as before, we want to distinguish between two different voltage signals, but now where those signals have time dependent magnitude and phase.
We showed above that for two voltage waves of equal magnitude $V$ the signal to noise ratio is
\begin{equation*}
  \text{SNR} = \frac{g^2 (V^2 / 2) T}{S_\xi^e}
\end{equation*}
Thinking of these voltage waves in their phasor representations
\begin{equation*}
  V_j(t)
  = \Re \left[ V_j \exp(i \phi_j) \exp(\Omega t) \right]
  \, ,
\end{equation*}
we can understand the quantity $(g V)^2$ as the square of the distance between the phasors.
Therefore, defining the distance between two phasors as $\delta V$, we can reinterpret our formula as
\begin{equation*}
  \text{SNR} = \frac{\left( T \delta V \right)^2 }{2 T S_\xi^e} \, .
\end{equation*}
Suppose now that the voltage phasors are time dependent, i.e. they have time varying magnitude and phase.
In that case, it's somewhat obvious that the SNR formula becomes
\begin{equation*}
  \text{SNR} = \frac{\abs{\int_0^T \delta V(t) \, dt}^2}{2 T S_\xi^e} \, .
\end{equation*}
Now suppose we multiply the voltage signals by a time dependent window $w(t)$ before integration.
The SNR formula is modified to
\begin{equation*}
  \text{SNR} = \frac{\abs{\int_0^T \delta V(t) \, w(t) \, dt}^2}{2 S_\xi^e \int_0^T \abs{w(t)}^2 \, dt} \, .
\end{equation*}

\leveldown{Quantum variables}
One final useful modification to our SNR formula is to convert to variables suited to quantum mechanical applications.
Instead of voltage amplitudes we use wave amplitudes $A$ normalized such that $\abs{A}$ is the flux of photons travelling through the coaxial cable.
The power corresponding to a voltage phasor $V$ is $P = V^2 / 2 Z_0$ where $Z_0$ is the transmission line impedance, but we can also express the power in terms of photon flux as $P = \abs{A}^2 \hbar \omega$.
Therefore, $\abs{V}^2 = 2 Z_0 \hbar \omega \abs{A}^2$.
Furthermore, the noise power spectral density in a quantum system is $\hbar \omega / 2 \eta$ where $\eta < 1$ is the so-called ``quantum efficiency'', which can be though of as the fraction of photons captured by the detector.
The voltage noise spectral density is therefore $S_\xi^e = Z_0 \hbar \omega / 2 \eta$.
Using these relations, we can write the SNR as
\begin{equation}
  \text{SNR}
  = \frac
    {2 \eta \abs{\int_0^T \delta A(t) \, w(t) \, dt}^2}
    {\int_0^T \abs{w(t)}^2 \, dt} \, .
\end{equation}
and the separation error as
\begin{equation}
  \epsilon = \frac{1}{2} \, \text{erfc} \left(
    \sqrt{\frac{\eta}{2}} \frac
      {\abs{\int_0^T \delta A(t) \, w(t) \, dt }}
      {\sqrt{\int_0^T \abs{w(t)}^2 \, dt}}
  \right)
  \, .
\end{equation}

\section{Introduction - The Big Picture}
In this section I'm going to explain the main message of this chapter, and actually this entire document, by dissecting an extremely simple idea in great detail.
In linear algebra class the archetypical vector space is the set of lists of $n$ numbers, $\textbf{R}^n$.
It's easy to see that things are indeed vectors: you can add them together and multiply them by scalars \footnote{For example, $(1,4,3)+(0,1,5)=(1,5,8)$}  $\textbf{R}^n$, but rather we use elements in $\textbf{R}^n$, namely lists of numbers, to represent the things we \emph{are} interested in.
For example, we can use lists of two numbers to represent displacement in a two dimensional plane.
In figure \ref{Fig} we draw a plane with a coordinate system and two arrows \textbf{v} and \textbf{w}, each of unit length.
If we let $\ket{x}$ represent a unit arrow along the $x$ axis and $\ket{y}$ represent a unit arrow along the $y$ axis, then our two arrows are represented as,
\begin{equation}
\ket{v} \sim \frac{1}{\sqrt{2}} \left[ \begin{array}{c} 1 \\ 1 \end{array} \right] \
\qquad \ket{w} \sim \frac{1}{\sqrt{2}} \left[ \begin{array}{c} 1 \\ -1 \end{array} \right]
\end{equation}
If you add the lists together you get $(\sqrt{2},0)$ which is of course the correct representation for the geometric sum of the two arrows.

We could choose another set of coordinate axes as in Figure \ref{Fig}.
Call the arrows with unit length along these axes $\ket{p}$ and $\ket{q}$.
Then our arrows are represented as
\begin{displaymath}
\ket{v}\sim \left[\begin{array}{c} 1 \\ 0 \end{array} \right] \qquad \
\ket{w}\sim \left[\begin{array}{c} 0 \\ -1 \end{array} \right]
\end{displaymath}
The arrow $\ket{v}$ didn't change but the vector in $\textbf{R}^2$ did.
This highlights the most essential philosophical point to be made in this document: 

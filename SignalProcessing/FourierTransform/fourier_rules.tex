\documentclass{article}
\usepackage{amstext}
\usepackage{amsmath}

\author{Daniel Sank}
\title{Fourier Rules}
\date{October 4, 2010}

\begin{document}

\maketitle
\tableofcontents

\section{Notation}

\begin{equation*}
  \mathcal{F}(f)(\omega) = \tilde{f}(\omega) = \int f(t) e^{-i \omega t} \, dt
\end{equation*}

\section{Trigonometric formulae}

\subsection*{Phase / amplitudes}
\begin{equation}
  A \cos(x) + B \sin(x) = M \cos(x+\phi)
\end{equation}
where $M^{2} = A^{2}+B^{2}$ and $\tan(\phi) = -B / A$.

\subsection*{Addition formulae}

\begin{align*}
  \cos ( \alpha + \beta ) & =
    \cos ( \alpha ) \cos ( \beta ) - \sin ( \alpha ) \sin ( \beta ) \\
  \sin ( \alpha + \beta ) & =
    \sin ( \alpha ) \cos ( \beta ) + \cos ( \alpha ) \sin ( \beta )
\end{align*}

\subsection*{Product formulae}

\begin{align*}
  \sin(\alpha) \sin(\beta) & =
    \frac{1}{2} \left[ \cos (\alpha - \beta) - \cos(\alpha + \beta) \right] \\
  \cos(\alpha) \cos(\beta) & =
    \frac{1}{2} \left[ \cos (\alpha - \beta) + \cos(\alpha + \beta) \right]\\
  \sin(\alpha) \cos(\beta) &=
    \frac{1}{2} \left[ \sin (\alpha + \beta) + \sin(\alpha - \beta) \right]
\end{align*}

\subsection*{sine cosine equivalence}

\begin{equation*}
  \cos(x) = \sin\left(x + \frac{\pi}{2} \right)
  \quad \textrm{and} \quad
  \sin(x) = \cos\left(x - \frac{\pi}{2} \right)
\end{equation*}

\subsection*{Modulation}

Given two modulation functions $I$ and $Q$ both band limited in the frequency range from $[0, B]$, and a carrier at frequency $\Omega$, the following relations are true:
\begin{align*}
  f(t) &= I(t) \cos(\Omega t) - Q(t) \sin(\Omega t) \\
  \tilde{f}(\omega) &= \frac{1}{2} \left[
      \tilde I(\omega - \Omega) + i \tilde Q(\omega - \Omega)
    + \tilde I(\omega + \Omega) - i \tilde Q(\omega + \Omega)
    \right] \\
  f(t) &= \Re \left[
    e^{i \Omega t} \int_{-B}^B \left(
      \tilde{I}(\omega) + i \tilde{Q}(\omega)
    \right) e^{i \omega t} \frac{d \omega}{2 \pi}
  \right]
\end{align*}

\section{Some other stuff}

\begin{align*}
  \cos(\Omega t) &\rightarrow \frac{1}{2} \left[ \delta(\omega - \Omega) + \delta(\omega + \Omega) \right] \\
  \sin(\Omega t) &\rightarrow \frac{1}{2} \left[ -i \delta(\omega - \Omega) + i \delta(\omega + \Omega) \right]
\end{align*}

\section{Fourier transform rules in $t$ and $\nu$}


\subsection{Fourier transform pair}

\begin{align*}
  \tilde{f}(\nu) &= \int f(t) e^{-i 2 \pi \nu t} \, dt \\
  f(t) &= \int \tilde{f}(\nu) e^{i 2 \pi \nu t} \, d\nu
\end{align*}

\subsection{Delta function}

\begin{align*}
  \int dt \, e^{i 2 \pi(\nu - \nu') t} &= \delta(\nu - \nu') \\
  \int d\nu \, e^{i 2 \pi(t - t') \nu} &= \delta(t - t')
\end{align*}

\subsection{Convolution}
\begin{align*}
  \widetilde{(fg)}
  &= \int f(t) g(t) \, e^{-i 2 \pi N t} \, dt
  =  \int \tilde{f}(\nu) \tilde{g}(N - \nu) \, d\nu
  = (\tilde{g} \otimes \tilde{f}) \\
  \widetilde{(f \otimes g)}
  &= \int (f \otimes g)(t) \, e^{-i 2 \pi \nu t} \, dt
  = \tilde{f}(\nu) \tilde{g}(\nu)
  = \tilde{f} \tilde{g} \\
  \mathcal{F}^{-1}(\tilde{f} \tilde{g})
  &= \int_{\nu} \tilde{f}(\nu) \tilde{g}(\nu) e^{i 2 \pi \nu \tau} \, d\nu
  = \int_{t} f(t) g(\tau - t) \, dt
  = (f \otimes g) \\
  \mathcal{F}^{-1}(\tilde{f} \otimes \tilde{f})
  &= \int_{\nu}(\tilde{f} \otimes \tilde{g})(\nu) \, e^{i 2 \pi \nu t} \, d\nu
  = f(t) g(t)
\end{align*}

Edited to here

\subsection{Cross Correlation}

\begin{equation}
  \int_{\tau} d\tau e^{-i 2 \pi N \tau}
  \left(f \star g\right)(\tau)
  = \int_{\tau} d\tau e^{-i 2 \pi N \tau}
  \left(\int_{t}f(t)^{*}g(t+\tau)dt\right)
  =\tilde{f}(N)^{*}\tilde{g}(N)
\end{equation}

\subsection{Weiner Kinchein}
Consider the cross correlation theorem
with $f=g$.
Defining the physicist's power spectrum as $S_{p}(\nu)=\tilde{f}^{*}(\nu)\tilde{f}(\nu)$
we have
\begin{eqnarray*}
  \int_{\tau}e^{-i2\pi N\tau}(f\star f)(\tau)d\tau & = & \int_{\tau}e^{-i2\pi N\tau}\left(\int_{t}f(t)^{*}f(t+\tau)dt\right)d\tau=\tilde{f}(N)^{*}\tilde{f}(N)=S_{p}(N)\\
  \int_{N}S_{p}(N)e^{i2\pi N\tau}dN & = & \int_{t}f(t)^{*}f(t+\tau)\: dt
  \end{eqnarray*}

\subsection*{Real Signals}

Consider $f(t)\in\Re$. Then $\tilde{f}(\nu)^{*}=\tilde{f}(-\nu)$.
Then $S_{p}(N)=\tilde{f}(N)^{*}\tilde{f}(N)=\tilde{f}(-N)\tilde{f}(N)=S_{p}(-N)$.
Therefore we only need to look at positive frequencies to have all
available information about $f$. Define the engineer power spectrum
$S_{e}(N)=2S_{p}(N)$. Then for example we get \begin{eqnarray*}
\int_{\tau}\left(\int_{t}f(t)f(t+\tau)~dt\right)\: e^{-i2\pi N\tau}\: d\tau & = & \frac{1}{2}S_{e}(N)\\
\int_{N=0}^{\infty}S_{e}(N)\:\cos(2\pi N\tau)\: dN & = & \int_{t}f(t)f(t+\tau)\: dt\end{eqnarray*}

\subsection*{Finite Time Series}

Consider a set of experimental data represented by $f(t)$ for $t\in[0,T]$,
ie. we aquire data for a time $T$. Then it makes sense to define
the Fourier transforms as\[
\tilde{f}(\nu)=\frac{1}{\sqrt{T}}\int_{t=0}^{T}f(t)e^{-i2\pi\nu t}dt\]
 We still take\[
S_{p}(\nu)=\tilde{f}(\nu)^{*}\tilde{f}(\nu)\]
 Note that $S_{p}$ now has units of $[f^{2}T]$ or $f^{2}/\textrm{Hz}$.
This is called a $\textbf{power spectral density}$. Now the WK theorem
reads\[
\int_{N=0}^{\infty}S_{e}(N)\:\cos\left(2\pi N\tau\right)\: dN=\frac{1}{T}\int_{t}f(t)f(t+\tau)\: dt\]
 The thing on the right hand side is the autocorrelation function
of $f$. The prefactor normalizes the correlation. Because we now
have this normalization we can replace this autocorrelation with an
$\emph{ensemble}$ average:\[
\frac{1}{T}\int_{t}f(t)f(t+\tau)\: dt=\langle f(0)f(\tau)\rangle\]
 where the equality of these two {}``averages'' is the definition
of an $\textbf{ergodic}$ process, and the enemble average can be
written as $\langle f(0)f(\tau)\rangle$ if we assume the process
is $\textbf{stationary}$. All in all this gives us\begin{eqnarray*}
\int_{N=0}^{\infty}S_{e}(N)\:\cos\left(2\pi N\tau\right)\: dN & = & \langle f(0)f(\tau)\rangle\\
2\int_{-\infty}^{\infty}e^{-i2\pi Nt}\langle f(0)f(t)\rangle dt & = & S_{e}(N)\end{eqnarray*}


\section*{Fourier transform rules in $t$ and $\omega$}

Checked every line February 2014.

\begin{itemize}
\item \textbf{Fourier transform pair} \begin{eqnarray*}
\tilde{f}(\omega) & = & \int f(t)e^{-i\omega t}~dt\\
f(t) & = & \int\tilde{f}(\omega)e^{i\omega t}~\frac{d\omega}{2\pi}\end{eqnarray*}

\item \textbf{Delta function} \begin{eqnarray*}
\int dt~e^{i(\omega-\omega')t} & = & 2\pi\delta(\omega-\omega')\\
\int\frac{d\omega}{2\pi}~e^{i\omega(t-t')} & = & \delta(t-t')\end{eqnarray*}

\item \textbf{Convolution} \begin{eqnarray*}
\int_{t}f(t)g(t)~e^{-i\Omega t}dt & = & \int_{\omega}\tilde{f}(\omega)\tilde{g}(\Omega-\omega)\frac{d\omega}{2\pi}=\frac{1}{2\pi}(\tilde{g}*\tilde{f})(\Omega)\\
\int_{\omega}\tilde{f}(\omega)\tilde{g}(\omega)~e^{i\omega\tau}\frac{d\omega}{2\pi} & = & \int_{t}f(t)g(\tau-t)dt=(f*g)(\tau)\\
\int_{t}(f*g)(t)~e^{-i\omega t}~dt & = & \tilde{f}(\omega)\tilde{g}(\omega)\\
\int_{\omega}(\tilde{f}*\tilde{g})(\omega)~e^{i\omega t}~\frac{d\omega}{2\pi} & = & 2\pi f(t)g(t)\end{eqnarray*}

\item \textbf{Cross Correlation}\begin{equation}
\int d\tau e^{-i\Omega\tau}\left(f\star g\right)(\tau) = \int d\tau e^{-i\Omega\tau}\left(\int f(t)^{*}g(t+\tau)dt\right)=\tilde{f}^{*}(\Omega)\tilde{g}(\Omega) \nonumber \end{equation}

\item \textbf{Weiner Kinchein} Consider the cross correlation theorem in
the case where $g=f$. Defining the physicist power spectrum by $S_{p}^{\infty}(\Omega)=\tilde{f}(\Omega)^{*}\tilde{f}(\Omega)$
we have \begin{eqnarray*}
\int_{\tau}\left(\int_{t}f^{*}(t)f(t+\tau)~dt\right)~e^{-i\Omega\tau}~d\tau & = & S_{p}^{\infty}(\Omega)\\
\int_{\Omega}S_{p}^{\infty}(\Omega)~e^{i\Omega\tau}~\frac{d\Omega}{2\pi} & = & \int_{t}f^{*}(t)f(t+\tau)~dt\end{eqnarray*}
\end{itemize}

\subsection*{Real Signals - $\tilde{f}(\omega)^{*}=\tilde{f}(-\omega)$}
All results in this subsection pertain to real functions of time.

\begin{itemize}

\item $\textbf{Weiner Kinchein:}$ Consider $f(t)\in\Re$. Then $\tilde{f}(\omega)^{*}=\tilde{f}(-\omega)$.
Then $S_{p}^{\infty}(\Omega)=\tilde{f}(\Omega)^{*}\tilde{f}(\Omega)=\tilde{f}(-\Omega)\tilde{f}(-\Omega)^{*}=S_{p}^{\infty}(-\Omega)$.
Therefore we only need to look at positive frequencies to have all
available information about $f$. Define the engineer power spectrum
$S_{e}^{\infty}(\Omega)=2S_{p}^{\infty}(\Omega)$. Then for example
we get \begin{eqnarray*}
\int_{\tau}\left(\int_{t}f(t)f(t+\tau)~dt\right)\: e^{-i\Omega\tau}\: d\tau & = & \frac{1}{2}S_{e}^{\infty}(\Omega)\\
\int_{\Omega=0}^{\infty}S_{e}^{\infty}(\Omega)\:\cos(\Omega\tau)\:\frac{d\Omega}{2\pi} & = & \int_{t}f(t)f(t+\tau)\: dt\\
\textrm{or}\qquad\lim_{T\rightarrow\infty}\frac{1}{T}\int_{\Omega=0}^{\infty}S_{e}^{T}(\Omega)\cos(\Omega\tau)\,\frac{d\Omega}{2\pi} & = & \lim_{T\rightarrow\infty}\frac{1}{T}\int_{-T/2}^{T/2}f(t)f(t+\tau)\, dt\end{eqnarray*}

\item $\textbf{Sine/Cosine Representation}$
For real functions $f$ there is a particularly useful decomposition into sine and cosine functions.
\begin{eqnarray*}
f(t) & = & \int\frac{d\omega}{2\pi}\tilde{f}(\omega)e^{i\omega t}\\
 & = & \int_{0}^{\infty}\frac{d\omega}{2\pi}\left[e^{i\omega t}\tilde{f}(\omega)+e^{-i\omega t}\tilde{f}(\omega)^{*}\right]\\
 & = & \int_{0}^{\infty}\frac{d\omega}{2\pi}\cos(\omega t)\left[\tilde{f}(\omega)^{*}+\tilde{f}(\omega)\right]+i\sin(\omega t)\left[\tilde{f}(\omega)-\tilde{f}(\omega)^{*}\right]\\
 & = & 2 \int_{0}^{\infty}\frac{d\omega}{2\pi}\left[\Re\tilde{f}(\omega)\cos(\omega t)-\Im\tilde{f}(\omega)\sin(\omega t)\right]\end{eqnarray*}
 or alternately\[
f(t)= 2 \int_{0}^{\infty}\frac{d\omega}{2\pi}\sqrt{S_{p}(\omega)}\cos\left(\omega t+\phi(\omega)\right)\]
 where\[
S_{p}(\omega)=\Re\tilde{f}(\omega)^{2}+\Im\tilde{f}(\omega)^{2}=|\tilde{f}(\omega)|^{2}\qquad\textrm{and}\qquad\tan\left(\phi(\omega)\right)=\frac{\Im\tilde{f}}{\Re\tilde{f}}\]

\item $\textbf{Weiner Kinchein revisit:}$ Consider a set of experimental data
represented by $f(t)$ for $t\in[0,T]$, ie. we aquire data for a
time $T$. Then it makes sense to define the Fourier transforms as\[
\tilde{f}(\omega)=\frac{1}{\sqrt{T}}\int_{t=0}^{T}f(t)e^{-i\omega t}dt\]
 We still take\[
S_{p}(\omega)=\tilde{f}(\omega)^{*}\tilde{f}(\omega)\]
 Note that $S_{p}$ now has units of $[f^{2}T]$ or $f^{2}/\textrm{Hz}$.
This is called a $\textbf{power spectral density}$. Now the WK theorem
reads\[
\int_{\Omega=0}^{\infty}S_{e}(\Omega)\:\cos\left(\Omega\tau\right)\:\frac{d\Omega}{2\pi}=\frac{1}{T}\int_{t}f(t)f(t+\tau)\: dt\]
 The thing on the right hand side is the autocorrelation function
of $f$. The prefactor normalizes the correlation. Because we now
have this normalization we can replace this autocorrelation with an
$\emph{ensemble}$ average:\[
\frac{1}{T}\int_{t}f(t)f(t+\tau)\: dt=\langle f(0)f(\tau)\rangle\]
 where the equality of these two {}``averages'' is the definition
of an $\textbf{ergodic}$ process, and the enemble average can be
written as $\langle f(0)f(\tau)\rangle$ if we assume the process
is $\textbf{stationary}$. All in all this gives us\[
\int_{\Omega=0}^{\infty}S_{e}(\Omega)\:\cos\left(\Omega\tau\right)\:\frac{d\Omega}{2\pi}=\langle f(0)f(\tau)\rangle\]

\end{itemize}

\section{Delta functions}

\begin{equation}
\int T \frac{\sin \left( \omega T / 2 \right)}{\left( \omega T / 2 \right)} \frac{d\omega}{2\pi} = 1 \end{equation}

\end{document}

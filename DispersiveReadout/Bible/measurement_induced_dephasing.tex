\levelstay{Measurement induced dephasing} \label{sec:measurementInducedDephasing}

As photons enter the resonator and acquire a qubit state dependent phase shift, they carry some information on the qubit state.
By transferring quantum information from the qubit to the photons, the measurement process partially collapses the qubit state \cite{Murch:trajectories2013}.
In dispersive measurement, the photon's phase shift carries information about the qubit's projection along the z-axis of the Bloch sphere.
Therefore, the partial collapse induced by the scattered photons can be understood as qubit dephasing, similar in principle to the example of the phonon-induced decoherence of the transistor in Chapter \ref{ch:Introduction}.
The dephasing is only partial because the state of the qubit cannot be unambiguously determined from a single scattered photon, as will become clear shortly.

In the following discussion, we derive a relation between the measurement visibility and the qubit dephasing induced by the measurement process.
We do this in two ways.
First, we use a general ``information theory'' approach.
We work from mathematical constraints on the form of the quantum density matrix with no reference to a particular qubit system or measurement strategy.
This approach is the most general, making no connection to the actual mechanism by which the qubit dephases.
Second, we work from explicit form of the dispersive interaction.
We compute the entangled qubit-photon state and understand the qubit dephasing as a consequence of the entanglement.
This approach offers a simple interpretation in which the qubit dephasing comes from the random ac Stark shift imposed by the uncertainty in the the number of photons in the resonator.

\subimportlevel{./}{information_dephasing}{1}

\subimportlevel{./}{photon_dephasing}{1}

\subimportlevel{./}{amplifier}{1}

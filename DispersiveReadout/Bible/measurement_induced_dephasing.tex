\section{Measurement induced dephasing} \label{sec:measurementInducedDephasing}

As photons enter the resonator and acquire a qubit state dependent phase shift, they carry some information on the qubit state.
By transferring quantum information from the qubit to the photons, the measurement process partially collapses the qubit state \cite{Murch:trajectories2013}.
In dispersive measurement, the photon's phase shift carries information about the qubit's projection along the z-axis of the Bloch sphere.
Therefore, the partial collapse induced by the scattered photons can be understood as qubit dephasing, similar in principle to the example of the phonon-induced decoherence of the transistor in Chapter \ref{ch:Introduction}.
The dephasing is only partial because the state of the qubit cannot be unambiguously determined from a single scattered photon, as will become clear shortly.

In the following discussion, we derive a relation between the measurement visibility and the qubit dephasing induced by the measurement process.
We do this in two ways.
First, we use a general ``information theory'' approach.
We work from mathematical constraints on the form of the quantum density matrix with no reference to a particular qubit system or measurement strategy.
This approach is the most general, making no connection to the actual mechanism by which the qubit dephases.
Second, we work from explicit form of the dispersive interaction.
We compute the entangled qubit-photon state and understand the qubit dephasing as a consequence of the entanglement.
This approach offers a simple interpretation in which the qubit dephasing comes from the random ac Stark shift imposed by the uncertainty in the the number of photons in the resonator.

\levelstay{Information theoretic approach}

The phase coherence of a state of a 2-level system is described by the off-diagonal terms in the density matrix, $\rho_{10}$ and $\rho_{01}$.
The diagonal terms $\rho_{00}$ and $\rho_{11}$ are just the probabilities $P(0)$ and $P(1)$ that the qubit is in $\ket{0}$ or $\ket{1}$, respectively.
From the requirement that the density matrix must be positive-semidefinite,\footnote{This guarantees positive eigenvalues} it can be shown that \begin{equation}
\left| \rho_{10} \right| \leq \sqrt{\rho_{00} \rho_{11}} = \sqrt{P(0)P(1)} . \label{eq:sec:measurementInducedDephasing:inequality} \end{equation}
Suppose we measure the qubit along the z-axis with a meter which yields a single real number with value $x$.
Equation (\ref{eq:sec:measurementInducedDephasing:inequality}) yields a new inequality conditional on the measured value $x$, \begin{equation}
\left| \rho_{10} \right|(x) \leq \sqrt{P(0|x) P(1|x)} \equiv I(x) . \label{eq:sec:measurementInducedDephasing:inequality_x} \end{equation}
To quantify the amount of information about the qubit state we have learned from the measurement, we consider the probability, given the result $x$, that the qubit is in $\ket{0}$.
Using Bayes's theorem, we can write \begin{equation}
P(0|x) = \frac{P(x|0)P(0)}{P(x)}. \label{eq:sec:measurementInducedDephasing:bayes0} \end{equation}
In English, Eq.\,(\ref{eq:sec:measurementInducedDephasing:bayes0}) reads ``The probability that the qubit is in $\ket{0}$ given that we measured $x$, is equal to the probability that we would measure $x$ if the qubit were in $\ket{0}$, times the probability that the qubit is in $\ket{0}$, divided by the probability that we would measure $x$.''
We write a similar equation for $P(1|x)$, \begin{equation}
P(1|x) = \frac{P(x|1)P(1)}{P(x)}. \label{eq:sec:measurementInducedDephasing:bayes1} \end{equation}
Suppose we are given a qubit in the state $\left( \ket{0} + \ket{1} \right) / \sqrt{2}$.
Then $P(0)=P(1)=1/2$.
Combining these results we can rewrite right right hand side of Eq.\,\ref{eq:sec:measurementInducedDephasing:inequality_x} as \begin{equation}
I(x) \equiv \sqrt{\frac{P(x|0)P(x|1)P(0)P(1)}{P(x)^2}} = \frac{\sqrt{P(x|0)P(x|1)}}{2 P(x)} . \end{equation}
The coherence of the qubit is limited by the total information contained in the scattered photon.
To recover this information, we must average over all possible detector values $x$, \begin{equation}
\left| \rho_{10} \right| \leq \int_{x = -\infty}^{\infty} I(x) P(x) \, dx = \int_{x = -\infty}^{\infty} \frac{1}{2}\sqrt{P(x|0)P(x|1)}\,dx . \label{eq:sec:measurementInducedDephasing:integralForm} \end{equation}

Suppose he measured voltages are Gaussian distributed, with the $\ket{0}$ and $\ket{1}$ having different means, \begin{equation}
P(x|0) = \frac{1}{\sqrt{2\pi \sigma^2}}\exp \left[ \frac{-(x-x_0)^2}{2\sigma^2} \right] \quad P(x|1) = \frac{1}{\sqrt{2\pi \sigma^2}}\exp \left[ \frac{-(x-x_1)^2}{2\sigma^2} \right] . \end{equation}
Plugging these expressions into Eq.\,(\ref{eq:sec:measurementInducedDephasing:integralForm}) yields \begin{eqnarray}
\left| \rho_{10} \right| &\leq& \int_{x = -\infty}^{\infty} \frac{1}{2}\frac{1}{\sqrt{2\pi \sigma^2}} \exp \left[ \frac{-(x-x_0)^2 - (x-x_1)^2}{4 \sigma^2} \right] \, dx \nonumber \\
&\leq& \frac{1}{2} \exp \left[- \frac{(x_0 - x_1)^2}{8\sigma^2} \right] \\
&\leq& \frac{1}{2} \exp \left[- \frac{\text{SNR}}{4} \right] . \label{eq:sec:measurementInducedDephasing:measurementInducedDephasing} \end{eqnarray}
In the last line we used Eq.\,(\ref{eq:sec:lollipops:SNR}) for the definition of SNR.
Equation (\ref{eq:sec:measurementInducedDephasing:measurementInducedDephasing}) provides the quantitative link between the measurement SNR and qubit phase decoherence.
As scattered photons are collected, the separation $x_0 - x_1$ increases and the upper bound on $\rho_{10}$ decreases.\footnote{Or, if the data is normalized to a constant value of $x_0 - x_1$, the widths $\sigma$ of the Gaussian curves decreases.}
In other words, increased visibility between the qubit states decreases phase coherence.


\levelstay{Physical mechanism approach}

In this section, we calculate the photon dephasing by explicitly accounting for the physical interaction between the qubit and measurement photons.
We start from the qubit-resonator interaction Hamiltonian \begin{equation}
H_I/\hbar = -\chi n \sigma_z . \end{equation}
Consider an initial quantum state \begin{align}
\ket{\Psi_i}
&= \ket{\alpha}(1/\sqrt{2})\left( \ket{g} + \ket{e} \right) . \end{align}
% &= e^{-|\alpha|^2 / 2}e^{\alpha a^{\dagger}}\ket{0} \otimes \frac{1}{\sqrt{2}} \left( 1 + \sigma_x \right)\ket{g} . \end{align}
While the resonator is in the ground state $\ket{0}$, the interaction Hamiltonian is identically zero.
When we turn on the probe signal, the resonator photon number increases and the resonator emits travelling waves with a phase $\phi$ determined by Eq.\,(\ref{eq:S21}) and the the qubit state.
Assuming we probe at a frequency in between the two possible resonator frequencies, the phases for the two qubit states have the same magnitude and opposite sign.
Therefore, the output state is \begin{align}
\ket{\Psi_f} &= \frac{1}{\sqrt{2}} \left( \ket{\alpha e^{i \phi}}\ket{g} + \ket{\alpha e^{-i \phi}}\ket{e} \right) . \label{eq:sec:measurementInducedDephasing:dispersedState} \end{align}

%We subject this initial state to the time evolution for time $t$.
%The time evolution operator is \begin{equation}
%U(t) = \exp \left[ i \chi n \sigma_z \right] \end{equation}
%which produces a final state \begin{align}
%\ket{\Psi_f}
%&= U \ket{\Psi_i} \\
%\frac{1}{\sqrt{2}} \left( \ket{\alpha e^{i\chi t}}\ket{g} + \ket{\alpha e^{-i \chi t}}\ket{e} \right) . \end{equation}

The distance between the two dispersed photon states is \begin{align}
\delta x &=
\langle x \rangle_{\alpha \exp(i \phi)} - \langle x \rangle_{\alpha \exp(-i \phi)} \\
&= \frac{1}{2}\langle a + a^{\dagger} \rangle_{\alpha \exp(i \phi)} - \frac{1}{2}\langle a + a^{\dagger} \rangle_{\alpha \exp(-i \phi)} \\
&= 2| \alpha | \sin \left( \phi \right) . \end{align}
From \citeinternaltype \citeinternalref{quantumOscillator}, the variance along any axis through the center of a coherent state is \begin{equation}
\sigma^2 = 1/4 . \end{equation}
Therefore, the signal to noise ratio for the two dispersed photon states is \begin{align}
\text{SNR}
&\equiv \frac{\delta x^2}{2 \sigma^2} \\
&= \frac{\left( 2 | \alpha | \sin \left( \phi \right) \right)^2}{1/2} \\
&= 8 \left| \alpha \right|^2 \sin \left(\phi \right)^2 . \label{eq:dispersedSNR}
\end{align}


\leveldown{Qubit dephasing - full calculation}

Now we would like to look at the phase coherence of the qubit.
To do this, we start with the density matrix for the entangled state $\ket{\Psi_f}$, and then find the reduced density matrix of the qubit with the resonator removed.
The part of the density describing the qubit phase coherence is \begin{align}
\rho_{01} &= \frac{1}{2} \left( \ket{\alpha e^{i \phi}} \bra{\alpha e^{-i \phi}} \otimes \ket{g}\bra{e} \right) \\
&= \frac{1}{2} \exp\left[-\left| \alpha \right|^2 \right]
\sum_{n,m}
\frac{ \left( \alpha e^{i \phi} \right)^n}{\sqrt{n!}}
\frac{ \left( \alpha^* e^{i \phi} \right)^m}{\sqrt{m!}}
\ket{n}\bra{m} \otimes \ket{g}\bra{e} . \end{align}
To find the reduced density matrix of the qubit, we trace over the resonator states \begin{align}
\text{Tr}_{\text{res}}\rho_{10} &=
\frac{1}{2} \exp \left[-\left| \alpha \right|^2 \right]
\sum_{n,m,k}
\frac{ \left( \alpha e^{i \phi} \right)^n}{\sqrt{n!}}
\frac{ \left( \alpha^* e^{i \phi} \right)^m}{\sqrt{m!}} \braket{k}{n}\braket{m}{k} \otimes \ket{g}\bra{e} \\
&= \frac{1}{2} \exp \left[- \left| \alpha \right|^2 \right] \exp \left[\left| \alpha \right|^2 e^{2 i \phi} \right] \ket{g}\bra{e} \\
&= \frac{1}{2} \exp \left[- \left| \alpha \right|^2 \right] \exp \left[\left| \alpha \right|^2
\left( \cos\left( 2 \phi \right) + i \sin \left( 2 \phi \right) \right) \right] \ket{g}\bra{e} . \end{align}
Note that the effect of the trace is to select only those terms for which the resonator ``has a definite photon number''.
To find the qubit phase coherence, we look for the magnitude of the off diagonal element \begin{align}
\left| \text{Tr}_{\text{res}}\rho_{10} \right| &=
\frac{1}{2} \exp \left[- \left| \alpha \right|^2 \right] \exp \left[\left| \alpha \right|^2 \cos\left( 2 \phi \right) \right] \\
&= \frac{1}{2} \exp \left[ -\left| \alpha \right|^2 \left( 1 - \cos \left( 2\phi\right) \right) \right] \\
&= \frac{1}{2} \exp \left[ - \frac{\text{SNR}}{4} \right] \label{eq:photonLimitSNR}
\end{align}
which is exactly the same expression we found in Eq.\,(\ref{eq:sec:measurementInducedDephasing:measurementInducedDephasing}) using the information theory approach.
Therefore, we have shown that the qubit dephasing incurred by dispersive measurement is equivalently understood as either an effect of the information extracted from the system, or as an effect of the entanglement between the qubit and the photon.

Note that we have demonstrated that decoherence is really just an a result of considering a sub-part of an entangled quantum system.


\levelstay{Qubit dephasing - simple calculation}

The trace over resonator states in the full calculation is just a mathematically rigorous way to select components of the state with definite photon number.
This suggests a simpler approach to the problem: we could just do a weighted average of the qubit density matrix over the resonator photon number states.
From this point of view, the qubit dephasing comes simply from the random ac Stark shift imposed by the ``uncertainty'' in the resonator photon number.
Here is the calculation:

\begin{align}
\rho_{01}
&= \sum_n \rho_{01}(n) P(n) \\
&= \sum_n \frac{1}{2} e^{i 2 n \phi} e^{-\bar{n}} \frac{\bar{n}^n}{n!} \\
&= \frac{1}{2} e^{-\bar{n}} \sum_n \frac{\left( e^{i 2 \phi}\bar{n}\right)^n}{n!} \\
&= \frac{1}{2} e^{-\bar{n}} \exp \left[ \bar{n} e^{i 2 \phi} \right] \\
&= \frac{1}{2} e^{-|\alpha|^2} \exp \left[ |\alpha|^2 e^{i 2 \phi} \right] \\
&= \frac{1}{2} \exp \left[ -|\alpha|^2 \left( 1 - e^{i 2 \phi} \right) \right] . \end{align}
Taking the aboslute value leaves \begin{align}
|\rho_{01}|
&= \frac{1}{2} \exp \left[ -|\alpha|^2 \left( 1 - \cos\left(2 \phi \right) \right) \right] \\
&= \frac{1}{2} \exp \left[ - \frac{\text{SNR}}{4} \right]  \end{align}
which matches the full calculation.


\levelstay{Amplifier}

In this section we study the effect of amplifying the dispersed photons.
Before investigating the effect of the amplifier on the photon signal to noise ratio, we explicitly show that the amplifier does not affect the qubit state.
Consider an arbitrary state $\ket{\Psi}$ for the qubit-resonator system \begin{equation}
\ket{\Psi} = \sum_{\alpha \beta} c_{\alpha \beta} \ket{\alpha} \otimes \ket{\beta} . \end{equation}
The density matrix for this state is \begin{equation}
\rho = \sum_{\alpha \beta \gamma \delta} c_{\alpha \beta}c_{\gamma \delta}^* \ket{\alpha}\bra{\gamma}\otimes \ket{\beta}\bra{\delta} . \end{equation}
Applying an arbitrary transformation $U$ to the resonator changes the density matrix to \begin{equation}
\rho = \sum_{\alpha \beta \gamma \delta} c_{\alpha \beta}c_{\gamma \delta}^* U\ket{\alpha}\bra{\gamma}U^{\dagger} \otimes \ket{\beta}\bra{\delta} . \end{equation}
Now we compute the reduced density matrix of the qubit by tracing over the resonator states \begin{align}
\rho_{\text{qubit}} &= \text{Tr}_{\text{res}} \rho \\
&= \sum_{n, \alpha \beta \gamma \delta} c_{\alpha \beta} c_{\gamma \delta}^* \bra{n}U\ket{\alpha}\bra{\gamma}U^{\dagger}\ket{n} \otimes \ket{\beta}\bra{\delta} \\
&= \sum_{n, \alpha \beta \gamma \delta} c_{\alpha \beta} c_{\gamma \delta}^* \bbraket{\gamma}{U^{\dagger}}{n} \bbraket{n}{U}{\alpha} \otimes \ket{\beta}\bra{\delta} \\
&= \sum_{\alpha \beta \gamma \delta} c_{\alpha \beta} c_{\gamma \delta}^* \bbraket{\gamma}{U^{\dagger}U}{\alpha} \otimes \ket{\beta}\bra{\delta} \\
&= \sum_{\alpha \beta \gamma \delta} c_{\alpha \beta} c_{\gamma \delta}^* \braket{\gamma}{\alpha} \otimes \ket{\beta}\bra{\delta} \\
&= \sum_{\alpha \beta \delta} c_{\alpha \beta} c_{\alpha \delta}^* \ket{\beta}\bra{\delta} .
\end{align}
The effect of $U$ has disappeared, indicating that the reduced density matrix for the qubit is unaffected by $U$.
Therefore, the qubit state is unchanged by any subsequent actions on the photon, such as the action of an amplifier.

\leveldown{Phase sensitive amplifier}

In this subsection, we calculate the signal to noise ratio of dispersed coherent states once they have been amplified by an ideal phase sensitive amplifier.
A phase sensitive amplifier amplifies only one of the $\sin$ and $\cos$ quadratures of a signal.
Representing the action of the amplifier by an operator $S$, the output for a single coherent state input is $S\ket{\alpha}$.
For the phase sensitive amplifier, the operator $S$ is the squeezing operator \begin{equation}
S(z) = \exp \left[ \frac{1}{2} \left( z^* a^2 - z a^{\dagger^2} \right) \right] \qquad z=re^{i\theta} . \end{equation}
In the case $\theta = 0$, $S$ transforms the annihilation operator in a simple way: \begin{equation}
S(r)aS(r)^{\dagger} = \mu a + \nu a^{\dagger} , \qquad \mu = \cosh(r) \quad \nu = \sinh(r) . \end{equation}
Note that $S(r)^{\dagger} = S(-r)$, so \begin{equation}
S(r)^{\dagger}aS(r) = S(-r)aS(-r)^{\dagger} = \mu a - \nu a^{\dagger} . \end{equation}
For simplicity, we assume in the following computations that $\phi = \pm \pi/2$, so the two dispersed photon states are $\ket{\alpha}$ and $\ket{-\alpha}$.
We compute the expectation value of $x$ for $S\ket{\alpha}$:
\begin{align}
\bbraket{S\alpha}{x}{S\alpha} &= \frac{1}{2} \bbraket{\alpha}{S^{\dagger} \left( a+a^{\dagger} \right)S}{\alpha} \\
&= \frac{1}{2} \bbraket{\alpha}{\mu a -\nu a^{\dagger} + \mu a^{\dagger} - \nu a}{\alpha} \\
&= \alpha ( \mu - \nu ) . \end{align}
Note that for $r \ll 0$, $\mu - \nu$ is a large number, indicating that the amplifier provides gain.
The signal is the distance between the two dispersed states, \begin{equation}
\delta x =2\langle x \rangle = 2\alpha(\mu - \nu) . \end{equation}

Next, we compute the expectation value of $x^2$:
\begin{align}
\bbraket{S\alpha}{x^2}{S\alpha} &= \frac{1}{4}\bbraket{\alpha}{S^{\dagger}\left[ \left( a+a^{\dagger} \right)^2 \right] S}{\alpha} \\
&= \frac{1}{4} \bbraket{\alpha}{\left( \mu a-\nu a^{\dagger} + \mu a^{\dagger} - \nu a \right)^2}{\alpha} \\
&= \frac{1}{4} \left( \mu - \nu \right)^2 + |\alpha|^2 \left(\mu - \nu \right)^2 . \end{align}
The noise for an amplified state $S\ket{\alpha}$ is therefore \begin{align}
\sigma^2 &\equiv \langle(x - \langle x \rangle)^2 \rangle \\
&= \langle x^2 \rangle - \langle x \rangle^2 \\
&= \frac{1}{4}(\mu - \nu)^2 . \end{align}
Finally, the signal to noise ratio is \begin{align}
\text{SNR} &= \frac{\delta x^2}{2 \sigma^2} \\
&= \frac{4 \alpha^2 (\mu - \nu)^2}{2 \frac{1}{4} (\mu - \nu)^2} \\
&= 8 \alpha^2 . \end{align}
This is the same as the SNR we found before the amplification, as given by Eq.\,(\ref{eq:dispersedSNR}) in the case $\phi = \pi/2$.
Therefore, the ideal phase sensitive amplifier does not change the SNR.


\levelstay{Phase insensitive amplifier}

Phase preserving amplifiers (also called phase-insensitive amplifiers) are amplifiers which, like a traditional electronic amplifiers, amplify both the $\sin$ and $\cos$ quadratures of a signal.
In other words, they preserve the phase of the input signal.
It turns out that an ideal noiseless linear phase preserving amplifier which independently amplifies each frequency cannot exist \cite{Caves:amplifiers1982}.
To preserve the commutation relations of the two quadratures of the photon state, the amplifier must mix at least two frequencies.
Thus, the action of the phase preserving amplifier is represented by the two-mode squeezing operator $S_2$
\begin{equation}
S_2(z) = \exp \left[ z^* a b - z a^\dagger b^\dagger \right] \end{equation}
where the $a$ and $a^{\dagger}$ operators correspond to the main mode called the ``signal'', and $b$ and $b^{\dagger}$ operators correspond to an auxiliary mode called the ``idler''.
The $S_2$ operator transforms the creation an annihilation operators as follows:
\begin{align}
S_2^\dagger(z) a S_2(z) &= \mu a - e^{i\theta} \nu b^\dagger \\
S_2^\dagger(z) b S_2(z) &= \mu b - e^{i\theta} \nu a^\dagger \\
S_2^\dagger(z) a^\dagger S_2(z) &= \mu a^\dagger - e^{-i\theta} \nu b \\
S_2^\dagger(z) b^\dagger S_2(z) &= \mu b^\dagger - e^{-i\theta} \nu a
\end{align}
where $z\equiv r e^{i \theta}$, $\mu = \cosh(r)$, and $\nu = \sinh(r)$.
For simplicity, we assume that $z$ is real so that $\theta=0$.

Now we calculate the gain and uncertainty in $x$ for the amplified state.
First, let us calculate the expectation value of $x$ for an amplified state $S_2 \ket{\alpha}$ on the real axis,
\begin{align}
\bbraket{S_2 \alpha}{x}{S_2 \alpha} =&
\frac{1}{2}\bbraket{\alpha}{S_2^\dagger (a + a^\dagger) S_2}{\alpha} \nonumber \\
=& \frac{1}{2} \bbraket{\alpha}{\mu a - \nu b^\dagger + \mu a^\dagger - \nu b}{\alpha} \nonumber \\
=& \mu \alpha .
\end{align}
Thus, the gain of the phase preserving amplifier is $\mu$.
Next, we calculate $\langle x^2 \rangle$,
\begin{align}
\bbraket{S_2 \alpha}{x^2}{S_2 \alpha} =& 
\bbraket{\alpha}{S_2^\dagger x S_2 S_2^\dagger x S_2}{\alpha} \nonumber \\
=& \frac{1}{4}\bbraket{\alpha}{(\mu a - \nu b^\dagger + \mu a^\dagger - \nu b)^2}{\alpha} \nonumber \\
=& \frac{1}{4}\bbraket{\alpha}{\mu^2 a^2 - 2\mu\nu a b^\dagger \nonumber \\
& + \mu^2 (2a^\dagger a + 1) - 2 \mu \nu a b \nonumber \\ 
& + \nu^2 b^\dagger b^\dagger - 2 \mu \nu b^\dagger a^\dagger + \nu^2 (2 b^\dagger b + 1) \nonumber \\
& + \mu^2 a^\dagger a^\dagger - \mu \nu a^\dagger b \nonumber \\
& + \nu^2 b}{\alpha} \nonumber \\
=& \frac{1}{4} \left( \mu^2 \alpha^2 + \mu^2 (2 \alpha^2 + 1) + \mu^2 \alpha^2 + \nu^2 \right) \nonumber \\
\approx & \mu^2 \left( \alpha^2 + \frac{1}{2} \right) .
\end{align}
The approximation in the last line is for large gain where $\mu \approx \nu$.
We now compute the variance,
\begin{equation}
\sigma^2 \equiv \langle x^2 \rangle - \langle x \rangle^2 = \frac{1}{2} \mu^2.
\end{equation}

Finally we calculate the SNR,
\begin{align}
\text{SNR} =& \frac{\delta x^2}{2 \sigma^2} \\
=& \frac{(2\mu \alpha)^2}{\mu^2} \\
=& 4\alpha^2.
\end{align}
For the phase preserving amplifier, the SNR is half that of the phase sensitive amplifier.
In particular, when using a phase preserving amplifier, the upper limit on the ratio of SNR to qubit dephasing is one half of the limit allowed by quantum mechanics.


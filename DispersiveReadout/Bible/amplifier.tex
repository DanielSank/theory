\subsection{Amplifier}

In this section we study the effect of amplifying the dispersed photons.
Before investigating the effect of the amplifier on the photon signal to noise ratio, we explicitly show that the amplifier does not affect the qubit state.
Consider an arbitrary state $\ket{\Psi}$ for the qubit-resonator system \begin{equation}
\ket{\Psi} = \sum_{\alpha \beta} c_{\alpha \beta} \ket{\alpha} \otimes \ket{\beta} . \end{equation}
The density matrix for this state is \begin{equation}
\rho = \sum_{\alpha \beta \gamma \delta} c_{\alpha \beta}c_{\gamma \delta}^* \ket{\alpha}\bra{\gamma}\otimes \ket{\beta}\bra{\delta} . \end{equation}
Applying an arbitrary transformation $U$ to the resonator changes the density matrix to \begin{equation}
\rho = \sum_{\alpha \beta \gamma \delta} c_{\alpha \beta}c_{\gamma \delta}^* U\ket{\alpha}\bra{\gamma}U^{\dagger} \otimes \ket{\beta}\bra{\delta} . \end{equation}
Now we compute the reduced density matrix of the qubit by tracing over the resonator states \begin{align}
\rho_{\text{qubit}} &= \text{Tr}_{\text{res}} \rho \\
&= \sum_{n, \alpha \beta \gamma \delta} c_{\alpha \beta} c_{\gamma \delta}^* \bra{n}U\ket{\alpha}\bra{\gamma}U^{\dagger}\ket{n} \otimes \ket{\beta}\bra{\delta} \\
&= \sum_{n, \alpha \beta \gamma \delta} c_{\alpha \beta} c_{\gamma \delta}^* \bbraket{\gamma}{U^{\dagger}}{n} \bbraket{n}{U}{\alpha} \otimes \ket{\beta}\bra{\delta} \\
&= \sum_{\alpha \beta \gamma \delta} c_{\alpha \beta} c_{\gamma \delta}^* \bbraket{\gamma}{U^{\dagger}U}{\alpha} \otimes \ket{\beta}\bra{\delta} \\
&= \sum_{\alpha \beta \gamma \delta} c_{\alpha \beta} c_{\gamma \delta}^* \braket{\gamma}{\alpha} \otimes \ket{\beta}\bra{\delta} \\
&= \sum_{\alpha \beta \delta} c_{\alpha \beta} c_{\alpha \delta}^* \ket{\beta}\bra{\delta} .
\end{align}
The effect of $U$ has disappeared, indicating that the reduced density matrix for the qubit is unaffected by $U$.
Therefore, the qubit state is unchanged by any subsequent actions on the photon, such as the action of an amplifier.

\subsubsection{Phase sensitive amplifier}

In this subsection, we calculate the signal to noise ratio of dispersed coherent states once they have been amplified by an ideal phase sensitive amplifier.
A phase sensitive amplifier amplifies only one of the $\sin$ and $\cos$ quadratures of a signal.
Representing the action of the amplifier by an operator $S$, the output for a single coherent state input is $S\ket{\alpha}$.
For the phase sensitive amplifier, the operator $S$ is the squeezing operator \begin{equation}
S(z) = \exp \left[ \frac{1}{2} \left( z^* a^2 - z a^{\dagger^2} \right) \right] \qquad z=re^{i\theta} . \end{equation}
In the case $\theta = 0$, $S$ transforms the annihilation operator in a simple way: \begin{equation}
S(r)aS(r)^{\dagger} = \mu a + \nu a^{\dagger} , \qquad \mu = \cosh(r) \quad \nu = \sinh(r) . \end{equation}
Note that $S(r)^{\dagger} = S(-r)$, so \begin{equation}
S(r)^{\dagger}aS(r) = S(-r)aS(-r)^{\dagger} = \mu a - \nu a^{\dagger} . \end{equation}
For simplicity, we assume in the following computations that $\phi = \pm \pi/2$, so the two dispersed photon states are $\ket{\alpha}$ and $\ket{-\alpha}$.
We compute the expectation value of $x$ for $S\ket{\alpha}$:
\begin{align}
\bbraket{S\alpha}{x}{S\alpha} &= \frac{1}{2} \bbraket{\alpha}{S^{\dagger} \left( a+a^{\dagger} \right)S}{\alpha} \\
&= \frac{1}{2} \bbraket{\alpha}{\mu a -\nu a^{\dagger} + \mu a^{\dagger} - \nu a}{\alpha} \\
&= \alpha ( \mu - \nu ) . \end{align}
Note that for $r \ll 0$, $\mu - \nu$ is a large number, indicating that the amplifier provides gain.
The signal is the distance between the two dispersed states, \begin{equation}
\delta x =2\langle x \rangle = 2\alpha(\mu - \nu) . \end{equation}

Next, we compute the expectation value of $x^2$:
\begin{align}
\bbraket{S\alpha}{x^2}{S\alpha} &= \frac{1}{4}\bbraket{\alpha}{S^{\dagger}\left[ \left( a+a^{\dagger} \right)^2 \right] S}{\alpha} \\
&= \frac{1}{4} \bbraket{\alpha}{\left( \mu a-\nu a^{\dagger} + \mu a^{\dagger} - \nu a \right)^2}{\alpha} \\
&= \frac{1}{4} \left( \mu - \nu \right)^2 + |\alpha|^2 \left(\mu - \nu \right)^2 . \end{align}
The noise for an amplified state $S\ket{\alpha}$ is therefore \begin{align}
\sigma^2 &\equiv \langle(x - \langle x \rangle)^2 \rangle \\
&= \langle x^2 \rangle - \langle x \rangle^2 \\
&= \frac{1}{4}(\mu - \nu)^2 . \end{align}
Finally, the signal to noise ratio is \begin{align}
\text{SNR} &= \frac{\delta x^2}{2 \sigma^2} \\
&= \frac{4 \alpha^2 (\mu - \nu)^2}{2 \frac{1}{4} (\mu - \nu)^2} \\
&= 8 \alpha^2 . \end{align}
This is the same as the SNR we found before the amplification, as given by Eq.\,(\ref{eq:dispersedSNR}) in the case $\phi = \pi/2$.
Therefore, the ideal phase sensitive amplifier does not change the SNR.


\subsubsection{Phase insensitive amplifier}

Phase preserving amplifiers (also called phase-insensitive amplifiers) are amplifiers which, like a traditional electronic amplifiers, amplify both the $\sin$ and $\cos$ quadratures of a signal.
In other words, they preserve the phase of the input signal.
It turns out that an ideal noiseless linear phase preserving amplifier which independently amplifies each frequency cannot exist \cite{Caves:amplifiers1982}.
To preserve the commutation relations of the two quadratures of the photon state, the amplifier must mix at least two frequencies.
Thus, the action of the phase preserving amplifier is represented by the two-mode squeezing operator $S_2$
\begin{equation}
S_2(z) = \exp \left[ z^* a b - z a^\dagger b^\dagger \right] \end{equation}
where the $a$ and $a^{\dagger}$ operators correspond to the main mode called the ``signal'', and $b$ and $b^{\dagger}$ operators correspond to an auxiliary mode called the ``idler''.
The $S_2$ operator transforms the creation an annihilation operators as follows:
\begin{align}
S_2^\dagger(z) a S_2(z) &= \mu a - e^{i\theta} \nu b^\dagger \\
S_2^\dagger(z) b S_2(z) &= \mu b - e^{i\theta} \nu a^\dagger \\
S_2^\dagger(z) a^\dagger S_2(z) &= \mu a^\dagger - e^{-i\theta} \nu b \\
S_2^\dagger(z) b^\dagger S_2(z) &= \mu b^\dagger - e^{-i\theta} \nu a
\end{align}
where $z\equiv r e^{i \theta}$, $\mu = \cosh(r)$, and $\nu = \sinh(r)$.
For simplicity, we assume that $z$ is real so that $\theta=0$.

Now we calculate the gain and uncertainty in $x$ for the amplified state.
First, let us calculate the expectation value of $x$ for an amplified state $S_2 \ket{\alpha}$ on the real axis,
\begin{align}
\bbraket{S_2 \alpha}{x}{S_2 \alpha} =&
\frac{1}{2}\bbraket{\alpha}{S_2^\dagger (a + a^\dagger) S_2}{\alpha} \nonumber \\
=& \frac{1}{2} \bbraket{\alpha}{\mu a - \nu b^\dagger + \mu a^\dagger - \nu b}{\alpha} \nonumber \\
=& \mu \alpha .
\end{align}
Thus, the gain of the phase preserving amplifier is $\mu$.
Next, we calculate $\langle x^2 \rangle$,
\begin{align}
\bbraket{S_2 \alpha}{x^2}{S_2 \alpha} =& 
\bbraket{\alpha}{S_2^\dagger x S_2 S_2^\dagger x S_2}{\alpha} \nonumber \\
=& \frac{1}{4}\bbraket{\alpha}{(\mu a - \nu b^\dagger + \mu a^\dagger - \nu b)^2}{\alpha} \nonumber \\
=& \frac{1}{4}\bbraket{\alpha}{\mu^2 a^2 - 2\mu\nu a b^\dagger \nonumber \\
& + \mu^2 (2a^\dagger a + 1) - 2 \mu \nu a b \nonumber \\ 
& + \nu^2 b^\dagger b^\dagger - 2 \mu \nu b^\dagger a^\dagger + \nu^2 (2 b^\dagger b + 1) \nonumber \\
& + \mu^2 a^\dagger a^\dagger - \mu \nu a^\dagger b \nonumber \\
& + \nu^2 b}{\alpha} \nonumber \\
=& \frac{1}{4} \left( \mu^2 \alpha^2 + \mu^2 (2 \alpha^2 + 1) + \mu^2 \alpha^2 + \nu^2 \right) \nonumber \\
\approx & \mu^2 \left( \alpha^2 + \frac{1}{2} \right) .
\end{align}
The approximation in the last line is for large gain where $\mu \approx \nu$.
We now compute the variance,
\begin{equation}
\sigma^2 \equiv \langle x^2 \rangle - \langle x \rangle^2 = \frac{1}{2} \mu^2.
\end{equation}

Finally we calculate the SNR,
\begin{align}
\text{SNR} =& \frac{\delta x^2}{2 \sigma^2} \\
=& \frac{(2\mu \alpha)^2}{\mu^2} \\
=& 4\alpha^2.
\end{align}
For the phase preserving amplifier, the SNR is half that of the phase sensitive amplifier.
In particular, when using a phase preserving amplifier, the upper limit on the ratio of SNR to qubit dephasing is one half of the limit allowed by quantum mechanics.

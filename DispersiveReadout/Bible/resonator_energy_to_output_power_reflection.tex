\levelstay{Resonator energy to output power ratio: reflection}

We derive an equation relating the energy in the measurement resonator to the power leaving the system for a reflection circuit.
This ratio is an important quantity as output power is a limited resource due to the finite saturation power of quantum limited amplifiers, and resonator energy is a limited resource because of readout-induced transitions.

A circuit diagram of the system is shown in Figure \ref{Fig:dispersive_readout.reflection_circuit}.

\quickfig{\columnwidth}
{reflection_circuit.pdf}
{Reflection circuit.}
{Fig:dispersive_readout.reflection_circuit}

\leveldown{Resonator internal energy}

Denote the impedance of the resonator as $Z_r$ and the impedance of the capacitor coupling the resonator to the outside by $Z_c$.
The incoming voltage wave with amplitude $V_\text{in}$ reflects from the coupling capacitor, resulting in an outgoing wave with amplitude $\Gamma V_\text{in}$ where reflection coefficient $\Gamma$ is
\begin{equation}
  \Gamma = \frac{Z_l - Z_0}{Z_l + Z_0}
\end{equation}
and where $Z_l$ is the load impedance $Z_l = Z_c + Z_r$.
Rearranging, we can write
\begin{equation}
  \frac{Z_l}{Z_0} = \frac{1 + \Gamma}{1 - \Gamma} \, .
\end{equation}
The voltage at the coupling capacitor is therefore
\begin{equation}
  V = V_\text{in} + \Gamma V_\text{in} = V_\text{in}(1 + \Gamma) \, .
\end{equation}
The voltage $V_d$ at the resonator is, by voltage division,
\begin{equation}
  V_d = V \left( \frac{Z_r}{Z_r + Z_c} \right) \, .
\end{equation}
The average energy in the resonator is therefore
\begin{equation}
  E
  = \frac{1}{2} C |V_d|^2
  = \frac{1}{2} C \abs{V_\text{in}}^2 \abs{1 + \Gamma}^2 \abs{\frac{Z_r}{Z_r + Z_c}}^2 \, .
\end{equation}

\levelstay{Reflected power}

The power reflecting from the system is
\begin{equation}
  P = \frac{1}{2} \abs{V_\text{in} \Gamma}^2 / Z_0 \, .
\end{equation}

\levelstay{Ratio}
We can now compute the ratio of output power to energy stored in the resonator:
\begin{align}
  \frac{P}{E}
    &= \frac{(1/2) \abs{V_\text{in}}^2 \abs{\Gamma}^2}
            {Z_0 (1/2) C \abs{V_\text{in}}^2 \abs{1 + \Gamma}^2}
       \abs{\frac{Z_r + Z_c}{Z_r}}^2 \\
    &= \frac{1}{Z_0 C} \frac{\abs{\Gamma}^2}{\abs{1 + \Gamma}^2}
       \abs{1 - \frac{Z_c}{Z_l}}^{-2} \\
    &= \frac{1}{Z_0 C} \frac{\abs{\Gamma}^2}{\abs{1 + \Gamma}^2}
       \abs{1 - \frac{Z_c}{Z_0} \left( \frac{1-\Gamma}{1+\Gamma} \right)}^{-2} \, .
\end{align}
From the appendix of Daniel's thesis, we know that the loaded $Q$ of a resonator that is capacitively coupled to a resistance $R$ through a capacitance $C_c$ is
\begin{equation}
  Q_l = \left( \frac{C}{C_c} \right)^2 \frac{Z_{LC}}{R}
\end{equation}
where $Z_{LC} \equiv \sqrt{L/C}$ is the characteristic impedance of the resonance.
Rearranging this for the impedance of the coupling capacitor $Z_c$, we get
\begin{equation}
  \frac{Z_c}{R} = i \sqrt{Q_l \frac{Z_{LC}}{R}} \, .
\end{equation}
Going back to the reflection measurement setup, we have $R \to Z_0$, so plugging in gives
\begin{align}
  \frac{P}{E}
    &= \frac{1}{Z_0 C} \frac{\abs{\Gamma}^2}{\abs{1 + \Gamma}^2}
       \abs{1 - i \sqrt{Q_l \frac{Z_{LC}}{Z_0}}
       \left( \frac{1-\Gamma}{1+\Gamma} \right)}^{-2} \, .
\end{align}

Now let us consider the case of optimized readout where $\Gamma = \pm i$.\footnote{I didn't check this and someone should!}
In the $\Gamma = i$ case, we get $(1-\Gamma)/(1+\Gamma)=-i$ and $\abs{1+\Gamma}^2=2$, so
\begin{align}
  \frac{P}{E}
  &= \frac{1}{Z_0 C} \frac{1}{2} \abs{1 - i \sqrt{Q_l \frac{Z_{LC}}{Z_0}} (-i)}^{-2} \\
  &= \frac{1}{Z_0 C} \frac{1}{2} \abs{1 - \sqrt{Q_l \frac{Z_{LC}}{Z_0}}}^{-2} \\
  (\text{assume } Q_l \gg 1) \quad &= \frac{1}{Z_0 C} \frac{1}{2} \frac{Z_0}{Q_l Z_{LC}} \\
  &= \frac{1}{2} \frac{\omega_r}{Q_l} \\
  &= \frac{\kappa}{2} \, .
\end{align}

For reflection measurement we have $g=2$, giving
\begin{align}
  \text{SNR}
    &= 4 \frac{PT}{S} \\
    (\text{use } P = \kappa E/2) \quad &= 4 \frac{\kappa E T}{2S} \\
    &= 2 \kappa n T
\end{align}
which is 8 times larger than the tranmission circuit case.

\documentclass{article}

% General physics constructs
\newcommand{\bra}[1]{\langle #1 |}
\newcommand{\ket}[1]{| #1 \rangle }
\newcommand{\braket}[2]{\langle #1|#2\rangle}
\newcommand{\bbraket}[3]{ \langle #1 | #2 | #3 \rangle }
\newcommand{\boltzmann}{k_b}

% Common math
\newcommand{\norm}[1]{\left \lvert #1 \right \rvert}
\newcommand{\abs}[1]{\left \lvert #1 \right \rvert}  % These two are redundant. Consider removing one.

\newcommand{\avg}[1]{\left \langle #1 \right \rangle}  % Should get rid of this, as "average" isn't specific.
\newcommand{\angavg}[1]{\left \langle #1 \right \rangle}

\newcommand{\VS}{\textit{\textbf{V}}}
\newcommand{\Tr}{\textrm{Tr}}
\renewcommand{\Re}{\textrm{Re}}
\renewcommand{\Im}{\textrm{Im}}
\newcommand{\basis}[1]{\{\ket{#1}\}}

% Quantum
\newcommand{\nboseeinstein}{n_\text{BE}}
\newcommand{\gammaup}{\Gamma_\uparrow}
\newcommand{\gammadown}{\Gamma_\downarrow}
\newcommand{\gammaupdown}{\Gamma_{\uparrow \downarrow}}
\newcommand{\gammaemission}{\Gamma_\text{loss}}
\newcommand{\qualityfactoremission}{Q_{d,\text{loss}}}

% Qubits
\newcommand{\omegaqubit}{\omega_{10}}

% Circuits
\newcommand{\impedance}{Z_0}
\newcommand{\resistorsource}{R_s}
\newcommand{\vsource}{V_s}
\newcommand{\vsourcerms}{V_{s,\text{rms}}}
\newcommand{\vloadrms}{V_{l,\text{rms}}}

% Signals and noise
\newcommand{\psdsingle}{S_\text{ss}}
\newcommand{\psddouble}{S_\text{ds}}
\newcommand{\noiseavailable}{S_{p,a}^e}
\newcommand{\spectralengineer}{S^e}
\newcommand{\spectralsymmetric}{S^\text{symm}}
\newcommand{\spectralattenuator}{\spectralengineer_{\poweravailable, \text{att.}}}

% Microwaves
\newcommand{\vright}{V_+}
\newcommand{\vleft}{V_-}
\newcommand{\iright}{I_+}
\newcommand{\ileft}{I_-}
\newcommand{\poweravailable}{P_a}

% Figures. Example usage:
% \quickfig{\columnwidth}{my_image}{This is the caption}{fig:my_fig}
\DeclareRobustCommand{\quickfig}[4]{
\begin{figure}
\begin{centering}
\includegraphics[width=#1]{#2}
\par\end{centering}
\caption{#3}
\label{#4}
\end{figure}
}

\DeclareRobustCommand{\quickwidefig}[4]{
\begin{figure*}[h]
\begin{centering}
\includegraphics[width=#1]{#2}
\par\end{centering}
\caption{#3}
\label{#4}
\end{figure*}
}

\DeclareRobustCommand{\quickfigcentered}[4]{
  \begin{figure}
  \makebox[\textwidth][c]{\includegraphics[width=#1]{#2}}
  \caption{#3}
  \label{#4}
  \end{figure}
}

%Packages
\usepackage{amsmath}
\usepackage{amstext}
\usepackage{amssymb}
\usepackage{appendix}
\usepackage{coseoul}
\usepackage{graphicx}
\usepackage{import}
\usepackage{lscape}
\usepackage{modular}

\usepackage[pdfpagemode=UseNone,pdfstartview=FitH,colorlinks=true,linkcolor=blue,citecolor=blue,urlcolor=blue]{hyperref}
\usepackage[all]{hypcap}



\begin{document}

\section*{Chapter 3}


\subsection*{Cooper Pair Wavefunction}


\subsubsection*{Wavefunction in second quantization}

We would like to understand where 3.2 comes from. The Hamiltonian
is\[
H=\sum_{\sigma,k}\epsilon_{k,\sigma}+\sum_{p,q,\xi,\alpha,\beta}V(\xi)a_{p-\xi,\alpha}^{\dagger}a_{q+\xi,\beta}^{\dagger}a_{q,\beta}a_{p,\alpha}\]
We would like to see what 3.1 looks like in $2^{\textrm{nd}}$ quantization.
The wavefunction we want to represent is a filled Fermi sea plus two
particles in a state with 2-body wavefunction given by\[
\Psi(r_{1},r_{2})=\sum_{k}g_{k}\cos\left(k\cdot(x-y)\right)\left(\alpha_{x}\beta_{y}-\beta_{x}\alpha_{y}\right)\]
where here $\alpha$ and $\beta$ are the spin states and $x$ and
$y$ refer to the two particles. This wavefunction was chosen so that
the two particles make up a zero momentum state, and have a symmetric
spatial dependence so that their overlap is maximized. This maximization
of spatial overlap was chosen to anticipate the attractive interaction
between electrons.

\begin{flushleft}
In second quantization we have\[
a_{\phi,\alpha}^{\dagger}a_{\psi,\beta}^{\dagger}|0\rangle\rightarrow\phi(r_{1})\psi(r_{2})\alpha_{1}\beta_{2}-\psi(r_{1})\phi(r_{2})\beta_{1}\alpha_{2}\]
We can use this relation to turn our 2-body wavefunction into its
second quantized form as follows here:\begin{eqnarray*}
\cos\left(k\cdot(x-y)\right)\left(\alpha_{1}\beta_{2}-\beta_{1}\alpha_{2}\right) & = & \frac{1}{2}\left[e^{ik\cdot x}e^{-ik\cdot y}+e^{-ik\cdot x}e^{-k\cdot y}\right]\left(\alpha_{1}\beta_{2}-\beta_{1}\alpha_{2}\right)\end{eqnarray*}
Rename $e^{ikx}=\phi(x)$ and $e^{-ikx}=\psi(x)$. Then we have\begin{eqnarray*}
\cos\left(k\cdot(x-y)\right)\left(\alpha_{1}\beta_{2}-\beta_{1}\alpha_{2}\right) & = & \frac{1}{2}\left[\phi(x)\psi(y)+\psi(x)\phi(y)\right]\left(\alpha_{1}\beta_{2}-\beta_{1}\alpha_{2}\right)\\
 & = & \frac{1}{2}\left[\left(\phi(x)\psi(y)\alpha_{x}\beta_{y}-\psi(x)\phi(y)\beta_{x}\alpha_{y}\right)+\left(\psi(x)\phi(y)\alpha_{x}\beta_{y}-\phi(x)\psi(y)\beta_{x}\alpha_{y}\right)\right]\\
 & \rightarrow & \frac{1}{2}\left[a_{\phi,\alpha}^{\dagger}a_{\psi,\beta}^{\dagger}+a_{\psi,\alpha}^{\dagger}a_{\phi,\beta}^{\dagger}\right]|0\rangle\\
 & = & \frac{1}{2}\left[a_{k,\alpha}^{\dagger}a_{-k,\beta}^{\dagger}+a_{-k,\alpha}^{\dagger}a_{k,\beta}^{\dagger}\right]\end{eqnarray*}
Notice that these terms are equal if the sign of $k$ is reversed.
Now we stuff this formula into the wavefunction,\begin{eqnarray*}
\Psi(x,y) & = & \sum_{k>k_{F}}g_{k}\cos\left(k\cdot(x-y)\right)\left(\alpha_{x}\beta_{y}-\beta_{x}\alpha_{y}\right)\\
 & = & \sum_{k>k_{F}}g_{k}\frac{1}{2}\left[a_{k,\alpha}^{\dagger}a_{-k,\beta}^{\dagger}+a_{-k,\alpha}^{\dagger}a_{k,\beta}^{\dagger}\right]\end{eqnarray*}
Let $\sum_{k_{F}^{+}}$ indicate a sum over half of $k$-space, ie.
the half with $k_{x}>0$. Then\[
\sum_{k>k_{F}}=\sum_{k^{+}>k_{F}}+\sum_{k^{-}>k_{F}}\]
So\begin{eqnarray*}
\sum_{k>k_{F}}g_{k}\frac{1}{2}\left[a_{k,\alpha}^{\dagger}a_{-k,\beta}^{\dagger}+a_{-k,\alpha}^{\dagger}a_{k,\beta}^{\dagger}\right] & = & \sum_{k^{+}>k_{F}}g_{k}\frac{1}{2}\left[a_{k,\alpha}^{\dagger}a_{-k,\beta}^{\dagger}+a_{-k,\alpha}^{\dagger}a_{k,\beta}^{\dagger}\right]+\sum_{k^{-}>k_{F}}g_{k}\frac{1}{2}\left[a_{k,\alpha}^{\dagger}a_{-k,\beta}^{\dagger}+a_{-k,\alpha}^{\dagger}a_{k,\beta}^{\dagger}\right]\\
 & = & \sum_{k^{+}>k_{F}}g_{k}\frac{1}{2}a_{k,\alpha}^{\dagger}a_{-k,\beta}^{\dagger}+\sum_{k^{+}>k_{F}}g_{k}\frac{1}{2}a_{-k,\alpha}^{\dagger}a_{k,\beta}^{\dagger}+\cdots\\
 &  & \qquad\cdots+\sum_{k^{-}>k_{F}}g_{k}\frac{1}{2}a_{k,\alpha}^{\dagger}a_{-k,\beta}^{\dagger}+\sum_{k^{-}>k_{F}}g_{k}\frac{1}{2}a_{-k,\alpha}^{\dagger}a_{k,\beta}^{\dagger}\\
 & = & \sum_{k^{+}>k_{F}}g_{k}\frac{1}{2}a_{k,\alpha}^{\dagger}a_{-k,\beta}^{\dagger}+\sum_{k^{+}>k_{F}}g_{k}\frac{1}{2}a_{-k,\alpha}^{\dagger}a_{k,\beta}^{\dagger}+\cdots\\
 &  & \cdots+\sum_{k^{+}>k_{F}}g_{-k}\frac{1}{2}a_{-k,\alpha}^{\dagger}a_{k,\beta}^{\dagger}+\sum_{k^{+}>k_{F}}g_{-k}\frac{1}{2}a_{k,\alpha}^{\dagger}a_{-k,\beta}^{\dagger}\end{eqnarray*}
Then, using the fact that $g_{k}=g_{-k}$ we have\begin{eqnarray*}
\sum_{k>k_{F}}g_{k}\frac{1}{2}\left[a_{k,\alpha}^{\dagger}a_{-k,\beta}^{\dagger}+a_{-k,\alpha}^{\dagger}a_{k,\beta}^{\dagger}\right] & = & \sum_{k^{+}>k_{F}}g_{k}a_{k,\alpha}^{\dagger}a_{-k,\beta}^{\dagger}+\sum_{k^{+}>k_{F}}g_{k}a_{-k,\alpha}^{\dagger}a_{k,\beta}^{\dagger}\\
 & = & \sum_{k>k_{F}}g_{k}a_{k,\alpha}^{\dagger}a_{-k,\beta}^{\dagger}\end{eqnarray*}
which agrees with 3.11.
\par\end{flushleft}


\subsubsection*{Schrodinger's equation with the cooper pair wavefunction}


\section*{BCS ground state}

The BCS ground state is\begin{eqnarray*}
|\Psi_{G}\rangle & = & \prod_{k}(u_{k}+v_{k}c_{k\uparrow}^{\dagger}c_{-k\downarrow}^{\dagger})|0\rangle\\
\langle\Psi_{G}| & = & \langle0|\prod_{k}(u_{k}^{*}+v_{k}^{*}c_{-k\downarrow}c_{k\uparrow})\end{eqnarray*}
You can see that this thing has terms with anywhere from zero to ??

We would like to calculate $\bar{N}$.\begin{eqnarray*}
\bar{N} & = & \langle\Psi_{G}|\hat{N}|\Psi_{G}\rangle=\sum_{k,\sigma}\langle\Psi_{G}|\hat{n}_{k,\sigma}|\Psi_{G}\rangle\\
 & = & \sum_{k,\sigma}\langle0|\prod_{p}(u_{p}^{*}+v_{p}^{*}c_{-p\downarrow}c_{p\uparrow})|n_{k,\sigma}|\prod_{q}(u_{q}+v_{q}c_{q\uparrow}^{\dagger}c_{-q\downarrow}^{\dagger})|0\rangle\\
 & = & \sum_{k,\sigma}\langle0|(u_{k}^{*}+v_{k}^{*}c_{-k\downarrow}c_{k\uparrow})\prod_{p\neq k}(u_{p}^{*}+v_{p}^{*}c_{-p\downarrow}c_{p\uparrow})|n_{k,\sigma}|(u_{k}+v_{k}c_{k\uparrow}^{\dagger}c_{-k\downarrow}^{\dagger})\prod_{q\neq k}(u_{q}+v_{q}c_{q\uparrow}^{\dagger}c_{-q\downarrow}^{\dagger})|0\rangle\\
 & = & \sum_{k,\sigma}\langle0|(u_{k}^{*}+v_{k}^{*}c_{-k\downarrow}c_{k\uparrow})n_{k,\sigma}(u_{k}+v_{k}c_{k\uparrow}^{\dagger}c_{-k\downarrow}^{\dagger})\times\prod_{{q\neq k \brace p\neq k}}(u_{p}^{*}+v_{p}^{*}c_{-p\downarrow}c_{p\uparrow})(u_{q}+v_{q}c_{q\uparrow}^{\dagger}c_{-q\downarrow}^{\dagger})|0\rangle\end{eqnarray*}
The two terms in parentheses in the product always commute because
they involve even numbers of fermion operators. Therefore, we can
group together terms where $p=q$ to get, \begin{eqnarray*}
\bar{N} & = & \sum_{k,\sigma}\langle0|(u_{k}^{*}+v_{k}^{*}c_{-k\downarrow}c_{k\uparrow})n_{k,\sigma}(u_{k}+v_{k}c_{k\uparrow}^{\dagger}c_{-k\downarrow}^{\dagger})\times\prod_{p\neq k}(u_{p}^{*}+v_{p}^{*}c_{-p\downarrow}c_{p\uparrow})(u_{p}+v_{p}c_{p\uparrow}^{\dagger}c_{-p\downarrow}^{\dagger})|0\rangle\\
 & = & \sum_{k,\sigma}\left[\langle0|(u_{k}^{*}+v_{k}^{*}c_{-k\downarrow}c_{k\uparrow})n_{k,\sigma}(u_{k}+v_{k}c_{k\uparrow}^{\dagger}c_{-k\downarrow}^{\dagger})|0\rangle\langle0|\prod_{p\neq k}(u_{p}^{*}+v_{p}^{*}c_{-p\downarrow}c_{p\uparrow})(u_{p}+v_{p}c_{p\uparrow}^{\dagger}c_{-p\downarrow}^{\dagger})|0\rangle\right]\\
 & = & \sum_{k,\sigma}\left[|v_{k}|^{2}\langle0|c_{-k\downarrow}c_{k\uparrow}n_{k,\sigma}c_{k\uparrow}^{\dagger}c_{-k\downarrow}^{\dagger}|0\rangle\langle0|\prod_{p\neq k}\left(|u_{p}|^{2}+|v_{p}|^{2}c_{-p\downarrow}c_{p\uparrow}c_{p\uparrow}^{\dagger}c_{-p\downarrow}^{\dagger}\right)|0\rangle\right]\\
 & = & \sum_{k}\left[2|v_{k}|^{2}\prod_{p\neq k}1\right]\\
\bar{N} & = & 2\sum_{k}|v_{k}|^{2}\end{eqnarray*}
An entirely similar computation shows that\[
\langle N^{2}\rangle=\]
 Therefore the variance in the particle number is $\sigma_{N}^{2}=\langle N^{2}\rangle-\langle N\rangle^{2}=$
\end{document}

\documentclass{article}

\usepackage{amsmath}
\usepackage{amssymb}
\usepackage{graphicx}

\newcommand{\ket}[1]{\left \lvert #1 \right \rangle}
\newcommand{\SNR}{\text{SNR}}

\DeclareRobustCommand{\quickfig}[4]{
  \begin{figure}
    \begin{centering}
      \includegraphics[width=#1]{#2}
      \par
    \end{centering}
    \caption{#3}
    \label{#4}
  \end{figure}
}


\author{Daniel Sank \\ \small{Google. LLC}}
\title{Can post selection improve supremacy \\ \small{tl,dr: no, it can't}}
\date{20 March 2019}

\begin{document}
\maketitle

\quickfig{0.8\columnwidth}{setup.pdf}{Gaussian readout distributions.}{fig:setup}

When we measure the state of a superconducting transmon qubit, we get a single number $x$ that is drawn from one of two probability distributions.
If the qubit was in $\ket{0}$ then we draw from $P_+(x)$, and if the qubit was in $\ket{1}$ then we draw from $P_-(x)$.
The probability distributions are Gaussian
\begin{equation}
  P_\pm (x) = \frac{1}{\sqrt{2 \pi \sigma^2}} \exp \left( \frac{(x \mp s/2)^2}{2 \sigma^2} \right)
\end{equation}
where $\sigma$ is the width of the Gaussians and $s$ is the separation between their centers.
The error rate for the measurement is simply the fraction of each Gaussian curve on the ``wrong'' side of $x=0$, i.e.
\begin{equation}
  \epsilon_\text{raw} \equiv \int_{-\infty}^0 P_+(x) \, dx
  = \frac{1}{2} \text{erfc} \left(
    \frac{s}{2 \sqrt{2} \sigma}
  \right)
  = \frac{1}{2} \text{erfc} \left(
    \frac{\sqrt{\SNR}}{2}
  \right)
  \, .
\end{equation}
See Figure \ref{fig:setup}.

Now suppose we try to do some post-selection to remove measurement shots where the qubit state is uncertain.
In particular, we throw out any values of $x$ that lie in a segment of length $r \, \sigma$ at the center of the two distributions.
With this area excluded, if the qubit is in $\ket{0}$, then only points in the range $[-\infty, -r \sigma/2]$ are errors.
Therefore the error rate is
\begin{align}
  \epsilon_\text{post-selected}
  \int_{-\infty}^{-r \sigma/2} P_+(x) \, dx
  &= \frac{1}{2} \text{erfc} \left(
    \frac{s}{2 \sqrt{2} \sigma} + \frac{r}{2 \sqrt{2}}
  \right) \\
  &= \frac{1}{2} \text{erfc} \left(
    \frac{\sqrt\SNR}{2} + \frac{r}{2 \sqrt{2}}
  \right)
  \, .
\end{align}
Of course, we are throwing away data.
The fraction of data kept is
\begin{align}
  F
  &= 1 - \int_{-r \sigma / 2}^{r \sigma / 2} P_+(x) \, dx \\
  &= 1 - \frac{1}{2} \left(
    \text{erf} \left(
      \frac{s}{2 \sqrt{2} \sigma} + \frac{r}{2 \sqrt{2}}
    \right)
    + \text{erf} \left(
      \frac{-s}{2 \sqrt{2} \sigma} + \frac{r}{2 \sqrt{2}}
    \right)
  \right) \\
  &= 1 - \frac{1}{2} \left(
    \text{erf} \left(
      \frac{\sqrt{\SNR}}{2} + \frac{r}{2 \sqrt{2}}
    \right)
    + \text{erf} \left(
      \frac{-\sqrt{\SNR}}{2} + \frac{r}{2 \sqrt{2}}
    \right)
  \right)
  \, .
\end{align}

The measure of success for supremacy is the so-called ``supremacy SNR'' defined as
\begin{equation}
  \text{supremacy SNR} \equiv \text{Fidelity} \times \sqrt{N}
\end{equation}
where $N$ is the number of random circuits we run.
The supremacy SNR gain by using post-selection is therefore
\begin{equation}
  G = \frac{1 - \epsilon_\text{post-selected}}{1 - \epsilon_\text{raw}} \sqrt{F} \, .
\end{equation}
Curves of $G$ versus $r$ for a few values of intrinsic SNR are shown in Figure \ref{fig:G}.

\quickfig{0.9\columnwidth}{snr_gain.pdf}{Supremacy SNR gain versus $r$ for a few values of intrinsic SNR}{fig:G}

It's also useful to look at the fidelity gain and fraction of data kept separately.
These are shown in Figure \ref{fig:compare}.

\quickfig{0.8\columnwidth}{compare.pdf}{Fractional fidelity gain (top) and fraction of data kept (bottom).}{fig:compare}

\end{document}

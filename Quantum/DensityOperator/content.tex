\levelstay{Why?}

I want to explicitly show that the density matrix formulation of quantum mechanics says nothing more or less than the state vector formulation. We have learned in school that the density matrix formulation should be used if

\begin{itemize}
\item We only have access to a sub-part of a quantum system
\item We have a quantum state specified with statistical information
\end{itemize}

I show in this note that in these two cases the density matrix and state vector formulations predict identical expectation values. That demonstration is necessary and sufficient to show that the theories are exactly equivalent.

\levelstay{Definition of density matrix}

Given a quantum state $\ket{\Psi}$ the density operator for the state is defined as \begin{equation}
\rho = \ket{\Psi}\bra{\Psi} \end{equation}
Note that this contains exactly the same amount of information as the state vector less the global phase which we know to be meaningless.
Given a probability distribution over pure states $\{\Psi_i\}$ with probabilities $p_i$, the density matrix is
\begin{equation*}
  \rho = \sum_i p_i \ket{\Psi_i}\bra{\Psi_i} \, .
\end{equation*}

\levelstay{Expectation values}

\leveldown{General operators}

In the state vector representation, the expectation value of an operator $A$ is computed as
\begin{equation}
  \avg{A}_\Psi = \bra{\Psi}A\ket{\Psi} \, .
\end{equation}
If we have a probability distribution over $\Psi_i$, then the expectation value is
\begin{equation*}
  \avg{A}_{\{\Psi_i\}} = \sum_i p_i \bbraket{\Psi_i}{A}{\Psi_i} \, .
\end{equation*}
In the density matrix representation, the expectation value of an operator $A$ is computed as
\begin{align*}
  \langle A \rangle_{\rho} &= \Tr \left[ A \rho \right] \\
  &= \sum_n \sum_i p_i \braket{n}{A|\Psi_i}\braket{\Psi_i}{n} \\
  &= \sum_i p_i \bbraket{\Psi_i}{A}{\Psi_i} \\
\end{align*}
which is identical to the state vector expression.
This proves that the density matrix and state vector representations yield the same expectation values for pure states or for probability distributions over pure states.

\levelstay{Subsystem operators}

The density matrix representation is really useful when we want to think about a sub-part of a quantum system.
Denote such a sub-part of the system as $X$ and the remainder of the total system by $Y$.
The state is denoted
\begin{equation}
  \ket{\Psi} = \sum_{\alpha \beta} c_{\alpha \beta} \ket{\alpha}\ket{\beta}
\end{equation}
where $\alpha$ indexes $X$ and $\beta$ indexes $Y$.
We compute the expectation value of the general operator $A$ acting only on $X$:
\begin{align*}
  \langle A \rangle_{\Psi}
  =& \braket{\Psi}{A \otimes 1|\Psi} \nonumber \\
  =& \sum_{\alpha \beta \delta \gamma} c^*_{\delta \gamma}c_{\alpha \beta} \bra{\delta \gamma} A \otimes 1 \ket{\alpha \beta} \\
  =& \sum_{\delta \alpha \beta} c^*_{\delta \beta} c_{\alpha \beta} A_{\delta \alpha} \, .
\end{align*}
Now we compute the same expectation value using the density matrix.
The density matrix is
\begin{equation*}
  \rho = \sum_{\alpha \beta \delta \gamma} c^*_{\delta \gamma}c_{\alpha \beta} \ket{\alpha \beta}\bra{\delta \gamma}
\end{equation*}
so the expectation value is
\begin{align*}
  \avg{A}_\rho
  =& \Tr \left[\rho A \right] \\
  =& \sum_{ij}\sum_{\alpha \beta \delta \gamma} c^*_{\delta \gamma}c_{\alpha \beta} \braket{ij}{\alpha \beta} \bra{\delta \gamma} A\otimes 1 \ket{ij} \\
  =& \sum_{ij}\sum_{\alpha \beta \delta \gamma} c^*_{\delta \gamma}c_{\alpha \beta} \bra{\delta \gamma} A\otimes 1 \ket{ij} \braket{ij}{\alpha \beta} \\
  =&\sum_{\alpha \beta \delta \gamma} c^*_{\delta \gamma}c_{\alpha \beta} \bra{\delta \gamma} A\otimes 1 \ket{\alpha \beta} \\
  =& \sum_{\alpha \delta \beta} c^*_{\delta \beta} c_{\alpha \beta} A_{\delta \alpha} \, .
\end{align*}
This is exactly equal to the expression from the state vector formulation, which is totally unsurprising as we already showed that the two formulations predict equal expectation values in the general case.
However the utility of the density matrix becomes apparent if we slightly rewrite the sums:
\begin{equation*}
  \avg{A}_\rho = \sum_{\alpha \delta} A_{\delta \alpha} \sum_\beta c^*_{\delta \beta} c_{\alpha \beta} \, .
\end{equation*}
The sum over $\beta$ looks something like a trace of a matrix.
In fact it is the trace taken precisely over subspace $Y$.
We now show this explicitly,
\begin{align*}
\rho_X
  =& \Tr_Y \, \rho \\
  =& \sum_n \sum_{\alpha \beta \delta \gamma} c^*_{\delta \gamma}c_{\alpha \beta} \braket{\cdot n}{\alpha \beta}\braket{\delta \gamma}{\cdot n} \\
  =& \sum_n \sum_{\alpha \beta \delta \gamma} c^*_{\delta \gamma}c_{\alpha \beta} \ket{\alpha}\bra{\delta}\delta_{n\beta}\delta{\gamma n} \\
  =& \sum_{\alpha \beta \delta} c^*_{\delta \beta}c_{\alpha \beta} \ket{\alpha}\bra{\delta} \\
  =& \sum_{\alpha \delta} \ket{\alpha}\bra{\delta} \sum_{\beta} c^*_{\delta \beta}c_{\alpha \beta} \\
  \avg{A}_{\rho_X}
  =& \Tr \left[ A \rho_X \right] \\
  =& \sum_n \sum_{\alpha \delta} \bra{n} A \ket{\alpha} \braket{\delta}{n} \sum_{\beta} c^*_{\delta \beta}c_{\alpha \beta} \\
  =& \sum_n \sum_{\alpha \delta} \braket{\delta}{n}\bra{n}A\ket{\alpha} \sum_{\beta} c^*_{\delta \beta}c_{\alpha \beta} \\
  =& \sum_{\alpha \delta} A_{\delta \alpha} \sum_{\beta} c^*_{\delta \beta}c_{\alpha \beta}
\end{align*}
This shows that
\begin{equation}
\avg{A}_\rho = \avg{A}_{\rho_X} = \avg{A}_{\Tr_y \rho} \, .
\end{equation}

We have now shown that by taking a trace over $Y$ we are left a density matrix that yields correct expectation values for any operator on $X$.

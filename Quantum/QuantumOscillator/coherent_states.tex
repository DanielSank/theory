\section{Coherent States}

Suppose we want to find an eigenvector of the lowering operator.
For starters we know that the state $\ket{0}$ is an eigenket of the lowering operator with eigenvalue $0$.
We also know that to translate $a$ by an amount $\phi$ we should apply the operator $\exp[\phi a^{\dagger}]$.
Therefore we guess that the general eigenket of $a$ is
\begin{equation*}
  \ket{\phi}_?
  = e^{\phi a^{\dagger}}\ket{0}
  = \sum_{n=0}^\infty \frac{\phi^n (a^\dagger)^n}{n!} \ket{0}
  \, .
\end{equation*}
With this definition we have
\begin{align*}
  a \ket{\phi}_?
  &= a e^{\phi a^\dagger} \ket{0} \\
  &= \sum_{n=1}^\infty \frac{\phi^{n}}{n!} a (a^{\dagger})^{n} \, \ket{0}
\end{align*}
where the $n=0$ term vanishes because $a\ket{0} = 0$.
As $[a,(a^{\dagger})^{n}] = \partial_{a^\dagger}(a^\dagger)^n = n(a^{\dagger})^{n-1}$ we have
\begin{equation}
  a(a^{\dagger})^{n}=(a^{\dagger})^{n}a+n(a^{\dagger})^{n-1}
\end{equation}
so then
\begin{equation}
  a\ket{\phi}_? = \sum_{n=1}^{\infty}\frac{\phi^{n}}{n!} \left((a^{\dagger})^{n}a+n(a^{\dagger})^{n-1} \right) \, \ket{0}
\end{equation}
and since $a\ket{0}=0$ we can drop the $(a^{\dagger})^{n}a$ term, leaving,
\begin{align*}
  a\ket{\phi}_?
  &= \sum_{n=1}^{\infty}\frac{\phi^{n}}{n!} \left(n(a^{\dagger})^{n-1} \right) \, \ket{0} \\
  &= \phi \, \sum_{n=1}^\infty \frac{\phi^{n-1}}{(n-1)!}(a^{\dagger})^{n-1} \, \ket{0} \\
  &= \phi \, \sum_{n=0}^\infty \frac{\phi^n}{n!}(a^{\dagger})^n \, \ket{0} \\
  &= \phi\ket{\phi}_?
\end{align*}
so the kets $\ket{\phi}_?$ are eigenkets of $a$ as we hoped.
However, they aren't propertly  normalized,
\begin{align*}
  _?\braket{\phi}{\phi}_?
  &= \sum_{m,n=0}^\infty \bra{0} a^m \frac{\phi^{*^m}}{m!} \frac{\phi^n}{n!}(a^\dagger)^n \ket{0} \\
  &= \sum_{n=0}^\infty \frac{\lvert \phi \rvert^{2n}}{n!} \\
  &= e^{\lvert \phi \rvert^2}
\end{align*}
so the properly normalized states called ``coherent states'' are
\begin{equation}
  \boxed{
    \ket{\phi} = e^{-\frac{1}{2} \lvert \phi \vert^2 } e^{\phi a^\dagger} \ket{0}
  }
  \, .
\end{equation}
Another way to write the coherent state is
\begin{equation*}
  \ket{\phi} = e^{-\frac{1}{2} \abs{\phi}^2} e^{\phi a^\dagger} e^{-\phi^* a} \ket{0}
\end{equation*}
which works because $\exp \left( -\phi a \right) \ket{0} = \ket{0}$.
Using the BCH theorem we find
\begin{equation*}
  e^{-\frac{1}{2} \abs{\phi}^2} e^{\phi a^\dagger} e^{-\phi^* a}
  = e^{\phi a^\dagger - \phi^* a}
\end{equation*}
so
\begin{equation}
  \boxed{
    \ket{\phi} = \underbrace{\exp \left( \phi a^\dagger - \phi^* a \right)}_{D(\phi)} \ket{0}
  }
\end{equation}
where $D(\phi)$ is called the ``displacement operator''.
The displacement operator is convenient for mathematical manipulations.
For example, the derivative of a coherent state with respect to its eigenvalue
\begin{align*}
  \partial_{\phi}\ket{\phi}
  &= \frac{\partial}{\partial \phi} \sum_{n=0}^\infty \frac{\phi^n (a^\dagger)^n}{n!} \ket{0} \\
  &= \sum_{n=1}^{\infty}\frac{\phi^{n-1}}{(n-1)!}a^{\dagger}(a^{\dagger})^{n-1} \, \ket{0} \\
  &= a^{\dagger}\ket{\phi}
\end{align*}
can be remembered easily as
\begin{align}
  \partial_\phi \ket{\phi}
  &= \partial_\phi D(\phi) \ket{0} \nonumber \\
  &= \partial_\phi \exp \left(\phi a^\dagger - \phi^* a \right) \nonumber \\
  &= a^\dagger \exp \left(\phi a^\dagger - \phi^* a \right) \nonumber \\
  &= a^\dagger \ket{\phi} \, .
\end{align}


A very important identity that is used in the context of Wigner functions
is the completeness relation
\begin{equation}
  \int \frac{d\Re(\phi)d\Im(\phi)}{\pi} e^{-\phi^{*}\phi}\ket{\phi}\bra{\phi}
  = \mathbb{I}
\end{equation}


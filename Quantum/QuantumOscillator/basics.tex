\section{General Form}

The general form of the Hamiltonian for a harmonic oscillator is \begin{equation}
H = \frac{1}{2} \alpha u^2 + \frac{1}{2} \beta v^2 \qquad [u,v]= i \gamma . \end{equation}
Using dimensionless operators
\begin{align}
X \equiv \frac{1}{\sqrt{2 \gamma}} \left( \frac{\alpha}{\beta}\right)^{1/4}\,u
&\quad \textrm{and} \quad
Y \equiv \frac{1}{\sqrt{2 \gamma}} \left( \frac{\beta}{\alpha} \right)^{1/4}\,v 
\end{align}
we get
\begin{equation}
H = \gamma \sqrt{\alpha \beta} \left[ X^{2} + Y^{2} \right],
\qquad
[X,Y] = i/2 \, .
\end{equation}
We also introduce raising and lowering operators $a$ and $a^{\dagger}$ defined by the following equations \begin{align}
a = X+iY &\qquad a^{\dagger} = X-iY \nonumber \\
X = \frac{1}{2}\left(a+a^{\dagger}\right) &\qquad Y = \frac{-i}{2}\left(a-a^{\dagger}\right) \\
[ a, a^{\dagger} ] &= 1 . \end{align}
Writing down the expression for $a^{\dagger}a$ and expanding it in terms of the $X$ and $Y$ operators, we find \begin{align*}
\gamma \sqrt{\alpha \beta} ( a^{\dagger} a ) & = \gamma\sqrt{\alpha\beta} (X-iY)(X+iY)\\
 & = \gamma\sqrt{\alpha\beta} \left(X^2 + iXY - iYX + Y^2 \right)\\
 & = \gamma\sqrt{\alpha\beta} \left(X^2 + Y^{2} + i \left[ X,Y \right] \right)\\
 & = \gamma\sqrt{\alpha\beta} \left(X^2 + Y^{2} - 1/2 \right)\\
 & = H - \frac{1}{2}\gamma\sqrt{\alpha\beta} \\
\textrm{so}\qquad H & = \gamma\sqrt{\alpha\beta} \left( a^{\dagger}a + \frac{1}{2} \right) \, .
\end{align*}
It will be shown below from Heisenberg's equations of motion that the frequency of oscillation $\omega$ is related to the other constants via
\begin{equation}
\hbar\omega=\gamma\sqrt{\alpha\beta} \, ,
\end{equation}
which means that the Hamiltonian can be written
\begin{equation}
H=\hbar\omega\left(a^{\dagger}a+\frac{1}{2}\right) \, .
\end{equation}

\subsection{Zero point fluctuation}

The zero point fluctuation of $X$ is
\begin{equation}
\bbraket{0}{X^2}{0} = \frac{1}{4}\bbraket{0}{a^2 + aa^{\dagger} + a^{\dagger}a + a^{\dagger^2}}{0} = 1/4
\end{equation}
which we write compactly as
\begin{equation}
\langle X^2 \rangle_0 = \langle Y^2 \rangle_0 = 1/4 \, .
\end{equation}
From this, we compute the zero point fluctuations of $u$ and $v$, \begin{equation}
\langle u^2 \rangle_0 = \frac{1}{2}\gamma \sqrt{\beta / \alpha} \quad \langle v^2 \rangle_0 = \frac{1}{2}\gamma \sqrt{\alpha / \beta} . \end{equation}
Defining $u_{\textrm{zpf}}^2 \equiv \langle u^2 \rangle_0 $ we have \begin{equation}
X = \frac{1}{2}\frac{u}{u_{\textrm{zpf}}} \quad Y = \frac{1}{2}\frac{v}{v_{\textrm{zpf}}} \end{equation}
With these definitions, we can write the lowering operator as
\begin{equation}
a = \frac{u}{2 u_\text{zpf}} + i \frac{v}{2 v_\text{zpf}}
\end{equation}
and
\begin{equation}
  u = u_\text{zpf}(a + a^\dagger) \qquad v = -i v_\text{zpf} (a - a^\dagger) \, .
\end{equation}

An interesting case is the electical LC oscillator with Hamiltonian
\begin{equation*}
  H = \frac{\Phi^2}{2L} + \frac{Q^2}{2C} \qquad [\Phi, Q] = i \hbar \, .
\end{equation*}
In our generalized language, we have $u = \Phi$, $v=Q$, $\alpha=1/L$, $\beta = 1/C$.
Using this and defining $R_K \equiv h/e^2$, this gives zero point fluctuation
\begin{align*}
  \avg{\Phi^2}
  &= \frac{\hbar}{2} \sqrt{\frac{L}{C}} \\
  &= 2 \left( \frac{\Phi_0}{2\pi} \right)^2 \frac{Z}{R_K/2\pi} \, .
\end{align*}

\section{Algebra}

From the commutator $ [a,a^{\dagger}]=1 $ and the conjugate variables formulae in \citeinternaltype \citeinternalref{quantumMechanics} it follows that \begin{equation}
[a,T] = \frac{\partial T}{\partial a^{\dagger}} \qquad [a^{\dagger},T] = -\frac{\partial T}{\partial a}\end{equation}
as long as $T$ is written in normal order form (all $a^{\dagger}$ operators to the left of all $a$ operators).
This is extremely useful when computing dynamics in the Heisenberg or interaction picture, as will be shown in the next section.

\section{Equations of Motion}

The Heisenberg equation of motion for the $a$ operator is \begin{eqnarray*}
i\hbar d_{t}a & = & [a,H] \\
& = & \gamma\sqrt{\alpha\beta}[a,a^{\dagger}a+\frac{1}{2}] \\
& = & \gamma\sqrt{\alpha\beta}\frac{\partial(aa^{\dagger})}{\partial a^{\dagger}} \\
& = & \gamma\sqrt{\alpha\beta}\,a \end{eqnarray*}
giving \begin{equation}
\dot{a} = -i\frac{\gamma\sqrt{\alpha\beta}}{\hbar}a .\end{equation}
Solving this simple differential equation yields
\begin{equation}
a(t) = a(0)\exp\left[-i \omega t \right],
\quad
a^{\dagger}(t) = a^{\dagger}(0)\exp\left[i \omega t \right] \, .
\end{equation}
where $\omega \equiv \gamma\sqrt{\alpha\beta}/\hbar$ as claimed above.
Note that the evolution of $a$ in the phase plane is $\emph{clockwise}$, ie. the phasor convention we inherit from Schrodinger's (Heisenberg's) equation has a $-i$.
This is important interpreting the meaning of positive and negative energy in a quantum calculation.


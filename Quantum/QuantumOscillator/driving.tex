\levelstay{Driven Oscillator}

Coherent states exist in nature because a classical harmonic drive applied to a harmonic oscillator system results in a coherent state.
Consider the energy associated with application of a spatially uniform force field: energy = -force $\times$ distance.
Therefore, the Hamiltonian term associated with this force is
\begin{equation*}
  H_{\textrm{drive}}
  = -x F(t)
  = -(a + a^\dagger) f_x(t)
\end{equation*}
where $f_{x} = x_\text{zpf} F(t)$ has dimensions of energy.
This kind of driving is called a ``linear'' or ``dipole'' drive.
We can, for free, add a term proportional to the conjugate variable:
\begin{align*}
  H_{\textrm{drive}}
  &=  -(a+a^{\dagger})f_{x}(t)-(-i)(a-a^{\dagger})f_{y}(t)\\
  &= a(-f_{x}+if_{y})+a^{\dagger}(-f_{x}-if_{y})
\end{align*}
and define
\begin{equation}
  f(t) = -f_x(t) + i f_y(t)
\end{equation}
so that the drive Hamiltonian is
\begin{equation*}
  H_{\textrm{drive}} = af(t)+a^{\dagger}f(t)^{*} \, .
\end{equation*}
The full Hamiltonian in the Schrodinger picture is
\begin{equation}
  H_S(t)
  = \hbar \omega_0 \left( a^{\dagger} a + \frac{1}{2} \right) + f(t) a(t) + f^*(t) a^\dagger(t)
  \, .
\end{equation}

\leveldown{Solution in the Heisenberg picture}

In this section we solve the driven oscillator problem in the Heisenberg.
We follow the approach in Merzbacher, starting at page 335.
The Hamiltonian in the Heisenberg picture is
\begin{equation}
  H_H(t)
  = \hbar \omega_0 \left( a_H^{\dagger}(t)a_H(t) + \frac{1}{2} \right) + f(t)a_H(t) + f^{*}(t)a_H^{\dagger}(t)
  \, .
\end{equation}
For the rest of this section, we drop the subscript $H$'s and denote $a_H(t) = a(t)$.
Heisenberg's equation of motion for $a(t)$ is
\begin{align*}
  i\hbar\dot{a}(t)
  &= \left[ a(t), H(t) \right] \\
  &= \frac{\partial H(t)}{\partial a^{\dagger}(t)} \\
  &= \hbar\omega_0 a(t) + f^{*}(t) \\
  \textrm{or} \qquad
  \left(d_{t}+i\omega_0\right)a(t) &= -\frac{i}{\hbar}f^{*}(t)
  \, .
\end{align*}
Note that this equation of motion is \emph{exactly} the same as the classical equation of motion for the resonator's action-angle variable (also called ``mode amplitude'' and a variety of other names).
That's one of the points of the Heisenberg picture: we solve for time dependence of the dynamical variables, just as we do in classical mechanics, instead of solving for the time dependence of the mysterios wave function.
The homogeneous solution, i.e. when $f(t) = 0$, is
\begin{equation}
  a_\text{hom.}(t) = a(0) \exp(-i \omega_0 t) \, .
\end{equation}
For the inhomogenous part, we convert to the frequency domain using the Fourier transform convention
\begin{equation}
  f(t) = \int e^{i\omega t} \tilde{f}(t) \frac{d\omega}{2\pi}
  \, .
\end{equation}
The equation for the Green's function can then be rewritten as
\begin{align*}
  \left(d_{t} + i \omega_0 \right) \ket{G_{t_0}}
  =& -\frac{i}{\hbar} \ket{\delta_{t_0}} \\
  \int \left(d_{t} + i \omega_0 \right) \ket{\omega} \braket{\omega}{G_{t_0}} \frac{d\omega}{2\pi}
  =& -\frac{i}{\hbar} \int \ket{\omega}\braket{\omega}{\delta_{t_0}} \frac{d\omega}{2\pi} \\
  \int i\left(\omega + \omega_0 \right) \tilde{G}_{t_0}(\omega) \ket{\omega} \frac{d\omega}{2\pi}
  =& -\frac{i}{\hbar} \int \ket{\omega} e^{-i\omega t_0} \frac{d\omega}{2\pi} \\
  \left(\omega + \omega_0 \right) \tilde{G}_{t_0}(\omega)
  =& -\frac{1}{\hbar} e^{-i\omega t_0} \\
  \tilde{G}_{t_0}(\omega)
  =& -\frac{1}{\hbar \left( \omega + \omega_0 \right)} e^{-i \omega t_0} \\
  G_{t_0}(t)
  =& -\int_{-\infty}^{\infty}\frac{e^{i\omega(t - t_0)}}{\hbar \left(\omega + \omega_0 \right)}\frac{d\omega}{2\pi}
  \, .
\end{align*}
There's a pole on the real line which means we have to impose damping to get a sensible result. We want the $\emph{retarded}$ Green's function to be zero if $t<t_0$ which means we want the poles to exist only when the imaginary part of $\omega$ is in the upper half plane. Therefore we add to the denominator a term $-i\epsilon$,
\begin{align*}
  G_{t_0}^{R}(t)
  =& -\int_{-\infty}^{\infty} \frac{e^{i\omega(t - t_0)}}{\hbar \left(\omega + \omega_0 -i \epsilon\right)} \frac{d\omega}{2\pi} \\
  =& -i \left( 2\pi \right) \frac{1}{\hbar 2\pi} e^{i(-\omega_0 + i \epsilon)(t - t_0)} \Theta(t-t_0) \\
  =& -\frac{i}{\hbar} e^{-i\omega_0(t - t_0)} \Theta(t - t_0)
  \, .
\end{align*}
You can check that the advanced Green's function, which you get by using $+i\epsilon$, is
\begin{equation}
  G_{t_0}^{A}(t) = \frac{i}{\hbar} e^{-i\omega_0 (t - t_0)} \Theta(t_0 - t)
  \, .
\end{equation}
Using the retarded Green's function, we can write the inhomogenious part of the solution as
\begin{align*}
  a_\text{inhom.}(t)
  &= \int_{-\infty}^{\infty}G_{t'}^{R}(t)f^{*}(t')dt' \\
  &= -\frac{i}{\hbar}\int_{-\infty}^{t}e^{-i\omega_0(t-t')}f^{*}(t')dt'
\end{align*}
Note that the retarded Green's function has the effect of including influence from the driving force at times $t'$ only earlier than the evaluation time $t$.
In words: the system is affected only by influences from the past.

Consider the case that the driving force turns on at $t_1$.
We can then write the full solution (homogeneous plus inhomogeneous) for $t > t_1$ as
\begin{equation}
  a(t)
  = a(t_1) e^{-i\omega_0(t - t_1)} \underbrace{- \frac{i}{\hbar} \int_{t_1}^t e^{-i \omega_0(t - t')} f^{*}(t') \, dt'}_{\zeta(t)} \, .
\end{equation}
This relation shows explicitly that during the driving period $a$ acquires the usual $e^{-i\omega_0(t - t_1)}$ dynamical phase from free oscillation, plus an added effect, namely the Fourier transform, of the driving.

\leveldown{Evolution of a coherent state}

We can now show that a coherent state, subject to driving, remains a coherent state.
If the system is in coherent state $\ket{\phi}$ at time $t=0$, then $a\ket{\Psi(0)} = \phi \ket{\Psi(0)}$, and we compute
\begin{align*}
  a \ket{\Psi(t)}
  &= a T_S(t) \ket{\Psi(0)} \\
  &= T_S(t) \underbrace{\left( T_S^\dagger(t) a T_S(t) \right)}_{a(t)} \ket{\Psi(0)} \\
  &= T_S(t) \left( a e^{-i \omega_0 t} + \zeta(t) \right) \ket{\Psi(0)} \\
  &= \left( \phi e^{-i \omega_0 t} + \zeta(t) \right) T_S(t) \ket{\Psi(0)} \\
  &= \left( \phi e^{-i \omega_0 t} + \zeta(t) \right) \ket{\Psi(t)} \\
\end{align*}
which shows that under the effect of driving, a coherent state evolves as
\begin{equation}
  \ket{\phi} \Rightarrow \ket{\phi e^{-i \omega_0 t} + \zeta(t)}
  \, .
\end{equation}

\levelup{Solution in a rotating frame}

In the Heisenberg picture, the equation of motion for the lowering operator was
\begin{equation*}
  \left( d_t + i \omega_0 \right) a(t) = -\frac{i}{\hbar} f^*(t)
  \, .
\end{equation*}
This equation involves two potentially large frequencies: the resonance frequency $\omega_0$ of the oscillator, and the time dependence $f^*(t)$.
In the typical case where the spectral content of $f^*$ is centered near $\omega_0$, we can use the rotating frame to remove the fast time dependence.
Consider the rotating frame defined by
\begin{equation*}
  R(t) = \exp(i \omega_R \, t \, n)
\end{equation*}
which implies $\tilde{H}_S = \hbar \omega_R n$.
The equation of motion for $a_R(t)$ is
\begin{align*}
  i \hbar \dot{a}_R(t)
  &= \left[ a, \tilde{H}_S \right]_R \\
  &= \left(
    \frac{\partial}{\partial a^\dagger} \left( \hbar \omega_R a^\dagger a \right)
    \right)_R \\
  &= \hbar \omega_R a_R(t) \\
  \dot{a}_R(t)
  &= -i \omega_R a_R(t)
\end{align*}
with solution $a_R(t) = \exp(-i \omega_R t) a$.
Meanwhile, the kets evolve in time according to the rotating frame Hamiltonian
\begin{align}
  H_R(t)
  &= i \hbar \dot{R}(t) R^\dagger(t) + R(t) H_S(t) R^\dagger(t) \nonumber \\
  &= \hbar \Delta \, n + \frac{1}{2} \hbar \omega_0 + f(t) a_R(t) + f^*(t) a_R^\dagger(t)
  \label{eq:rotating_frame_hamiltonian}
\end{align}
where $\Delta \equiv \omega_0 - \omega_R$ and we dropped the subscript on $n$ because $n_R(t) = n$.
Since we already solved for $a_R(t)$, it remains only to solve the Schrodinger equation for $H_R(t)$, and then we would be able to find all matrix elements of interest.
However, it's more convenient to take the point of view that Eq.~(\ref{eq:rotating_frame_hamiltonian}) itself is a Schrodinger equation whos matrix elements can be found in the Heisenberg picture.
We can do this by lumping the time dependence of $a_R(t)$ into the drive terms, re-expressing the Hamiltonian as
\begin{equation}
  H_R(t) = \hbar \Delta n + \frac{1}{2}\hbar \omega_0 + f(t)e^{-i \omega_R t} a + f^*(t) e^{i \omega_r t} a^\dagger
\end{equation}
which looks just like the Hamiltonian in the Schrodinger picture (here we write $a$ instead of $a_S$), but with modified time dependence in the drive.
Formally, this Hamiltonian has a propagator $T_R$, and to solve for the time dependence of the kets we'd have to compute $T_R$.
But of course, we can apply the Heisenberg picture to \emph{this} Hamiltonian instead.
Formally, and matrix element can be computed as
\begin{align*}
  \bbraket{\Psi_S(t)}{\mathcal{O}_S}{\Phi_S(t)}
  &= \bbraket{\Psi_R(t)}{R(t) \mathcal{O}_S R^\dagger(t)}{\Phi_R(t)} \\
  &= \bbraket{\Psi_R(0)}{T_R^\dagger(t) R(t) \mathcal{O}_S R^\dagger(t) T_R(t)}{\Phi_R(0)}
  \, .
\end{align*}
In the present case, we have $R(t) a R^\dagger(t) = \exp(-i \omega_R t) a$, and of course $\ket{\Psi_R(0} = \ket{\Psi_S(0)}$, so
\begin{align*}
  \bbraket{\Psi_S(t)}{\mathcal{O}_S}{\Phi_S(t)}
  &= \exp(-i \omega_R t) \bbraket{\Psi_S(0)}{T_R^\dagger(t) a T_R(t)}{\Phi_S(0)}
  \, .
\end{align*}
Therefore, to find matrix elements, we can solve the Heisenberg equation for $a(t)$ with respect to $H_R$, and add the time dependent phase $\exp(-i \omega_R t)$ arising from the rotating frame to that result.
The form of $H_R(t)$ is exactly the same as $H_S(t)$ with the modifications $\omega_0 \rightarrow \Delta$ and $f(t) \rightarrow f(t) e^{-i \omega_R t}$, so the equation of motion is
\begin{equation}
  (d_t + i \Delta) a(t) = - \frac{i}{\hbar} f^*(t) e^{i \omega_R t}
\end{equation}
and the solution for $a(t)$, in the case that the drive turns on at time $t_1$ is
\begin{equation}
  a(t) = a(t_1) e^{-i \Delta (t - t_1)}
  \underbrace{- \frac{i}{\hbar} \int_{t_1}^t e^{-i \Delta (t - t')} f^*(t') e^{i \omega_R t'} \, dt'}_{\zeta_R(t)}
  \, .
\end{equation}
Now we come to the main advantage of the rotating frame.
If the complex function $f(t)$ has spectral weight over a limited band centered at frequency $\omega_d$, then it can be written as $f(t) = \varepsilon(t) \exp(i \omega_t t)$ where $\varepsilon(t)$ has a limited bandwidth.
If we take our rotating frame frequency $\omega_R = \omega_d$, then our equation of motion is
\begin{equation}
  \left( d_t + i \Delta \right) a(t) = - \frac{i}{\hbar} \varepsilon^*(t)
  \, .
\end{equation}
The point of this is that $\Delta = \omega_0 - \omega_d$ and $\varepsilon(t)$ are both slow, so numerics, plots, etc. are easier to understand.

\leveldown{Evolution of a coherent state}

Suppose the state at $t=0$ is a coherent state $\ket{\Psi_S(0)} = \ket{\phi}$.
Then
\begin{align*}
  a \ket{\Psi_S(t)}
  &= a R^\dagger(t) \ket{\Psi_R(t)} \\
  &= a R^\dagger(t) T_R(t) \ket{\Psi_R(0)} \\
  &= R^\dagger(t) \underbrace{R(t) a R^\dagger(t)}_{\exp(-i \omega_R t) a} T_R(t) \ket{\Psi_R(0)} \\
  &= e^{-i \omega_R t} R^\dagger(t) T_R(t) \left(T^\dagger(t) a T_R(t) \right) \ket{\Psi_R(0)} \\
  &= e^{-i \omega_R t} R^\dagger(t) T_R(t) \left(a e^{-i \Delta t} + \zeta_R(t) \right) \ket{\Psi_R(0)} \\
  &= e^{-i \omega_R t} R^\dagger(t) T_R(t) \left(\phi e^{-i \Delta t} + \zeta_R(t) \right) \ket{\Psi_R(0)} \\
  &= \left(\phi e^{-i \omega_0 t} + e^{-i \omega_R t} \zeta_R(t) \right) \ket{\Psi_S(t)} \\
  &= \left(\phi e^{-i \omega_0 t} + \zeta(t) \right) \ket{\Psi_S(t)}
\end{align*}
which is identical to our result in the Heisenberg picture.
So once again we have shown that under the effect of driving, a coherent state remains a coherent state, and the eigenvalue of that coherent state evolves according to the classical equation of motion of the action-angle variable.

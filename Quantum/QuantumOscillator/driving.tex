\levelstay{Driven Oscillator}

Coherent states exist in nature because a classical harmonic drive applied to a harmonic oscillator system results in a coherent state.
Consider the energy associated with application of a spatially uniform force field: energy = -force $\times$ distance.
Therefore, the Hamiltonian term associated with this force is
\begin{equation}
  H_{\textrm{drive}}
  = -x F(t)
  = -(a + a^\dagger) f_x(t)
\end{equation}
where $f_{x} = x_\text{zpf} F(t)$ has dimensions of energy.
This kind of driving is called a ``linear'' or ``dipole'' drive.
We can, for free, add a term proportional to the conjugate variable:
\begin{align*}
  H_{\textrm{drive}}
  &=  -(a+a^{\dagger})f_{x}(t)-(-i)(a-a^{\dagger})f_{y}(t)\\
  &= a(-f_{x}+if_{y})+a^{\dagger}(-f_{x}-if_{y})
\end{align*}
so that defining
\begin{equation}
  f(t) = -f_x(t) + i f_y(t)
\end{equation}
the drive Hamiltonian is
\begin{equation*}
  H_{\textrm{drive}} = af(t)+a^{\dagger}f(t)^{*} = af(t)+\textrm{H.C.}
\end{equation*}
The full Hamiltonian in the Schrodinger picture is
\begin{equation}
  H_S(t)
  = \hbar\Omega\left[a^{\dagger} a + \frac{1}{2} \right] + f(t) a(t) + f^*(t) a^\dagger(t)
  \, .
\end{equation}
and in the Heisenberg picture it is
\begin{equation}
  H_H(t)
  = \hbar\Omega\left[a_H^{\dagger}(t)a_H(t) + \frac{1}{2}\right] + f(t)a_H(t) + f^{*}(t)a_H^{\dagger}(t)
  \, .
\end{equation}
From now on, we drop the subscript $H$'s and denote $a_H(t) = a(t)$.
Heisenberg's equation of motion for $a(t)$ is
\begin{align*}
  i\hbar\dot{a}(t)
  &= \left[ a(t), H(t) \right] \\
  &= \frac{\partial H(t)}{\partial a^{\dagger}(t)} \\
  &= \hbar\Omega a(t) + f^{*}(t) \\
  \textrm{or} \qquad
  \left(d_{t}+i\Omega\right)a(t) &= -\frac{i}{\hbar}f^{*}(t)
\end{align*}

\leveldown{Solution by Green's function: Heisenberg Picutre (Merzbacher 335)}

Fourier convention: \begin{equation}
f(t)=\int e^{i\omega t}\tilde{f}(t)d\omega \end{equation}
The equation for the Green's function is \begin{eqnarray*}
\left(d_{t}+i\Omega\right)|G_{t_{0}}\rangle & = & -\frac{i}{\hbar}|\delta_{t_{0}}\rangle\\
\int\left(d_{t}+i\Omega\right)|\omega\rangle\langle\omega|G_{t_{0}}\rangle\frac{d\omega}{2\pi} & = & -\frac{i}{\hbar}\int|\omega\rangle\langle\omega|\delta_{t_{0}}\rangle\frac{d\omega}{2\pi}\\
\int i\left(\omega+\Omega\right)\tilde{G}_{t_{0}}(\omega)|\omega\rangle\frac{d\omega}{2\pi} & = & -\frac{i}{\hbar}\int|\omega\rangle e^{-i\omega t_{0}}\frac{d\omega}{2\pi}\\
\left(\omega+\Omega\right)\tilde{G}_{t_{0}}(\omega) & = & -\frac{1}{\hbar}e^{-i\omega t_{0}}\\
\tilde{G}_{t_{0}}(\omega) & = & -\frac{1}{\hbar\left(\omega+\Omega\right)}e^{-i\omega t_{0}}\\
G_{t_{0}}(t) & = & -\int_{-\infty}^{\infty}\frac{e^{i\omega(t-t_{0})}}{\hbar\left(\omega+\Omega\right)}\frac{d\omega}{2\pi}\end{eqnarray*}
There's a pole on the real line which means we have to impose damping to get a sensible result. We want the $\emph{retarded}$ Green's function to be zero if $t<t_{0}$ which means we want the poles to exist only when the imaginary part of $\omega$ is in the upper half plane. Therefore we add to the denominator a term $-i\epsilon$, \begin{eqnarray*}
G_{t_{0}}^{R}(t) & = & -\int_{-\infty}^{\infty}\frac{e^{i\omega(t-t_{0})}}{\hbar\left(\omega+\Omega-i\epsilon\right)}\frac{d\omega}{2\pi}\\
G_{\textrm{t}_{0}}^{R}(t) & = & -i2\pi\frac{1}{\hbar2\pi}e^{i(-\Omega+i\epsilon)(t-t_{0})}\Theta(t-t_{0})\\
G_{\textrm{t}_{0}}^{R}(t) & = & -\frac{i}{\hbar}e^{-i\Omega(t-t_{0})}\Theta(t-t_{0}) \end{eqnarray*}
You can check that the advanced Green's function, which you get by using $+i\epsilon$, is \begin{equation}
G_{\textrm{t}_{0}}^{A}(t)=\frac{i}{\hbar}e^{-i\Omega(t-t_{0})}\Theta(t_{0}-t) \end{equation}
The inhomogenious part of the solution is \begin{equation}
\int_{-\infty}^{\infty}G_{t'}^{R}(t)f^{*}(t')dt' \end{equation}
\begin{equation}
a(t)=-\frac{i}{\hbar}\int_{-\infty}^{t}e^{-i\Omega(t-t')}f^{*}(t')dt' \end{equation}
Note that the retarded Green's function has the effect of including influence from the driving force at intermittent times $t'$ only less than the evaluation time $t$. In words this means that the system cannot be affected by influences from the future.

Consider the case that the driving force turns on at $T_{1}$ and turns off again at $T_{2}$. We can then write the full solution (homogeneous plus inhomogeneous) during the driving period as \begin{equation}
a(t)=a(T_{1})e^{-i\Omega(t-T_{1})}-\frac{i}{\hbar}\int_{-\infty}^{t}e^{-i\Omega(t-t')}f^{*}(t')dt' \end{equation}
This can be understood in a nice way in the context of coherent states. Recall that a coherent state has the property $a|\phi\rangle=\phi|\phi\rangle$. In free evolution $a(t)=a(0)e^{-i\Omega t}$ so $a(t)|\phi\rangle=\phi e^{-i\Omega t}|\phi\rangle$, which means that the eigenvalue associated to the coherent state is travelling in circles in the phase plane. If we drive the system then $a(t)$ gets the usual free evolution part, but also picks up the additional integral term which is just a complex number. Therefore the influence of driving is to translate the coherent state by an amount equal to the integral term we just derived.

Another way to phrase this is found through the use of the advanced Green's function. Using it we can write a new expression for $a(t)$,\begin{eqnarray*}
a(t) & = & a(T_{2})e^{-i\Omega(t-T_{2})}+\int_{-\infty}^{\infty}G_{t'}^{A}(t)f^{*}(t')dt'\\
a(t) & = & a(T_{2})e^{-i\Omega(t-T_{2})}+\frac{i}{\hbar}\int_{t}^{\infty}e^{-i\Omega(t-t')}f^{*}(t')dt'\end{eqnarray*}
Equating this with our previous expression for $a(t)$ yields\begin{eqnarray*}
a(T_{2})e^{-i\Omega(t-T_{2})}+\frac{i}{\hbar}\int_{t}^{\infty}e^{-i\Omega(t-t')}f^{*}(t')dt' & = & a(T_{1})e^{-i\Omega(t-T_{1})}-\frac{i}{\hbar}\int_{-\infty}^{t}e^{-i\Omega(t-t')}f^{*}(t')dt'\\
a(T_{2})e^{i\Omega T_{2}} & = & a(T_{1})e^{i\Omega T_{1}}-\frac{i}{\hbar}\int_{-\infty}^{\infty}e^{i\Omega t'}f^{*}(t')dt'\\
a(T_{2}) & = & a(T_{1})e^{-i\Omega(T_{2}-T_{1})}-\frac{i}{\hbar}\int_{-\infty}^{\infty}e^{i\Omega t'}f^{*}(t')dt'\end{eqnarray*}
This relation shows explicitly that during the driving period $a$ acquires the usual $e^{-i\Omega(T_{2}-T_{1})}$ dynamical phase we expect from free oscillation, plus the added effect of the driving.

As a useful special case let's look at what happens if we drive only the $X$ quadrature with a sinusoidal drive, \begin{equation}
f(t)=A\sin\left(\Omega t+\theta\right) \end{equation}
from time $0$ to time $T$. The translation of the coherent state is then\begin{eqnarray*}
\zeta & = & -\frac{i}{\hbar}\frac{A}{2i}\int_{0}^{T}e^{i\Omega t'}\left[e^{i(\Omega t'+\theta)}-e^{-i(\Omega t'+\theta)}\right]dt'\\
\zeta & = & -\frac{A}{2\hbar}\int_{0}^{T}\left[e^{i(2\Omega t'+\theta)}-e^{-i\theta}\right]dt'\\
\zeta & = & -\frac{A}{2\hbar}T\left[\frac{1}{\Omega T}\left(e^{i(\Omega T+\theta)}\sin\left(\Omega T\right)\right)-e^{-i\theta}\right]\end{eqnarray*}
In the case $\Omega T\gg1$ this becomes \begin{equation}
\zeta=\frac{1}{2}\frac{AT}{\hbar}e^{-i\theta} \end{equation}
The phase of the drive determines the direction of the displacement, and the amplitude and time of the drive determine its magnitude.


\levelstay{Solution by propogator: Interaction Picture (Mezbacher 338)}

Coming soon.

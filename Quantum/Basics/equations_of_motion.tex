\levelstay{Equations of Motion}

\leveldown{Schrodinger Picture}

The time evolution of the quantum state $\ket{\Psi(t)}$ is determined by
\begin{equation}
  i \hbar \partial_t \ket{\Psi_S(t)} = H_S(t) \ket{\Psi_S(t)}
  \, .
\end{equation}
called the ``Schrodinger equation''.
$H_S(t)$ is the Hamiltonian, here written with a subscript $S$ to mean ``Schrodinger'', which will distinguish it from other versions of the Haniltonian to be described below.
If $H_S(t)$ is Hermitian, then there is a formal solution to the Schrodinger equation
\begin{equation}
  \ket{\Psi_S(t)} = T_S(t,t_0) \ket{\Psi_S(t_0)}
\end{equation}
where $T_S(t, t_0)$, called the ``propogator'', is unitary, meaning that $T_S(t, t_0)^{-1} = T_S(t, t_0)^\dagger$.
In the special case that $H_S(t)$ commutes with itself at different times, it is easily verified that
\begin{equation}
  \ket{\Psi_S(t)}
  = \exp \left[ \frac{-i}{\hbar}\int_{t_0}^{t}H_S(t') \, dt' \right] \ket{\Psi_S(t_0)}
\end{equation}
meaning that
\begin{equation}
  T(t,t_0) = \exp \left[\frac{-i}{\hbar}\int_{t_0}^t H_{S}(t') \, dt' \right]
  \, .
\end{equation}
In any case, the Schrodinger equation has the same form as our generic differential equation treated earlier, so we know that
\begin{equation}
  i \hbar \partial_t T_S(t, t_0) = H_S(t) T_S(t, t_0) \, .
\end{equation}

\levelstay{Rotating frames}

Consider a spin with a magnetic field oriented along the z axis.
The Hamiltonian is
\begin{equation}
  H_{0, S}/\hbar = -\frac{\omega_0}{2}\sigma_z
\end{equation}
where we've added a subscript $0$ to denote that this is the Hamiltonian with just the z-axis field and no extra x or y fields.
In particular, there is no time dependent drive.
Under $H_{0, S}$ the problem is simple and we can find that the spin precesses around the z axis.
However, suppose we add a driving term so that the Hamiltonian is
\begin{equation}
  H_S(t) = H_{0, S} + V \cos(\omega t) \sigma_x \, .
\end{equation}
How do we analyze this case?
Intuitively, the motion of the spin may be simpler if we go into a frame of reference that is rotating along with the precession induced by $H_{0, S}$.
Indeed, this idea of a ``rotating frame'' is pervasive.
The basic idea is to ``rotate'' the state vector backwards in time at the cost of rotating the operators forwards in time.
Some problems become much easier to solve, and the associated equations of motion look a lot like classical equaitons of motion, in the rotating frame.

\leveldown{Heisenberg Picture}

Recall the Schrodinger equation
\begin{equation}
  i \hbar \partial_t \ket{\Psi_S(t)} = H_S(t) \ket{\Psi_S(t)}
\end{equation}
with formal solution
\begin{equation}
  \ket{\Psi_S(t)} = T_S(t, t_0) \ket{\Psi_S(t_0)} \, .
\end{equation}
Intuitively, if we want to undo the ``rotating'' motion of $\ket{\Psi_S}$ we should just apply the inverse of the evolution, i.e. we define a new vector
\begin{equation}
  \ket{\Psi_H} = T_S(t, t_0)^{-1} \ket{\Psi_S(t)} = \ket{\Psi_S(t_0)}
\end{equation}
which is constant.
Now consider a matrix element:
\begin{align}
  \mathcal{O}_{S,\Psi\Phi} =
  \bbraket{\Psi_{S}(t)}{\mathcal{O}_S}{\Phi_{S}(t)}
  =& \bbraket{T_S(t, t_0) \Psi_S(t_0)}{\mathcal{O}_S}{T_S(t, t_0) \Phi_S(t_0)} \nonumber \\
  =& \bbraket{\Psi_H}{\underbrace{T_S(t, t_0)^\dagger \mathcal{O}_S T_S(t, t_0)}_{\mathcal{O}_H(t)}}{\Phi_H}
\end{align}
As we can see, time evolved expectation values can be computed using the constant vector $\ket{\Psi_H} = \ket{\Psi_S(t_0)}$ and the time evelved operator $\mathcal{O}_H(t)$ defined as
\begin{equation}
  \mathcal{O}_H(t) = T_S(t, t_0)^\dagger \mathcal{O}_S T_S(t, t_0) \, .
\end{equation}
Just as we did in the for the general differential equation treated earlier, we can differentiate this definition of $\mathcal{O}_H$ with respect to time and using the fact that $T(t, t_0)$ obeys the Schrodinger equation to reveal \emph{Heisenberg's equation}
\begin{equation}
  i \hbar \partial_t \mathcal{O}_H(t)
  = \left[ \mathcal{O}_H(t), H_H(t) \right]
  = \left[ \mathcal{O}_S, H_S(t) \right]_H
  \, .
\end{equation}
This equation defines one of the several ``pictures'' of quantum mechanics.
In the Heisenberg picture, the quantum states are static but the operators evolve in time.
This picture is much more similar to classical physics where dynamical variables such as $x$ and $p$ evolve in time, and there is no notion of a quantum state.

\levelstay{The general rotating frame}

In the Heisenberg picture, we de-rotate our state vectors completely so that they are constant in time and all the time dependence is in the operators.
However, more generally we can de-rotate only part of the state vector evolution.
Define a rotating frame vector
\begin{equation}
  \ket{\Psi_R(t)} = R(t) \ket{\Psi_S(t)}
\end{equation}
where $R(t)$ is an arbitrary unitary operator with $R(0) = \mathbb{I}$.\footnote{$R(t) = T_S(t)^{-1}$ in the Heisenberg picture.}
Computing the time evolution of this new state we get
\begin{align}
  i\hbar\partial_t \ket{\Psi_R(t)} &= i\hbar \dot{R}(t) \ket{\Psi_S(t)} + R(t) i \hbar\partial_t\ket{\Psi_S(t)} \nonumber \\
  &= i\hbar \dot{R}(t)R^{\dagger}(t) \ket{\Psi_R(t)} + R(t) H_S(t) R^{\dagger}(t) \ket{\Psi_R(t)} \nonumber \\
  &= \left( i\hbar \dot{R}(t)R^\dagger(t) + R(t) H_S(t) R^{\dagger}(t) \right) \ket{\Psi_R(t)}.
\end{align}
This can be interpreted as a Schrodinger equation for a system with Hamiltonian
\begin{equation}
  H_R(t) = i\hbar \dot{R}(t)R^{\dagger}(t) + R(t) H_S(t) R^{\dagger}(t) \, .
\end{equation}
Note that this result is correct for any unitary $R$.
Note also that we have made no assumptions at all about $H_S(t)$; in particular we haven't assumed that it commutes with itself at different times.
Matrix elements are computed as
\begin{align}
  \mathcal{O}_{S,\Psi\Phi}(t) = \bbraket{\Psi_S(t)}{\mathcal{O}_S}{\Phi_S(t)}
  &= \bbraket{\Psi_R(t)}{R(t) \mathcal{O}_S R^\dagger(t)}{\Phi_R(t)} \nonumber \\
  &= \bbraket{\Psi_R(t)}{\mathcal{O}_R}{\Phi_R(t)}
\end{align}
where we defined $\mathcal{O}_R(t) = R(t) \mathcal{O}_S R^\dagger(t)$.
Of course, $\mathcal{O}_R(t)$ obeys a Heisenberg type equation of motion, but with what Hamiltonian?
Differentiating the definition of $\mathcal{O}_R(t)$ gives us an equation involving $\dot{R}$ and $\dot{R}^\dagger$, which can be solved in principle but doesn't reveal useful structure.
In most practical applications, $R$ is chosen to unwind some part of the original Hamiltonian $H_S$.
Various common choices of which part to unwind are called ``pictures'' of quantum mechanics.
As we already noted, unwinding all of $H_S$ is called the Heisenberg picture.
More generally, if $R$ is chosen to unwind the evolution from a Hamiltonian $\tilde{H}_S(t)$,
\begin{equation}
  R(t) = \tilde{T}(t)^{-1}
  \quad \text{where} \quad
  i \hbar \partial_t \tilde{T}(t) = \tilde{H}_S(t) \tilde{T}(t)
  \, ,
\end{equation}
then the equation of motion for $R$ is
\begin{equation}
  i \hbar \partial_t R(t) = - R(t) \tilde{H}_S(t)
  \, .
\end{equation}
It follows that the equation of motion for $\mathcal{O}_R(t)$ is
\begin{equation}
  i \hbar \partial_t \mathcal{O}_R(t)
  = \left[ \mathcal{O}_R(t), \tilde{H}_R(t) \right]
  = \left[ \mathcal{O}_S, \tilde{H}_S(t) \right]_R
  \, .
\end{equation}
Note that the equation of motion for $\mathcal{O}_R$ is formally identical to the equation of motion for $\mathcal{O}_H$.
In this rotated frame, matrix element computation takes place in two steps: 1) We solve the Schrodinger equation with Hamiltonian $H_R(t)$ to find the states $\ket{\Psi_R(t)}$, and then 2) We solve the Heisenberg equation for $\mathcal{O}_R(t)$.

\levelstay{Interaction picture}

The ``interaction picture'' typically describes a situation where the Schrodinger frame Hamiltonian is
\begin{equation}
  H_S(t) = H_0 + V_S(t)
\end{equation}
and we take
\begin{equation}
  R(t) = \exp(i H_0 t / \hbar)
\end{equation}
i.e. $R(t)$ unwinds the evolution generated by $H_0$.
The rotating frame Hamiltonian, here denoted $H_I$, is then
\begin{align*}
  H_I(t)
  &= i \hbar \dot{R}(t) R^\dagger(t) + R(t) H_S(t) R^\dagger(t) \\
  &= i \hbar (i H_0 / \hbar) e^{i H_0 t / \hbar} e^{-i H_0 t / \hbar}
    + e^{i H_0 t / \hbar} \left( H_0 + V(t) \right) e^{-i H_0 t/ \hbar} \\
  &= - H_0 + H_0 +
    \underbrace{\exp \left(i H_0 t / \hbar \right) V_S(t) \exp \left(-i H_0 t / \hbar \right)}_{V_I(t)}
  \, .
\end{align*}
\begin{equation}
  H_I(t) = V_I(t) \, .
\end{equation}
Therefore, in the interaction picture, the two equations of motion are
\begin{equation}
  i \hbar \partial_t \ket{\Psi_I(t)} = V_I(t)
  \qquad
  i \hbar \partial_t \mathcal{O}_I(t) = \left[ \mathcal{O}_I(t), H_0 \right]
  \, .
\end{equation}


\levelup{Invariance in the various pictures}

As illustrated in the preceding text, the various pictures of quantum mechanics are each defined by a time dependent unitary transformation $R(t)$.
In the Heisenberg picture $R(t) = T_S^{-1}(t)$ and in the interaction picture $R(t) = \exp(i H_0 t/ \hbar)$, and in all cases we have time dependent operators defined by
\begin{equation*}
  \mathcal{O}_R(t) = R(t) \mathcal{O}_S R^\dagger(t)
  \, .
\end{equation*}
Consider three operators $A_S$, $B_S$ and $C_S$ with $[A_S, B_S] = C_S$.
The form of this commutation relation is invariant across all pictures:
\begin{align*}
  \left[ A_R(t), B_R(t) \right]
  &= A_R(t) B_R(t) - B_R(t) A_R(t) \\
  &= R(t) A_S R^\dagger(t) R(t) B_S R^\dagger(t) - R(t) B_S R^\dagger(t) R(t) A_s R^\dagger(t) \\
  &= R(t) \left( A_S B_S - B_S A_S \right) R^\dagger(t) \\
\end{align*}
\begin{equation}
  \boxed{
  \left[ A_R(t), B_R(t) \right] = \left[ A_S, B_S \right]_R = C_R(t)
  }
  \, .
\end{equation}
The invariance of commutators is critical for actually using these pictures, as it allows us to write down the right hand side of the operator equations of motion.
For example, for a free particle with
\begin{equation*}
  H_S = \frac{p^2}{2m}
\end{equation*}
the Heisenberg equation of motion for $x$ is
\begin{align*}
  i \hbar \dot{x}_H(t)
  &= \left[ x_H(t), H_H(t) \right] \\
  &= \left[ x, H_S \right]_H \\
  &= \left( i \hbar \left(\frac{\partial}{\partial p_S} \right) H_S \right)_H \\
  &= \left( i \hbar \frac{p_S}{m} \right)_H \\
  \dot{x}_H(t) &= \frac{p_H(t)}{m}
  \, .
\end{align*}
In other words, you can work out the commutators you need in the Schrodinger picture, and slap the subscripts for whatever picture you want onto the result.

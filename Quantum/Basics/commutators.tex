\levelstay{Commutators and equations of motion}

\leveldown{Products}

\begin{align}
  \left[A,BC\right] &= ABC - BCA \nonumber \\
  & = ABC - BAC + BAC - BCA \nonumber \\
  & = [A,B]C + B[A,C].
\end{align}
This can be remembered by noting that $[A,\cdot]$ is like a derivative with respect to $A$.

\levelstay{Conjugation by an operator}

A very common expression involves the translation of an operator $A$ by another operator $B$.
The translation can be expressed as a sum
\begin{equation}
  e^{A}Be^{-A}=\sum_{n=0}^{\infty}\frac{1}{n!}\underbrace{[A,[A,[A,\ldots[A,B]]]]}_{n\textrm{ times}}. \label{eq:translationSum}
\end{equation}
Defining
\begin{equation}
  B(\lambda) = e^{\lambda A} B e^{-\lambda A}
  \, ,
\end{equation}
differentiating by $\lambda$ we find
\begin{equation}
  \frac{d}{d\lambda}B(\lambda) = [A, B(\lambda)] \label{eq:translation_diff_eq}  % export
\end{equation}
which again shows a relationship between $[A, \cdot]$ and differentiation.

\levelstay{Translation}

Suppose we have a linear differential equation
\begin{equation}
  i \partial_t \ket{\Psi(t)} = H(t) \ket{\Psi(t)} \, .
\end{equation}
The resemblance of this equation to the Schrodinger equation is obvious, but we remain agnostic about applications for now.
This equation has a formal solution
\begin{equation}
  \ket{\Psi(t)} = T(t) \ket{\Psi(0)}
\end{equation}
for some linear transformation $T(t)$, called the ``time translation operator'' or ``propagator''.
By differentiating this formal soluation with respect to time, we find
\begin{equation}
  i \partial_t T(t) = H(t) T(t)
\end{equation}
i.e. the time translation operator satisfies the same equation of motion as the original vector $\ket{\Psi(t)}$.
Suppose we'd like to compute a matrix element
\begin{align}
  \mathcal{O}(t)_{\Psi,\Phi}
  &= \bbraket{\Psi(t)}{\mathcal{O}}{\Phi(t)} \nonumber \\
  &= \bbraket{\Psi(0)}{T^\dagger(t) \mathcal{O}T(t)}{\Phi(0)}
  \, .
\end{align}
Therefore, rather than computing the time evolution of the vectors, we can compute the time evolution of the matrix\footnote{The subscript $H$ here intentionally suggests a relation to the Heisenberg picture of quantum mechanics.}
\begin{equation}
  \mathcal{O}_H(t) \equiv T^\dagger(t) \mathcal{O} T(t)
  \, .
\end{equation}
In the case that $H(t)$ is Hermitian, $T(t)$ is unitary and we can differentiate both sides with respect to $t$ to get
\begin{align}
  i \partial_t \mathcal{O}_H(t)
  &= -T^\dagger(t) H(t) \mathcal{O} T(t)
     + T^\dagger(t) \mathcal{O} H(t) T(t) \nonumber \\
  &= \left[ \mathcal{O}_H(t), H_H(t) \right] \\
  &= \left[ \mathcal{O}, H(t) \right]_H
  \, .
\end{align}
In a common case where $H(t) = H$ is constant, $T(t) = \exp(-i t H)$ and $H_H(t) = H$.
We will see later that this overall structure of going from a linear differential equation for a vector, to an equation of motion for matrices in a different coordinate frame (like $\mathcal{O}_H$), is the idea behind the so called ``interaction'' and ``Heisenberg'' pictures of quantum mechanics.


\subsection{Baker-Campbell-Hausdorff}

The BCH formula provides a summation representation of the product of two exponentiated operators,
\begin{equation}
e^{A}e^{B}=e^{A+B+\frac{1}{2}[A,B]+\frac{1}{12}\left([A,[A,B]]-[B,[A,B]]\right)+\cdots}. \end{equation}

\subsection{Conjugate Variables}

Two operators $\alpha$ and $\beta$ are ``conjugate'' if they have the commutator
\begin{equation}
  [\alpha,\beta]=\eta
\end{equation}
where $\eta$ is a complex number.
If an operator $A$ is normal ordered (all $\beta$'s to the left of all $\alpha$'s) then we have the following extremely useful formulae,
\begin{equation}
  [\alpha,A] = \eta\frac{\partial A}{\partial\beta}
  \qquad\textrm{and}\qquad
  [\beta,A]=-\eta\frac{\partial A}{\partial\alpha}
  \, .
\end{equation}
Conjugate variables also have a very simple translation property
\begin{equation}
  e^{\lambda \hat{\alpha}}\hat{\beta}e^{-\lambda \hat{\alpha}}=\hat{\beta} + \lambda \eta.
\end{equation}

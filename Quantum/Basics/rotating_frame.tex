\section{Rotating Frame}

Consider a spin with a magnetic field oriented along the z-axis.
The Hamiltonian is
\begin{equation}
  H_0/\hbar = -\frac{\omega_q}{2}\sigma_z
  \, .
\end{equation}
Under this Hamiltonian, the spin precesses around the Z axis.

Consider a Schrodinger equation
\begin{equation}
  i \hbar (d/dt) \ket{\Psi_S(t)} = H_S(t) \ket{\Psi_S(t)}
\end{equation}
with formal solution
\begin{equation}
  \ket{\Psi_S(t)} = U_S(t) \ket{\Psi_S(0)} \, .
\end{equation}
Intuitively, we should just apply the inverse of this evolution to the Schrodinger state vector in order to remove the precession.
So, we define a new vector
\begin{equation}
  \ket{\Psi_R(t)} = U_S(t)^{-1} \ket{\Psi_S(t)}
\end{equation}
and we then study the behavior of $\ket{\Psi_R(t)}$.
However, before proceding, let's generalize a bit and instead define
\begin{equation}
  \ket{\Psi_R(t)} = R(t) \ket{\Psi_S(t)}
\end{equation}
where $R(t)$ is an arbitrary unitary operator ($R(t) = U_S(t)^{-1}$ in the special case mentioned above).
Computing the time evolution of this new state we get
\begin{align}
  i\hbar\partial_t \ket{\Psi_R(t)} &= i\hbar \dot{R}(t) \ket{\Psi_S(t)} + R(t) i \hbar\partial_t\ket{\Psi_S(t)} \nonumber \\
  &= i\hbar \dot{R}(t)R^{\dagger}(t) \ket{\Psi_R(t)} + R(t) H_S(t) R^{\dagger}(t) \ket{\Psi_R(t)} \nonumber \\
  \text{($t$'s dropped for brevity)} \qquad
  &= \left( i\hbar \dot{R}R^{\dagger} + R H_S R^{\dagger} \right) \ket{\Psi_R}.
\end{align}
This can be interpreted as a Schrodinger equation for a system with Hamiltonian
\begin{equation}
  H_R(t) = i\hbar \dot{R}(t)R^{\dagger}(t) + R(t) H_S(t) R^{\dagger}(t) \, .
\end{equation}
Note that this result is correct for \emph{any} $R$, not necessarily inverse of the Schrodinger evolution operator.
Note also that we have made no assumptions at all about $H_S(t)$; in particular we haven't assumed that it commutes with itself at different times.
We have an equation of motion for $\ket{\Psi_R(t)}$, but so what... what can we do with it?
Some times, physics can be extracted from $\ket{\Psi_R(t)}$ directly.
In other cases, we may want to compute matrix elements explicitly, so let's see how to do that.
Suppose we have a Schrodinger frame operator $\mathcal{O}_S$ and states $\ket{m_S(t)}$ and $\ket{n_S(t)}$ and we want to find
\begin{equation*}
  \mathcal{O}_{m n}(t) = \bbraket{m_S(t)}{\mathcal{O}_S}{n_S(t)} \, .
\end{equation*}
Direct substitution shows
\begin{equation*}
  \mathcal{O}_{m n}(t) = \bbraket{m_R(t)}{R(t) \mathcal{O}_S R(t)^{-1}}{n_R(t)}
\end{equation*}
so it's natural to define
\begin{equation*}
  \mathcal{O}_R(t) = R(t) \mathcal{O}_S R(t)^{-1}
\end{equation*}
so that matrix elements have the same form in the original and rotated frames:
\begin{equation}
  \mathcal{O}_{mn}(t)
  = \bbraket{m_S(t)}{\mathcal{O}_S}{n_S(t)}
  = \bbraket{m_R(t)}{\mathcal{O}_R(t)}{n_R(t)}
  \, .
\end{equation}

Now let's study the evolution of $\mathcal{O}_R(t)$.


In the case that $R=T^{\dagger}$ the resulting Hamiltonian is particularly simple, as expected \begin{eqnarray}
H_0' &=& i\hbar (iH_0/\hbar)RR^{\dagger} + H_0 \\
&=& 0. \end{eqnarray}
Here we used the fact that $R$, like $T$ is unitary, and that $\partial_t T = -i(H_0/\hbar)T$. The point is that if we rotate the frame at the same rate as the Hamiltonian was rotating the states, the effective Hamiltonian becomes zero.

An extremely important fact to note is that if we take the rotation operator $R$ to be the inverse of the time translation operator induced by the original Hamiltonian, then the effect on any other perturbation or coupling terms $V$ is $V\rightarrow RVR^{\dagger} = T^{\dagger}VT$, which is identical to the transformation found in the interaction picture.

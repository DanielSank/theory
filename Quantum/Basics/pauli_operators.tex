\section{Pauli operators}

\subsection{Representation}
The Pauli operators can be represented as \begin{eqnarray}
\sigma_x &=& \left( \begin{array}{cc} 0 & 1 \\ 1 & 0 \end{array} \right) \\
\sigma_y &=& \left( \begin{array}{cc} 0 & -i \\ i & 0 \end{array} \right) \\
\sigma_z &=& \left( \begin{array}{cc} 1 & 0 \\ 0 & -1 \end{array} \right) \end{eqnarray}

\subsection{Products and commutators}
The Pauli operators anticommute \begin{equation}
\sigma_i \sigma_j = - \sigma_j \sigma_i \qquad (i \neq j) \end{equation}
and have a convenient product property \begin{equation}
\sigma_i \sigma_j = i \epsilon_{ijk} \sigma_k. \end{equation}
From the product and anticommutation follows the commutation relation \begin{equation}
\left[ \sigma_i, \sigma_j \right] = 2i \epsilon_{ijk} \sigma_k. \end{equation}

\subsection{Translation}

The problem of translating one Pauli operator by another arises frequently when analyzing qubit systems.
We wish to evaluate \begin{equation}
S(Q) = e^{-iQ\sigma_i}\sigma_j e^{iQ\sigma_i}. \end{equation}
We use the differential equation (\ref{eq:translationDiffEq}) with $A=-iQ\sigma_i$ to get \begin{equation}
\frac{dS}{dQ} = i[S(Q),\sigma_i]. \end{equation}
We postulate the solution \begin{equation}
S(Q) = \alpha(Q)\sigma_j + \beta(Q)\sigma_k. \end{equation}
First work out the commutator $i[S(Q),\sigma_i]$ \begin{eqnarray*} 
i[S(Q),\sigma_i] &=& i[\alpha\sigma_j + \beta\sigma_k, \sigma_i] \\
&=& i \left( -2i\alpha \sigma_k + 2i\beta \sigma_j \right) \\
&=& 2\alpha \sigma_k - 2\beta \sigma_j. \end{eqnarray*}
Equating the right hand side with the explicit derivative of $S$ yields \begin{equation}
\dot{\alpha}(Q) = -2\beta(Q) \qquad \dot{\beta}(Q) = 2\alpha(Q) \nonumber \end{equation}
with solution \begin{equation}
\alpha(Q) = \cos \left( 2Q \right) \qquad \beta(Q) = \sin \left( 2Q \right). \nonumber \end{equation}
Therefore \begin{equation}
e^{-iQ\sigma_i}\sigma_j e^{iQ\sigma_i} = \cos \left( 2Q \right) \sigma_j + \sin \left( 2Q \right) \sigma_k. \end{equation}

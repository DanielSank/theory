% -- begin imports --
% import commutators as comm
% -- end imports --

\section{Pauli operators}

\subsection{Representation}
The Pauli operators can be represented as \begin{align}
  \sigma_x &= \left( \begin{array}{cc} 0 & 1 \\ 1 & 0 \end{array} \right) \nonumber \\
  \sigma_y &= \left( \begin{array}{cc} 0 & -i \\ i & 0 \end{array} \right) \nonumber \\
  \sigma_z &= \left( \begin{array}{cc} 1 & 0 \\ 0 & -1 \end{array} \right)
\end{align}

\subsection{Products and commutators}
The Pauli operators anticommute \begin{equation}
\sigma_i \sigma_j = - \sigma_j \sigma_i \qquad (i \neq j) \end{equation}
and have a convenient product property \begin{equation}
\sigma_i \sigma_j = i \epsilon_{ijk} \sigma_k. \end{equation}
From the product and anticommutation follows the commutation relation \begin{equation}
\left[ \sigma_i, \sigma_j \right] = 2i \epsilon_{ijk} \sigma_k. \end{equation}

\subsection{Translation}

The problem of translating one Pauli operator by another arises frequently when analyzing qubit systems.
We wish to evaluate
\begin{equation}
  \mathcal{O}(\lambda) = e^{-i\lambda\sigma_i}\sigma_j e^{i\lambda\sigma_i}.
\end{equation}
We use the differential equation (\ref{comm.eq:translation_diff_eq}) with $A=-i\lambda\sigma_i$ to get
\begin{equation}
  \frac{d\mathcal{O}}{d\lambda} = i[\mathcal{O}(\lambda),\sigma_i].
\end{equation}
Think of the Pauli matrices as the $x$, $y$, and $z$ axes in three dimensional space.
The geometric meaning of this differential equation is that we're rotating a vector aligned along $j$-axis about the $i$-axis.
It's clear then that as we rotate, the original vector along $j$ should pick up a component along $k$.
For example, if you rotate a vector aligned along the $x$-axis about the $y$-axis, you start to get some contribution along the $z$-axis.
With this in mind, we postulate the solution
\begin{equation}
  \mathcal{O}(\lambda) = \alpha(\lambda)\sigma_j + \beta(\lambda)\sigma_k \, .
\end{equation}
This solution is trivially correct for $\lambda = 0$.
To make progress, we work out the commutator $i[\mathcal{O}(\lambda),\sigma_i]$
\begin{align*}
  i[\mathcal{O}(\lambda),\sigma_i]
    &= i[\alpha(\lambda) \sigma_j + \beta(\lambda) \sigma_k, \sigma_i] \\
    &= i \left( -2i\alpha(\lambda) \sigma_k + 2i\beta(\lambda) \sigma_j \right) \\
    &= 2\alpha(\lambda) \sigma_k - 2\beta(\lambda) \sigma_j \, .
\end{align*}
Now we have
\begin{align*}
  \frac{d\mathcal{O}}{d\lambda}
    &= i \left[ \mathcal{O}(\lambda), \sigma_i \right] \\
  \dot{\alpha}(\lambda)\sigma_j + \dot{\beta}(\lambda) \sigma_k
    &= 2 \alpha(\lambda)\sigma_k - 2 \beta(\lambda) \sigma_j \, .
\end{align*}
Matching terms gives $\dot{\alpha}(\lambda) = -2\beta(\lambda)$ and $\dot{\beta}(\lambda) = 2\alpha(\lambda)$, with solution
\begin{equation}
  \alpha(\lambda) = \cos \left( 2\lambda \right) \qquad \beta(\lambda) = \sin \left( 2\lambda \right). \nonumber
\end{equation}
Therefore
\begin{equation}
  e^{-i\lambda\sigma_i}\sigma_j e^{i\lambda\sigma_i} = \cos \left( 2\lambda \right) \sigma_j + \sin \left( 2\lambda \right) \sigma_k \, .
\end{equation}

\levelstay{Time Dependent Perturbation Theory}

In the interaction picture, the equation of motion for $\ket{\Psi_{I}(t)}$ can be formally solved
\begin{equation}
  \ket{\Psi_I(t)}
  = \ket{\Psi_I(0)} + \frac{1}{i \hbar}\int_0^t dt' V_I(t') \ket{\Psi_I(t')}
  \, .
\end{equation}
Plugging the left side into the right side gives
\begin{equation}
  \ket{\Psi_I(t)}
  = \ket{\Psi_I(0)} + \frac{1}{i \hbar} \int_0^t dt' \, V_I(t') \left[ \ket{\Psi_{I}(0)} + \frac{1}{i \hbar} \int_0^{t'}dt'' \, V_I(t'') \ket{\Psi_I(t'')} \right]
\end{equation}
which can be iterated any number of times to produce a perturbation series.

\leveldown{First order}
Keeping only terms to first order in the interaction $V$ we have
\begin{equation}
  \ket{\Psi_I(t)} = \ket{\Psi_I(0)} + \frac{1}{i \hbar} \int_0^t dt' \, V_I(t') \ket{\Psi_I(0)}
\end{equation}
which can be understood in terms of a propogator,
\begin{equation}
  \ket{\Psi_I(t)} = T_I(t,0) \ket{\Psi_I(0)}
  \qquad \text{where} \qquad
  T_I(t,0) = 1 - \frac{i}{\hbar} \int_0^t dt' \, V_I(t')
  \, .
\end{equation}
Let $\ket{n}$ label a complete set of states, for example those which diagonalize $H_0$, and consider the case where $\ket{\Psi_I(0)}$ is one of these states $\ket{m}$.
There is no loss of generality, as any initial state can be resolved into a superposition of such states.
We compute the components of $\ket{\Psi_I(t)}$ in this basis,
\begin{align}
  \braket{n}{\Psi_I(t)} = c^{(1)}_{n,m}(t)
  =& \braket{n}{\Psi_I(0)} + \frac{1}{i\hbar} \int_{-\infty}^t dt' \bra{n} V_I(t') \ket{\Psi_I(0)} \nonumber \\
  =& \braket{n}{m} + \frac{1}{i \hbar}\int_{-\infty}^t dt' \bra{n}T^{\dagger}(t')V_S(t')T(t)\ket{m} \nonumber \\
  =& \delta_{n,m} + \frac{1}{i \hbar}\int_{-\infty}^t dt' e^{-i (\omega_m-\omega_n) t} \bra{n}V_S(t')\ket{m}
  \, .
\end{align}
A common situation is one in which $V_S(t)$ can be written as a time dependant scalar times a time independent operator, ie. $V_S(t) = f(t)V_S$. In this case the transition amplitude becomes
\begin{equation}
  c^{(1)}_{n,m}(t) =
    \delta_{n,m} + \frac{\braket{n}{V_S|m}}{i \hbar}\int_{-\infty}^t dt' f(t') e^{i(\omega_n - \omega_m)t'}
  \, .
\end{equation}
This equation conveys important physical meaning.
Under the influence of a weak drive, the probability that a system transitions from $\ket{m}$ to $\ket{n}$ is proportional to the Fourier transform of the time dependent drive function, evaluated at the frequency given by the difference $\omega_n - \omega_m$.
In other words, the drive has to supply precisely the energy difference required to make the transition.
Of course, the integral is taken only as long as the drive is applied, so even a monochromatic drive e.g. $f(f) = \cos(\omega t)$ does not provide perfectly sharp frequency resolution.
This is just the usual case where a longer drive leads to more sharply resolved peaks.

\levelstay{Second order}
The second order term in the series is
\begin{align}
  c^{(2)}_{n,m}(t)
  =& \left(\frac{1}{i\hbar}\right)^2 \int_{-\infty}^t dt' \int_{-\infty}^{t'} dt'' \bbraket{n}{V_I(t')V_I(t'')}{m} \nonumber \\
  =& \left(\frac{1}{i\hbar}\right)^2 \int_{-\infty}^t dt' \int_{-\infty}^{t'} dt'' \bbraket{n}{T^{\dagger}(t')V_S(t')T(t')T^{\dagger}(t'')V_S(t'')T(t'')}{m} \nonumber \\
  =& \left(\frac{1}{i\hbar}\right)^2 \sum_{k} \int_{-\infty}^t dt' \int_{-\infty}^{t'} dt'' e^{i\omega_n t'} \bbraket{n}{V_S(t')}{k} e^{-i \omega_k (t'-t'')} \bbraket{k}{V_S(t'')}{m} e^{-i\omega_m t''} \nonumber \\
  =& \left(\frac{1}{i\hbar}\right)^2 \sum_{k} \int_{-\infty}^t dt' \int_{-\infty}^{t'} dt''
  \left( e^{i (\omega_n - \omega_k) t'} \bbraket{n}{V_S(t')}{k} \right)
  \left( e^{i (\omega_k - \omega_m) t''}\bbraket{k}{V_S(t'')}{m} \right)
  \, .
\end{align}
This equation again has an important physical meaning.
The transition probability contains two factors as shown in parentheses.
Each factor looks just like one of the first order transition probabilities we found above, with similar meaning.
However, now instead of transitioning directly from the initial state to the final state, we have a sum over all possible intermediate states $\ket{k}$.
We imagine that, on its way from initial state $\ket{m}$ to final state $\ket{n}$, the system may make an intermediate stop at any of the various states $\ket{k}$, and it may do this at any time $t'$.
These states $\ket{k}$ are colloquially referred to as ``virtual states'' because the system is never actually found occupying them, but they show up in the perturbation expansion.


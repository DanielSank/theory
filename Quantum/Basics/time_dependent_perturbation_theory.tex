\levelstay{Time Dependent Perturbation Theory}

In the interaction picture, the equation of motion for $\ket{\Psi_{I}(t)}$ can be formally solved
\begin{equation}
  \ket{\Psi_I(t)}
  = \ket{\Psi_I(0)} + \frac{1}{i \hbar}\int_0^t dt' V_I(t') \ket{\Psi_I(t')}
  \, .
\end{equation}
Plugging the left side into the right side gives
\begin{equation}
  \ket{\Psi_I(t)}
  = \ket{\Psi_I(0)} + \frac{1}{i \hbar} \int_0^t dt' \, V_I(t') \left[ \ket{\Psi_{I}(0)} + \frac{1}{i \hbar} \int_0^{t'}dt'' \, V_I(t'') \ket{\Psi_I(t'')} \right]
  \, .
\end{equation}
This can be iterated any number of times to produce a perturbation series.

\leveldown{First order}
Keeping only terms to first order in the interaction $V$ we have
\begin{equation}
  \ket{\Psi_I(t)} = \ket{\Psi_I(0)} + \frac{1}{i \hbar} \int_0^t dt' \, V_I(t') \ket{\Psi_I(0)}
  \, .
\end{equation}
Note that this can be understood in terms of a propogator,
\begin{equation}
  \ket{\Psi_I(t)} = T_I(t,0) \ket{\Psi_I(0)}
  \qquad \text{where} \qquad
  T_I(t,0) = 1 - \frac{i}{\hbar} \int_0^t dt' \, V_I(t')
  \, .
\end{equation}
Let $\ket{n}$ label a complete set of states, for example, those which diagonalize $H_0$, and consider the case where $\ket{\Psi_I(0)}$ is one of these states $\ket{m}$.
There is no loss of generality, as any initial state can be resolved into a superposition of such states.
We compute the components of $\ket{\Psi_I(t)}$ in this basis,
\begin{align}
  \braket{n}{\Psi_I(t)} = c^{(1)}_{n,m}(t)
  =& \braket{n}{\Psi_I(0)} + \frac{1}{i\hbar} \int_{-\infty}^t dt' \bra{n} V_I(t') \ket{\Psi_I(0)} \nonumber \\
  =& \delta_{n,m} + \frac{1}{i \hbar}\int_{-\infty}^t dt' \bra{n}T^{\dagger}(t')V_S(t')T(t)\ket{m} \nonumber \\
  =& \delta_{n,m} + \frac{1}{i \hbar}\int_{-\infty}^t dt' e^{-i (\omega_m-\omega_n) t} \bra{n}V_S(t')\ket{m}
  \, .
\end{align}
A common situation is one in which the time dependent perturbation can be written as a time dependant scalar times a time independent operator, ie. $V_S(t) = f(t)V_S$. In this case the transition amplitude becomes
\begin{equation}
c^{(1)}_{n,m}(t) = \delta_{n,m} + \frac{\braket{n}{V_S|m}}{i \hbar}\int_{-\infty}^t dt' f(t') e^{-i(\omega_m - \omega_n)t'} \end{equation}
From this expression you can derive Fermi's Golden Rule for transition rates.

\levelstay{Second order}
The second order term in the series is
\begin{align}
  c^{(2)}_{n,m}(t)
  =& \left(\frac{1}{i\hbar}\right)^2 \int_{-\infty}^t dt' \int_{-\infty}^{t'} dt'' \bbraket{n}{V_I(t')V_I(t'')}{m} \nonumber \\
  =& \left(\frac{1}{i\hbar}\right)^2 \int_{-\infty}^t dt' \int_{-\infty}^{t'} dt'' \bbraket{n}{T^{\dagger}(t')V_S(t')T(t')T^{\dagger}(t'')V_S(t'')T(t'')}{m} \nonumber \\
  =& \left(\frac{1}{i\hbar}\right)^2 \sum_{k} \int_{-\infty}^t dt' \int_{-\infty}^{t'} dt'' e^{i\omega_n t'} \bbraket{n}{V_S(t')}{k} e^{-i \omega_k (t'-t'')} \bbraket{k}{V_S(t'')}{m} e^{-i\omega_m t''}
  \, .
\end{align}

\levelstay{Kubo Formula (not finished)}
In this language we can derive a useful expression for expectation
values of operators in the interaction picture, \begin{eqnarray*}
\braket{\Psi_{I}(t)}{A_{I}(t)|\Psi_{I}(t)} & = & \bra{T(t,0)\Psi_{I}(0)}A_{I}(t)\ket{T(t,0)\Psi_{I}(0)}\\
& = & \bra{\Psi_{I}(0)}T^{\dagger}(t,0)A_{I}(t)T(t,0)\ket{\Psi_{I}(0)}\\
& = & \bra{\Psi_{I}(0)}\left[1+\frac{1}{\hbar}\int_{0}^{t}dt'V_{I}(t')\right]A_{I}(t)\left[1-\frac{i}{\hbar}\int_{0}^{t}dt'V_{I}(t')\right]\ket{\Psi_{I}(0)}\end{eqnarray*}
To first order in the interaction this is \begin{eqnarray*}
\braket{\Psi_{I}(t)}{A_{I}(t)|\Psi_{I}(t)} & = & \braket{\Psi_{I}(0)}{A_{I}(t)-A_{I}(t)\left[\frac{i}{\hbar}\int_{0}^{t}dt'V_{I}(t')\right]+\left[\frac{i}{\hbar}\int_{0}^{t}dt'V_{I}(t')\right]A_{I}(t)|\Psi_{I}(0)}\\
\braket{\Psi_{I}(t)}{A_{I}(t)|\Psi_{I}(t)} & = & \braket{\Psi_{I}(0)}{AI(t)-\frac{i}{\hbar}\int_{0}^{t}}\end{eqnarray*}

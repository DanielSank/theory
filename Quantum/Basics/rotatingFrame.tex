\section{Rotating Frame}

The basic qubit Hamiltonian is \begin{equation}
H_q/\hbar = -\frac{\omega_q}{2}\sigma_z \end{equation}
A quantum state under this Hamiltonian precesses around the Z axis. In the lab we are used to thinking about a rotating frame in which this precession is absent. We now show how do we do this mathematically.

In the Schrodinger picture the time evolution operator for a Hamiltonian $H_0$ is \begin{equation}
T = \exp \left[ -\frac{i}{\hbar}H_0 t \right]. \end{equation}
Intuitively, we should just apply the inverse of this evolution to the Schrodinger state vector in order to remove the precession. We can then define a state in the rotating frame as \begin{equation}
\ket{\Psi'(t)} = R\ket{\Psi(t)} \end{equation}
where $R=T^{\dagger}$. Computing the time evolution of this new state we get \begin{eqnarray}
i\hbar\partial_t \ket{\Psi'(t)} &=& i\hbar \dot{R}\ket{\Psi(t)} + Ri\hbar\partial_t\ket{\Psi(t)} \\
&=& i\hbar \dot{R}R^{\dagger} \ket{\Psi'(t)} + R H_0 R^{\dagger} \ket{\Psi'(t)} \\
&=& \left( i\hbar \dot{R}R^{\dagger} + RH_0R^{\dagger} \right) \ket{\Psi'(t)}. \end{eqnarray}
This can be interpreted as a Schrodinger equation for a system with Hamiltonian \begin{equation}
H_0' = i\hbar \dot{R}R^{\dagger} + RH_0R^{\dagger}. \end{equation}
Note that this result is correct for \emph{any} $R$, not necessarily inverse of the Schrodinger evolution operator.

In the case that $R=T^{\dagger}$ the resulting Hamiltonian is particularly simple, as expected \begin{eqnarray}
H_0' &=& i\hbar (iH_0/\hbar)RR^{\dagger} + H_0 \\
&=& 0. \end{eqnarray}
Here we used the fact that $R$, like $T$ is unitary, and that $\partial_t T = -i(H_0/\hbar)T$. The point is that if we rotate the frame at the same rate as the Hamiltonian was rotating the states, the effective Hamiltonian becomes zero.

An extremely important fact to note is that if we take the rotation operator $R$ to be the inverse of the time translation operator induced by the original Hamiltonian, then the effect on any other perturbation or coupling terms $V$ is $V\rightarrow RVR^{\dagger} = T^{\dagger}VT$, which is identical to the transformation found in the interaction picture.

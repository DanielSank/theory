\documentclass[twocolumn]{article}

% General physics constructs
\newcommand{\bra}[1]{\langle #1 |}
\newcommand{\ket}[1]{| #1 \rangle }
\newcommand{\braket}[2]{\langle #1|#2\rangle}
\newcommand{\bbraket}[3]{ \langle #1 | #2 | #3 \rangle }
\newcommand{\boltzmann}{k_b}

% Common math
\newcommand{\norm}[1]{\left \lvert #1 \right \rvert}
\newcommand{\abs}[1]{\left \lvert #1 \right \rvert}  % These two are redundant. Consider removing one.

\newcommand{\avg}[1]{\left \langle #1 \right \rangle}  % Should get rid of this, as "average" isn't specific.
\newcommand{\angavg}[1]{\left \langle #1 \right \rangle}

\newcommand{\VS}{\textit{\textbf{V}}}
\newcommand{\Tr}{\textrm{Tr}}
\renewcommand{\Re}{\textrm{Re}}
\renewcommand{\Im}{\textrm{Im}}
\newcommand{\basis}[1]{\{\ket{#1}\}}

% Quantum
\newcommand{\nboseeinstein}{n_\text{BE}}
\newcommand{\gammaup}{\Gamma_\uparrow}
\newcommand{\gammadown}{\Gamma_\downarrow}
\newcommand{\gammaupdown}{\Gamma_{\uparrow \downarrow}}
\newcommand{\gammaemission}{\Gamma_\text{loss}}
\newcommand{\qualityfactoremission}{Q_{d,\text{loss}}}

% Qubits
\newcommand{\omegaqubit}{\omega_{10}}

% Circuits
\newcommand{\impedance}{Z_0}
\newcommand{\resistorsource}{R_s}
\newcommand{\vsource}{V_s}
\newcommand{\vsourcerms}{V_{s,\text{rms}}}
\newcommand{\vloadrms}{V_{l,\text{rms}}}

% Signals and noise
\newcommand{\psdsingle}{S_\text{ss}}
\newcommand{\psddouble}{S_\text{ds}}
\newcommand{\noiseavailable}{S_{p,a}^e}
\newcommand{\spectralengineer}{S^e}
\newcommand{\spectralsymmetric}{S^\text{symm}}
\newcommand{\spectralattenuator}{\spectralengineer_{\poweravailable, \text{att.}}}

% Microwaves
\newcommand{\vright}{V_+}
\newcommand{\vleft}{V_-}
\newcommand{\iright}{I_+}
\newcommand{\ileft}{I_-}
\newcommand{\poweravailable}{P_a}

% Figures. Example usage:
% \quickfig{\columnwidth}{my_image}{This is the caption}{fig:my_fig}
\DeclareRobustCommand{\quickfig}[4]{
\begin{figure}
\begin{centering}
\includegraphics[width=#1]{#2}
\par\end{centering}
\caption{#3}
\label{#4}
\end{figure}
}

\DeclareRobustCommand{\quickwidefig}[4]{
\begin{figure*}[h]
\begin{centering}
\includegraphics[width=#1]{#2}
\par\end{centering}
\caption{#3}
\label{#4}
\end{figure*}
}

\DeclareRobustCommand{\quickfigcentered}[4]{
  \begin{figure}
  \makebox[\textwidth][c]{\includegraphics[width=#1]{#2}}
  \caption{#3}
  \label{#4}
  \end{figure}
}

%Packages
\usepackage{amsmath}
\usepackage{amstext}
\usepackage{amssymb}
\usepackage{appendix}
\usepackage{coseoul}
\usepackage{graphicx}
\usepackage{import}
\usepackage{lscape}
\usepackage{modular}

\usepackage[pdfpagemode=UseNone,pdfstartview=FitH,colorlinks=true,linkcolor=blue,citecolor=blue,urlcolor=blue]{hyperref}
\usepackage[all]{hypcap}



\title{Jaynes-Cummings Hamiltonian}
\author{Daniel Sank}
\date{9 August 2011}

\begin{document}

\maketitle
The Hamiltonian is\begin{eqnarray*}
H & = & H_{q}+H_{r}+H_{I}\\
H_{q} & = & -\frac{1}{2}\omega_{q}\sigma_{z}\\
H_{r} & = & \omega_{r}\left(a^{\dagger}a+\frac{1}{2}\right)\\
H_{I} & = & \frac{\Omega}{2}\left(a^{\dagger}\sigma^{-}+a\sigma^{+}\right)\end{eqnarray*}



\section{Schrodinger Picture}

We first work the problem in the Schrodinger picture. To do this we
regroup the Hamiltonian as follows\[
H=\underbrace{\left[\omega_{r}\left(a^{\dagger}a-\frac{1}{2}\sigma^{z}\right)\right]}_{H_{1}}+\underbrace{\left[-\frac{1}{2}\delta\sigma^{z}+\frac{\Omega}{2}\left(a\sigma^{+}+a^{\dagger}\sigma^{-}\right)\right]}_{H_{2}}\]
where $\delta\equiv\omega_{q}-\omega_{r}$. Look first at $H_{1}$.
It does not couple the qubit and resonator, so its eigenstates are
trivially the products of the eigenstates of the two systems independently.
Now consider $H_{2}$. This term couples the systems together. However,
$H_{1}$ and $H_{2}$ commute with one another, which means that there
exists a basis of eigenvectors of the total Hamiltonian which simultaneously
diagonalizes each of them individually. The eigenvectors of $H_{1}$
are of the form $|g,n\rangle$ meaning {}``ground state of the qubit,
n photons in the resonator,'' or $|e,m\rangle$ meaning {}``excited
state of the qubit with m photons in the resonator.'' In order to
build linear combinations of these states that are still eigenvectors
of $H_{1}$ we have to make sure that the operator $a^{\dagger}a+\frac{1}{2}\sigma^{z}$
has the same value on all terms in each linear combination. In other
words, we have to work within the degenerate subspaces of $H_{1}$.
A generic two term linear combination would be\[
|g,n\rangle+|e,m\rangle\]
and the values of $H_{1}$ acting on each term are\[
H_{1}|g,n\rangle=n-\frac{1}{2}\qquad H_{1}|e,m\rangle=m+\frac{1}{2}\]
Setting these two values equal to one another gives\[
n-\frac{1}{2}=m+\frac{1}{2}\longrightarrow n=m+1\]
Therefore, the eigenvectors of the total Hamiltonian can be written
as pairwise linear combinations of states $|g,n\rangle$ and $|e,n-1\rangle$.
In more physical terms, the interaction Hamiltonian conserves excitation
number, so the eigenstates are combinations of states with equal total
numbers of excitations.

Now that we know that our eigenstates will be combinations of states
with equal numbers of excitations, we have to find out exactly what
the combinations are. To do this we write out the matrix form of $H$
in the subspace of $|g,n\rangle$ and $|e,n-1\rangle$\[
H^{(n)}=\underbrace{\left[\begin{array}{cc}
\omega_{r}\left(n-1\right)+\frac{\omega_{q}}{2} & \frac{\Omega}{2}\sqrt{n}\\
\frac{\Omega}{2}\sqrt{n} & \omega_{r}n-\frac{\omega_{q}}{2}\end{array}\right]}_{|e,n-1\rangle\qquad|g,n\rangle}\]


in order to diagonalize this two by two matrix we break it down in
terms of pauli operators.\[
H^{(n)}=\omega_{r}\left(n-\frac{1}{2}\right)I+\frac{\delta}{2}\sigma^{z}+\frac{\Omega}{2}\sqrt{n}\sigma^{x}\]
The eigenvalues of this Hamiltonian are given simply by the part proportional
to identity plus or minus the length of the rest,\[
E_{\pm}=\omega_{r}\left(n-\frac{1}{2}\right)\pm\frac{1}{2}\sqrt{\Omega^{2}n+\delta^{2}}\]
and the eigenstates are\begin{eqnarray*}
|\Psi^{-}\rangle & = & \cos\left(\frac{\alpha_{n}}{2}\right)|e,n-1\rangle+\sin\left(\frac{\alpha_{n}}{2}\right)|g,n\rangle\\
|\Psi^{+}\rangle & = & -\sin\left(\frac{\alpha_{n}}{2}\right)|e,n-1\rangle+\cos\left(\frac{\alpha_{n}}{2}\right)|g,n\rangle\end{eqnarray*}
where $\alpha_{n}\equiv\arctan\left(\Omega\sqrt{n}/\delta\right)$.
If the qubit and resonator are far off resonance then $\delta$ is
large, so $\alpha_{n}$ is small and the eigenstates are just products
of the eigenstates we would have if the systems were uncoupled. If
the systems are exactly on resonance then $\delta=0$ and $\alpha_{n}=\pi/2$
so the eigenstates are equal superpositions of the original two.


\section{Interaction picture}

We work in the interaction picture, taking the intrinsic Hamiltonian
to be $H_{0}=H_{r}+H_{q}$ and the interaction term to be $H_{I}$.
The time dependent operators in the interaction picture are determined
by the Heisenberg equation of motion, $id_{t}\mathcal{O}(t)=[\mathcal{O}(t),H_{0}]$.
Applying this equation to the qubit and resonator operators yields\begin{eqnarray*}
a(t) & = & a(0)e^{-i\omega_{r}t}\\
\sigma^{-}(t) & = & \sigma^{-}(0)e^{-i\omega_{q}t}\end{eqnarray*}
All other time dependences can be derived from these. For example,
the time dependence of $\sigma^{x}$ is found to be\begin{eqnarray*}
\sigma^{x}(t) & = & \sigma^{+}(t)+\sigma^{-}(t)\\
& = & e^{i\omega_{q}t}\sigma^{+}+e^{-i\omega_{q}t}\sigma^{-}\\
& = & e^{i\omega_{q}t}\frac{1}{2}\left(\sigma^{x}-i\sigma^{y}\right)+e^{-i\omega_{q}t}\frac{1}{2}\left(\sigma^{x}+i\sigma^{y}\right)\\
& = & \cos\left(\omega_{q}t\right)\sigma^{x}+\sin\left(\omega_{q}t\right)\sigma^{y}\end{eqnarray*}
Here all operators without time dependence are implied to be the Schrodinger
picture operators, ie the operators at $t=0$.

The interaction Hamiltonian must be written in the interaction picture.
We do this using the time dependences listed above, \begin{equation}
H_{I}(t)=\frac{\Omega}{2}\left(a\sigma^{+}e^{i\delta t}+a^{\dagger}\sigma^{-}e^{-i\delta t}\right) \end{equation}
where $\delta\equiv\omega_{q}-\omega_{r}$. In the case that $\delta=0$
the interaction Hamiltonian is time independent and we can solve the
evolution easily. The interaction $H_{I}$ in this case is block diagonal,
linking only states $|g,n\rangle$ and $|e,n-1\rangle$. The matrix
is \begin{equation}
H_{I}^{(n)}=\frac{\Omega}{2}\left[\begin{array}{cc}
0 & \sqrt{n}\\
\sqrt{n} & 0\end{array}\right]=\frac{\Omega}{2}\sqrt{n}\sigma^{x} \end{equation}
Computing the evolution operator we get \begin{equation}
U(t)=\exp\left[-iH_{I}t\right]=\cos\left(\frac{\Omega}{2}\sqrt{n}t\right)I-i\sin\left(\frac{\Omega}{2}\sqrt{n}t\right)\sigma^{x} \end{equation}
Note that this is the evolution operator only for states with nonzero excitations. In the subspace of $|g,0\rangle$ the Hamiltonian is zero and the evolution operator is the identity.

If the system starts out in the state $\ket{g \alpha}$ the effect of this evolution produces a state with two parts: one where the qubit remains in the ground state, and another with the qubit in the excited state. After time $t$ the term with the qubit in the ground state is \begin{eqnarray*}
\ket{\Psi} &=& \ket{g0} + \sum_{n=1}^{\infty} \frac{\alpha^n}{\sqrt{n!}} \cos \left( \frac{\Omega}{2}\sqrt{n} t \right) \ket{gn} \\
&=& \ket{g0} + \sum_{n=1}^{\infty} \frac{\alpha^n}{\sqrt{n!}} \left( 1 - \frac{\Omega^2 n t^2}{8}\right) \ket{gn} \\
&=& \ket{g0} + \sum_{n=1}^{\infty} \frac{\left[ \alpha \left(1 - \Omega^2 t^2 / 8 \right) \right]^n}{\sqrt{n!}} \ket{gn} \\
&=& \ket{g \beta} \end{eqnarray*}
where $\beta = \alpha [1 - (\Omega t)^2 /8]$. Thus, the amplitude of the coherent state is reduced in the part of the system state in which the qubit is in the ground state.

Now we compute the part with the qubit in the excited state. By 
\end{document}
\documentclass{article}

% General physics constructs
\newcommand{\bra}[1]{\langle #1 |}
\newcommand{\ket}[1]{| #1 \rangle }
\newcommand{\braket}[2]{\langle #1|#2\rangle}
\newcommand{\bbraket}[3]{ \langle #1 | #2 | #3 \rangle }
\newcommand{\norm}[1]{\| #1\|}
\newcommand{\avg}[1]{\left \langle #1 \right \rangle}
\newcommand{\angavg}[1]{\left \langle #1 \right \rangle}
\newcommand{\abs}[1]{\left \lvert #1 \right \rvert}
\newcommand{\VS}{\textit{\textbf{V}}}
\newcommand{\Tr}{\textrm{Tr}}
\renewcommand{\Re}{\textrm{Re}}
\renewcommand{\Im}{\textrm{Im}}
\newcommand{\basis}[1]{\{\ket{#1}\}}

\newcommand{\omegaqubit}{\omega_{10}}

% Figures. Example usage:
% \quickfig{\columnwidth}{my_image}{This is the caption}{fig:my_fig}
\DeclareRobustCommand{\quickfig}[4]{
\begin{figure}
\begin{centering}
\includegraphics[width=#1]{#2}
\par\end{centering}
\caption{#3}
\label{#4}
\end{figure}
}

\DeclareRobustCommand{\quickwidefig}[4]{
\begin{figure*}[h]
\begin{centering}
\includegraphics[width=#1]{#2}
\par\end{centering}
\caption{#3}
\label{#4}
\end{figure*}
}

%Packages
\usepackage{amsmath}
\usepackage{amstext}
\usepackage{amssymb}
\usepackage{appendix}
\usepackage{coseoul}
\usepackage{enumerate}
\usepackage{graphicx}
\usepackage{import}
\usepackage{lscape}
\usepackage{modular}

\usepackage[pdfpagemode=UseNone,pdfstartview=FitH,colorlinks=true,linkcolor=blue,citecolor=blue,urlcolor=blue]{hyperref}
\usepackage[all]{hypcap}



\begin{document}

\title{Ramsey Decay Envelope}
\author{Daniel Sank\\University of California Santa Barbara\\presently at Google}
\date{October 23, 2009}

\maketitle


\section{General Description}

The sequence for a Ramsey experiment is
\begin{equation}
X_{\pi/2}\rightarrow\textrm{wait for time }t\rightarrow X_{-\pi/2} \, .
\end{equation}
The first pulse rotates the qubit from $\ket{0}$ to a point on the equator of the Bloch sphere.
The qubit state then idles on the equator for time $t$.
In the absence of noise the state is unchanged during this time, so the final pulse rotates the state back to $\ket{0}$.
When there is noise the final state is not exactly $\ket{0}$.
It is shown elsewhere that energy relaxation effects ($T_1$) and frequency fluctuation effects $(T_2)$ can be considered separately because their effects on the decay envelopes of various experiments factor apart; for example the decay of a Ramsey fringe for a qubit under the influence of energy decay with time constant $T_1$ and phase decay with time constant $T_\phi$ is $\exp(-t(2T_1 + T_\phi)$.
Therefore, we consider only the effect of dephasing, ie. no energy relaxation.

During the part of the sequence in which the qubit idles at the equator of the Bloch sphere, the $\ket{0}\rightarrow\ket{1}$ transition frequency, and therefore the precession frequency, fluctuates and causes the qubit state to diffuse (make a random walk)
on the equator.
Therefore, the second rotation results in a qubit state different from $|0\rangle$ and so the measured probability of finding $\ket{0}$ is not 1.
The probability is what we can measure, and the decrease away from 1 is what we call the decay of the Ramsey curve.

Note that if the qubit state is completely diffused on the equator then the expected value of the zero state probability after the second pulse is $\frac{1}{2}$.
Therefore, we expect our calculations to yield a curve that goes from 1 to $\frac{1}{2}$ as the wait time goes from zero to infinity.

\subsection{Note on conventions}

In this document, spectral densities, e.g. $S(\omega)$, are single sided spectral densities.
This is the thing you read off of a spectrum analyzer.
It's defined for positive frequencies only.
For real signals where the double sided spectral density is an even function, we have
\begin{equation}
S_{\text{single sided}}(\omega) = 2 S_{\text{double sided}}(\omega) \, .
\end{equation}
Importantly for the calculations done here, with single sided spectral density $S$ the Wiener-Khinchin theorem is
\begin{equation}
\langle f(t) f(0) \rangle = \int_0^\infty \frac{d\omega}{2\pi} S_f(\omega) \cos(\omega t)
\end{equation}
and conversely
\begin{equation}
S(\omega) = 2 \int_{-\infty}^\infty dt \, \langle f(0) f(t) \rangle \cos(\omega t) \, . 
\end{equation}


\section{Ramsey decay}

\subsection{General noise spectral density}

The fluctuations in the qubit frequency lead to phase noise according to the following formula
\begin{equation}
\phi(t)=\frac{\partial\omega_{01}}{\partial\lambda}\int_0^t\lambda(t')\, dt'
\end{equation}
where $\lambda$ is the physical quantity that is actually causing the qubit frequency noise, ie. $\lambda$ could be magnetic flux in the flux biased phase qubit.
If $\lambda$ is Gaussian noise then $\phi(t)$ is also Gaussian distributed.
Note that when we say ``distributed'' here we're implicitly referring to an ensemble of $\emph{experiments}$.
In other words, if we were to measure $\phi(t)$ many times and then make a histogram, that curve would be a Gaussian.

Now let's write down the measured probability as a function of $\phi$.
The measured population is the projection of the Block vector on the $z$ axis, so we get
\begin{equation}
\textrm{z-Projection}(\phi)=\frac{1}{2}\left(1+\cos(\phi)\right) \, .
\end{equation}
When we measure the probability for a certain wait time $t$, we actually repeat the experiment many times and average them together.
This averages over the possible values of $\phi$.
Therefore we can say\begin{equation}
p(t)
= \langle\frac{1}{2}\left(1+\cos(\phi)\right)\rangle_{\phi}
= \frac{1}{2}+\frac{1}{2}\langle\cos(\phi)\rangle_{\phi} \, .
\end{equation}
Denote the probability distribution of $\phi$ by $\Phi(\phi)$.
Then we have
\begin{align}
\langle\cos\left(\phi\right)\rangle_{\phi}
& \equiv \int_{-\infty}^{\infty}\cos(\phi)\Phi(\phi)\, d\phi \\
& = \int_{-\infty}^{\infty}\exp(i\phi)\Phi(\phi)\, d\phi\qquad\textrm{because }\Phi\textrm{ is even} \\
& = \langle\exp(i\phi)\rangle_{\phi} \, .
\end{align}
To proceed we need to develop a little math identity.

Consider a variable $x$ with Gaussian distribution.
Then averages of function of $x$ are given by
\begin{equation}
\langle f(x)\rangle_{x}=\frac{\int f(x)\exp(-\frac{1}{2}ax^{2})\, dx}{\int\exp(-\frac{1}{2}ax^{2})\, dx}
\end{equation}
It's easy to show that $\langle x^{2}\rangle_{x}=1/a$.
It is also relatively easy to show, using elementary integrals, that $\langle\exp(Ax)\rangle_{x}=\exp(\frac{1}{2}A^{2}\frac{1}{a})$.
Therefore,\begin{equation}
\langle \exp(Ax)\rangle_{x}
= \exp\left(\frac{1}{2}A^{2}\frac{1}{a}\right)
= \exp \left( \frac{1}{2}\langle(Ax)^{2}\rangle_{x} \right) \, .
\end{equation}

Coming now back to the main calculation, we have
\begin{equation}
\langle\cos\left(\phi\right)\rangle_{\phi}
= \langle\exp(i\phi)\rangle_{\phi}
= \exp\left(-\frac{1}{2}\langle\phi^{2}\rangle_{\phi}\right) \, .
\end{equation}
Then, using the integral expression for $\phi$ from above we can
get $\langle\phi^{2}\rangle_{\phi}$ as follows:
\begin{align}
\langle \phi^2 \rangle_{\phi}
& = \left( \frac{\partial \omega_{01}}{\partial\lambda} \right)^2 \left\langle \int_{0}^{t}\lambda(t')\, dt' \, \int_0^t \lambda(t'') \, dt'' \right\rangle _{\textrm{experiments}} \\
& = \left( \frac{\partial\omega_{01}} {\partial\lambda} \right)^2 \int_0^t dt' \int_0^t dt'' \langle\lambda(t') \lambda(t'') \rangle_{\textrm{experiments}} \\
& = \left(\frac{\partial\omega_{01}} {\partial\lambda}\right)^2 \int_0^t dt' \int_0^t dt'' \int_0^{\infty} S_{\lambda}(f) \cos \left( 2\pi f(t''-t') \right) \, df \\
& = \left(\frac{\partial\omega_{01}}{\partial\lambda} \right)^2 \int_0^{\infty} S_{\lambda}(f) \frac{\sin\left(\pi ft\right)^{2}}{\left(\pi f\right)^2} \, df \\
\langle\phi^2 \rangle_{\phi} & = \left(\frac{\partial\omega_{01}}{\partial\lambda} \right)^2 t^2 \int_0^{\infty} S_{\lambda}(f) \textrm{sinc} \left(ft\right)^2 \, df
\end{align}
where $\textrm{sinc}(x)\equiv\sin(\pi x)/\pi x$.
Putting it all together we get
\begin{equation}
p(t) = \frac{1}{2}+\frac{1}{2}\exp\left(-\frac{1}{2}\left(\frac{\partial\omega_{01}}{\partial\lambda}\right)^{2}t^{2}\int_{0}^{\infty}S_{\lambda}(f)\textrm{sinc}(ft)^{2}\, df\right)
\end{equation}
or writing $z=ft$
\begin{equation}
p(t) = \frac{1}{2} + \frac{1}{2} \exp \left( -\frac{1}{2} \left( \frac{\partial\omega_{01}}{\partial\lambda} \right)^2 t \int_0^\infty S_{\lambda} \left( z/t \right) \textrm{sinc}(z)^2 \, dz \right) \, .
\end{equation}


\subsection{White noise spectrum}

Consider the case $S_{\lambda}(f) = S_{\lambda}$ over a frequency range $f_{min}$ to $f_{max}$.
We then have
\begin{equation}
p(t) = \frac{1}{2} + \frac{1}{2}\exp \left( -\frac{1}{2}\left( \frac{d\omega_{10}}{d\lambda} \right)^2 t S_{\lambda} \int_{f_{min}t}^{f_{max}t} \left( \frac{\sin(\pi z)} {\pi z} \right)^2 dz \right) \, .
\end{equation}
Spectral densities almost always decrease with increasing frequency, so it's safe for the purposes of real computations to deal only with the case where we take $f_{\textrm{min}}=0$.
In this case the integral can be approximated as
\begin{equation}
\int_0^{f_{\textrm{max}}t} \left( \frac{\sin(\pi z)}{\pi z} \right)^2 dz \approx \frac{1}{2+(f_{\textrm{max}}t)^{-1}} \, .
\end{equation}
The approximation has a maximum error of 20\% occuring near $f_{\textrm{max}}t = 1$. 
Plugging this in, the argument of the exponential in the decay function becomes \begin{equation}
-\frac{1}{2} \left( \frac{d\omega_{01}}{d\lambda} \right)^2 t S_{\lambda} \frac{1}{2+(f_{\textrm{max}}t)^{-1}} \, .
\end{equation}
To precisely predict the shape of the decay curve one would have to include this full time dependence 


\subsection{1/f noise spectrum}

Now consider the case where $S_{\lambda}(f)=S^{*}/f$.
The integral we have to do is then
\begin{equation}
\int_{z=z_{min}}^{z_{max}}S^{*}\frac{t}{z}\textrm{sinc}^{2}\left(z\right)dz=tS^{*} \times I
\end{equation}
where $I$ is the integral over frequency,
\begin{equation}
I = \int_{z=z_{\textrm{min}}}^{z_{\textrm{max}}} \textrm{sinc}(z)^{2}\frac{dz}{z} \, .
\end{equation}
Making the substitution $x=\log z$ we get
\begin{equation}
I=\int_{x_{min}}^{x_{max}}\textrm{sinc}^{2}(e^{x})\, dx \, .
\end{equation}
Note that $x_{min}$ is a large negative number because $z_{min}\ll1$.
We make one last change, taking $x\rightarrow-x$ we get
\begin{align}
I & = \int_{-x_{max}}^{-x_{min}}\textrm{sinc}^{2}\left(e^{-x}\right)\, dx\\
& = \int_{-x_{max}}^{0}\textrm{sinc}^{2}\left(e^{-x}\right)\, dx+\int_{0}^{-x_{min}}\textrm{sinc}^{2}\left(e^{-x}\right) \, dx \, .
\end{align}
Let's look at the first integral.
The question is: what value do we take for $x_{max}$?
We can do the integral explicitly for the case $x_{max}=\infty$,
\begin{equation}
\int_{-\infty}^{0}\textrm{sinc}^{2}\left(e^{-x}\right)\, dx=-Ci(2\pi)=0.0226
\end{equation}
where $Ci(x)\equiv-\int_{x}^{\infty}\cos(x')/x'\, dx'$.
This corresponds to the case where the upper cutoff frequency of the 1/f spectrum is high compared to the times in the experiment.
We work in this limit for now.

The second integral is\begin{align*}
\int_{0}^{N}\textrm{sinc}^{2}\left(e^{-x}\right)\, dx & =
-Ci(2\pi e^{-x})|_{0}^{N} \\
&+ \frac{1}{4\pi^{2}}e^{2x}\left[1-\cos\left(2\pi e^{-x}\right)\right]|_{0}^{N} \\
&+ \frac{1}{2\pi}e^{x}\sin\left(2\pi e^{-x}\right)|_{0}^{N}
\end{align*}
 where $N\equiv-x_{min}=-\log(f_{min}t)\gg1$. We have to look at
each of these three terms for $x=0$ and as $x=N\rightarrow\infty$. 

\begin{itemize}

\item $-Ci(2\pi e^{-x})$
\begin{itemize}
\item $x=0$: $-Ci(2\pi)$ 
\item $x=N\rightarrow\infty$: $-\left[\gamma+\log\left[2\pi e^{-N}\right]+O^{2}(e^{-N})\right]$
where $\gamma$ is Euler's constant. 
\end{itemize}

\item $\frac{1}{4\pi^{2}}e^{2x}\left(1-\cos(2\pi e^{-x})\right)$
\begin{itemize}
\item $x=0$: The terms clearly goes to 0. 
\item $x=N\rightarrow\infty$: Then we get
\begin{align*}
\frac{1}{4\pi^{2}}e^{2x}\left[1-\cos\left(2\pi e^{-x}\right)\right]
& = \frac{1}{4\pi^{2}}e^{2N}\left[1-\left(1-\frac{1}{2}4\pi^{2}e^{-2N}+O^{4}(e^{-N})\right)\right] \\
&= \frac{1}{2}+O^{4}(e^{-N})
\end{align*}
\end{itemize}

\item $\frac{1}{2\pi}\sin\left(2\pi e^{-x}\right)$
\begin{itemize}
\item $x=0$: This terms also goes to 0. 
\item $x=N\rightarrow\infty$: Expanding again,
\begin{align*}
\frac{1}{2\pi}\sin\left(2\pi e^{-N}\right) &= \frac{1}{2\pi}\left(2\pi e^{-N}+O^{3}(e^{-N})\right) \\
& = 1+O^{3}(e^{-N})
\end{align*}
\end{itemize}
\end{itemize}
Putting this all together we get for the second integral,
\begin{equation}
\int_{0}^{-x_{min}}\textrm{sinc}^{2}\left(e^{-x}\right)\, dx=\frac{3}{2}-\gamma-\log\left(2\pi\right)-\log\left(e^{-N}\right)-\left[-Ci\left(2\pi\right)\right] \, .
\end{equation}
Adding this to the first integral cancels the $Ci(2\pi)$ term and
we're left with,
\begin{align}
I & = \frac{3}{2}-\gamma-\log\left(2\pi\right) + N \\
\textrm{substitute } N = -\log\left(f_{min}t\right) \qquad
& = \frac{3}{2}-\gamma-\log\left(2\pi\right)-\log\left(f_{min}t\right) \\
& = \log\left(\frac{\exp(\frac{3}{2}-\gamma)/2\pi}{f_{min}t}\right)\\
& = \log\left(\frac{0.4005}{f_{min}t}\right)
\end{align}
This matches the result obtained numerically in John's bias noise
paper.
Putting everything together the Ramsey decay envelope is
\begin{equation}
p(t) = \frac{1}{2} + \frac{1}{2} \exp \left( -\frac{1}{2} \left( \frac{\partial\omega_{01}}{\partial\lambda} \right)^2 t^2 S^* \ln\left(\frac{0.4005}{f_{\textrm{min}} t} \right) \right) \, .
\end{equation}
We define $T_2$ by the equation $p(t)\propto\exp\left[-(t/T_{2})^{2}\right]$.
The log factor is in the neighborhood of 24 for values of $t$ encountered
in real experiments.
Therefore $T_2$ is given by
\begin{equation}
T_2 = \left( \frac{\partial\omega_{10}}{\partial\lambda} \right)^{-1} \frac{1}{\sqrt{12S^{*}}} = \left( \frac{\partial\omega_{10}}{\partial\lambda} \right)^{-1} S^{*-1/2} * 0.289 \, .
\end{equation}


\section{Spin Echo}

\subsection{General noise spectral density}

The spin echo sequence works similarly to the Ramsey except that the
qubit experiences a $\pi$-pulse halfway through.
This means that any frequency detuning essentially switches sign halfway through the sequence.
Therefore, the total phase noise experienced during the sequence is
\begin{equation}
\phi = \frac{\partial\omega_{10}}{\partial\lambda} \left( \int_0^{t/2} \lambda(t_1) dt_1 -\int_{t/2}^t \lambda(t_2) dt_2 \right) \, .
\end{equation}
Using the same arguments as for the Ramsey sequence we can write down the measured probability at the end of the sequence
\begin{equation}
p(t) = \frac{1}{2}+\frac{1}{2}\exp\left[-\frac{1}{2}\langle\phi^{2}\rangle_{\phi}\right] \, .
\end{equation}
The computation of $\langle\phi^{2}\rangle_{\phi}$ proceeds similarly to the Ramsey case:
\begin{align}
\langle\phi^{2}\rangle_{\phi}
&= \left( \frac{\partial\omega_{10}}{\partial\lambda} \right)^2 \left\langle \left( \int_0^{t/2} \lambda(t_1) dt_1 - \int_{t/2}^t \lambda(t_2) dt_2 \right) \right. \nonumber \\
& \left. \left( \int_0^{t/2} \lambda(t_3) dt_{3} - \int_{t/2}^t \lambda(t_4) dt_4 \right) \right \rangle \\
& = \left( \frac{\partial \omega_{10}}{\partial\lambda} \right)^2
\int_{f_{\textrm{min}}}^{\infty} df S_{\lambda}(f)
\left[
\int_0^{t/2} \int_0^{t/2} dt_1 \, dt_3 \cos[2\pi f (t_1 - t_3)] \right. \nonumber \\
& + \int_{t/2}^t \int_{t/2}^t dt_{2} \, dt_{4} \cos[2\pi f (t_2-t_4)] \nonumber \\
& - 2 \left. \int_{0}^{t/2}\int_{t/2}^{t} dt'dt'' \cos[2\pi f (t'-t'')] \right] \\
& = \left( \frac{\partial\omega_{10}}{\partial\lambda} \right)^2 2 \int_{f_{\textrm{min}}}^{\infty} S_\lambda (f) \left[ \frac{\sin\left(\pi ft/2\right)^2}{\left(\pi f\right)^2} \left( 1 - \cos \left( \pi f t \right) \right) \right]df \label{eq:spinEchoRawIntegral} \\
& = \left( \frac{\partial\omega_{10}}{\partial\lambda}\right)^2 t \int_{t f_{\textrm{min}}/2}^{\infty} S_\lambda \left( \frac{2z}{t} \right) \left[ \frac{\sin\left(\pi z\right)^{2}}{\left(\pi z\right)^2} \left(1 - \cos\left(2\pi z\right)\right)\right] dz \, .
\end{align}
We note that (\ref{eq:spinEchoRawIntegral}) matches equation (35b) in John's bias noise paper.
Plugging this into the formula for $p(t)$ yields
\begin{align}
p(t)
&= \frac{1}{2} + \frac{1}{2} \exp \left( - \frac{1}{2} \left( \frac{\partial\omega_{10}}{\partial\lambda} \right)^{2} t \int_{tf_{\textrm{min}}/2}^{\infty} \right. \nonumber \\
& \left. S_{\lambda} \left( \frac{2z}{t} \right) \left[ \frac{\sin\left(\pi z\right)^{2}}{\left(\pi z\right)^{2}} \left(1 - \cos\left(2\pi z\right)\right) \right] dz \right) \, .
\end{align}


\subsection{1/f noise spectrum}

In the case of 1/f noise where $S(f)=S^{*}/f$ the integral becomes
\begin{equation}
S^{*} \frac{t}{2} \int_{t f_{\textrm{min}}/2}^{\infty} \left[ \frac{\sin\left(\pi z\right)^2}{\left(\pi z\right)^2} \left(1 - \cos\left(2\pi z\right)\right) \right] \frac{dz}{z} = \frac{1}{2} t S^{*} \times I \, .
\end{equation}
We need to evaluate the integral $I$.
Make the change of variables $x=\ln z$.
Then $dz/z=dx$ and we get
\begin{equation}
I=\int_{\ln\left(tf_{\textrm{min}}/2\right)}^{\ln\left(tf_{\textrm{max}}/2\right)}\left(\frac{\sin\left(\pi e^{x}\right)}{\left(\pi e^{x}\right)}\right)^{2}\left(1-\cos\left(2\pi e^{x}\right)\right)dx \, .
\end{equation}
We now break the integral into two pieces
\begin{align}
I &=
\underbrace{\int_{\ln\left(tf_{\textrm{min}}/2\right)}^{0}\left(\frac{\sin\left(\pi e^{x}\right)}{\left(\pi e^{x}\right)}\right)^{2}\left(1-\cos\left(2\pi e^{x}\right)\right)dx}_{I_{1}} \nonumber \\
&+ \underbrace{\int_{0}^{\ln\left(tf_{\textrm{max}}/2\right)}\left(\frac{\sin\left(\pi e^{x}\right)}{\left(\pi e^{x}\right)}\right)^{2}\left(1-\cos\left(2\pi e^{x}\right)\right)dx}_{I_{2}} \, .
\end{align}
The second term $I_{2}$ is very small even if the upper limit of integration is extended to infinity.
Numerically we find $I_{2}\approx0.032$.
The first term requires only slightly more attention.
Taking the experiment times $t$ to be in the range 10ns-1$\mu$s, and the total time of the experiment to be on the order of minutes or an hour\footnote{This is the total time over which all points in the experiment are averaged}, we find $\ln(tf_{\textrm{min}}/2)<-18$.
Inspection of a plot of the integrand shows that this is sufficiently far to the left of zero that the integral is completely insensitive to the value of this lower limit.
We therefore numerically integrate in $\textit{Mathematica}$ from $-10$ to with result
\begin{equation}
I_1 \approx 1.35
\end{equation}
The Spin-Echo decay curve for 1/f noise is therefore
\begin{equation}
p(t) = \frac{1}{2} + \frac{1}{2} \exp \left(-\frac{1}{4} \left( \frac{\partial\omega_{10}}{\partial\lambda} \right)^2 t^2 S^{*} * 1.38 \right) \, .
\end{equation}
This again has a Gaussian envelope but now with a different decay constant.
Defining the decay rate by the equation
\begin{equation}
p(t) \propto \exp \left[ - \left( t / T_{\textrm{echo}} \right)^2 \right]
\end{equation}
we find
\begin{equation}
T_{\textrm{echo}}
= \left( \frac{\partial\omega_{10}}{\partial\lambda} \right)^{-1} \frac{2}{\sqrt{1.35*S^{*}}}
= \left( \frac{\partial\omega_{10}}{\partial\lambda} \right)^{-1} S^{*-1/2} * 1.72 \, .
\end{equation}
The echo time is therefore calculated to be approximately $1.72/0.289=6$ times larger than the Ramsey decay times.


\section{Notes}

Note that the white spectral density leads to exponential decay while $1/f$ noise leads to Gaussian decay.


\section{Dimensionless versions}

Consider noise which scales as $1/f^{\alpha}$.
The Ramsey integral can be written as
\begin{equation}
\langle \phi^2 \rangle_{\textrm{Ramsey}}
= t^{1+\alpha} \left( \frac{\partial\omega_{10}} {\partial\lambda} \right)^2 S^*_{\lambda} \int_{-\ln f_{\textrm{max}}t}^{-\ln f_{\textrm{min}}t} \left(\frac{\sin(\pi e^{-x})}{\pi e^{-x}}\right)^2 e^{x(\alpha-1)} dx \, .
\end{equation}

The spin-echo integral can be written as
\begin{align}
\langle \phi^2 \rangle_{\textrm{Echo}} =
& t^{1+\alpha} \left( \frac{\partial\omega_{10}} {\partial\lambda} \right)^2 S^*_{\lambda} \nonumber \\
& \int_{-\ln f_{\textrm{max}}t/2}^{-\ln f_{\textrm{min}}t/2} \frac{1}{2^{\alpha}} \left( \frac{\sin(\pi e^{-x})}{\pi e^{-x}}\right)^2 (1-\cos[2\pi e^{-x}]) e^{x(\alpha-1)} dx
\end{align}

\end{document}

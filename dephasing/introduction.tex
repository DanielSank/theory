The various methods are divided into two basic types.
Methods of the first type use control pulse sequences such that the probability to find the qubit in the excited state at the end of the sequence is a function of the length of the sequence and of a particular integral of the noise spectral density.
In general, if the qubit is subjected to a particular pulse sequence like a spin-echo or Rabi sequence, the probability of finding the qubit in $\ket{1}$ is
\begin{equation}
p_{\ket{1}}(t) = \frac{1}{2} \left( 1 + \exp \left[ -\frac{1}{2} \avg{\phi^2}(t) \right] \right)
\end{equation} 
where
\begin{equation}
\avg{\phi^2 (t)} \equiv \left( \frac{d \omega_q}{d \lambda} \right)^2 t \int_{f_\text{min} t}^{f_\text{max} t} S_\lambda (z/t) W(f,t) \, dz
\end{equation}
and $W(f, t)$ is a weight function which depends on the pulse sequence.
For example, for the Rabi sequence with a Rabi oscillation frequency of $f_\text{Rabi}$ we have
\begin{equation}
W_\text{Rabi}(f,t) = \left( \frac{z / (f_\text{Rabi} t)}{1 - \left(z / f_\text{Rabi} t \right)^2} \right) \left( \frac{\sin(\pi z)}{\pi z} \right)^2 \, .
\end{equation}

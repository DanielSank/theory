\section{Quantum noise with classical intuition}

In a wide class of problems, notably circuits coupled with noisy resistors as we'll see shortly,\footnote{See Reference \cite{Nyquist:noise1928} for an excellent explanation of why resistors produce noise according to the Bose-Einstein distribution.} the noise spectrum and Hamiltonian have forms such that
\begin{equation}
  g^2 S_{FF}(\omega) = A^2 \omega (\nboseeinstein(\omega, T) + 1) \label{eq:noise_standard_form}
\end{equation}
where here $A^2$ is a dimensionless constant describing the strength of the noise multiplied by the coupling between the noise and the system of interest, and $\nboseeinstein$ is the \href{https://en.wikipedia.org/wiki/Bose\%E2\%80\%93Einstein_statistics}{Bose-Einstein distribution} for the number of quanta in a mode at frequency $\omega$ and temperature $T$
\begin{align}
  \nboseeinstein(\omega, T)
  &= \frac{1}{\exp(\hbar \omega / \boltzmann T) - 1} \\
  & = \frac{1}{2} \left[ \coth \left( \frac{\beta \hbar \omega}{2} \right) - 1 \right] \\
  (\boltzmann T \gg \hbar \omega) \quad & \approx \frac{\boltzmann T}{\hbar \omega} - \frac{1}{2}
  \, .
\end{align}
In this case, the up and down rates are
\begin{align}
    \gammaup & = A^2 \omegaqubit ~ \nboseeinstein(\omegaqubit, T) \\
    \gammadown &= A^2 \omegaqubit ~ (\nboseeinstein(\omegaqubit, T) + 1)
    \, .
\end{align}
These equations express the up and down rates in terms of physical parameters of the system.
Interestingly, we can express the rates in terms of each other.
Notice that $\gammadown = \gammaup + A^2 \omegaqubit$ so we can interpret $\gammadown$ as having two contributions.
The first contribution is the ``noise'' contribution $A^2 \omegaqubit ~ \nboseeinstein(\omegaqubit, T)$, which goes to zero at zero temperature and contributes equally to $\gammaup$ and $\gammadown$.
The second contribution is ``spontaneous emission'' contribution $A^2 \omegaqubit$ which contributes only to $\gammadown$ and is independent of temperature.
Renaming the common noise contribution as $\gammaupdown$ and the loss contribution as $\gammaemission$, we can re-express the up and down rates as
\begin{align}
  \gammaup &= \gammaupdown \\
  \gammadown &= \gammaupdown + \gammaemission \\
  \text{where} \qquad
  \gammaupdown &= \gammaemission ~ \nboseeinstein(\omegaqubit, T) \label{eq:fluctuation_dissipation} \\
  \text{and} \qquad
  \gammaemission &= A^2 \omegaqubit
  \, .
\end{align}
The value in expressing the rates this way is that it admits an intuitive interpretation.
It makes sense that ``noise'' should drive the system up and down with equal strength, while ``loss'' should lead only to transitions down.
Furthermore, it's sensible that ``noise'' should depend on temperature while ``loss'' should not.
The two rates $\gammaupdown$ and $\gammaemission$ correspond respectively to stimulated emission/absorption and spontaneous emission.

Equation (\ref{eq:fluctuation_dissipation}) is an expression of the fluctuation dissipation law\footnote{More common terminology would be ``the fluctuation dissipation theorem''. However, Michel Devoret has rightly pointed out that a theorem is an inevitable logical consequence of a set of axioms, while a law is a fundamental pattern in physics, and what is usually called the fluctuation dissipation theorem is an example of the latter.} relating the motion of a system due to contact with a noisy environment to the loss of energy from the system into that environment.
Another useful expression of Eq.~(\ref{eq:fluctuation_dissipation}) is
\begin{equation}
  \gammaupdown = \frac{\omegaqubit}{\qualityfactoremission} ~ \nboseeinstein(\omegaqubit, T)
  \label{eq:fluctuation_dissipation_quality_factor}
\end{equation}
where we've introduced the quality factor $\qualityfactoremission = \omegaqubit / \gammaemission$.
Note that $A^2 = 1 / \qualityfactoremission$.

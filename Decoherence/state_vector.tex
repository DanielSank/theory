\subsection{State vector approach}

Suppose the system begins in a pure state
\begin{displaymath}
\ket{\Psi(0)} = \ket{m} \otimes \ket{\phi(0)}
\end{displaymath}
where $\ket{m}$ is the initial state of the environment and $\ket{\phi(0)}$ is the initial state of the qubit.
Schrodinger's equation of motion is formally solved by the implicit equation
\begin{equation}
\ket{\Psi(t)} = \ket{\Psi(0)} - \frac{i}{\hbar} \int_0^t \tilde{V}(t') \ket{\Psi(t')} \, dt' \, .
\end{equation}
Plugging the expression for $\ket{\Psi(t)}$ into the right hand side and keeping only terms first order in $\int dt' \, \tilde{V}(t') / \hbar$ gives
\begin{equation}
\ket{\Psi(t)} = \ket{\Psi(0)} - \frac{i}{\hbar} \int_0^t \tilde{V}(t') \ket{\Psi(0)} \, dt' \, .
\end{equation}
We now suppose the system starts in the ground state, i.e. $\ket{\psi(0)}=\ket{g}$ and compute the amplitude to be in the excited state $\ket{e}$:
\begin{align}
\ket{E(t)}\braket{e}{\phi(t)}
&= -\frac{i}{\hbar} \int_0^t \bbraket{e}{\tilde{V}(t')}{\Psi(0)} \, dt' \\
&= - ig \int_0^t F(t')\ket{m} \otimes \bbraket{e}{e^{i \Omega t'} \sigma_+ + e^{-i \Omega t'} \sigma_-}{g} \, dt' \\
&= - ig \int_0^t F(t')\ket{m} e^{i \Omega t'} \, dt' \, .
\end{align}

So far we have considered only a single initial environment state.
In order to average over the states of the environment we form the density operator (already conditioned such that the qubit is in the excited state):
\begin{equation}
\rho(t) = g^2 \int_0^t \int_0^t \, dt' \, dt'' \, F(t')\ket{m}\bra{m}F(t'') e^{i \Omega (t'-t'')} \, .
\end{equation}
We actually only care about the state of the qubit, so we trace over the states of the environment.
We also now average over the possible initial states of the environment by the replacement
\begin{displaymath}
\ket{m}\bra{m} \rightarrow \sum_m \rho_{mm} \ket{m}\bra{m}
\end{displaymath}
which gives us
\begin{align}
p_e(t)
&= g^2 \sum_{nm} \int_0^t \int_0^t \, dt' \, dt'' \, \rho_{mm} \bbraket{n}{F(t')}{m} \bbraket{m}{F(t'')}{n} e^{i \Omega (t' - t'')} \\
&= g^2 \sum_{nm} \int_0^t \int_0^t \, dt' \, dt'' \, \rho_{mm} \bbraket{m}{F(t'')}{n} \bbraket{n}{F(t')}{m} e^{i \Omega (t' - t'')} \\
&= g^2 \sum_m \int_0^t \int_0^t \, dt' \, dt'' \, \rho_{mm} \bbraket{m}{F(t'') F(t')}{m} e^{i \Omega (t' - t'')} \\
&= g^2 \int_0^t \int_0^t \, dt' \, dt'' \, \langle F(t'') F(t') \rangle e^{i \Omega (t' - t'')} \, .
\end{align}
Changing variables to $\tau \equiv t'' - t'$ and assuming the average of $F$ is stationary gives
\begin{equation}
p_e(t)
= g^2 \int_0^t \int_{-t'}^{t-t'} \, dt' \, d\tau \, \langle F(\tau) F(0) \rangle e^{-i \Omega \tau} \, .
\end{equation}
If the correlation function of $F$ is very short then we can extend the limits of integration of $\tau$ to positive and negative infinity.
Doing this and defining the spectral density as
\begin{equation}
S_{FF}(\Omega) \equiv \int_{-\infty}^\infty \, dt \, \langle F(t) F(0) \rangle e^{i \Omega t}
\end{equation}
we find
\begin{align}
p_e(t) 
&= g^2 \int_0^t S_{FF}(-\Omega) \\
\Gamma_{\uparrow} = \frac{dp_e}{dt} &= g^2 S_{FF}(-\Omega) \, .
\end{align}
Similar arguments can be used to find
\begin{equation}
\Gamma_\downarrow = g^2 S_{FF}(\Omega) \, . \label{eq:ch.decoherence.sec.noise_from_a_quantum_environment:gamma_down}
\end{equation}

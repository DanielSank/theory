\subsection{Density matrix approach}

The difference here is that we use density matrices in order to handle the environment, and we will have to work to second order in the interaction.

We assume the environment starts in a state which is diagonal in its own energy basis:
\begin{equation}
\rho_E(0) = \sum_n \rho_{nn} \ket{n}\bra{n}
\end{equation}
and we assume that $S$ begins in the ground state
\begin{equation}
\rho_S(0) = \ket{0}\bra{0}
\end{equation}
giving an initial state
\begin{equation}
\rho(0) = \sum_n \rho_{nn} \ket{n}\bra{n} \otimes \ket{0}\bra{0} \, .
\end{equation}
The equation of motion for the density matrix, in the rotating frame, is
\begin{equation}
i \hbar \dot{\rho} = \left[ \tilde{V}(t), \rho(t) \right]
\end{equation}
This equation is formally solved as follows:
\begin{align}
\rho(t)
&= \rho(0) - \frac{i}{\hbar}
\int_0^t dt' \, \left[ \tilde{V}(t'), \rho(t') \right] \\
&= \rho(0) - \frac{i}{\hbar}
\int_0^t dt' \, \left[ \tilde{V}(t'), \rho(0) - \frac{i}{\hbar} \int_0^{t'} dt'' \, \left[ \tilde{V}(t''), \rho(t'') \right] \right] \\
&= \rho(0) - \frac{i}{\hbar} \int_0^{t} dt' \, \left[ \tilde{V}(t'), \rho(0) \right]
- \frac{1}{\hbar^2} \int_0^{t} \int_0^{t'} dt' \, dt'' \nonumber \\
& \qquad \left[ \tilde{V}(t'), \left[ \tilde{V}(t''), \rho(t'') \right] \right] \, .
\end{align}
Note that this equation is exact.
In order to compute a transition rate we differentiate with respect to $t$:
\begin{equation}
\dot{\rho}(t)
= -\frac{i}{\hbar} \left[ \tilde{V}(t), \rho(0) \right]
- \frac{1}{\hbar^2} \int_0^t dt' \, \left[ \tilde{V}(t), \left[ \tilde{V}(t'), \rho(t') \right] \right] \, .
\end{equation}
As we are interested in the transition rates of the two level system we trace over the environment:
\begin{align}
\dot{\rho}_S(t) = 
\Tr_E \dot{\rho}(t) =
&-\frac{i}{\hbar} \underbrace{ \Tr_E \left[ \tilde{V}(t), \rho(0) \right] }_{A} \nonumber \\
&- \frac{1}{\hbar^2} \underbrace{ \int_0^t dt' \, \Tr_E \left[ \tilde{V}(t), \left[ \tilde{V}(t'), \rho(t') \right] \right]}_{B} \, .
\end{align}


\subsubsection{Term $A$}

This term is actually identically zero as long as $F$ has no diagonal terms.
This is the case for most coupling operators which are usually something like the $x \propto a + a^\dagger$ operator of a harmonic oscillator.
To see this we just write everything out
\begin{align}
A
&= \sum_{n,m} \rho_{mm} \bra{n} \left[ \tilde{V}(t), \ket{m}\bra{m}\otimes \rho_S(0) \right] \ket{n} \\
= \sum_{n,m} &\rho_{mm}
\bbraket{n}{T^\dagger_E(t) F T_E(t)}{m}\braket{m}{n}
\otimes \tilde{V}_S(t) \rho_S(0) \nonumber \\
- &\rho_{mm} \braket{n}{m} \bbraket{m}{T_E^\dagger(t) F T_E(t)}{n} \otimes \rho_S(0) \tilde{V}_S(t) \\
= \sum_n &\rho_{nn} \bbraket{n}{T_E^\dagger(t) F T_E(t)}{n} \otimes \tilde{V}_S(t) \rho_S(0) \nonumber \\
- &\rho_{nn} \bbraket{n}{T_E^\dagger(t) F T_E(t)}{n} \otimes \rho_S(0) \tilde{V}_S(t) \\
= \sum_n &\rho_{nn} F_{nn} \left[ \tilde{V}_S(t), \rho_S(0) \right] \, .
\end{align}
Assuming $F_{nn}=0$ we have $A = 0$.


\subsubsection{Term $B$}

\begin{align}
B
&= \int_0^t \Tr_E \left[ \tilde{V}(t), \left[ \tilde{V}(t-t'), \rho(t-t') \right] \right] \\
&= \hbar g \int_0^t \Tr_E \left[ \tilde{V}(t),
\left[ F(t-t') \right. \right. \nonumber \\
& \left. \left. \qquad \otimes (e^{i \Omega (t-t')} \sigma_+ + e^{-i \Omega (t-t')} \sigma_-), 
\rho_E \otimes \ket{0}\bra{0} ) \right] \right] \\
&= \hbar g \int_0^t \Tr_E \left[ \tilde{V}(t),
\left( F(t-t')\rho_E \otimes e^{i \Omega (t-t')} \ket{1}\bra{0} \right) \right. \nonumber \\
& \left. \qquad - \left( \rho_E F(t-t') \otimes e^{-i \Omega (t-t')} \ket{0}\bra{1} \right)
\right] \, . 
\end{align}
Now we expand the second commutator, keeping only terms which put the system in $\ket{1}\bra{1}$.
\begin{align}
B_{11}
&= - (\hbar g)^2 \int_0^t dt' \, \Tr_E \left(
F(t) \rho_E F(t-t') e^{i \Omega t'} \right. \nonumber \\
& \left. \qquad + F(t-t') \rho_e F(t) e^{-i \Omega t'}
\right) \otimes \ket{1}\bra{1} \, .
\end{align}
Next we insert the diagonal form of $\rho_E$ and expand the $Tr_E$ operation
\begin{align}
B_{11}
&= - (\hbar g)^2 \sum_{m,n} \rho_{nn} \int_0^t dt' \,
\left(
\bbraket{m}{F(t)}{n} \bbraket{n}{F(t-t')}{m}e^{i \Omega t'} \right. \nonumber \\
& \left. \qquad \qquad + \bbraket{m}{F(t-t')}{n} \bbraket{n}{F(t)}{m} e^{-i \Omega t'}
\right) \ket{1}\bra{1} \\
&= - (\hbar g)^2 \sum_n \rho_{nn} \int_0^t dt' \, \left(
\bbraket{n}{F(t-t')F(t)}{n} e^{i \Omega t'} \right. \nonumber \\
& \left. \qquad \qquad + \bbraket{n}{F(t)F(t-t')}{n} e^{-i \Omega t'}
\right) \ket{1}\bra{1} \\
&= - (\hbar g)^2 \int_0^t dt' \, \left(
\langle F(t-t') F(t) \rangle_E e^{i \Omega t'} \right. \nonumber \\
& \left. \qquad \qquad \langle F(t) F(t-t') \rangle_E e^{-i \Omega t'}
\right) \ket{1}\bra{1} \, .
\end{align}
If we now assume that the quantum-thermal averages over the bath are stationary in time, we can rewrite \mbox{$\langle F(t) F(t-t') \rangle_E$} as \mbox{$\langle F(t')F(0) \rangle_E$}.
Doing this and switching the sign of the integration variable in the first term gives
\begin{equation}
B_{11} = - 2 (\hbar g)^2 \int_0^t \left(
\langle F(t')F(0) \rangle_E e^{-i \Omega t'} \right)
\ket{1}\bra{1} \, .
\end{equation}


\subsubsection{Transition rate}

Plugging $B$ into the expression for $\dot{\rho}$ we get
\begin{equation}
\dot{\rho}_{S,11} =
2 g^2 \int_0^t dt' \, \langle F(t') F(0) \rangle_E \, e^{-i \Omega t'} \, .
\end{equation}
If the correlation time of $F$ is very short compared to $t$ we can extending the limits of integration to positive infinity.
Also extending to negative infinity requires multiplying by $1/2$.
Doing this and defining the spectral density as
\begin{equation}
S_{FF}(\omega) \equiv \int_{-\infty}^\infty dt \, \langle F(t) F(0) \rangle
\end{equation}
we get
\begin{equation}
\dot{\rho}_{S,11} = g^2 S_{FF}(-\Omega) \, .
\end{equation}
This is interpreted as a transition rate, giving us the often cited formula
\begin{equation}
\Gamma_\uparrow = g^2 S_{FF}(-\Omega) \, .
\end{equation}
Entirely similar arguments show
\begin{equation}
\Gamma_\downarrow = g^2 S_{FF}(\Omega) \, .
\end{equation}
This calculation shows that the positive and negative frequency parts of the spectral density of the environment correspond to absorption and emission of a quantum of energy by the environment from the qubit.

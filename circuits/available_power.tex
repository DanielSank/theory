\documentclass[twocolumn]{article}

% General physics constructs
\newcommand{\bra}[1]{\langle #1 |}
\newcommand{\ket}[1]{| #1 \rangle }
\newcommand{\braket}[2]{\langle #1|#2\rangle}
\newcommand{\bbraket}[3]{ \langle #1 | #2 | #3 \rangle }
\newcommand{\norm}[1]{\| #1\|}
\newcommand{\avg}[1]{\left \langle #1 \right \rangle}
\newcommand{\angavg}[1]{\left \langle #1 \right \rangle}
\newcommand{\abs}[1]{\left \lvert #1 \right \rvert}
\newcommand{\VS}{\textit{\textbf{V}}}
\newcommand{\Tr}{\textrm{Tr}}
\renewcommand{\Re}{\textrm{Re}}
\renewcommand{\Im}{\textrm{Im}}
\newcommand{\basis}[1]{\{\ket{#1}\}}

\newcommand{\omegaqubit}{\omega_{10}}

% Figures. Example usage:
% \quickfig{\columnwidth}{my_image}{This is the caption}{fig:my_fig}
\DeclareRobustCommand{\quickfig}[4]{
\begin{figure}
\begin{centering}
\includegraphics[width=#1]{#2}
\par\end{centering}
\caption{#3}
\label{#4}
\end{figure}
}

\DeclareRobustCommand{\quickwidefig}[4]{
\begin{figure*}[h]
\begin{centering}
\includegraphics[width=#1]{#2}
\par\end{centering}
\caption{#3}
\label{#4}
\end{figure*}
}

%Packages
\usepackage{amsmath}
\usepackage{amstext}
\usepackage{amssymb}
\usepackage{appendix}
\usepackage{coseoul}
\usepackage{enumerate}
\usepackage{graphicx}
\usepackage{import}
\usepackage{lscape}
\usepackage{modular}

\usepackage[pdfpagemode=UseNone,pdfstartview=FitH,colorlinks=true,linkcolor=blue,citecolor=blue,urlcolor=blue]{hyperref}
\usepackage[all]{hypcap}


\newcommand{\citeinternaltype}{Ref.\,}
\newcommand{\Citeinternaltype}{Reference}
\newcommand{\citeinternalref}[1]{\cite{Sank:#1}}

\usepackage{tikz}
\usepackage{circuitikz}
\allowdisplaybreaks

\title{Voltage at a mismatched load:\\
Thinking in terms of available power}
\author{Daniel Sank \\
\small Google Quantum AI}
\date{23 May 2015}

\begin{document}

\maketitle

\section{introduction}
In this note, we relate the voltage on a mismatched load to the available power of an attenuated voltage source.
\textbf{All voltages in this document are RMS}.

\section{Lumped source with matched load}

Consider, as shown in Fig.\,\ref{fig:lumped_matched_load}, a voltage source $V_s$ with output impedance $Z_0$ connected to a matched resistor $R_l=Z_0$.
The voltage measured across the load is $V_l = V_s/2$.
The power delivered to this matched load, called the ``available power'' $P_a$, is
\begin{equation}
P_a = V_l^2 / Z_0 = (V_s/2)^2 / Z_0 \, .
\end{equation}
The term ``available power'' indicates that this is the largest amount of power which can be extracted from the source.\footnote{It is a simple exercise to prove that a voltage source delivers the maximum power when connected to a matched load.}
Most importantly, $P_a$ is the power you would measure at room temperature with a spectrum analyzer.

Similarly, $V_l$ here is the voltage you would measure with a matched voltage probe (oscilloscope).
Therefore, we denote $V_l$ for the matched load case by the symbol $V_{\text{meas}}$.
Note that $P_a = V_{\text{meas}}^2 / Z_0$.

\begin{figure}
\begin{centering}
\begin{tikzpicture}
\draw (0,0)
to[V,v=$V_s$] (0,2)
to[R=$Z_0$] (0,4)
to[short,-*] (3,4) node[label={above:$V_l=V_s/2$}]{}
to[R=${R_l=Z_0}$] (3,2)
to[short] (3,0)
to[short] (0,0);
\end{tikzpicture}
\par\end{centering}
\caption{Voltage source with a matched load.}
\label{fig:lumped_matched_load}
\end{figure}

\section{Transmission line source with matched load}

In this section we connect the load voltage to the travelling wave voltage.
If the source is connected to a transmission line with characteristic impedance $Z_0$, as shown in Fig.\,\ref{fig:tline_matched_load}, then the output impedance of the source plus line is still $Z_0$.
Suppose we connect a load $R_l=Z_0$.
By voltage division, the voltage at the load is $V_l = V_s/2$.
The power dissipated into the load is therefore
\begin{equation}
P_l = V_l^2/Z_0 = (V_s/2)^2 / Z_0 \, .
\end{equation}

Because the reflection coefficient $\Gamma$ is zero,
\begin{equation}
\Gamma = \frac{R_l - Z_0}{R_l + Z_0} = 0 \, ,
\end{equation}
the load power $P_l$ must be equal the input power flow $P_{\text{in}}$ carried by the input travelling voltage wave with amplitude $V_{\text{in}}$.
In other words,
\begin{align}
\frac{(V_s/2)^2}{Z_0} = P_l &= P_{\text{in}} = \frac{V_{\text{in}}^2}{Z_0} \\
\text{so} \qquad V_{\text{in}} &= V_s/2 \, .
\end{align}

\begin{figure}
\begin{centering}
\begin{tikzpicture}
\draw (0,0)
to[V,v=$V_s$] (0,2)
to[R=$Z_0$] (0,4)
to[transmission line, n=tline, -*] (3,4) node[label={right:$V_m=V_s/2$}]{}
to[R=${R_l=Z_0}$] (3,2)
to[short] (3,0)
to[short] (0,0)
(tline.s) node[below]{$\begin{array}{c}V_{\text{in}}\longrightarrow \\ V_{\text{out}}\longleftarrow \end{array}$}
;
\end{tikzpicture}
\par\end{centering}
\caption{Voltage source with transmission line and matched load.}
\label{fig:tline_matched_load}
\end{figure}

\subsection{Attenuator}

If we add an attenuator with attenuation factor $A$ to the transmission line, then the power flowing into the load goes down by a factor of $A$.
The source output impedance is still $Z_0$ so the system is equivalent to the one without the attenuator as long as we scale the source voltage to $V_s'=V_s \sqrt{A}$.

\section{Arbitrary load}

Now suppose we terminate the line with an arbitrary load $Z_l$.
The voltage at the load is
\begin{align}
V_l
&= V_{\text{in}} + V_{\text{out}} \\
&= V_{\text{in}} (1 + \Gamma) \\
&= V_{\text{in}} \frac{2 Z_l}{Z_l + Z_0} \\
&= \frac{V_s'}{2} \frac{2 Z_l}{Z_l + Z_0} \\
&= V_s \sqrt{A} \frac{Z_l}{Z_l + Z_0} \\
&= 2 V_{\text{meas}} \sqrt{A} \frac{Z_l}{Z_l + Z_0} \\
&= 2 \sqrt{P_a Z_0 A} \frac{Z_l}{Z_l + Z_0} \, . \label{eq:V_l}
\end{align}

\section{Thinking in terms of available power}

There's an easy way to remember all of this by thinking only in terms of available power.
The load voltage is related to the input wave voltage by
\begin{equation}
V_l = V_{\text{in}}(1 + \Gamma) = V_{\text{in}} \frac{2 Z_l}{Z_l + Z_0} \,
\end{equation}
The input wave voltage is related to the \emph{power available at the load}, denoted $P_{a,l}$ by
\begin{equation}
P_{a,l} = V_{\text{in}}^2 / Z_0 \, .
\end{equation}
Finally, the power available at the load is related to the source power by $P_{a,l} = P_a A$.
Combining these results gives
\begin{equation}
V_l = 2 \sqrt{P_a Z_0 A} \frac{Z_l}{Z_l + Z_0}
\end{equation}
which agrees with Eq.\,(\ref{eq:V_l}).

\end{document}


% General physics constructs
\newcommand{\bra}[1]{\langle #1 |}
\newcommand{\ket}[1]{| #1 \rangle }
\newcommand{\braket}[2]{\langle #1|#2\rangle}
\newcommand{\bbraket}[3]{ \langle #1 | #2 | #3 \rangle }
\newcommand{\norm}[1]{\| #1\|}
\newcommand{\avg}[1]{\left \langle #1 \right \rangle}
\newcommand{\abs}[1]{\left \lvert #1 \right \rvert}
\newcommand{\VS}{\textit{\textbf{V}}}
\newcommand{\Tr}{\textrm{Tr}}
\renewcommand{\Re}{\textrm{Re}}
\renewcommand{\Im}{\textrm{Im}}

% Enable encapsulation of relative heading imports.
% http://tex.stackexchange.com/questions/245348/recursively-import-sub-document-at-arbitrary-level
\makeatletter
\newcounter{currentimportdepth}
\setcounter{currentimportdepth}{0}
\newcommand{\subimportlevel}[3]{
  \expandafter\edef\csname @currentlevel\thecurrentimportdepth\endcsname{\thecurrentlevel}
  \addtocounter{currentimportdepth}{1}
  \addtocounter{currentlevel}{-#3}
  \subimport*{#1}{#2}
  \addtocounter{currentimportdepth}{-1}
  \setcounter{currentlevel}{\csname  @currentlevel\thecurrentimportdepth\endcsname}
}
\makeatother

% Figures. Example usage:
% \quickfig{\columnwidth}{my_image}{This is the caption}{fig:my_fig}
\DeclareRobustCommand{\quickfig}[4]{
\begin{figure}
\begin{centering}
\includegraphics[width=#1]{#2}
\par\end{centering}
\caption{#3}
\label{#4}
\end{figure}
}

\DeclareRobustCommand{\quickwidefig}[4]{
\begin{figure*}[h]
\begin{centering}
\includegraphics[width=#1]{#2}
\par\end{centering}
\caption{#3}
\label{#4}
\end{figure*}
}

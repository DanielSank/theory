\documentclass{article}
\author{Daniel Sank}
\date{2024 August 7}
\title{The Mathematics and Physics of Music Theory}

\begin{document}

\maketitle

\section{The physics of sound}

Sounds is a wave.
Vibrating physical objects push on air, and the ensuing vibration of the air pushes on our ear drum.
The vibrating ear drum induces electrical signals into the auditory nerve, which is carried to the brain and perceived as sound.

What objects create sound?
The vibrations of a person's vocal cords create the sound we call speech.
Similarly, the vibrating vocal cords of birds create bird song.
When we play recorded music, it is the vibration of the membrane of a speaker or headphones that vibrate the air to carry that music to our ears.
In each of these cases, the sound originates from a vibrating object.
So let's study how objects vibrate.

Figure \ref{fig:guitar} illustrates a guitar, which is a musical instrument with six strings.
The bottom part of the figure focuses on a single string, and the shape it makes when it vibrates.
The three different colors show three of natural the shapes the string can make.
Each of these shapes vibrates at a single frequency, i.e. a single musical note.

\end{document}

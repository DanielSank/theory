\levelstay{Hamiltonian approach}
The Hamiltonian for two LC oscillators coupled through a coupling capacitor $C_g$ is
\begin{align}
  H
  &= \frac{\Phi_1^2}{2 L_1} + \frac{\Phi_2^2}{2 L_2} \\
  &+ \frac{Q_1^2}{2 C_1} + \frac{Q_2^2}{2 C_2} \\
  &+ \frac{Q_1 Q_2}{C_g''}
\end{align}
where
\begin{align}
  C_g'' &\equiv \frac{C_1' C_2' - C_g^2}{C_g} \\
  C_i' &\equiv C_i + C_g \, .
\end{align}
We transform the Hamiltonian in two steps.
First we define
\begin{align*}
  X_i &\equiv \frac{1}{\sqrt{2 \hbar}} \frac{1}{\sqrt{Z_i}} \Phi_i \\
  Y_i &\equiv \frac{1}{\sqrt{2 \hbar}} \sqrt{Z_i} Q_i
\end{align*}
where $Z_i \equiv \sqrt{L_i / C_i''}$.
With these new variables, the Hamiltonian is
\begin{equation}
  H/\hbar =
    \omega_1 \left(X_1^2 + Y_1^2) \right)
  + \omega_2 \left(X_2^2 + Y_2^2) \right)
  + 2 \left( \frac{C_1'' C_2''}{C_g''} \right) \sqrt{\omega_1'' \omega_2''} Y_1 Y_2 \, .
\end{equation}
Note that $[X, Y] = i/2$ so Hamilton's equations of motion are a little different than usual.
Anyway, we next use $a = X_1 + i Y_1$ and $b = X_2 + i Y_2$ to get
\begin{equation}
  H / \hbar =
    \omega_1'' a^\dagger a
  + \omega_2'' b^\dagger b
  - g (a - a^\dagger) (b - b^\dagger)
\end{equation}
where
\begin{equation}
  g \equiv \frac{1}{2} \frac{\sqrt{C_1'' C_2''}}{C_g''} \sqrt{\omega_1'' \omega_2''} \, .
\end{equation}
Hamilton's equations of motion in this representation are
\begin{equation*}
  \dot{a} = -i \frac{\partial H/\hbar}{\partial a^\dagger}
  \quad \text{and} \quad
  \dot{a}^\dagger = i \frac{\partial H/\hbar}{\partial a}
\end{equation*}
and similarly for $b$.
Therefore the equations of motion for the system are
\begin{align*}
  \dot a &= -i \omega_1'' a -i g b + i g b^\dagger \\
  \dot b &= -i \omega_2'' b -i g a + i g a^\dagger \, .
\end{align*}
These equations can be written in matrix form as
\begin{equation}
  \frac{d}{dt}
  \left( \begin{array}{c}
    a \\ b \\ a^\dagger \\ b^\dagger
  \end{array} \right)
  =
  -i \left( \begin{array}{cccc}
    \omega_1'' & g & 0 & -g \\
    g & \omega_2'' & -g & 0 \\
    0 & g & - \omega_1'' & -g \\
    g & 0 & -g & - \omega_2''
  \end{array} \right)
  \left( \begin{array}{c}
    a \\ b \\ a^\dagger \\ b^\dagger
  \end{array} \right)
  \, .
\end{equation}
The matrix can be conveniently expressed as
\begin{equation}
  \sigma_z \otimes \left(
    g \sigma_x + \frac{\Delta}{2} \sigma_z + \frac{S}{2} \mathbb{I}
  \right)
  -i g \left( \sigma_y \otimes \sigma_x \right)
\end{equation}
where $\Delta \equiv \omega_1'' - \omega_2''$ and $S \equiv \omega_1'' + \omega_2''$.
The eigenvalues are
\begin{equation}
  \lambda^2 =
  \pm \frac{1}{2}
  \left(
    \omega_1''^2 + \omega_2''^2
    \pm \sqrt{(\omega_1''^2 - \omega_2''^2)^2 + 16 g^2 \omega_1'' \omega_2''}
  \right)
\end{equation}

In the rotating wave approximation, we drop the $ab$ and $a^\dagger b^\dagger$ terms in the Hamilonian.
Dropping those terms is equivalent to dropping the $-i g \left( \sigma_y \otimes \sigma_x \right)$ term in the algebraic representation!
Doing that and computing the eigenvalues again gives
\begin{equation}
  \lambda_\pm =
  \frac{1}{2} \left( S \pm \sqrt{\Delta^2 + 4 g^2} \right) \, .
\end{equation}

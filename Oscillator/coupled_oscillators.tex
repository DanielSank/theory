Consider two LC oscillators coupled through a coupling capacitor $C_g$.
Kirchhoff's equations of motion for this circuit can be written in matrix form
as
\begin{equation}
  \left( \begin{array}{c} V_1 \\ V_2 \end{array} \right)
  =
  \left( \begin{array}{cc}
    (1 + \epsilon_1)/\omega_1^2 & - \epsilon_1  \omega_1^2 \\
    - \epsilon_2  \omega_2^2 & (1 + \epsilon_2)/\omega_2^2
  \end{array} \right)
  \left( \begin{array}{c} \ddot{V}_1 \\ \ddot{V}_2 \end{array} \right)
\end{equation}
where $\epsilon_i \equiv C_g / C_i$ and $\omega_i^2 \equiv 1 / L_i C_i$.
The normal mode frequencies of the system are the square roots of the reciprocals of the eigenvalues of the matrix.
We could compute those eigenvalues explicitly, but it's a mess and there's a better way to do it that we'll show below.
However, in the particular case that $L_1 = L_2$ and $C_1 = C_2$, then $\epsilon_1 = \epsilon_2 \equiv \epsilon$, $\omega_1 = \omega_2 \equiv \omega_0$ and the matrix takes the form
\begin{equation}
  \frac{1}{\omega_0^2} \left( 
    (1 + \epsilon) \mathbb{I} - \epsilon \sigma_x
  \right)
\end{equation}
and the eigenvalues are
\begin{equation}
  \lambda_{\pm} = \frac{1}{\omega_0^2} ( 1 + \epsilon \pm \epsilon)
\end{equation}
so the normal mode frequencies are
\begin{equation}
  \omega_+ = \omega_0 \quad
  \text{and} \quad
  \omega_- = \frac{\omega_0}{\sqrt{1 + 2 \epsilon}}
    \approx \omega_0 (1 - \epsilon) \, .
\end{equation}
Note that the difference bewteen the two normal mode frequencies is
\begin{equation}
  \omega_+ - \omega_- \approx \omega_0 \epsilon
  = \omega_0 \frac{C_g}{C}
  = 2 \times \underbrace{\left(\frac{\omega}{2} \frac{C_g}{C} \right)}_g
  = 2 g
\end{equation}
which justifies our definition of the coupling strength $g$ because the frequency splitting of two oscillators on resonance is supposed to be $2g$.

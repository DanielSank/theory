\levelstay{Frequency Response}

Now we investigate the case in which $j(t)=A\cos(\Omega t)$.
In this case
\begin{equation}
  j(\omega)=\frac{A}{2}(2\pi)\left[\delta(\omega-\Omega)+\delta(\omega+\Omega)\right] \, .
\end{equation}
Therefore, by modifying equations from the previous section we have
\begin{align}
  \phi(\omega)
  & = \frac{-A}{2}(2\pi) \int \frac{d\omega}{2\pi} e^{i \omega t} \frac{\delta(\omega-\Omega)+\delta(\omega+\Omega)}{\omega^{2} - 2i\beta\omega - \omega_0^2} \nonumber \\
  & =
    \frac{-A}{2}
    \left[
      \frac{e^{i \Omega t}}{\Omega^{2} - 2i\beta\Omega - \omega_0^2} +
      \frac{e^{-i \Omega t}}{\Omega^2 + 2i\beta\Omega - \omega_0^2}
    \right] \nonumber \\
  &= \Re \left[ e^{i \Omega t} \frac{-A}{\Omega^2 - 2i\beta\Omega - \omega_0^2} \right]
  \label{eq:phasor_form}
\end{align}
This is a very special form known as {}``phasor'' form.
The idea of phasors is that in a linear system driven by a sinusoidal drive, all variables will oscillate with the same frequency as the drive.
Any dynamical variable, in our case $\phi$, differs from the drive only in amplitude and phase.
We see this here: the time dependence of $\phi$ is sinusoidal at angular frequency $\Omega$.
The crucial relation is that if we have two signals at the same frequency expressed as phasors
\begin{equation}
  f(t) = \Re[ e^{i\omega t} \hat{f}] \qquad
  g(t) = \Re[ e^{i\omega t} \hat{g}]
\end{equation}
where $\hat{f}$ and $\hat{g}$ are complex numbers, then we have
\begin{equation}
  \langle f(t)g(t)\rangle_{t}=\frac{1}{2}\Re[\hat{f}^{*}\hat{g}]
\end{equation}
where the left hand is a time average.
Our expression for $\phi(t)$ has this form if we take
\begin{equation}
  \hat{\phi} = \frac{-A}{\Omega^{2} - 2i\beta\Omega - \omega_0^2} \, .
\end{equation}
Therefore,
\begin{equation}
  \langle\phi(t)^{2}\rangle_{t}
  = \frac{1}{2} \frac{A^{2}}{(\Omega^{2}-\omega_0^{2})^{2}+(2\beta\Omega)^2}
  \, .
\end{equation}
This function maximizes for $\Omega=\pm\sqrt{\omega_{0}^{2}-2\beta^{2}}$.
Note that this is not the same as the free oscillation frequency found from the impulse response.

\leveldown{Resonance}

We define the resonance frequency as that frequency at which power flow from the drive to the system is unidirectional.
This happens when $\dot{\phi}(t)$ is in phase with $j(t)$.
Velocity and position are a quarter cycle out of phase, so the resonance occurs when the position is a quarter cycle shifted from the drive.
In other words, the resonance happens when the denominator of the phasor is imaginary, which occurs for
\begin{equation}
(\text{resonance}) \qquad \Omega = \pm \omega_0 \, .
\end{equation}

\levelstay{Phase shift}

It is useful to look at the relative phase shift between the drive and the response.
To do this we need to find the argument of $\hat{\phi}$.
Note that
\begin{equation}
f(t)
= \Re[e^{i \Omega t}\hat{f}]
= \Re[e^{i \Omega t}|f|e^{i\theta}]
=|f|\cos(\Omega t + \theta)]
\end{equation}
where $\theta$ is the phase of $\hat{f}$.
Therefore, a positive $\theta$ indicates that the response leads the source, and a negative $\theta$ indicates that the response lags the source.

\levelstay{Lorentzian approximation}

We now express the resonance in the so-called Lorentzian form.
\begin{align}
\phi(t)
&= \Re \left[ e^{i\Omega t} \frac{-A}{\Omega^2 - \omega_0^2 - 2i\beta \Omega} \right] \nonumber \\
&= \frac{A \cos \left( \Omega t + \delta \right)}{\sqrt{\left( \Omega^2 - \omega_0^2 \right)^2 + (2 \beta \Omega)^2}}
\end{align}
where $\delta$ is a phase we don't care about.
We can re-express this approximately as
\begin{equation}
\phi(t) \approx \frac{(A \beta / \Omega) \cos(\Omega t + \delta)}{\left( \Omega - \omega_0 \right)^2 + 2\beta^2} \, .
\end{equation}

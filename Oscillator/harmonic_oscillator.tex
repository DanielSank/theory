\documentclass{article}

%Packages
\usepackage{amsmath}
\usepackage{amstext}
\usepackage{amssymb}
\usepackage{appendix}
\usepackage{coseoul}
\usepackage{enumerate}
\usepackage{graphicx}
\usepackage{import}
\usepackage{lscape}
\usepackage{modular}

\usepackage[pdfpagemode=UseNone,pdfstartview=FitH,colorlinks=true,linkcolor=blue,citecolor=blue,urlcolor=blue]{hyperref}
\usepackage[all]{hypcap}


% General physics constructs
\newcommand{\bra}[1]{\langle #1 |}
\newcommand{\ket}[1]{| #1 \rangle }
\newcommand{\braket}[2]{\langle #1|#2\rangle}
\newcommand{\bbraket}[3]{ \langle #1 | #2 | #3 \rangle }
\newcommand{\norm}[1]{\| #1\|}
\newcommand{\avg}[1]{\left \langle #1 \right \rangle}
\newcommand{\angavg}[1]{\left \langle #1 \right \rangle}
\newcommand{\abs}[1]{\left \lvert #1 \right \rvert}
\newcommand{\VS}{\textit{\textbf{V}}}
\newcommand{\Tr}{\textrm{Tr}}
\renewcommand{\Re}{\textrm{Re}}
\renewcommand{\Im}{\textrm{Im}}
\newcommand{\basis}[1]{\{\ket{#1}\}}

\newcommand{\omegaqubit}{\omega_{10}}

% Figures. Example usage:
% \quickfig{\columnwidth}{my_image}{This is the caption}{fig:my_fig}
\DeclareRobustCommand{\quickfig}[4]{
\begin{figure}
\begin{centering}
\includegraphics[width=#1]{#2}
\par\end{centering}
\caption{#3}
\label{#4}
\end{figure}
}

\DeclareRobustCommand{\quickwidefig}[4]{
\begin{figure*}[h]
\begin{centering}
\includegraphics[width=#1]{#2}
\par\end{centering}
\caption{#3}
\label{#4}
\end{figure*}
}


\title{Harmonic Oscillator}
\author{Daniel Sank \\ \small{University of California Santa Barbara} \\ \small{Presently Google Quantum AI}}
\date{23 February 2009}

\begin{document}
\maketitle

\section{Impulse Response}

The equation for a damped harmonic oscillator is \begin{equation}
\ddot{\phi}(t)+2\beta\dot{\phi}(t)+\omega_{0}^{2}\phi(t)=j(t) \end{equation}
In the usual way we turn this into an abstract equation \begin{equation}
(D_{t}^{2}+2\beta D_{t}+\omega_{0}^{2})\ket{\phi}=\ket{j} \end{equation}
Our Fourier convention will be
\begin{equation}
\braket{t}{\omega}=e^{i\omega t}
\end{equation}
which is the common choice in electrical engineering.
However, it is the \emph{opposite} of what is normally used in physics, notably in Schrodinger's equation.
With this convention $-iD_t$ is Hermitian.
Going to the $\ket{\omega}$ basis we get
\begin{align}
\bbraket{\omega}{D_t^2 + 2\beta D_t + \omega_0^2}{\phi} &= \braket{\omega}{j} \\
(-\omega^{2} + 2 i \beta\omega + \omega_{0}^{2}) \phi(\omega) & = j(\omega)\\
\phi(\omega) & = \frac{j(\omega)}{-\omega^{2} + 2 i \beta \omega + \omega_{0}^{2}} \, .
\end{align}
Taking $j(t)=A\delta(t-t_{0})$ we get $\braket{\omega}{j} = A e^{-i \omega t_{0}}$.
We can factor the denominator as $-(\omega-\omega_{+})(\omega-\omega_{-})$
where
\begin{equation}
\omega_{+} = i \beta + \omega_0' \qquad \omega_{-} = i \beta - \omega_0' \label{eq:omega_plus_minus}
\end{equation}
and
\begin{equation}
\omega_0' = \omega_0 \sqrt{ 1 - \left( \beta / \omega_0 \right) ^2 } \, .
\end{equation}
Therefore we have \begin{equation}
\phi(\omega) = \frac{-A e^{-i \omega t_0}}{(\omega - \omega_{+})(\omega-\omega_{-})} \, .
\end{equation}
Upon Fourier transform
\begin{equation}
\phi(t) = 
\int\frac{-Ae^{i \omega(t - t_0)}}{(\omega-\omega_{+})(\omega-\omega_{-})} \,
\frac{d\omega}{2\pi} \, .
\end{equation}
Consider the case $t < t_0$.
In that case we close the contour in the lower half plane.
Since both poles $\omega_\pm$ are in the upper half plane the integral is zero.
This reflects causality: the oscillator can't know about the impulse before it happens.
For the case $t > t_{0}$ we close in the upper half plane.
Now the contour encloses the poles and we will get a nonzero result.
Let $\tau \equiv t - t_0$.
Then the residue theorem gives
\begin{align}
\phi(t) &=
\frac{-A 2\pi i}{2 \pi}
\left(
\frac{e^{i \omega_+ \tau}}{\omega_+ - \omega_-} +
\frac{e^{i \omega_- \tau}}{\omega_- - \omega_+}
\right) \\
&= \frac{A}{\omega_0'} e^{- \beta \tau} \sin(\omega_0' \tau)
\end{align}
We see that the system oscillates with a frequency close to $\omega_0$ and decays with a rate given by the amplitude damping parameter $\beta$.
Note that this solution is valid only when $\omega_0 > \beta$.

In the limit $\beta \ll \omega_0$ we find $E(t) = E_{0}e^{-2 \beta t}$, where $E_0=\frac{1}{2} m A^{2}$ is the initial energy of the system.
Note that $A$ is the initial velocity.

Defining the quality factor $Q$ by
\begin{equation}
Q \equiv \frac{\textrm{Energy stored}}{\textrm{Energy dissipated per radian}} \, ,
\end{equation}
we find
\begin{equation}
Q
= \frac{E(t)}{dE/d\textrm{rad}}
= \frac{E(t)}{-(dE/dt)(dt/d\textrm{rad})}
= \frac{E(t)}{2\beta E(t)\frac{1}{\Delta\omega}}
= \frac{\Delta\omega}{2\beta}\approx\frac{\omega_{0}}{2\beta} \, .
\end{equation}
This relation is also written
\begin{equation}
Q \approx \omega_0 / \kappa
\end{equation}
where $\kappa$ is the energy decay rate of the oscillator.


\section{Frequency Response}

Now we investigate the case in which $j(t)=A\cos(\Omega t)$.
In this case
\begin{equation}
j(\omega)=\frac{A}{2}(2\pi)\left[\delta(\omega-\Omega)+\delta(\omega+\Omega)\right] \, .
\end{equation}
Therefore, by modifying equations from the previous section we have
\begin{align}
\phi(\omega)
& = \frac{-A}{2}(2\pi) \int \frac{d\omega}{2\pi} e^{i \omega t} \frac{\delta(\omega-\Omega)+\delta(\omega+\Omega)}{\omega^{2} - 2i\beta\omega - \omega_0^2} \\
& =
\frac{-A}{2}
\left[
\frac{e^{i \Omega t}}{\Omega^{2} - 2i\beta\Omega - \omega_0^2} +
\frac{e^{-i \Omega t}}{\Omega^2 + 2i\beta\Omega - \omega_0^2}
\right] \\
&= \Re \left[ e^{i \Omega t} \frac{-A}{\Omega^2 - 2i\beta\Omega - \omega_0^2}
\right] \label{eq:phasor_form}
\end{align}
This is a very special form known as {}``phasor'' form.
The idea of phasors is that in a linear system driven by a sinusoidal drive, all variables will oscillate with the same frequency as the drive.
Any dynamical variable, in our case $\phi$, differs from the drive only in amplitude and phase.
We see this here: the time dependence of $\phi$ is sinusoidal at angular frequency $\Omega$.
The crucial relation is that if we have two signals at the same frequency which can be expressed as \begin{equation}
f(t) = \Re[ e^{i\omega t} \hat{f}] \qquad
g(t) = \Re[ e^{i\omega t} \hat{g}]
\end{equation}
where $\hat{f}$ and $\hat{g}$ are complex numbers, then we have \begin{equation}
\langle f(t)g(t)\rangle_{t}=\frac{1}{2}\Re[\hat{f}^{*}\hat{g}] \end{equation}
where the left hand is a time average.
Our expression for $\phi(t)$ has this form if we take
\begin{equation}
\hat{\phi} = \frac{-A}{\Omega^{2} - 2i\beta\Omega - \omega_0^2} \, .
\end{equation}
Therefore, \begin{equation}
\langle\phi(t)^{2}\rangle_{t} = \frac{1}{2} \frac{A^{2}}{(\Omega^{2}-\omega_0^{2})^{2}+(2\beta\Omega)^{2}} \, .
\end{equation}
This function maximizes for $\Omega=\pm\sqrt{\omega_{0}^{2}-2\beta^{2}}$.
Note that this is not the same as the free oscillation frequency found from the impulse response.

\subsection{Resonance}

We define the resonance frequency as that frequency at which power flow from the drive to the system is unidirectional.
This happens when $\dot{\phi}(t)$ is in phase with $j(t)$.
Velocity and position are a quarter cycle out of phase, so the resonance occurs when the position is a quarter cycle shifted from the drive.
In other words, the resonance happens when the denominator of the phasor is imaginary, which occurs for
\begin{equation}
(\text{resonance}) \qquad \Omega = \pm \omega_0 \, .
\end{equation}

\subsection{Phase shift}

It is useful to look at the relative phase shift between the drive and the response.
To do this we need to find the argument of $\hat{\phi}$.
Note that
\begin{equation}
f(t)
= \Re[e^{i \Omega t}\hat{f}]
= \Re[e^{i \Omega t}|f|e^{i\theta}]
=|f|\cos(\Omega t + \theta)]
\end{equation}
where $\theta$ is the phase of $\hat{f}$.
Therefore, a positive $\theta$ indicates that the response leads the source, and a negative $\theta$ indicates that the response lags the source.

\subsection{Lorentzian approximation}

We now express the resonance in the so-called Lorentzian form.
\begin{align}
\phi(t)
&= \Re \left[ e^{i\Omega t} \frac{-A}{\Omega^2 - \omega_0^2 - 2i\beta \Omega} \right] \nonumber \\
&= \frac{A \cos \left( \Omega t + \delta \right)}{\sqrt{\left( \Omega^2 - \omega_0^2 \right)^2 + (2 \beta \Omega)^2}}
\end{align}
where $\delta$ is a phase we don't care about.
We can re-express this approximately as
\begin{equation}
\phi(t) \approx \frac{(A \beta / \Omega) \cos(\Omega t + \delta)}{\left( \Omega - \omega_0 \right)^2 + 2\beta^2} \, .
\end{equation}


\section{Phasor form}

Let us go back to the phasor form of the solution from Eq.\,(\ref{eq:phasor_form})
\begin{equation}
\phi(t) = \Re \left[ e^{i \Omega t} \frac{-A}{\Omega^2 - 2 i \beta \Omega - \omega_0^2} \right] \, .
\end{equation}
Factoring the denominator we get
\begin{equation}
\phi(t) = \Re \left[ e^{i \Omega t} \frac{-A}{(\Omega - \omega_+)(\Omega - \omega_-)} \right]
\end{equation}
where $\omega_{\pm}$ are defined by Eq.\,(\ref{eq:omega_plus_minus}).
We now use the method of partial fractions to split the double-pole into a sum of two single poles.
Assume the double pole can be written as
\begin{equation}
\frac{1}{(\Omega - \omega_+)(\Omega - \omega_-)}
= \frac{x}{\Omega - \omega_+} + \frac{y}{\Omega - \omega_-} \, .
\end{equation}
Multiplying through by $(\Omega - \omega_+)(\Omega - \omega_-)$ and setting either $x$ or $y$ to 1 yields two equations:
\begin{align}
1 &= x (\omega_+ - \omega_-) = x 2 \omega_0' \\
1 &= y (\omega_- - \omega_+) = -y 2 \omega_0'
\end{align}
from which we find
\begin{equation}
\phi(t) = -A \Re
\left[ e^{i \Omega t} \left(
\frac{1}{2 \omega_0' (\Omega - \omega_+)} -
\frac{1}{2 \omega_0' (\Omega - \omega_-)}
\right) \right] \, .
\end{equation}
If the drive frequency $\Omega$ is near the resonance we can drop the second term,
\begin{equation}
\phi(t) \approx -A \Re
\left[
\frac{e^{i \Omega t}}{2 \omega_0' (\Omega - i\beta - \omega_0')}
\right] \, .
\end{equation}
Define $\alpha(t)$ as the complex quantity of which $\phi(t)$ is the real part,
\begin{align}
\alpha(t) &= \frac{-A e^{i \Omega t}}{2 \omega_0'(\Omega - i \beta - \omega_0')} \\
(\Omega - i \beta - \omega_0') \alpha(t) &= - \frac{A}{2\omega_0'} e^{i \Omega t} \, .
\end{align}
Noting that $\dot{\alpha}(t) = i \Omega \alpha(t)$, we can write
\begin{equation}
\dot{\alpha}(t) = i \omega_0' \alpha(t) - \beta \alpha(t) - \frac{A}{2 \omega_0'} e^{i \Omega t} \, .
\end{equation}



\section{Rotating Mode Formalism}

There's an interesting way to write down the equations for a harmonic oscillator that's useful when you have coupled problems or problems with time varied parameters.
An as example consider a parallel LRC circuit.
As dynamical variables take the flux in the inductor, $\Phi$, and the charge on the capacitor, $Q$.
See figure ??.
The two first order equations of motion are\begin{equation}
\dot{\Phi}=Q/C\qquad\dot{Q}+Q/(RC)=-\Phi/L \end{equation}
Note that if $R\rightarrow\infty$ these can be written in a nearly Hamiltonian form \begin{equation}
\frac{d}{dt}\left[ \begin{array}{c} \Phi \\ Q\end{array} \right] =
\left[\begin{array}{cc} 0 & 1/C \\ -1/L & 0 \end{array} \right]
\left[\begin{array}{c} \Phi\\ Q\end{array}\right] \end{equation}
If the off-diagonal elements were equal to one another this would be a true Hamiltonian.
This is easy to do if we simply re-scale the variables, \begin{equation}
X \equiv \left( C/L \right)^{1/4} \Phi \qquad Y \equiv \left( L/C \right)^{1/4} Q \end{equation}
Now we find, in the case that $R\rightarrow\infty$ \begin{displaymath}
\frac{d}{dt}\left[ \begin{array}{c} X \\ Y \end{array} \right] = \omega_{0} \left[ \begin{array}{cc} 0 & 1 \\ -1 & 0 \end{array}\right]
\left[ \begin{array}{c} X \\ Y \end{array} \right] \end{displaymath}
where $\omega_{0}\equiv1/\sqrt{LC}$.
This is great because we have the anti-symmetric Hamiltonian matrix.
Adding the loss back in gives \begin{displaymath}
\frac{d}{dt} \left[ \begin{array}{c} X \\ Y \end{array} \right] =
\omega_0 \left[ \begin{array}{cc} 0 & 1 \\ -1 & -1/Q \end{array} \right]
\left[ \begin{array}{c} X \\ Y \end{array} \right] \end{displaymath}
where $Q\equiv \omega_0 R C$.

We can make one further simplification if we want to get rid of the $\omega_0$ out front.
Make one more rescaling,\begin{equation}
\xi\equiv\omega_{0}t \end{equation}
Then \begin{equation}
\frac{d}{dt}=\omega_{0}\frac{d}{d\xi}\end{equation}
Therefore the equations of motion for $X$ and $Y$ are \begin{equation}
\frac{d}{d\xi}\left[\begin{array}{c} X\\ Y\end{array}\right] = 
\left[\begin{array}{cc} 0 & 1\\ -1 & -1/Q \end{array}\right]
\left[\begin{array}{c} X\\ Y\end{array}\right]\end{equation}

\subsection{Normal Modes}
We can now find the normal modes of this equation.
To do so we must find the eigenvalues and eigenvectors of the matrix, \begin{equation}
\left[ \begin{array}{cc} 0 & 1 \\ -1 & -\epsilon \end{array} \right] \end{equation}
where here $\epsilon \equiv 1/Q$.
The eigenvalue equation is \begin{equation}
\lambda^2 + \epsilon \lambda + 1 = 0 \end{equation}
from which the eigenvalues are found to be \begin{equation}
\lambda_{\pm} = \pm i \sqrt{1 - \frac{\epsilon^2}{4}} - \frac{\epsilon}{2} \end{equation}
Now, we want to find eigenvectors $a^+$ and $a^-$ that satisfy the equations \begin{equation}
\dot{a}^+=\lambda_+ a^+ \quad \textrm{and} \quad \dot{a}^-=\lambda_- a^- \end{equation}
where the overdots denote differentiation with respect to $\xi$.
Write $a$ (denoting here either one of the eigenvectors) as $a = X+\alpha Y$.
Then taking the derivative with respect to $\xi$ of both sides gives \begin{eqnarray}
\dot{a} &=& \dot{X}+\alpha \dot{Y}\\
&=& Y+\alpha(-X-\epsilon Y)\\
&=& -\alpha \left( X+\left[ \epsilon - \frac{1}{\alpha} \right] Y \right) \end{eqnarray}
Since we're expecting $a$ to satisfy $\dot{a} = \lambda a$ we get \begin{eqnarray}
\dot{a} &=& \lambda a \\
-\alpha \left( X+\left[ \epsilon - \frac{1}{\alpha} \right] Y \right) &=& \lambda \left( X + \alpha Y \right) \\
\rightarrow \quad \lambda=-\alpha &\quad& 1-\epsilon \alpha = \lambda \alpha \end{eqnarray}
where the last line follows by equating coefficients of $X$ and $Y$.
We can check that these relations are self consistent by plugging the first into the second \begin{eqnarray}
1+\epsilon \lambda &=& -\lambda^2 \\
1+\epsilon \left( -\frac{\epsilon}{2} \pm i \sqrt{1-\frac{\epsilon^2}{4}} \right) &=&
-\left(-\frac{\epsilon}{2}\pm i \sqrt{1-\frac{\epsilon^2}{4}} \right) \\
1-\frac{\epsilon^2}{2}\pm i \epsilon \sqrt{\cdots} &=&
-\frac{\epsilon^2}{4}\pm i \epsilon\sqrt{\cdots} + 1 -\frac{\epsilon^2}{4}\\
0 &=& 0 \end{eqnarray}
so the relations are indeed consistent.

Now we can write down the eigenvectors along with their time dependence, \begin{eqnarray}
a^- = \frac{1}{\sqrt{2}}\left( X -\lambda_- Y \right) &\qquad&
a^+ = \frac{1}{\sqrt{2}}\left( X -\lambda_+ Y \right) \\
a^-(t) = a^-(0) e^{\omega_0 \lambda_- t} &\qquad& a^+(t) = a^+(0) e^{\omega_0 \lambda_+ t} \end{eqnarray}
Here we've put the $\omega_0$'s back in because we're using $t$ instead of $\xi$.

In the case that $Q \gg 1$ the eigenvalues simplify to \begin{equation}
\lambda_{\pm} = \pm i -\frac{\epsilon}{2} \end{equation}
so that we get \begin{eqnarray}
a^- = \frac{1}{\sqrt{2}}\left[ X + \left( i + \frac{\epsilon}{2} \right) Y \right] &\qquad&
a^+ = \frac{1}{\sqrt{2}}\left[ X + \left(-i + \frac{\epsilon}{2} \right) Y \right] \\
a^-(t) = a^-(0) \exp \left[ \omega_0 \left(-i - \frac{\epsilon}{2}\right) t \right]
& \qquad &
a^+(t) = a^+(0) \exp \left[ \omega_0 \left( i - \frac{\epsilon}{2}\right) t \right] \end{eqnarray}
Note that the time evolution has the same sense from what is typically used in quantum mechanics for the $a$ and $a^{\dagger}$ operators.

\end{document}

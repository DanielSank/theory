\documentclass{article}

%Packages
\usepackage{amsmath}
\usepackage{amstext}
\usepackage{amssymb}
\usepackage{appendix}
\usepackage{coseoul}
\usepackage{enumerate}
\usepackage{graphicx}
\usepackage{import}
\usepackage{lscape}
\usepackage{modular}

\usepackage[pdfpagemode=UseNone,pdfstartview=FitH,colorlinks=true,linkcolor=blue,citecolor=blue,urlcolor=blue]{hyperref}
\usepackage[all]{hypcap}


% General physics constructs
\newcommand{\bra}[1]{\langle #1 |}
\newcommand{\ket}[1]{| #1 \rangle }
\newcommand{\braket}[2]{\langle #1|#2\rangle}
\newcommand{\bbraket}[3]{ \langle #1 | #2 | #3 \rangle }
\newcommand{\norm}[1]{\| #1\|}
\newcommand{\avg}[1]{\left \langle #1 \right \rangle}
\newcommand{\angavg}[1]{\left \langle #1 \right \rangle}
\newcommand{\abs}[1]{\left \lvert #1 \right \rvert}
\newcommand{\VS}{\textit{\textbf{V}}}
\newcommand{\Tr}{\textrm{Tr}}
\renewcommand{\Re}{\textrm{Re}}
\renewcommand{\Im}{\textrm{Im}}
\newcommand{\basis}[1]{\{\ket{#1}\}}

\newcommand{\omegaqubit}{\omega_{10}}

% Figures. Example usage:
% \quickfig{\columnwidth}{my_image}{This is the caption}{fig:my_fig}
\DeclareRobustCommand{\quickfig}[4]{
\begin{figure}
\begin{centering}
\includegraphics[width=#1]{#2}
\par\end{centering}
\caption{#3}
\label{#4}
\end{figure}
}

\DeclareRobustCommand{\quickwidefig}[4]{
\begin{figure*}[h]
\begin{centering}
\includegraphics[width=#1]{#2}
\par\end{centering}
\caption{#3}
\label{#4}
\end{figure*}
}


\title{Harmonic Oscillator}
\author{Daniel Sank \\ \small{University of California Santa Barbara} \\ \small{Presently Google Quantum AI}}
\date{23 February 2009}

\begin{document}
\maketitle

\section{Equations of motion}

In this section we derive the equation of motion for a driven, damped Josephson oscillator with parametric modulation of its inductance.

\subsection{Josephson oscillator}

Consider a parallel resistor $R$, capacitor $C$, and Josephson junction of critical current $I_c$, driven by an external current source $I_s(t) = I_s \cos(\omega_s t + \theta)$.
The equation of motion for this circuit is
\begin{equation}
\ddot{\phi} + \frac{1}{RC}\dot{\phi} + \frac{1}{L_{J_0}C}\sin \phi = \frac{2\pi}{C \Phi_0} I_s \cos(\omega_s t + \theta)
\end{equation}
where $L_{J_0} \equiv \Phi_0 / 2\pi I_c$ is the un-biased inductance of the junction.
Defining $\omega_0 \equiv 1 / \sqrt{L_{J_0}C}$, $2 \Gamma \equiv 1/RC$, and $J \equiv  2\pi I_s / C \Phi_0$, we can re-write the equation of motion as
\begin{equation}
\ddot{\phi} + 2\Gamma \dot{\phi} + \omega_0^2 \sin \phi = J \cos(\omega_s t + \theta) .
\end{equation}

\subsection{Parametric pumping}

Suppose the junction were instead a DC SQUID. We could then modulate the inductance by driving flux through the SQUID loop. The inductance of the SQUID is then \begin{equation}
L(t) = L_{\textrm{dc}} + \Phi_p \frac{dL}{d\Phi}\cos(\omega_p t) \end{equation}
where $L_{\textrm{dc}}$ is the inductance of the SQUID under a constant flux bias, and $\Phi_p$ is the amplitude of the flux drive signal. This leads to a time dependent oscillation frequency \begin{eqnarray}
\omega_0(t) &=& (L_{\textrm{dc}}C)^{-1/2} \left(1 + \cos(\omega_p t) \frac{dL/L_{\textrm{dc}}} {d\Phi/\Phi_p} \right)^{-1/2} \\
&\approx& \omega_{0,\textrm{dc}} \left( 1 -\cos(\omega_p t) \frac{1}{2}\frac{dL/L_{\textrm{dc}}}{d\Phi/\Phi_p} \right) \\
&\equiv& \omega_{0,\textrm{dc}}\left( 1 - A\cos(\omega_p t) \right) \end{eqnarray}

Now the equation of motion is \begin{equation}
\ddot{\phi} + 2\Gamma \dot{\phi} + \omega_{0,\textrm{dc}}^2(1 - A \cos(\omega_p t))\sin(\phi) = J\cos(\omega_s t + \theta) \end{equation}
Taking only the linear term of the $\sin$ we get \begin{equation}
\ddot{\phi} + 2\Gamma \dot{\phi} + \omega_{0,\textrm{dc}}^2(1 - A \cos(\omega_p t))\phi = J\cos(\omega_s t + \theta) \label{eq:motion} \end{equation}
which is a driven, damped harmonic oscillator with time dependent frequency.

\section{Solution of equations of motion}

Equation (\ref{eq:motion}) is best studied in the frequency domain.
Making the transformation to frequency is easy using the following facts \begin{eqnarray}
\cos(\Omega t + \theta) &\rightarrow& \frac{1}{2}(2\pi)\left(e^{i \theta}\delta(\omega - \Omega) + e^{-i \theta} \delta(\omega + \Omega) \right) \nonumber \\
\phi(t)\cos(\Omega t) &\rightarrow& \frac{1}{2}\left( \tilde{\phi}(\omega-\Omega) + \tilde{\phi}(\omega+\Omega) \right) \nonumber \\
\dot{\phi}(t) &\rightarrow& i\omega \tilde{\phi}(\omega) \nonumber \end{eqnarray}
Using these equations and defining $L(\omega) \equiv (\omega_0^2 - \omega^2 +i2\omega\Gamma)$, we get \begin{equation}
\underbrace{L(\omega) \tilde{\phi}}_{\text{linear response}}
- \underbrace{\frac{1}{2} A \omega_0^2 \left[ \tilde{\phi}(\omega - \omega_p) + \tilde{\phi}(\omega + \omega_p) \right]}_{\text{modulated response}}
= \underbrace{\frac{1}{2}J(2\pi)\left[e^{i \theta}\delta(\omega-\omega_s) + e^{-i\theta}\delta(\omega+\omega_s) \right]}_{\text{source}} . \label{eq:eqOfMotionFrequency} \end{equation}

We want to find what $\tilde{\phi}(\omega)$ solves this equation.
If the parametric modulation were absent ($A=0$), then we would get the standard driven harmonic oscillator equation and $\phi$ would be a single tone at $\omega_s$.
However, the parametric drive produces frequency-shifted copies of $\tilde\phi$ so a single tone at $\omega_s$ does not work.

\begin{figure}
\begin{centering}
\includegraphics[width=12cm]{response_balance.pdf} 
\par\end{centering}
\caption{The three terms in Eq. (\ref{eq:eqOfMotionFrequency}).
The source term is a pure tone at $\omega_s$.
The modulated response produces a copy of $\phi$ shifted by $\omega_p$.
a) Case where $\phi$ is a pure tone at $\omega_s$ with amplitude $\alpha$ (blue).
In this case, the linear and modulated response terms do not sum up to match the source.
b) Here $\phi$ includes another tone at $\omega_i$ with amplitude $\beta$ (red).
Now we can choose $\alpha$ and $\beta$ such that the linear and modulated responses sum to match the source.}
\label{Fig:responseBalance}
\end{figure}

To understand how to fix this, we construct a diagram representing the right and left hand sides of Eq. (\ref{eq:eqOfMotionFrequency}), as shown in Figure \ref{Fig:responseBalance}.
Figure \ref{Fig:responseBalance}\,a shows the frequency space representation of the source and response terms in Eq. \ref{eq:eqOfMotionFrequency}) assuming that $\tilde\phi$ were a single tone at $\omega_s$ with amplitude $\alpha$.
Adding the linear and modulated responses gives a shifted tone with amplitude $\alpha^*$ that has no counterpart in the source.
The only solution in this case is $\alpha=0$ which is trivial.
Now suppose the response has another tone at frequency $\omega_i$ with amplitude $\beta$ as shown by the red peaks in Figure \ref{Fig:responseBalance}\,b.
Now the frequency shifted copies can add both to match the source at $\omega_s$ and, if the amplitudes are chosen correctly, to cancel at $\omega_i$.
The response peak at $\omega_s$ is called the ``signal'' and the peak at $\omega_i$ is called the ``idler''.

Balancing the terms yields two equations: \begin{eqnarray}
L(\omega_s)\alpha - \frac{1}{2}A\omega_0^2 \beta^* &=& \frac{1}{2}J(2\pi)e^{i\phi} \\
L(\omega_i)\beta - \frac{1}{2}A\omega_0^2 \alpha^* &=& 0 . \end{eqnarray}
Solving the two simultaneous equations gives
\begin{equation}
\beta^* = - \frac{1}{2}A\omega_0^2 \frac{J'}{\left(\frac{1}{2}A\omega_0^2 \right)^2 + L(\omega_s)^2} \qquad
\alpha = L(\omega_s) \frac{J'}{\left(\frac{1}{2}A\omega_0^2 \right)^2 + L(\omega_s)^2} . \label{eq:alphabeta}
\end{equation}
In order to use the circuit as an amplifier we need gain.
From Eq. (\ref{eq:alphabeta}) we see that gain is achieved when
\begin{equation}
\left| \frac{1}{2}A\omega_0^2 + L(\omega_s)^2 \right| \ll 1 . \label{eq:gainCondition} \end{equation}
Equation (\ref{eq:gainCondition}) is satisfied when $L(\omega_s)$ is approximately purely imaginary, which occurs for $\omega_s \approx \omega_0$.
This makes sense, as we know the amplifier has large gain when the signal frequency is near the resonator mode center frequency.
%This also implies that $\alpha$ and $\beta$ are related by complex conjugation and an approximately 90 degree phase shift, which is the same as reflection about the line $\Im z = \Re z$.

\subsection{Drive at $\omega_i$}
If the drive is at the idler frequency the equations are \begin{eqnarray}
L(\omega_i)\beta - \frac{1}{2}A\omega_0^2 \alpha^* &=& \frac{1}{2}J(2\pi)e^{i\phi} \\
L(\omega_s)\alpha - \frac{1}{2}A\omega_0^2 \beta^* &=& 0 \end{eqnarray}
Solving gives \begin{eqnarray}
\beta^* &=& -\frac{1}{2}A\omega_0^2 \frac{J'}{\left( \frac{1}{2}A\omega_0^2 \right)^2 + L(\omega_s)^2} \\
\alpha &=& L(\omega_s) \frac{J'}{\left( \frac{1}{2}A\omega_0^2 \right)^2 + L(\omega_s)^2} \end{eqnarray}


\section{Rotating Mode Formalism}

There's an interesting way to write down the equations for a harmonic oscillator that's useful when you have coupled problems or problems with time varied parameters.
An as example consider a parallel LRC circuit.
As dynamical variables take the flux in the inductor, $\Phi$, and the charge on the capacitor, $Q$.
See figure ??.
The two first order equations of motion are\begin{equation}
\dot{\Phi}=Q/C\qquad\dot{Q}+Q/(RC)=-\Phi/L \end{equation}
Note that if $R\rightarrow\infty$ these can be written in a nearly Hamiltonian form \begin{equation}
\frac{d}{dt}\left[ \begin{array}{c} \Phi \\ Q\end{array} \right] =
\left[\begin{array}{cc} 0 & 1/C \\ -1/L & 0 \end{array} \right]
\left[\begin{array}{c} \Phi\\ Q\end{array}\right] \end{equation}
If the off-diagonal elements were equal to one another this would be a true Hamiltonian.
This is easy to do if we simply re-scale the variables, \begin{equation}
X \equiv \left( C/L \right)^{1/4} \Phi \qquad Y \equiv \left( L/C \right)^{1/4} Q \end{equation}
Now we find, in the case that $R\rightarrow\infty$ \begin{displaymath}
\frac{d}{dt}\left[ \begin{array}{c} X \\ Y \end{array} \right] = \omega_{0} \left[ \begin{array}{cc} 0 & 1 \\ -1 & 0 \end{array}\right]
\left[ \begin{array}{c} X \\ Y \end{array} \right] \end{displaymath}
where $\omega_{0}\equiv1/\sqrt{LC}$.
This is great because we have the anti-symmetric Hamiltonian matrix.
Adding the loss back in gives \begin{displaymath}
\frac{d}{dt} \left[ \begin{array}{c} X \\ Y \end{array} \right] =
\omega_0 \left[ \begin{array}{cc} 0 & 1 \\ -1 & -1/Q \end{array} \right]
\left[ \begin{array}{c} X \\ Y \end{array} \right] \end{displaymath}
where $Q\equiv \omega_0 R C$.

We can make one further simplification if we want to get rid of the $\omega_0$ out front.
Make one more rescaling,\begin{equation}
\xi\equiv\omega_{0}t \end{equation}
Then \begin{equation}
\frac{d}{dt}=\omega_{0}\frac{d}{d\xi}\end{equation}
Therefore the equations of motion for $X$ and $Y$ are \begin{equation}
\frac{d}{d\xi}\left[\begin{array}{c} X\\ Y\end{array}\right] = 
\left[\begin{array}{cc} 0 & 1\\ -1 & -1/Q \end{array}\right]
\left[\begin{array}{c} X\\ Y\end{array}\right]\end{equation}

\subsection{Normal Modes}
We can now find the normal modes of this equation.
To do so we must find the eigenvalues and eigenvectors of the matrix, \begin{equation}
\left[ \begin{array}{cc} 0 & 1 \\ -1 & -\epsilon \end{array} \right] \end{equation}
where here $\epsilon \equiv 1/Q$.
The eigenvalue equation is \begin{equation}
\lambda^2 + \epsilon \lambda + 1 = 0 \end{equation}
from which the eigenvalues are found to be \begin{equation}
\lambda_{\pm} = \pm i \sqrt{1 - \frac{\epsilon^2}{4}} - \frac{\epsilon}{2} \end{equation}
Now, we want to find eigenvectors $a^+$ and $a^-$ that satisfy the equations \begin{equation}
\dot{a}^+=\lambda_+ a^+ \quad \textrm{and} \quad \dot{a}^-=\lambda_- a^- \end{equation}
where the overdots denote differentiation with respect to $\xi$.
Write $a$ (denoting here either one of the eigenvectors) as $a = X+\alpha Y$.
Then taking the derivative with respect to $\xi$ of both sides gives \begin{eqnarray}
\dot{a} &=& \dot{X}+\alpha \dot{Y}\\
&=& Y+\alpha(-X-\epsilon Y)\\
&=& -\alpha \left( X+\left[ \epsilon - \frac{1}{\alpha} \right] Y \right) \end{eqnarray}
Since we're expecting $a$ to satisfy $\dot{a} = \lambda a$ we get \begin{eqnarray}
\dot{a} &=& \lambda a \\
-\alpha \left( X+\left[ \epsilon - \frac{1}{\alpha} \right] Y \right) &=& \lambda \left( X + \alpha Y \right) \\
\rightarrow \quad \lambda=-\alpha &\quad& 1-\epsilon \alpha = \lambda \alpha \end{eqnarray}
where the last line follows by equating coefficients of $X$ and $Y$.
We can check that these relations are self consistent by plugging the first into the second \begin{eqnarray}
1+\epsilon \lambda &=& -\lambda^2 \\
1+\epsilon \left( -\frac{\epsilon}{2} \pm i \sqrt{1-\frac{\epsilon^2}{4}} \right) &=&
-\left(-\frac{\epsilon}{2}\pm i \sqrt{1-\frac{\epsilon^2}{4}} \right) \\
1-\frac{\epsilon^2}{2}\pm i \epsilon \sqrt{\cdots} &=&
-\frac{\epsilon^2}{4}\pm i \epsilon\sqrt{\cdots} + 1 -\frac{\epsilon^2}{4}\\
0 &=& 0 \end{eqnarray}
so the relations are indeed consistent.

Now we can write down the eigenvectors along with their time dependence, \begin{eqnarray}
a^- = \frac{1}{\sqrt{2}}\left( X -\lambda_- Y \right) &\qquad&
a^+ = \frac{1}{\sqrt{2}}\left( X -\lambda_+ Y \right) \\
a^-(t) = a^-(0) e^{\omega_0 \lambda_- t} &\qquad& a^+(t) = a^+(0) e^{\omega_0 \lambda_+ t} \end{eqnarray}
Here we've put the $\omega_0$'s back in because we're using $t$ instead of $\xi$.

In the case that $Q \gg 1$ the eigenvalues simplify to \begin{equation}
\lambda_{\pm} = \pm i -\frac{\epsilon}{2} \end{equation}
so that we get \begin{eqnarray}
a^- = \frac{1}{\sqrt{2}}\left[ X + \left( i + \frac{\epsilon}{2} \right) Y \right] &\qquad&
a^+ = \frac{1}{\sqrt{2}}\left[ X + \left(-i + \frac{\epsilon}{2} \right) Y \right] \\
a^-(t) = a^-(0) \exp \left[ \omega_0 \left(-i - \frac{\epsilon}{2}\right) t \right]
& \qquad &
a^+(t) = a^+(0) \exp \left[ \omega_0 \left( i - \frac{\epsilon}{2}\right) t \right] \end{eqnarray}
Note that the time evolution has the same sense from what is typically used in quantum mechanics for the $a$ and $a^{\dagger}$ operators.

\section{Coupled oscillators}

\levelstay{Coupled oscillators}

Consider two LC oscillators coupled through a coupling capacitor $C_g$.
Kirchhoff's equations of motion for this circuit can be written in matrix form
as
\begin{equation}
  \left( \begin{array}{c} V_1 \\ V_2 \end{array} \right)
  =
  \left( \begin{array}{cc}
    (1 + \epsilon_1)/\omega_1^2 & - \epsilon_1  \omega_1^2 \\
    - \epsilon_2  \omega_2^2 & (1 + \epsilon_2)/\omega_2^2
  \end{array} \right)
  \left( \begin{array}{c} \ddot{V}_1 \\ \ddot{V}_2 \end{array} \right)
\end{equation}
where $\epsilon_i \equiv C_g / C_i$ and $\omega_i^2 \equiv 1 / L_i C_i$.
The normal mode frequencies of the system are the square roots of the reciprocals of the eigenvalues of the matrix.
We could compute those eigenvalues explicitly, but it's a mess and there's a better way to do it that we'll show below.
However, in the particular case that $L_1 = L_2$ and $C_1 = C_2$, then $\epsilon_1 = \epsilon_2 \equiv \epsilon$, $\omega_1 = \omega_2 \equiv \omega_0$ and the matrix takes the form
\begin{equation}
  \frac{1}{\omega_0^2} \left( 
    (1 + \epsilon) \mathbb{I} - \epsilon \sigma_x
  \right)
\end{equation}
and the eigenvalues are
\begin{equation}
  \lambda_{\pm} = \frac{1}{\omega_0^2} ( 1 + \epsilon \pm \epsilon)
\end{equation}
so the normal mode frequencies are
\begin{equation}
  \omega_+ = \omega_0 \quad
  \text{and} \quad
  \omega_- = \frac{\omega_0}{\sqrt{1 + 2 \epsilon}}
    \approx \omega_0 (1 - \epsilon) \, .
\end{equation}
Note that the difference bewteen the two normal mode frequencies is
\begin{equation}
  \omega_+ - \omega_- \approx \omega_0 \epsilon
  = \omega_0 \frac{C_g}{C}
  = 2 \times \underbrace{\left(\frac{\omega}{2} \frac{C_g}{C} \right)}_g
  = 2 g
\end{equation}
which justifies our definition of the coupling strength $g$ because the frequency splitting of two oscillators on resonance is supposed to be $2g$.

\subimportlevel{./}{coupled_oscillators_hamiltonian.tex}{1}


\subsection{Hamiltonian approach}

The Hamiltonian for two LC oscillators coupled through a coupling capacitor $C_g$ is
\begin{align}
  H
  &= \frac{\Phi_1^2}{2 L_1} + \frac{\Phi_2^2}{2 L_2} \\
  &+ \frac{Q_1^2}{2 C_1} + \frac{Q_2^2}{2 C_2} \\
  &+ \frac{Q_1 Q_2}{C_g''}
\end{align}
where
\begin{align}
  C_g'' &\equiv \frac{C_1' C_2' - C_g^2}{C_g} \\
  C_i' &\equiv C_i + C_g \, .
\end{align}
We transform the Hamiltonian in two steps.
First we define
\begin{align*}
  X_i &\equiv \frac{1}{\sqrt{2 \hbar}} \frac{1}{\sqrt{Z_i}} \Phi_i \\
  Y_i &\equiv \frac{1}{\sqrt{2 \hbar}} \sqrt{Z_i} Q_i
\end{align*}
where $Z_i \equiv \sqrt{L_i / C_i''}$.
With these new variables, the Hamiltonian is
\begin{equation}
  H/\hbar =
    \omega_1 \left(X_1^2 + Y_1^2) \right)
  + \omega_2 \left(X_2^2 + Y_2^2) \right)
  + 2 \left( \frac{C_1'' C_2''}{C_g''} \right) \sqrt{\omega_1'' \omega_2''} Y_1 Y_2 \, .
\end{equation}
Note that $[X, Y] = i/2$ so Hamilton's equations of motion are a little different than usual.
Anyway, we next use $a = X + i Y$ and $a^\dagger = X - i Y$ to get
\begin{equation}
  H / \hbar =
    \omega_1'' a_1^\dagger a_1
  + \omega_2'' a_2^\dagger a_2
  - g (a_1 - a_1^\dagger) (a_2 - a_2^\dagger)
\end{equation}
where
\begin{equation}
  g \equiv \frac{1}{2} \frac{\sqrt{C_1'' C_2''}}{C_g''} \sqrt{\omega_1'' \omega_2''} \, .
\end{equation}
Hamilton's equations of motion in this representation are
\begin{equation*}
  \dot{a} = -i \frac{\partial H/\hbar}{\partial a^\dagger}
  \quad \text{and} \quad
  \dot{a}^\dagger = i \frac{\partial H/\hbar}{\partial a} \, .
\end{equation*}
Therefore the equations of motion for the system are
\begin{align*}
  \dot a_1 &= -i \omega_1'' a_1 -i g a_2 + i g a_2^\dagger \\
  \dot a_2 &= -i \omega_2'' a_2 -i g a_1 + i g a_1^\dagger
\end{align*}
and the equations for $a^\dagger$ follow by conjugation.
These equations can be written in matrix form as
\begin{equation}
  \frac{d}{dt}
  \left( \begin{array}{c}
    a_1 \\ a_1^\dagger \\ a_2 \\ a_2^\dagger
  \end{array} \right)
  =
  \left( \begin{array}{cccc}
    -i \omega_1'' & 0 & -i g & i g \\
    0 & i \omega_1'' & -i g & ig \\
    -i g & i g & -i \omega_2'' & 0 \\
    -i g & i g & 0 & i \omega_2''
  \end{array} \right)
  \left( \begin{array}{c}
    a_1 \\ a_1^\dagger \\ a_2 \\ a_2^\dagger
  \end{array} \right)
  \, .
\end{equation}
The eigenvalues are
\begin{equation}
  \lambda^2 =
  \pm \frac{1}{2}
  \left(
    \omega_1''^2 + \omega_2''^2
    \pm \sqrt{(\omega_1''^2 - \omega_2''^2)^2 + 16 g^2 \omega_1'' \omega_2''}
  \right)
\end{equation}


\section{Coupled transmission lines}

% artifact header: coupled_lines
WARNING: This discussion probably has errors. In particular, I'm not sure what the sign of the couopling inductance should be. I'm also worried that I messed up the definitions of $L$ and $C$ by somehow not taking into account the effects of coupling on each mode's ``canonical'' inductance and capacitance.

Consider the coupled transmission lines shown in Figure \ref{fig:coupled_lines:circuit_diagram}.
The capacitances $C_a$ and $C_b$ and the inductaces $L_1$ and $L_2$ are the capcitances and inductances \emph{per length} of each transmission line
The coupling capacitance $C_g$ and coupling inductance $L_g$ are similarly per length quantities.

\quickfig{\columnwidth}{coupled_lines.pdf}{Two coupled transmission lines.}{fig:coupled_lines:circuit_diagram}

Of course, we can start an analysis via Kichhoff's laws, which for the illustrated section of line are
\begin{align*}
  V_a(x) - V_a(x + dx) =& L_a \dot I_a + L_g \dot I_b \\
  I_a(x) - I_a(x + dx) =& \dot V_a C_a + \left( \dot V_a - \dot V_b \right) C_g \\
  V_b(x) - V_b(x + dx) =& L_b \dot I_b + L_g \dot I_a \\
  I_b(x) - I_b(x + dx) =& \dot V_b C_b + \left( \dot V_b - \dot V_a \right) C_g
  \, .
\end{align*}
In the limit $dx \rightarrow 0$, these equations become differential equations
\begin{align}
  - \frac{\partial V_a}{\partial x} &= \frac{\partial}{\partial t}
    \left( L_a I_a + L_g I_b \right) \nonumber \\
  - \frac{\partial I_a}{\partial x} &= \frac{\partial}{\partial t}
    \left( (C_a + C_g)V_a - C_g V_b \right) \nonumber \\
  - \frac{\partial V_b}{\partial x} &= \frac{\partial}{\partial t}
    \left( L_b I_b + L_g I_a \right) \nonumber \\
  - \frac{\partial I_b}{\partial x} &= \frac{\partial}{\partial t}
    \left( (C_b + C_g)V_b - C_g V_a \right)
  \, .
\end{align}
Assuming sinusoidal time dependence at frequency $\omega$, and therefore converting time derivatives to $-i \omega$\footnote{The choice of sign in $-i \omega$ makes positive values of the wave vector correspond to right-moving waves.}, these equations can be written in matrix form as
\begin{equation}
  \frac{d}{dx} \left(
    \begin{array}{c} V_a \\ I_a \\ V_b \\ I_b \end{array}
  \right)
  =
  i \omega \left( \begin{array}{cccc}
    0 & L_a & 0 & L_g \\
    C_a & 0 & -C_g & 0 \\
    0 & L_g & 0 & L_b \\
    -C_g & 0 & C_b & 0
  \end{array} \right)
  \left(
    \begin{array}{c} V_a \\ I_a \\ V_b \\ I_b \end{array}
  \right) \, .
\end{equation}
If we convert to the ingoing and outcoming wave amplitudes
\begin{align}
  a_\pm \equiv& \frac{V_a}{\sqrt{Z_a}} \pm \sqrt{Z_a} I_a \\
  b_\pm \equiv& \frac{V_b}{\sqrt{Z_b}} \pm \sqrt{Z_b} I_b \, ,
\end{align}
the equations of motion take on a particular nice form:
\begin{equation}
  \frac{d}{dx} \left(
    \begin{array}{c}
      a_+ \\ b_- \\ a_- \\ b_+
    \end{array}
  \right)
  =
  i \left(
    \begin{array}{cccc}
      \beta_a & -\chi & 0 & \kappa \\
      \chi & - \beta_b & -\kappa & 0 \\
      0 & -\kappa & -\beta_a & \chi \\
      \kappa & 0 & - \chi & \beta_b
    \end{array}
  \right)
  \left(
    \begin{array}{c}
      a_+ \\ b_- \\ a_- \\ b_+
    \end{array}
  \right)
\end{equation}
where $\beta \equiv \omega / v$ is the wave vector and
\begin{align}
  \kappa &\equiv \frac{\omega}{2} \left( \frac{L_g}{\sqrt{Z_a Z_b}} - C_g \sqrt{Z_a Z_b} \right) \\
  \chi &\equiv \frac{\omega}{2} \left( \frac{L_g}{\sqrt{Z_a Z_b}} + C_g \sqrt{Z_a Z_b} \right) \, .
\end{align}
Note that the general structure of this matrix is very similar to the matrix we had for two coupled oscillators.
We can learn a few important things just from this matrix representation:
\begin{itemize}
    \item If we want to make a directional coupler, i.e. remove the coupling between $a_\pm$ and $b_\pm$, then we must make $\kappa = 0$ which requires $\sqrt{L_g / C_g} = \sqrt{Z_a Z_b}$. In other words, to make a directional coupler, the coupling impedance must be the \emph{geometric mean} of the two line impedances.
    \item We discussed above the connection between the anti-diagonal terms and the rotating wave approximation. Therefore, we can in fact ignore the anti-diagonal terms pricisely when the rotating wave approximation should be valid, i.e. when $\kappa \ll \{\beta_a, \beta_b\}$ which corresponds to the coupling inductance and capacitance being a small fraction of the lines' self-capacitance and self-inductance.
\end{itemize}


\end{document}

\documentclass{article}

%Packages
\usepackage{amsmath}
\usepackage{amstext}
\usepackage{amssymb}
\usepackage{appendix}
\usepackage{coseoul}
\usepackage{enumerate}
\usepackage{graphicx}
\usepackage{import}
\usepackage{lscape}
\usepackage{modular}

\usepackage[pdfpagemode=UseNone,pdfstartview=FitH,colorlinks=true,linkcolor=blue,citecolor=blue,urlcolor=blue]{hyperref}
\usepackage[all]{hypcap}


% General physics constructs
\newcommand{\bra}[1]{\langle #1 |}
\newcommand{\ket}[1]{| #1 \rangle }
\newcommand{\braket}[2]{\langle #1|#2\rangle}
\newcommand{\bbraket}[3]{ \langle #1 | #2 | #3 \rangle }
\newcommand{\norm}[1]{\| #1\|}
\newcommand{\avg}[1]{\left \langle #1 \right \rangle}
\newcommand{\angavg}[1]{\left \langle #1 \right \rangle}
\newcommand{\abs}[1]{\left \lvert #1 \right \rvert}
\newcommand{\VS}{\textit{\textbf{V}}}
\newcommand{\Tr}{\textrm{Tr}}
\renewcommand{\Re}{\textrm{Re}}
\renewcommand{\Im}{\textrm{Im}}
\newcommand{\basis}[1]{\{\ket{#1}\}}

\newcommand{\omegaqubit}{\omega_{10}}

% Figures. Example usage:
% \quickfig{\columnwidth}{my_image}{This is the caption}{fig:my_fig}
\DeclareRobustCommand{\quickfig}[4]{
\begin{figure}
\begin{centering}
\includegraphics[width=#1]{#2}
\par\end{centering}
\caption{#3}
\label{#4}
\end{figure}
}

\DeclareRobustCommand{\quickwidefig}[4]{
\begin{figure*}[h]
\begin{centering}
\includegraphics[width=#1]{#2}
\par\end{centering}
\caption{#3}
\label{#4}
\end{figure*}
}


\title{Harmonic Oscillator}
\author{Daniel Sank \\ \small{University of California Santa Barbara} \\ \small{Presently Google Quantum AI}}
\date{23 February 2009}

\begin{document}
\maketitle

\section{Impulse Response}

The equation for a damped harmonic oscillator is \begin{equation}
\ddot{\phi}(t)+2\beta\dot{\phi}(t)+\omega_{0}^{2}\phi(t)=j(t) \end{equation}
In the usual way we turn this into an abstract equation \begin{equation}
(D_{t}^{2}+2\beta D_{t}+\omega_{0}^{2})\ket{\phi}=\ket{j} \end{equation}
Our Fourier convention will be
\begin{equation}
\braket{t}{\omega}=e^{i\omega t}
\end{equation}
which is the common choice in electrical engineering.
However, it is the \emph{opposite} of what is normally used in physics, notably in Schrodinger's equation.
With this convention $-iD_t$ is Hermitian.
Going to the $\ket{\omega}$ basis we get
\begin{align}
\bbraket{\omega}{D_t^2 + 2\beta D_t + \omega_0^2}{\phi} &= \braket{\omega}{j} \\
(-\omega^{2} + 2 i \beta\omega + \omega_{0}^{2}) \phi(\omega) & = j(\omega)\\
\phi(\omega) & = \frac{j(\omega)}{-\omega^{2} + 2 i \beta \omega + \omega_{0}^{2}} \, .
\end{align}
Taking $j(t)=A\delta(t-t_{0})$ we get $\braket{\omega}{j} = A e^{-i \omega t_{0}}$.
We can factor the denominator as $-(\omega-\omega_{+})(\omega-\omega_{-})$
where
\begin{equation}
\omega_{+} = i \beta + \omega_0' \qquad \omega_{-} = i \beta - \omega_0' \label{eq:omega_plus_minus}
\end{equation}
and
\begin{equation}
\omega_0' = \omega_0 \sqrt{ 1 - \left( \beta / \omega_0 \right) ^2 } \, .
\end{equation}
Therefore we have \begin{equation}
\phi(\omega) = \frac{-A e^{-i \omega t_0}}{(\omega - \omega_{+})(\omega-\omega_{-})} \, .
\end{equation}
Upon Fourier transform
\begin{equation}
\phi(t) = 
\int\frac{-Ae^{i \omega(t - t_0)}}{(\omega-\omega_{+})(\omega-\omega_{-})} \,
\frac{d\omega}{2\pi} \, .
\end{equation}
Consider the case $t < t_0$.
In that case we close the contour in the lower half plane.
Since both poles $\omega_\pm$ are in the upper half plane the integral is zero.
This reflects causality: the oscillator can't know about the impulse before it happens.
For the case $t > t_{0}$ we close in the upper half plane.
Now the contour encloses the poles and we will get a nonzero result.
Let $\tau \equiv t - t_0$.
Then the residue theorem gives
\begin{align}
\phi(t) &=
\frac{-A 2\pi i}{2 \pi}
\left(
\frac{e^{i \omega_+ \tau}}{\omega_+ - \omega_-} +
\frac{e^{i \omega_- \tau}}{\omega_- - \omega_+}
\right) \\
&= \frac{A}{\omega_0'} e^{- \beta \tau} \sin(\omega_0' \tau)
\end{align}
We see that the system oscillates with a frequency close to $\omega_0$ and decays with a rate given by the amplitude damping parameter $\beta$.
Note that this solution is valid only when $\omega_0 > \beta$.

In the limit $\beta \ll \omega_0$ we find $E(t) = E_{0}e^{-2 \beta t}$, where $E_0=\frac{1}{2} m A^{2}$ is the initial energy of the system.
Note that $A$ is the initial velocity.

Defining the quality factor $Q$ by
\begin{equation}
Q \equiv \frac{\textrm{Energy stored}}{\textrm{Energy dissipated per radian}} \, ,
\end{equation}
we find
\begin{equation}
Q
= \frac{E(t)}{dE/d\textrm{rad}}
= \frac{E(t)}{-(dE/dt)(dt/d\textrm{rad})}
= \frac{E(t)}{2\beta E(t)\frac{1}{\Delta\omega}}
= \frac{\Delta\omega}{2\beta}\approx\frac{\omega_{0}}{2\beta} \, .
\end{equation}
This relation is also written
\begin{equation}
Q \approx \omega_0 / \kappa
\end{equation}
where $\kappa$ is the energy decay rate of the oscillator.


\section{Frequency Response}

Now we investigate the case in which $j(t)=A\cos(\Omega t)$.
In this case
\begin{equation}
j(\omega)=\frac{A}{2}(2\pi)\left[\delta(\omega-\Omega)+\delta(\omega+\Omega)\right] \, .
\end{equation}
Therefore, by modifying equations from the previous section we have
\begin{align}
\phi(\omega)
& = \frac{-A}{2}(2\pi) \int \frac{d\omega}{2\pi} e^{i \omega t} \frac{\delta(\omega-\Omega)+\delta(\omega+\Omega)}{\omega^{2} - 2i\beta\omega - \omega_0^2} \\
& =
\frac{-A}{2}
\left[
\frac{e^{i \Omega t}}{\Omega^{2} - 2i\beta\Omega - \omega_0^2} +
\frac{e^{-i \Omega t}}{\Omega^2 + 2i\beta\Omega - \omega_0^2}
\right] \\
&= \Re \left[ e^{i \Omega t} \frac{-A}{\Omega^2 - 2i\beta\Omega - \omega_0^2}
\right] \label{eq:phasor_form}
\end{align}
This is a very special form known as {}``phasor'' form.
The idea of phasors is that in a linear system driven by a sinusoidal drive, all variables will oscillate with the same frequency as the drive.
Any dynamical variable, in our case $\phi$, differs from the drive only in amplitude and phase.
We see this here: the time dependence of $\phi$ is sinusoidal at angular frequency $\Omega$.
The crucial relation is that if we have two signals at the same frequency which can be expressed as \begin{equation}
f(t) = \Re[ e^{i\omega t} \hat{f}] \qquad
g(t) = \Re[ e^{i\omega t} \hat{g}]
\end{equation}
where $\hat{f}$ and $\hat{g}$ are complex numbers, then we have \begin{equation}
\langle f(t)g(t)\rangle_{t}=\frac{1}{2}\Re[\hat{f}^{*}\hat{g}] \end{equation}
where the left hand is a time average.
Our expression for $\phi(t)$ has this form if we take
\begin{equation}
\hat{\phi} = \frac{-A}{\Omega^{2} - 2i\beta\Omega - \omega_0^2} \, .
\end{equation}
Therefore, \begin{equation}
  \langle\phi(t)^{2}\rangle_{t}
  = \frac{1}{2} \frac{A^{2}}{(\Omega^{2}-\omega_0^{2})^{2}+(2\beta\Omega)^2}
  \, .
\end{equation}
This function maximizes for $\Omega=\pm\sqrt{\omega_{0}^{2}-2\beta^{2}}$.
Note that this is not the same as the free oscillation frequency found from the impulse response.

\subsection{Resonance}

We define the resonance frequency as that frequency at which power flow from the drive to the system is unidirectional.
This happens when $\dot{\phi}(t)$ is in phase with $j(t)$.
Velocity and position are a quarter cycle out of phase, so the resonance occurs when the position is a quarter cycle shifted from the drive.
In other words, the resonance happens when the denominator of the phasor is imaginary, which occurs for
\begin{equation}
(\text{resonance}) \qquad \Omega = \pm \omega_0 \, .
\end{equation}

\subsection{Phase shift}

It is useful to look at the relative phase shift between the drive and the response.
To do this we need to find the argument of $\hat{\phi}$.
Note that
\begin{equation}
f(t)
= \Re[e^{i \Omega t}\hat{f}]
= \Re[e^{i \Omega t}|f|e^{i\theta}]
=|f|\cos(\Omega t + \theta)]
\end{equation}
where $\theta$ is the phase of $\hat{f}$.
Therefore, a positive $\theta$ indicates that the response leads the source, and a negative $\theta$ indicates that the response lags the source.

\subsection{Lorentzian approximation}

We now express the resonance in the so-called Lorentzian form.
\begin{align}
\phi(t)
&= \Re \left[ e^{i\Omega t} \frac{-A}{\Omega^2 - \omega_0^2 - 2i\beta \Omega} \right] \nonumber \\
&= \frac{A \cos \left( \Omega t + \delta \right)}{\sqrt{\left( \Omega^2 - \omega_0^2 \right)^2 + (2 \beta \Omega)^2}}
\end{align}
where $\delta$ is a phase we don't care about.
We can re-express this approximately as
\begin{equation}
\phi(t) \approx \frac{(A \beta / \Omega) \cos(\Omega t + \delta)}{\left( \Omega - \omega_0 \right)^2 + 2\beta^2} \, .
\end{equation}


\section{Phasor form}

It is possible to find approximation equations of motion for the driven oscillator in the case that the spectral content of the drive is reasonably well localized around the resonance frequency.
Consider the frequency domain equation of motion
\begin{equation}
  \phi(\omega) = \frac{-J(\omega)}{\omega^2 - 2 i \beta \omega - \omega_0^2}
  = - \frac{J(\omega)}{(\omega - \omega_+)(\omega - \omega_-)} \, .
\end{equation}
Using the method of partial fractions, we can rewrite the double pole as a sum of single poles
\begin{equation}
  \frac{1}{(\omega - \omega_+)(\omega - \omega_-)}
  = \frac{X}{\omega - \omega_+} + \frac{Y}{\omega - \omega_-}
\end{equation}
with unknown constants $X$ and $Y$.
Multiplying through by $(\omega - \omega_+)(\omega - \omega_-)$ and solving gives
\begin{align*}
  X =& \frac{1}{\omega_+ - \omega_-} =   \frac{1}{2 \omega_0'} \\
  Y =& \frac{1}{\omega_- - \omega_+} = - \frac{1}{2 \omega_0'}
\end{align*}
so
\begin{equation}
  \phi(\omega) = - \frac{J(\omega)}{2 \omega_0'}
  \left[ \frac{1}{\omega - \omega_+} - \frac{1}{\omega - \omega_-} \right] \, .
\end{equation}
If the drive is a narrow band signal near a central frequency $\Omega$, we can write it as
\begin{equation}
  J(t)
  = I(t) \cos(\Omega t) - Q(t) \sin(\Omega t)
  = \Re \left[ z(t) \exp \left( i \Omega t \right) \right] \, ,
\end{equation}
where $z(t) \equiv I(t) + i Q(t)$.
In this case we have
\begin{align*}
  \tilde{J}(\omega)
  &= \frac{1}{2} \left(
      \tilde{I}(\omega - \Omega) + i \tilde{Q}(\omega - \Omega)
    + \tilde{I}(\omega + \Omega) - i \tilde{Q}(\omega + \Omega)
  \right) \\
  &= \frac{1}{2} \left(
      \tilde{z}(\omega - \Omega) + \tilde{z}(-\omega - \Omega)^*
  \right)
\end{align*}
where we've used the fact that $\tilde{I}(\omega)^* = \tilde{I}(-\omega)$ and $\tilde{Q}(\omega)^* = \tilde{Q}(-\omega)$ because $I(t)$ and $Q(t)$ are real.
Combining results, we have
\begin{align*}
  \tilde{\phi}(\omega)
  =& - \frac{1}{4 \omega_0'}
  \left( \tilde{z}(\omega - \Omega) + \tilde{z}(-\omega - \Omega)^* \right)
  \left(
    \frac{1}{\omega - \omega_+} - \frac{1}{\omega - \omega_-}
  \right) \\
  =& - \frac{1}{4 \omega_0'}
  \left(
      \frac{\tilde{z}(\omega - \Omega)}{\omega - \omega_+}
    - \frac{\tilde{z}(-\omega - \Omega)^*}{\omega - \omega_-}
    \underbrace{
      - \frac{\tilde{z}(\omega - \Omega)}{\omega - \omega_-}
      + \frac{\tilde{z}(-\omega - \Omega)^*}{\omega - \omega_+}
    }_\text{small}
  \right)
\end{align*}
The second two terms in parentheses are small because e.g. $\tilde{z}(\omega - \Omega)$ is large when $\omega$ is near $+\Omega$, but $1 / (\omega - \omega_-)$ is large for $\omega$ near $-\Omega$.
Keeping only the larger terms, we get
\begin{equation}
  \tilde{\phi}(\omega) =
  - \frac{1}{4 \omega_0'}
  \left(
      \frac{\tilde{z}(\omega - \Omega)}{\omega - \omega_+}
    - \frac{\tilde{z}(-\omega - \Omega)^*}{\omega - \omega_-}
  \right) \, .
\end{equation}
Note that $\tilde{\phi}(-\omega) = \tilde{\phi}(\omega)^*$ as appropriate for $\phi(t) \in \mathbb{R}$.
Because of that symmetry, we can write it as a Fourier transform over only positive frequencies:
\begin{align*}
  \phi(t)
  &= 2 \Re \int_0^\infty \frac{d\omega}{2\pi} \, e^{i \omega t} \tilde{\phi}(\omega) \\
  &= 2 \Re \int_0^\infty \frac{d\omega}{2\pi} \, e^{i \omega t} \left( \frac{-1}{4 \omega_0'} \right)
    \left(
      \frac{\tilde{z}(\omega - \Omega)}{\omega - \omega_+}
      - \underbrace{\frac{\tilde{z}(-\omega - \Omega)^*}{\omega - \omega_-}}_\text{small}
    \right) \\
  (\text{define } \omega \equiv \Omega + \delta \omega)
  &= \Re
  \left(
    e^{i \Omega t}
    \int_{-\Omega}^\infty \frac{d \delta \omega}{2\pi} \,
    e^{i \delta \omega t}
    \left( \frac{-1}{2 \omega_0'} \right)
    \frac{\tilde{z}(\delta \omega)}{\delta \omega - \Delta - i \beta}
  \right)
\end{align*}
where $\Delta \equiv \omega_0' - \Omega$.
If $\tilde{z}$ is band limited to $[-B, B]$ where $B \ll \omega_0'$, then we can rewrite the limits of the integration as $\pm \infty$:
\begin{equation}
  \phi(t) = \Re
  \left(
    e^{i \Omega t}
    \underbrace{
      \int_{-\infty}^\infty \frac{d \delta \omega}{2\pi} \,
      e^{i \delta \omega t}
      \left( \frac{-1}{2 \omega_0'} \right)
      \frac{\tilde{z}(\delta \omega)}{\delta \omega - \Delta - i \beta}
    }_{\psi(t)}
  \right)
\end{equation}
The underbraced part $\psi(t)$ is the part of the oscillator's motion that comes on top of a free oscillation at frequency $\Omega$.
In other words, we're in the rotating frame of the drive.
By inspection, the time dynamics of $\psi(t)$ are governed by the equation
\begin{equation}
  \dot{\psi}(t) = \left(\frac{-i}{2 \omega_0'}\right) z(t) + (i \Delta - \beta ) \psi(t)
\end{equation}

\section{Rotating Mode Formalism}

There's an interesting way to write down the equations for a harmonic oscillator that's useful when you have coupled problems or problems with time varied parameters.
An as example consider a parallel LRC circuit.
As dynamical variables take the flux in the inductor, $\Phi$, and the charge on the capacitor, $Q$.
See figure ??.
The two first order equations of motion are\begin{equation}
\dot{\Phi}=Q/C\qquad\dot{Q}+Q/(RC)=-\Phi/L \end{equation}
Note that if $R\rightarrow\infty$ these can be written in a nearly Hamiltonian form \begin{equation}
\frac{d}{dt}\left[ \begin{array}{c} \Phi \\ Q\end{array} \right] =
\left[\begin{array}{cc} 0 & 1/C \\ -1/L & 0 \end{array} \right]
\left[\begin{array}{c} \Phi\\ Q\end{array}\right] \end{equation}
If the off-diagonal elements were equal to one another this would be a true Hamiltonian.
This is easy to do if we simply re-scale the variables, \begin{equation}
X \equiv \left( C/L \right)^{1/4} \Phi \qquad Y \equiv \left( L/C \right)^{1/4} Q \end{equation}
Now we find, in the case that $R\rightarrow\infty$ \begin{displaymath}
\frac{d}{dt}\left[ \begin{array}{c} X \\ Y \end{array} \right] = \omega_{0} \left[ \begin{array}{cc} 0 & 1 \\ -1 & 0 \end{array}\right]
\left[ \begin{array}{c} X \\ Y \end{array} \right] \end{displaymath}
where $\omega_{0}\equiv1/\sqrt{LC}$.
This is great because we have the anti-symmetric Hamiltonian matrix.
Adding the loss back in gives \begin{displaymath}
\frac{d}{dt} \left[ \begin{array}{c} X \\ Y \end{array} \right] =
\omega_0 \left[ \begin{array}{cc} 0 & 1 \\ -1 & -1/Q \end{array} \right]
\left[ \begin{array}{c} X \\ Y \end{array} \right] \end{displaymath}
where $Q\equiv \omega_0 R C$.

We can make one further simplification if we want to get rid of the $\omega_0$ out front.
Make one more rescaling,\begin{equation}
\xi\equiv\omega_{0}t \end{equation}
Then \begin{equation}
\frac{d}{dt}=\omega_{0}\frac{d}{d\xi}\end{equation}
Therefore the equations of motion for $X$ and $Y$ are \begin{equation}
\frac{d}{d\xi}\left[\begin{array}{c} X\\ Y\end{array}\right] = 
\left[\begin{array}{cc} 0 & 1\\ -1 & -1/Q \end{array}\right]
\left[\begin{array}{c} X\\ Y\end{array}\right]\end{equation}

\subsection{Normal Modes}
We can now find the normal modes of this equation.
To do so we must find the eigenvalues and eigenvectors of the matrix, \begin{equation}
\left[ \begin{array}{cc} 0 & 1 \\ -1 & -\epsilon \end{array} \right] \end{equation}
where here $\epsilon \equiv 1/Q$.
The eigenvalue equation is \begin{equation}
\lambda^2 + \epsilon \lambda + 1 = 0 \end{equation}
from which the eigenvalues are found to be \begin{equation}
\lambda_{\pm} = \pm i \sqrt{1 - \frac{\epsilon^2}{4}} - \frac{\epsilon}{2} \end{equation}
Now, we want to find eigenvectors $a^+$ and $a^-$ that satisfy the equations \begin{equation}
\dot{a}^+=\lambda_+ a^+ \quad \textrm{and} \quad \dot{a}^-=\lambda_- a^- \end{equation}
where the overdots denote differentiation with respect to $\xi$.
Write $a$ (denoting here either one of the eigenvectors) as $a = X+\alpha Y$.
Then taking the derivative with respect to $\xi$ of both sides gives \begin{eqnarray}
\dot{a} &=& \dot{X}+\alpha \dot{Y}\\
&=& Y+\alpha(-X-\epsilon Y)\\
&=& -\alpha \left( X+\left[ \epsilon - \frac{1}{\alpha} \right] Y \right) \end{eqnarray}
Since we're expecting $a$ to satisfy $\dot{a} = \lambda a$ we get \begin{eqnarray}
\dot{a} &=& \lambda a \\
-\alpha \left( X+\left[ \epsilon - \frac{1}{\alpha} \right] Y \right) &=& \lambda \left( X + \alpha Y \right) \\
\rightarrow \quad \lambda=-\alpha &\quad& 1-\epsilon \alpha = \lambda \alpha \end{eqnarray}
where the last line follows by equating coefficients of $X$ and $Y$.
We can check that these relations are self consistent by plugging the first into the second \begin{eqnarray}
1+\epsilon \lambda &=& -\lambda^2 \\
1+\epsilon \left( -\frac{\epsilon}{2} \pm i \sqrt{1-\frac{\epsilon^2}{4}} \right) &=&
-\left(-\frac{\epsilon}{2}\pm i \sqrt{1-\frac{\epsilon^2}{4}} \right) \\
1-\frac{\epsilon^2}{2}\pm i \epsilon \sqrt{\cdots} &=&
-\frac{\epsilon^2}{4}\pm i \epsilon\sqrt{\cdots} + 1 -\frac{\epsilon^2}{4}\\
0 &=& 0 \end{eqnarray}
so the relations are indeed consistent.

Now we can write down the eigenvectors along with their time dependence, \begin{eqnarray}
a^- = \frac{1}{\sqrt{2}}\left( X -\lambda_- Y \right) &\qquad&
a^+ = \frac{1}{\sqrt{2}}\left( X -\lambda_+ Y \right) \\
a^-(t) = a^-(0) e^{\omega_0 \lambda_- t} &\qquad& a^+(t) = a^+(0) e^{\omega_0 \lambda_+ t} \end{eqnarray}
Here we've put the $\omega_0$'s back in because we're using $t$ instead of $\xi$.

In the case that $Q \gg 1$ the eigenvalues simplify to \begin{equation}
\lambda_{\pm} = \pm i -\frac{\epsilon}{2} \end{equation}
so that we get \begin{eqnarray}
a^- = \frac{1}{\sqrt{2}}\left[ X + \left( i + \frac{\epsilon}{2} \right) Y \right] &\qquad&
a^+ = \frac{1}{\sqrt{2}}\left[ X + \left(-i + \frac{\epsilon}{2} \right) Y \right] \\
a^-(t) = a^-(0) \exp \left[ \omega_0 \left(-i - \frac{\epsilon}{2}\right) t \right]
& \qquad &
a^+(t) = a^+(0) \exp \left[ \omega_0 \left( i - \frac{\epsilon}{2}\right) t \right] \end{eqnarray}
Note that the time evolution has the same sense from what is typically used in quantum mechanics for the $a$ and $a^{\dagger}$ operators.

\section{Coupled oscillators}

\levelstay{Coupled oscillators}

Consider two LC oscillators coupled through a coupling capacitor $C_g$.
Kirchhoff's equations of motion for this circuit can be written in matrix form
as
\begin{equation}
  \left( \begin{array}{c} V_1 \\ V_2 \end{array} \right)
  =
  \left( \begin{array}{cc}
    (1 + \epsilon_1)/\omega_1^2 & - \epsilon_1  \omega_1^2 \\
    - \epsilon_2  \omega_2^2 & (1 + \epsilon_2)/\omega_2^2
  \end{array} \right)
  \left( \begin{array}{c} \ddot{V}_1 \\ \ddot{V}_2 \end{array} \right)
\end{equation}
where $\epsilon_i \equiv C_g / C_i$ and $\omega_i^2 \equiv 1 / L_i C_i$.
The normal mode frequencies of the system are the square roots of the reciprocals of the eigenvalues of the matrix.
We could compute those eigenvalues explicitly, but it's a mess and there's a better way to do it that we'll show below.
However, in the particular case that $L_1 = L_2$ and $C_1 = C_2$, then $\epsilon_1 = \epsilon_2 \equiv \epsilon$, $\omega_1 = \omega_2 \equiv \omega_0$ and the matrix takes the form
\begin{equation}
  \frac{1}{\omega_0^2} \left( 
    (1 + \epsilon) \mathbb{I} - \epsilon \sigma_x
  \right)
\end{equation}
and the eigenvalues are
\begin{equation}
  \lambda_{\pm} = \frac{1}{\omega_0^2} ( 1 + \epsilon \pm \epsilon)
\end{equation}
so the normal mode frequencies are
\begin{equation}
  \omega_+ = \omega_0 \quad
  \text{and} \quad
  \omega_- = \frac{\omega_0}{\sqrt{1 + 2 \epsilon}}
    \approx \omega_0 (1 - \epsilon) \, .
\end{equation}
Note that the difference bewteen the two normal mode frequencies is
\begin{equation}
  \omega_+ - \omega_- \approx \omega_0 \epsilon
  = \omega_0 \frac{C_g}{C}
  = 2 \times \underbrace{\left(\frac{\omega}{2} \frac{C_g}{C} \right)}_g
  = 2 g
\end{equation}
which justifies our definition of the coupling strength $g$ because the frequency splitting of two oscillators on resonance is supposed to be $2g$.

\subimportlevel{./}{coupled_oscillators_hamiltonian.tex}{1}


\subsection{Hamiltonian approach}

The Hamiltonian for two LC oscillators coupled through a coupling capacitor $C_g$ is
\begin{align}
  H
  &= \frac{\Phi_1^2}{2 L_1} + \frac{\Phi_2^2}{2 L_2} \\
  &+ \frac{Q_1^2}{2 C_1} + \frac{Q_2^2}{2 C_2} \\
  &+ \frac{Q_1 Q_2}{C_g''}
\end{align}
where
\begin{align}
  C_g'' &\equiv \frac{C_1' C_2' - C_g^2}{C_g} \\
  C_i' &\equiv C_i + C_g \, .
\end{align}
We transform the Hamiltonian in two steps.
First we define
\begin{align*}
  X_i &\equiv \frac{1}{\sqrt{2 \hbar}} \frac{1}{\sqrt{Z_i}} \Phi_i \\
  Y_i &\equiv \frac{1}{\sqrt{2 \hbar}} \sqrt{Z_i} Q_i
\end{align*}
where $Z_i \equiv \sqrt{L_i / C_i''}$.
With these new variables, the Hamiltonian is
\begin{equation}
  H/\hbar =
    \omega_1 \left(X_1^2 + Y_1^2) \right)
  + \omega_2 \left(X_2^2 + Y_2^2) \right)
  + 2 \left( \frac{C_1'' C_2''}{C_g''} \right) \sqrt{\omega_1'' \omega_2''} Y_1 Y_2 \, .
\end{equation}
Note that $[X, Y] = i/2$ so Hamilton's equations of motion are a little different than usual.
Anyway, we next use $a = X + i Y$ and $a^\dagger = X - i Y$ to get
\begin{equation}
  H / \hbar =
    \omega_1'' a_1^\dagger a_1
  + \omega_2'' a_2^\dagger a_2
  - g (a_1 - a_1^\dagger) (a_2 - a_2^\dagger)
\end{equation}
where
\begin{equation}
  g \equiv \frac{1}{2} \frac{\sqrt{C_1'' C_2''}}{C_g''} \sqrt{\omega_1'' \omega_2''} \, .
\end{equation}
Hamilton's equations of motion in this representation are
\begin{equation*}
  \dot{a} = -i \frac{\partial H/\hbar}{\partial a^\dagger}
  \quad \text{and} \quad
  \dot{a}^\dagger = i \frac{\partial H/\hbar}{\partial a} \, .
\end{equation*}
Therefore the equations of motion for the system are
\begin{align*}
  \dot a_1 &= -i \omega_1'' a_1 -i g a_2 + i g a_2^\dagger \\
  \dot a_2 &= -i \omega_2'' a_2 -i g a_1 + i g a_1^\dagger
\end{align*}
and the equations for $a^\dagger$ follow by conjugation.
These equations can be written in matrix form as
\begin{equation}
  \frac{d}{dt}
  \left( \begin{array}{c}
    a_1 \\ a_1^\dagger \\ a_2 \\ a_2^\dagger
  \end{array} \right)
  =
  \left( \begin{array}{cccc}
    -i \omega_1'' & 0 & -i g & i g \\
    0 & i \omega_1'' & -i g & ig \\
    -i g & i g & -i \omega_2'' & 0 \\
    -i g & i g & 0 & i \omega_2''
  \end{array} \right)
  \left( \begin{array}{c}
    a_1 \\ a_1^\dagger \\ a_2 \\ a_2^\dagger
  \end{array} \right)
  \, .
\end{equation}
The eigenvalues are
\begin{equation}
  \lambda^2 =
  \pm \frac{1}{2}
  \left(
    \omega_1''^2 + \omega_2''^2
    \pm \sqrt{(\omega_1''^2 - \omega_2''^2)^2 + 16 g^2 \omega_1'' \omega_2''}
  \right)
\end{equation}


\section{Coupled transmission lines}

% artifact header: coupled_lines
WARNING: This discussion probably has errors. In particular, I'm not sure what the sign of the couopling inductance should be. I'm also worried that I messed up the definitions of $L$ and $C$ by somehow not taking into account the effects of coupling on each mode's ``canonical'' inductance and capacitance.

Consider the coupled transmission lines shown in Figure \ref{fig:coupled_lines:circuit_diagram}.
The capacitances $C_a$ and $C_b$ and the inductaces $L_1$ and $L_2$ are the capcitances and inductances \emph{per length} of each transmission line
The coupling capacitance $C_g$ and coupling inductance $L_g$ are similarly per length quantities.

\quickfig{\columnwidth}{coupled_lines.pdf}{Two coupled transmission lines.}{fig:coupled_lines:circuit_diagram}

Of course, we can start an analysis via Kichhoff's laws, which for the illustrated section of line are
\begin{align*}
  V_a(x) - V_a(x + dx) =& L_a \dot I_a + L_g \dot I_b \\
  I_a(x) - I_a(x + dx) =& \dot V_a C_a + \left( \dot V_a - \dot V_b \right) C_g \\
  V_b(x) - V_b(x + dx) =& L_b \dot I_b + L_g \dot I_a \\
  I_b(x) - I_b(x + dx) =& \dot V_b C_b + \left( \dot V_b - \dot V_a \right) C_g
  \, .
\end{align*}
In the limit $dx \rightarrow 0$, these equations become differential equations
\begin{align}
  - \frac{\partial V_a}{\partial x} &= \frac{\partial}{\partial t}
    \left( L_a I_a + L_g I_b \right) \nonumber \\
  - \frac{\partial I_a}{\partial x} &= \frac{\partial}{\partial t}
    \left( (C_a + C_g)V_a - C_g V_b \right) \nonumber \\
  - \frac{\partial V_b}{\partial x} &= \frac{\partial}{\partial t}
    \left( L_b I_b + L_g I_a \right) \nonumber \\
  - \frac{\partial I_b}{\partial x} &= \frac{\partial}{\partial t}
    \left( (C_b + C_g)V_b - C_g V_a \right)
  \, .
\end{align}
Assuming sinusoidal time dependence at frequency $\omega$, and therefore converting time derivatives to $-i \omega$\footnote{The choice of sign in $-i \omega$ makes positive values of the wave vector correspond to right-moving waves.}, these equations can be written in matrix form as
\begin{equation}
  \frac{d}{dx} \left(
    \begin{array}{c} V_a \\ I_a \\ V_b \\ I_b \end{array}
  \right)
  =
  i \omega \left( \begin{array}{cccc}
    0 & L_a & 0 & L_g \\
    C_a & 0 & -C_g & 0 \\
    0 & L_g & 0 & L_b \\
    -C_g & 0 & C_b & 0
  \end{array} \right)
  \left(
    \begin{array}{c} V_a \\ I_a \\ V_b \\ I_b \end{array}
  \right) \, .
\end{equation}
If we convert to the ingoing and outcoming wave amplitudes
\begin{align}
  a_\pm \equiv& \frac{V_a}{\sqrt{Z_a}} \pm \sqrt{Z_a} I_a \\
  b_\pm \equiv& \frac{V_b}{\sqrt{Z_b}} \pm \sqrt{Z_b} I_b \, ,
\end{align}
the equations of motion take on a particular nice form:
\begin{equation}
  \frac{d}{dx} \left(
    \begin{array}{c}
      a_+ \\ b_- \\ a_- \\ b_+
    \end{array}
  \right)
  =
  i \left(
    \begin{array}{cccc}
      \beta_a & -\chi & 0 & \kappa \\
      \chi & - \beta_b & -\kappa & 0 \\
      0 & -\kappa & -\beta_a & \chi \\
      \kappa & 0 & - \chi & \beta_b
    \end{array}
  \right)
  \left(
    \begin{array}{c}
      a_+ \\ b_- \\ a_- \\ b_+
    \end{array}
  \right)
\end{equation}
where $\beta \equiv \omega / v$ is the wave vector and
\begin{align}
  \kappa &\equiv \frac{\omega}{2} \left( \frac{L_g}{\sqrt{Z_a Z_b}} - C_g \sqrt{Z_a Z_b} \right) \\
  \chi &\equiv \frac{\omega}{2} \left( \frac{L_g}{\sqrt{Z_a Z_b}} + C_g \sqrt{Z_a Z_b} \right) \, .
\end{align}
Note that the general structure of this matrix is very similar to the matrix we had for two coupled oscillators.
We can learn a few important things just from this matrix representation:
\begin{itemize}
    \item If we want to make a directional coupler, i.e. remove the coupling between $a_\pm$ and $b_\pm$, then we must make $\kappa = 0$ which requires $\sqrt{L_g / C_g} = \sqrt{Z_a Z_b}$. In other words, to make a directional coupler, the coupling impedance must be the \emph{geometric mean} of the two line impedances.
    \item We discussed above the connection between the anti-diagonal terms and the rotating wave approximation. Therefore, we can in fact ignore the anti-diagonal terms pricisely when the rotating wave approximation should be valid, i.e. when $\kappa \ll \{\beta_a, \beta_b\}$ which corresponds to the coupling inductance and capacitance being a small fraction of the lines' self-capacitance and self-inductance.
\end{itemize}


\end{document}

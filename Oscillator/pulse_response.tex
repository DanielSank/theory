%% LyX 1.6.6 created this file.  For more info, see http://www.lyx.org/.
%% Do not edit unless you really know what you are doing.
\documentclass[english]{revtex4}
\usepackage[T1]{fontenc}
\usepackage[latin9]{inputenc}
\usepackage{amstext}
\usepackage{esint}

\makeatletter
%%%%%%%%%%%%%%%%%%%%%%%%%%%%%% Textclass specific LaTeX commands.
\@ifundefined{textcolor}{}
{%
 \definecolor{BLACK}{gray}{0}
 \definecolor{WHITE}{gray}{1}
 \definecolor{RED}{rgb}{1,0,0}
 \definecolor{GREEN}{rgb}{0,1,0}
 \definecolor{BLUE}{rgb}{0,0,1}
 \definecolor{CYAN}{cmyk}{1,0,0,0}
 \definecolor{MAGENTA}{cmyk}{0,1,0,0}
 \definecolor{YELLOW}{cmyk}{0,0,1,0}
 }

%%%%%%%%%%%%%%%%%%%%%%%%%%%%%% User specified LaTeX commands.
%Some extra stuff to use in the document
\newcommand{\bra}[1]{\langle #1|}
\newcommand{\ket}[1]{|#1\rangle}
\newcommand{\braket}[2]{\langle #1|#2\rangle}

\makeatother

\usepackage{babel}

\begin{document}

\title{Oscillator Pulse Response}


\author{Daniel Sank}


\date{19 April 2011}

\maketitle

\section{Conventions}

The fourier convention used here is\begin{eqnarray*}
f(\omega) & = & \int f(t)e^{-i\omega t}dt\\
f(t) & = & \int f(\omega)e^{i\omega t}\frac{d\omega}{2\pi}\end{eqnarray*}



\section{Input-output}

If you have a wave travelling down a transmission with some impedance
$Z(\omega)$ at the termination, then you have the following relation\[
V_{\textrm{in}}(\omega)+V_{\textrm{out}}(\omega)=V(\omega)\]
where $V(\omega)$ is the voltage accross the termination impedance
and $V_{\textrm{in/out}}$ are the ingoing and outgoing voltage amplitudes.
The input and output amplitudes are related by\[
V_{\textrm{out}}(\omega)=\Gamma(\omega)V_{\textrm{in}}(\omega)\]
with the reflection coefficient $\Gamma$ given by\[
\Gamma(\omega)=\frac{Z(\omega)/Z_{0}-1}{Z(\omega)/Z_{0}+1}\]
where $Z_{0}$ is the characteristic impedance of the line. Using
these relations we can write down the voltage accross the termination
impedance in terms of the incoming voltage amplitude,\[
V(\omega)=V_{\textrm{in}}(\omega)\left[1+\Gamma(\omega)\right]\]
To find $V(t)$ we basically just inverse fourier transform this expression.
To do that we have to work out $\Gamma(\omega)$ and $V_{\textrm{in}}(\omega)$.


\section{Harmonic oscillator termination}

Consider the case that the termination is a harmonic oscillator. Then
the termination impedance is \begin{equation}
Z(\omega)/Z_{0}=\frac{ix}{1-x^{2}+i\epsilon x} \end{equation}
where $x\equiv\omega/\omega_{0}$, $\omega_{0}\equiv1/\sqrt{LC}$, and $\epsilon\equiv1/Q$. We can plug this into the expression for gamma, (ALL SUBSEQUENT OCURRANCES OF $\eta$ should be replaced by 1)\begin{eqnarray*}
\Gamma(\omega) & = & \frac{Z(\omega)/Z_{0}-1}{Z(\omega)/Z_{0}+1}\\
\Gamma(\omega) & = & \frac{\frac{ix\eta}{1-x^{2}+i\epsilon x}-1}{1+\frac{ix\eta}{1-x^{2}+i\epsilon x}}\\
\Gamma(\omega) & = & -\frac{1-x^{2}+i\epsilon x-i\eta x}{1-x^{2}+i\epsilon x+i\eta x}\\
1+\Gamma(\omega) & = & 2\frac{i\eta x}{1-x^{2}+i\epsilon x+i\eta x}\end{eqnarray*}
This is actually a very interesting equation. The imaginary parts
in the denominator are the effective damping terms. Note that one
of them depends on the intrinsic quality factor of the resonator,
expressed by $\epsilon$, while the other term comes from the transmission
line characteristic impedance via $\eta\equiv\omega_{0}L/Z_{0}$.
This means that there is an effective daming that gets larger as the
line impedance gets smaller because the line is damping the resonator.
This is usually discussed via the {}``coupling $Q$,'' which would
be $1/\eta$. This suggests that we should collect the two damping
terms into one, which we define as\[
\gamma\equiv\epsilon+\eta\]
giving\[
1+\Gamma(\omega)=2\frac{i\eta x}{1-x^{2}+i\gamma x}\]
For calculation it's extremely useful to factorize the denominator
in order to find the poles of the expression. The denominator factors
as follows,\begin{eqnarray*}
1-x^{2}+i\gamma x & = & 0\\
x^{2}-i\gamma x-1 & = & 0\\
x_{\pm} & =\frac{1}{2} & \left[i\gamma\pm\sqrt{-\gamma^{2}+4}\right]\\
x_{\pm} & = & \frac{i\gamma}{2}\pm\sqrt{1-\frac{\gamma^{2}}{4}}\end{eqnarray*}
Therefore the denominator is\[
-(x^{2}-i\gamma x-1)=-(x-x_{-})(x-x_{+})\]
So in the end we get\[
1+\Gamma(\omega)=-2\frac{i\eta x}{(x-x_{+})(x-x_{-})}\]



\section{Gaussian input}

For a gaussian input voltage pulse given by\[
V(t)=V_{0}\frac{1}{\sqrt{2\pi\sigma^{2}}}\exp\left[-t/2\sigma^{2}\right]\]
the frequency space representation turns out to be\[
V(\omega)=V_{0}\sqrt{2\pi\sigma^{2}}\exp\left[-\omega^{2}/2\Sigma^{2}\right]\]
where $\Sigma\equiv1/\sigma$. Now, usually our pulse will be modulated
by a high frequency (GHz) sinewave, ie,\[
V(t)=V_{0}\frac{1}{\sqrt{2\pi\sigma^{2}}}\exp\left[-t/2\sigma^{2}\right]\cos\left[\Omega t\right]\]
We can easily figure out what this looks like in frequency space using
the convolution theorem. Let $G_{\sigma}$stand for a gaussian with
width $\sigma$. Then we get \begin{eqnarray*}
V(t) & = & \textrm{const}*G_{\sigma}(t)*\frac{1}{2}\left[e^{i\Omega t}+e^{-i\Omega t}\right]\\
V(\omega) & = & (1/\textrm{const})*\frac{1}{2}\int G_{\Sigma}(\omega')2\pi\left[\delta(\omega'-\Omega-\omega)+\delta(\omega'+\Omega-\omega)\right]\frac{d\omega'}{2\pi}\\
V(\omega) & = & (1/\textrm{const})*\frac{1}{2}\left[G_{\Sigma}(\omega+\Omega)+G_{\Sigma}(\omega-\Omega)\right]\end{eqnarray*}
Plugging in the value of the constant and writing everything explicitly
we get\[
V(\omega)=V_{0}\sqrt{2\pi\sigma^{2}}\frac{1}{2}\left(\exp\left[-(\omega-\Omega)^{2}/2\Sigma^{2}\right]+\exp\left[-(\omega+\Omega)^{2}/2\Sigma^{2}\right]\right)\]



\section{Computation}

Now we can stuff everything into the inverse fourier transform. From
above we had\begin{eqnarray*}
V(\omega) & = & V_{\textrm{in}}(\omega)\left[1+\Gamma(\omega)\right]\\
V(t) & = & \int V_{\textrm{in}}(\omega)\left[1+\Gamma(\omega)\right]e^{i\omega t}\frac{d\omega}{2\pi}\\
V(t) & = & -2\int V_{\textrm{in}}(\omega)\frac{i\eta x}{(x-x_{-})(x-x_{+})}e^{i\omega t}\frac{d\omega}{2\pi}\\
V(t) & = & -2\omega_{0}\int V_{\textrm{in}}(\omega_{0}x)\frac{i\eta x}{(x-x_{-})(x-x_{+})}e^{ix\tau}\frac{dx}{2\pi}\end{eqnarray*}
 where $x\equiv\omega/\omega_{0}$ and $\tau\equiv\omega_{0}t$.


\subsection{Impulse response}

Here we work out the response for the case of an impulse input, $V_{0}\delta(t)$.
Note that $V_{0}$ has units of voltage/time. In this case the termination
voltage is\begin{eqnarray*}
V(t) & = & -2V_{0}\omega_{0}\int e^{-ix\tau_{0}}\frac{i\eta x}{(x-x_{+})(x-x_{-})}e^{ix\tau}\frac{dx}{2\pi}\\
V(t) & = & -2V_{0}\omega_{0}\int\frac{i\eta x}{(x-x_{+})(x-x_{-})}e^{ix(\tau-\tau_{0})}\frac{dx}{2\pi}\end{eqnarray*}
Recall the values of $x_{\pm}$,\[
x_{\pm}=\frac{i\gamma}{2}\pm\sqrt{1-\frac{\gamma^{2}}{4}}\]
Both of the poles are therefore in the upper half of the complex plane.
This means that in the case $\tau<\tau_{0}$, where we must close
the integration contour in the lower half plane, the integral is zero.
This makes sense because there shouldn't be any response of the system
before the delta function impulse occurs at $\tau_{0}$. In the case
that $\tau>\tau_{0}$ we just use the residue theorem,\begin{eqnarray*}
V(t) & = & -2V_{0}\omega_{0}\int\frac{i\eta x}{(x-x_{+})(x-x_{-})}e^{ix(\tau-\tau_{0})}\frac{dx}{2\pi}\\
\frac{V(t)}{-2V_{0}\omega_{0}} & = & i\left(\frac{i\eta x_{-}}{(x_{-}-x_{+})}\right)e^{ix_{-}\tau}+i\left(\frac{i\eta x_{+}}{(x_{+}-x_{-})}\right)e^{ix_{+}\tau}\\
\frac{V(t)}{-2V_{0}\omega_{0}} & = & \frac{i}{2\sqrt{1-\frac{\gamma^{2}}{4}}}\left[-\left(i\eta x_{-}\right)e^{ix_{-}\tau}+\left(i\eta x_{+}\right)e^{ix_{+}\tau}\right]\end{eqnarray*}
where here $\tau\equiv\tau-\tau_{0}$. Since $\left(ix_{+}\right)^{*}=ix_{-}$
you can see that the the stuff in square brackets is a difference
of two expressions which are complex conjugates of one another. Therefore,
the expression in brackets is purely imaginary. Since there's a factor
of $i$ out front the entire expression is real as it should be. In
fact we can use these observations to simplify the expression,\begin{eqnarray*}
\frac{V(t)}{-2V_{0}\omega_{0}} & = & \frac{i}{2\sqrt{1-\frac{\gamma^{2}}{4}}}2i\Im\left[i\eta x_{+}e^{ix_{+}\tau}\right]\\
\frac{V(t)}{2V_{0}\omega_{0}} & = & \frac{1}{\sqrt{1-\frac{\gamma^{2}}{4}}}\Im\left[i\eta x_{+}e^{ix_{+}\tau}\right]\end{eqnarray*}
Writing $x_{+}$ as $i\gamma/2+r$ we get\begin{eqnarray*}
\frac{V(t)}{2V_{0}\omega_{0}} & = & \frac{1}{\sqrt{1-\frac{\gamma^{2}}{4}}}\Im\left[i\eta x_{+}e^{i(i\gamma/2+r)\tau}\right]\\
\frac{V(t)}{2V_{0}\omega_{0}} & = & \frac{1}{r}e^{-\gamma\tau/2}\Im\left[i\eta x_{+}e^{ir\tau}\right]\end{eqnarray*}
Now writing the stuff in parentheses as $\alpha+i\beta$ the bracketed
expression becomes $\Im\left\{ \left[\alpha+i\beta\right]\left[\cos\left(r\tau\right)+i\sin\left(r\tau\right)\right]\right\} =\alpha\sin\left(r\tau\right)+\beta\cos\left(r\tau\right)$.
Therefore the voltage signal is\[
V(t)=2V_{0}\omega_{0}\frac{1}{r}e^{-\gamma\tau/2}\left[\alpha\sin\left(r\tau\right)+\beta\cos\left(r\tau\right)\right]\]
which is just a decaying sinusoid with phase and initial amplitude
determined by $\alpha$ and $\beta$.

Now we determine $\alpha$ and $\beta$.\begin{eqnarray*}
i\eta x_{+} & = & \frac{-\gamma\eta}{2}+i\eta\sqrt{1-\gamma^{2}/4}\\
\textrm{so}\qquad\alpha=-\gamma\eta/2 & \quad\textrm{and}\quad & \beta=\eta\sqrt{1-\gamma^{2}/4}\end{eqnarray*}
Now using simple trig identities we can write $\alpha\sin\left(r\tau\right)+\beta\cos\left(r\tau\right)$
as $M\cos\left(r\tau+\phi\right)$ where\begin{eqnarray*}
M & = & \sqrt{\alpha^{2}+\beta^{2}}=\eta\\
\tan\left(\phi\right) & = & -\frac{\alpha}{\beta}=\frac{\gamma\eta/2}{\eta\sqrt{1-\gamma^{2}/4}}=\frac{\gamma}{2\sqrt{1-\gamma^{2}/4}}\end{eqnarray*}
The full expression is therefore \[
V(t)=\left\{ \begin{array}{cc}
2V_{0}\omega_{0}\frac{\eta}{\sqrt{1-\gamma^{2}/4}}e^{-\gamma\tau/2}\cos\left(\sqrt{1-\gamma^{2}/4}\tau+\phi\right) & t>0\\
0 & t<0\end{array}\right.\]
This is a very sensible result. The oscillator voltage oscillates
with a frequency given by the damped frequency of the oscillator,
$\omega_{0}\sqrt{1-\gamma^{2}/4}$ where the damping coefficient $\gamma$
is the combination of the intrinsic oscillator damping and the damping
from the transmission line. There is an overall decay given by the
same constant. Remember that $V_{0}$ has units of voltage/time because
we used a delta function input pulse.


\subsection{Gaussian input}

Now that we've solved the case of a delta function input we should
be able to get the response to a Gaussian input pulse by use of the
convolution theorem.


\subsection{Fourier transform from time domain convolution}

Since we've found the delta function response $g(t)$ we can write
the solution to the problem as a convolution\begin{eqnarray*}
V(t) & = & \int_{t'=-\infty}^{t}V_{\textrm{in}}(t')g(t-t')dt'\\
V(t) & = & \int_{t'=-\infty}^{\infty}V_{\textrm{in}}(t')g(t-t')\Theta(t-t')dt'\\
\mathrm{or}\quad V(t) & = & \int_{t'=-\infty}^{\infty}V_{\mathrm{in}}(t-t')g(t')\Theta(t')dt'\end{eqnarray*}
Rather than actually do this convolution we can go to the frequency
domain,\begin{eqnarray*}
V_{\mathrm{in}}(t-t') & = & \int_{\omega}V_{\mathrm{in}}(\omega)e^{i(t-t')\omega}\frac{d\omega}{2\pi}\\
g(t') & = & \int_{\omega}g(\omega)e^{i\omega t'}\frac{d\omega}{2\pi}\\
\Theta(t') & = & \int_{\omega}\Theta(\omega)e^{i\omega t'}\frac{d\omega}{2\pi}\end{eqnarray*}
Stuffing these into the convolution integral gives\begin{eqnarray*}
V(t) & = & \int_{\omega}\int_{\omega'}\int_{\omega''}\int_{t'}V_{\mathrm{in}}(\omega)e^{i(t-t')\omega}g(\omega')e^{i\omega't'}\Theta(\omega'')e^{i\omega''t'}\frac{d\omega}{2\pi}\frac{d\omega'}{2\pi}\frac{d\omega''}{2\pi}\\
V(t) & = & \int_{\omega}\int_{\omega'}\int_{\omega''}V_{\mathrm{in}}(\omega)e^{i\omega t}g(\omega')\Theta(\omega'')\frac{d\omega}{2\pi}\frac{d\omega'}{2\pi}\frac{d\omega''}{2\pi}\int_{t'}e^{it'(-\omega+\omega'+\omega'')}\\
V(t) & = & \int_{\omega}\int_{\omega''}e^{i\omega t}V_{\mathrm{in}}(\omega)g(\omega-\omega'')\Theta(\omega'')\frac{d\omega}{2\pi}\frac{d\omega''}{2\pi}\\
\mathrm{or}\qquad V(t) & = & \int_{\omega}\int_{\omega'}e^{i\omega t}V_{\mathrm{in}}(\omega)g(\omega-\omega')\Theta(\omega')\frac{d\omega}{2\pi}\frac{d\omega'}{2\pi}\end{eqnarray*}
where in the last line we just relabeled an integration index. We
need to write down these three functions in frequency space. We already
know two of them,\begin{eqnarray*}
V(\omega) & = & V_{0}\sqrt{2\pi\sigma^{2}}\frac{1}{2}\left(\exp\left[-(\omega-\Omega)^{2}/2\Sigma^{2}\right]+\exp\left[-(\omega+\Omega)^{2}/2\Sigma^{2}\right]\right)\\
g(\omega) & = & -2\frac{i\eta x}{(x-x_{+})(x-x_{-})}\end{eqnarray*}
and the Fourier transform of the theta function is easy to compute\[
\Theta(\omega)=\lim_{\xi\rightarrow0}\left(\frac{-i}{\omega-i\xi}\right)\]
If we now eliminate the dimensions in the integral we have\begin{eqnarray*}
V(t) & = & \omega_{0}^{2}\int_{x}\int_{x'}e^{ix\tau}V_{\mathrm{in}}(\omega_{0}x)g(\omega_{0}x-\omega_{0}x')\Theta(\omega_{0}x')\frac{dx}{2\pi}\frac{dx'}{2\pi}\end{eqnarray*}
This is almost exactly the same integral we had at the beginning of
this section with the only difference being the presence of the extra
convolution between the green's function and the theta function. Putting
everything together we get\begin{eqnarray*}
V(t) & = & -2\omega_{0}^{2}\lim_{\xi\rightarrow0}\int_{x}\int_{x'}e^{ix\tau}V_{\mathrm{in}}(\omega_{0}x)\frac{-i}{\omega_{0}(x-x')-i\xi}\frac{i\eta x'}{(x'-x_{+})(x'-x_{-})}\frac{dx}{2\pi}\frac{dx'}{2\pi}\\
V(t) & = & -2\omega_{0}\lim_{\xi\rightarrow0}\int_{x}\int_{x'}e^{ix\tau}V_{\mathrm{in}}(\omega_{0}x)\frac{1}{(x-x')-i\xi}\frac{\eta x'}{(x'-x_{+})(x'-x_{-})}\frac{dx}{2\pi}\frac{dx'}{2\pi}\end{eqnarray*}
where we've changed $\xi/\omega_{0}\rightarrow\xi$ because we're
going to take it to zero anyway. Let's do the integral over $x$ first.\[
V(t)=-2\omega_{0}\lim_{\xi\rightarrow0}\int_{x'}\frac{\eta x'}{(x'-x_{+})(x'-x_{-})}\frac{dx'}{2\pi}\int_{x}e^{ix\tau}V_{\mathrm{in}}(\omega_{0}x)\frac{1}{(x-x')-i\xi}\frac{dx}{2\pi}\]
This is a bit weird because the poles in the $x$ integral depend
on the current value of $x'$.
\end{document}

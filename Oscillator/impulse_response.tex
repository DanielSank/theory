\levelstay{Impulse Response}

The equation for a damped harmonic oscillator is
\begin{equation}
  \ddot{\phi}(t)+2\beta\dot{\phi}(t)+\omega_{0}^{2}\phi(t)=j(t)
\end{equation}
In the usual way we turn this into an abstract equation \begin{equation}
(D_{t}^{2}+2\beta D_{t}+\omega_{0}^{2})\ket{\phi}=\ket{j} \end{equation}
Our Fourier convention will be
\begin{equation}
  \braket{t}{\omega}=e^{i\omega t}
\end{equation}
which is the common choice in electrical engineering.
However, it is the \emph{opposite} of what is normally used in physics, notably in Schrodinger's equation.
With this convention $-iD_t$ is Hermitian.
Going to the $\ket{\omega}$ basis we get
\begin{align}
  \bbraket{\omega}{D_t^2 + 2\beta D_t + \omega_0^2}{\phi} &= \braket{\omega}{j} \nonumber \\
  (-\omega^{2} + 2 i \beta\omega + \omega_{0}^{2}) \phi(\omega) & = j(\omega) \nonumber \\
  \phi(\omega) & = \frac{j(\omega)}{-\omega^{2} + 2 i \beta \omega + \omega_{0}^{2}}
  \, .
\end{align}
Taking $j(t)=A\delta(t-t_{0})$ we get $\braket{\omega}{j} = A e^{-i \omega t_{0}}$.
We can factor the denominator as $-(\omega-\omega_{+})(\omega-\omega_{-})$
where
\begin{equation}
  \omega_{+} = i \beta + \omega_0' \qquad \omega_{-} = i \beta - \omega_0' \label{eq:omega_plus_minus}
\end{equation}
and
\begin{equation}
  \omega_0' = \omega_0 \sqrt{ 1 - \left( \beta / \omega_0 \right) ^2 } \, .
\end{equation}
Therefore we have
\begin{equation}
  \phi(\omega) = \frac{-A e^{-i \omega t_0}}{(\omega - \omega_{+})(\omega-\omega_{-})} \, .
\end{equation}
Upon Fourier transform
\begin{equation}
  \phi(t) = 
  \int\frac{-Ae^{i \omega(t - t_0)}}{(\omega-\omega_{+})(\omega-\omega_{-})} \,
  \frac{d\omega}{2\pi} \, .
\end{equation}
Consider the case $t < t_0$.
In that case we close the contour in the lower half plane.
Since both poles $\omega_\pm$ are in the upper half plane the integral is zero.
This reflects causality: the oscillator can't know about the impulse before it happens.
For the case $t > t_{0}$ we close in the upper half plane.
Now the contour encloses the poles and we will get a nonzero result.
Let $\tau \equiv t - t_0$.
Then the residue theorem gives
\begin{align}
  \phi(t) &=
  \frac{-A 2\pi i}{2 \pi}
  \left(
  \frac{e^{i \omega_+ \tau}}{\omega_+ - \omega_-} +
  \frac{e^{i \omega_- \tau}}{\omega_- - \omega_+}
  \right) \\
  &= \frac{A}{\omega_0'} e^{- \beta \tau} \sin(\omega_0' \tau)
\end{align}
We see that the system oscillates with a frequency close to $\omega_0$ and decays with a rate given by the amplitude damping parameter $\beta$.
Note that this solution is valid only when $\omega_0 > \beta$.

In the limit $\beta \ll \omega_0$ we find $E(t) = E_{0}e^{-2 \beta t}$, where $E_0=\frac{1}{2} m A^{2}$ is the initial energy of the system.
Note that $A$ is the initial velocity.

Defining the quality factor $Q$ by
\begin{equation}
  Q \equiv \frac{\textrm{Energy stored}}{\textrm{Energy dissipated per radian}} \, ,
\end{equation}
we find
\begin{equation}
  Q
  = \frac{E(t)}{dE/d\textrm{rad}}
  = \frac{E(t)}{-(dE/dt)(dt/d\textrm{rad})}
  = \frac{E(t)}{2\beta E(t)\frac{1}{\Delta\omega}}
  = \frac{\Delta\omega}{2\beta}\approx\frac{\omega_{0}}{2\beta} \, .
\end{equation}
This relation is also written
\begin{equation}
  Q \approx \omega_0 / \kappa
\end{equation}
where $\kappa$ is the energy decay rate of the oscillator.


% Artifact header: coupled_resonators_L_and_C
\documentclass{article}

%Packages
\usepackage{amsmath}
\usepackage{amstext}
\usepackage{amssymb}
\usepackage{appendix}
\usepackage{coseoul}
\usepackage{enumerate}
\usepackage{graphicx}
\usepackage{import}
\usepackage{lscape}
\usepackage{modular}

\usepackage[pdfpagemode=UseNone,pdfstartview=FitH,colorlinks=true,linkcolor=blue,citecolor=blue,urlcolor=blue]{hyperref}
\usepackage[all]{hypcap}


% General physics constructs
\newcommand{\bra}[1]{\langle #1 |}
\newcommand{\ket}[1]{| #1 \rangle }
\newcommand{\braket}[2]{\langle #1|#2\rangle}
\newcommand{\bbraket}[3]{ \langle #1 | #2 | #3 \rangle }
\newcommand{\norm}[1]{\| #1\|}
\newcommand{\avg}[1]{\left \langle #1 \right \rangle}
\newcommand{\angavg}[1]{\left \langle #1 \right \rangle}
\newcommand{\abs}[1]{\left \lvert #1 \right \rvert}
\newcommand{\VS}{\textit{\textbf{V}}}
\newcommand{\Tr}{\textrm{Tr}}
\renewcommand{\Re}{\textrm{Re}}
\renewcommand{\Im}{\textrm{Im}}
\newcommand{\basis}[1]{\{\ket{#1}\}}

\newcommand{\omegaqubit}{\omega_{10}}

% Figures. Example usage:
% \quickfig{\columnwidth}{my_image}{This is the caption}{fig:my_fig}
\DeclareRobustCommand{\quickfig}[4]{
\begin{figure}
\begin{centering}
\includegraphics[width=#1]{#2}
\par\end{centering}
\caption{#3}
\label{#4}
\end{figure}
}

\DeclareRobustCommand{\quickwidefig}[4]{
\begin{figure*}[h]
\begin{centering}
\includegraphics[width=#1]{#2}
\par\end{centering}
\caption{#3}
\label{#4}
\end{figure*}
}


\author{Daniel Sank}
\date{14 March 2019}
\title{Coupled Oscillators}

\begin{document}
\maketitle
\tableofcontents

\section{Kirchhoff's laws}

\quickfig{\columnwidth}{coupled_circuits.pdf}{Two coupled resonators.}{fig:coupledOscillatorsLAndC:diagram}

Consider two coupled LC oscillators as shown in Figure \ref{fig:coupledOscillatorsLAndC:diagram}.
Kirchhoff's laws for this circuit give us four equations
\begin{align}
  I_{C_a} + I_{L_a} + I_g &= 0 \nonumber \\
  I_{C_b} + I_{L_b} - I_g &= 0 \nonumber \\
  L_a \dot I_{L_a} + L_g \dot I_{L_b} &= V_a \nonumber \\
  L_b \dot I_{L_b} + L_g \dot I_{L_a} &= V_b
  \, .
\end{align}
In order to get useful equations of motion, we need to work with all voltages or all currents.
We'll use voltages.
The capacitor currents are easily related to voltage via the constitutive equation for a capacitor: $C_i \dot V_i = I_i$.
Note that for the coupling capacitor, the constitutive relation gives $I_g = C_g (\dot V_a - \dot V_b)$.
Relating the inductor currents to voltages requires more work.
The bottom two of the equations from Kirchhoff's laws can be written in matrix form as
\begin{equation*}
  \left( \begin{array}{c}
    V_a \\ V_b
  \end{array}\right)
  = \underbrace{ \left( \begin{array}{cc}
    L_a & L_g \\
    L_g & L_b
  \end{array} \right)}_{T_L}
  \left( \begin{array}{c}
    \dot I_a \\ \dot I_b
  \end{array}\right)
  \, .
\end{equation*}
Inverting $T_L$ gives
\begin{align*}
  T_L^{-1}
  &= \frac{1}{L_a L_b - L_g^2} \left( \begin{array}{cc}
    L_b & -L_g \\ -L_g & L_a
  \end{array} \right) \\
  & \equiv \left( \begin{array}{cc}
    1 / L_a' & -1 / L_g' \\ -1 / L_g' & 1 / L_b'
  \end{array} \right)
\end{align*}
so that
\begin{align*}
  \dot I_{L_a} &= \frac{V_a}{L_a'} - \frac{V_b}{L_g'} \\
  \dot I_{L_b} &= \frac{V_b}{L_b'} - \frac{V_a}{L_g'} \, .
\end{align*}
Note that
\begin{align*}
  L_a' & \equiv L_a - \frac{L_g^2}{L_b} \\
  \text{and} \quad
  L_b' & \equiv L_b - \frac{L_g^2}{L_a}
\end{align*}
are the inductances to ground from each resonator's signal node, including the inductance through the mutual.
Now we can rewrite all of the currents in the first two of Kirchhoff's laws entirely in terms of $V_a$ and $V_b$:
\begin{align*}
  C_a \ddot V_a + \frac{V_a}{L_a'} - \frac{V_b}{L_g'} + C_g (\ddot V_a - \ddot V_b) &= 0 \\
  C_b \ddot V_b + \frac{V_b}{L_b'} - \frac{V_a}{L_g'} + C_g (\ddot V_b - \ddot V_a) &= 0 \, .
\end{align*}
Traditional analysis of circuits in the physics literature use flux and charge instead of current and voltage, so defining $\Phi = \int V \, dt$, we can write our equations of motion as
\begin{align*}
  \ddot \Phi_a (C_a + C_g) - C_g \ddot \Phi_b + \frac{\Phi_a}{L_a'} - \frac{\Phi_b}{L_g'} &= 0 \\
  \ddot \Phi_b (C_b + C_g) - C_g \ddot \Phi_a + \frac{\Phi_b}{L_b'} - \frac{\Phi_a}{L_g'} &= 0 \, .
\end{align*}

\section{Hamiltonian form}

By inspection and a bit of fiddling around, you can check that these equations of motion come from the Lagrangian
\begin{align}
  \mathcal{L}
  =& \underbrace{\frac{C_g}{2} \left(\dot \Phi_a - \dot \Phi_b \right)^2
   + \frac{C_a}{2} \dot \Phi_a^2 + \frac{C_b}{2} \dot \Phi_b^2}_\text{kinetic} \nonumber \\
  & \underbrace{- \frac{\Phi_a^2}{2 L_a'} - \frac{\Phi_b^2}{2 L_b'} + \frac{\Phi_a \Phi_b}{L_g'}}_\text{potential} \, .
\end{align}
The momenta conjugate to $\Phi_a$ and $\Phi_b$ are
\begin{align*}
  Q_a & \equiv \frac{\partial \mathcal{L}}{\partial \dot \Phi_a} = (C_a + C_g) \dot \Phi_a - C_g \dot \Phi_b \\
  Q_b & \equiv \frac{\partial \mathcal{L}}{\partial \dot \Phi_b} = (C_b + C_g) \dot \Phi_b - C_g \dot \Phi_a
  \, .
\end{align*}
or
\begin{equation*}
  \left( \begin{array}{c} Q_a \\ Q_b \end{array} \right)
  =
  \underbrace{
    \left( \begin{array}{cc}
       (C_a + C_g) & -C_g \\
       -C_g & (C_b + C_g) \end{array} \right)
  }_{T_C}
  \left( \begin{array}{c} \dot \Phi_a \\ \dot \Phi_b \end{array} \right)
  \, .
\end{equation*}
The inverse of $T_C$ is
\begin{align*}
  T_C^{-1} =& \frac{1}{C_a C_b + C_g (C_a + C_g)}
  \left( \begin{array}{cc}
    (C_b + C_g) & C_g \\
    C_g & (C_a + C_g)
  \end{array} \right) \\
  \equiv& \left( \begin{array}{cc}
    1 / C_a' & 1 / C_g' \\
    1 / C_g' & 1 / C_b'
  \end{array} \right)
\end{align*}
Note that $C_a'$ and $C_b'$ are the total capacitances to ground from each resonator's signal node.

The Hamiltonian for the system is defined formally by the equation
\begin{equation*}
  H \equiv \left( \sum_{i = \{a, b\}} Q_i \dot \Phi_i \right) - \mathcal{L} \, .
\end{equation*}
Therefore, in order to find the Hamiltonian, we have to replace the $\dot \Phi$'s in both the $Q \dot \Phi$ terms and in Lagrangian itself with the appropriate combinations of $Q$'s.
To do this, first note that the kinetic term in the Lagrangian can be expressed as\footnote{Because matrix transposition and inversion commute, and because the matrix $T_C$ is symmetric, we can bring $T_C^{-1}$ from the bra onto the ket for free.}
\begin{align*}
  \mathcal{L}_\text{kinetic}
  =& \frac{1}{2} \bbraket{\dot \Phi}{T_C}{\dot \Phi} \\
  =& \frac{1}{2} \bbraket{T_C^{-1} Q}{T_C}{T_C^{-1} Q} \\
  =& \frac{1}{2} \bbraket{Q}{T_C^{-1}}{Q}
  \, .
\end{align*}
Note also that $\sum_i \dot \Phi_i Q_i = \bbraket{Q}{T_C^{-1}}{Q}$, so we can write the Hamiltonian as
\begin{align}
  H
  =& \left( \sum_{i = \{a, b\}} Q_i \dot \Phi_i \right) - \mathcal{L} \nonumber \\
  =& \bbraket{Q}{T_C^{-1}}{Q} - \left( \mathcal{L}_\text{kinetic} + \mathcal{L}_\text{potential} \right) \nonumber \\
  =& \bbraket{Q}{T_C^{-1}}{Q} - \left( \frac{1}{2} \bbraket{Q}{T_C^{-1}}{Q} + \mathcal{L}_\text{potential} \right) \nonumber \\
  =& \frac{Q_a^2}{2 C_a'} + \frac{Q_b^2}{2 C_b'} + \frac{\Phi_a^2}{2 L_a'} + \frac{\Phi_b^2}{2 L_b'} \nonumber \\
  &+ \underbrace{\frac{Q_a Q_b}{C_g'} - \frac{\Phi_a \Phi_b}{L_g'}}_{\text{coupling Hamiltonian }H_g}
  \, .
\end{align}

\subsection{Dimensionless variables}

We will now simplify this Hamiltonian so that we can easily find its normal modes.
First, we define
\begin{align*}
  X_i &\equiv \frac{1}{\sqrt{2 \hbar}} \frac{1}{\sqrt{Z_i'}} \Phi_i \\
  Y_i &\equiv \frac{1}{\sqrt{2 \hbar}} \sqrt{Z_i'} Q_i
\end{align*}
where $Z_i' \equiv \sqrt{L_i' / C_i'}$, and we've added the constant $\hbar$ with dimentions of action to make $X$ and $Y$ dimensionless.
In these new variables, the Hamiltonian is
\begin{align}
  H / \hbar
  =& \omega_a' \left(X_a^2 + Y_a^2 \right)
  +  \omega_b' \left(X_b^2 + Y_b^2 \right) \nonumber \\
  &+ 2 \frac{1}{C_g' \sqrt{Z_a' Z_b'}} Y_a Y_b
   - 2 \frac{\sqrt{Z_a Z_b}}{L_g'} X_a X_b
  \, .
\end{align}


\subsection{Rotating modes}

Finally we define
\begin{align}
  a &\equiv X_a + i Y_a \nonumber \\
  b &\equiv X_b + i Y_b
\end{align}
to arrive at
\begin{align}
  H / \hbar
  &= \omega_a' a^* a + \omega_b' b^* b \nonumber \\
  & - \frac{1}{2} \frac{1}{C_g' \sqrt{Z_a' Z_b'}} (ab + a^* b^* - a b^* - a^* b) \nonumber \\
  &- \frac{1}{2} \frac{\sqrt{Z_a' Z_b'}}{L_g'} (ab + a^* b^* + a^* b + a b^*)
  \, .
\end{align}
The two frequencies $\omega_a'$ and $\omega_b'$ are called the \textbf{partial frequencies} and play an important role in the analysis of the system, particularly when making approximations.
The coupling term can be reorganized in a very useful form:
\begin{align}
  H_g / \hbar =
  &- \left( a b + a^* b^* \right)
    \frac{1}{2} \left(
      \frac{1}{C_g' \sqrt{Z_a' Z_b'}} + \frac{\sqrt{Z_a' Z_b'}}{L_g'}
    \right) \nonumber \\
  &+ \left( a b^* + a^* b \right)
    \frac{1}{2} \left(
      \frac{1}{C_g' \sqrt{Z_a' Z_b'}} - \frac{\sqrt{Z_a' Z_b'}}{L_g'}
    \right) \nonumber \\
  &= -g_+ (ab + a^* b^*) + g_- (ab^* + a^* b)
\end{align}
where we defined
\begin{align}
  g_c \equiv \frac{1}{2} \frac{1}{C_g' \sqrt{Z_a' Z_b'}} &\qquad
  g_l \equiv \frac{1}{2} \frac{\sqrt{Z_a' Z_b'}}{L_g'} \nonumber \\
  \text{and} \qquad \qquad
  g_+ \equiv g_c + g_l &\qquad g_- \equiv g_c - g_l
  \, .
\end{align}
Hamilton's equations of motion can now be expressed in matrix form
\begin{equation*}
  \frac{d}{dt}
  \left( \begin{array}{c} a \\ b \\ a^* \\ b^* \end{array} \right)
  = -i \left( \begin{array}{cccc}
    \omega_a' & g_- & 0 & -g_+ \\
    g_- & \omega_b' & -g_+ & 0 \\
    0 & g_+ & -\omega_a' & -g _- \\
    g_+ & 0 & -g_- & -\omega_b'
  \end{array} \right)
  \left( \begin{array}{c} a \\ b \\ a^* \\ b^* \end{array} \right) \, .
\end{equation*}


\section{Rotating wave approximation}

The matrix in these equations of motion can be expressed algebraically as
\begin{equation}
  -i \left[
    \sigma_z \otimes
      \left(
        g_- \sigma_x + \frac{\Delta}{2} \sigma_z + \frac{S}{2} \mathbb{I}
      \right)
    -i g_+ (\sigma_y \otimes \sigma_x)
  \right]
\end{equation}
where $S \equiv \omega_a' + \omega_b'$ and $\Delta \equiv \omega_a' - \omega_b'$.
The rotating wave approximation drops the $ab$ and $a^* b^*$ terms of the coupling Hamiltonian, i.e. the terms proportional to $g_+$.

That's equivalent to dropping the antidiagonal part of the matrix representation of Hamilton's equations of motion and therefore the $\sigma_y \otimes \sigma_x$ term in the algebraic representation.
In the matrix, we can see that dropping these terms is precisely equivalent to decoupling the clockwise and counterclockwise rotating modes.
In the rotating wave aproximation, the algebraic representation is reduced to
\begin{equation*}
  -i \sigma_z \otimes \left(
    g_- \sigma_x + \frac{\Delta}{2} \sigma_z + \frac{S}{2} \mathbb{I}
  \right)
\end{equation*}
which makes finding the eigenvalues particularly simple because the eigenvalues of a tensor product are just the products of the eigenvalues of the individual tensors being multiplied.
The eigenvalues of $\sigma_z$ are $\pm 1$.
The eigenvalues of the stuff in parentheses, and therefore the normal mode frequencies, are\footnote{The eigenvalues of the matrix are equal to $i$ times the normal mode frequencies.}
\begin{equation}
  \omega_\pm
  = \frac{\omega_a' + \omega_b'}{2} \pm \sqrt{g_-^2 + (\Delta / 2)^2 } \, .
\end{equation}
These eigenvalues are shown in Figure \ref{Figure:coupled_resonators_L_and_C:avoided_crossing} as a function of both $g_-$ and $\omega_b$.
There we can see the famous avoided level crossing of coupled resonators.
\quickfig{0.8\columnwidth}{avoided_crossing.pdf}{Normal frequencies $\omega_+$ (red) and $\omega_-$ (blue) as a function of $\omega_b'$ for $\omega_a' = 10$. Dashed lines are for $g_-=0.02$ and solid lines are for $g_-=0.1$.}{Figure:coupled_resonators_L_and_C:avoided_crossing}

The rotating wave approximation (RWA) works in the limit $g_+ \ll \omega_a', \omega_b'$.
This analysis has demonstrated the usefulness of the Hamiltonian formalism.
by working with and Hamiltonian, with its partial frequencies and impedances, we found the systematic and accurate approximation, namely the RWA, for the normal mode frequencies.
The lovely symmetry of Figure \ref{Figure:coupled_resonators_L_and_C:avoided_crossing} exists because we plot the normal frequencies against the partial frequency $\omega_b'$ rather than the uncoupled frequency $\omega_b$.

As shown above, the RWA is equivalent to dropping the Hamiltonian terms proportional to $g_+$.
Therefore, in a device where $g_+$ is constructed to be exactly zero the RWA would be exact!

\section{Relation between coupling and damping}

Consider the case of two resonators coupled through a coupling capacitor $C_g$.
The coupling gives rise to a coupling strength $g_c$ as described above.
If we replace resonator $b$ by a resistor $R$, then resonator $a$ would experience damping described by a quality factor $Q_a$.
Similarly, if we were to replace resonator $a$ by resistor $R$, then resonator $b$ would experience damping described by a quality factor $Q_b$.
Interestingly, the quality factors are related to the coupling strength $g_c$ via
\begin{equation*}
  \frac{(2 g_c)^2}{\omega_a' \omega_b'} \frac{R}{\sqrt{Z_a' Z_b'}} \sqrt{Q_a Q_b} = 1 \, .
\end{equation*}

\end{document}

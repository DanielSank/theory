\levelstay{Driven resonator in a rotating frame}

It is possible to find approximation equations of motion for the driven oscillator in the case that the spectral content of the drive is reasonably well localized around the resonance frequency.
Consider the frequency domain equation of motion
\begin{equation}
  \phi(\omega) = \frac{-J(\omega)}{\omega^2 - 2 i \beta \omega - \omega_0^2}
  = - \frac{J(\omega)}{(\omega - \omega_+)(\omega - \omega_-)} \, .
\end{equation}
Using the method of partial fractions, we can rewrite the double pole as a sum of single poles
\begin{equation}
  \frac{1}{(\omega - \omega_+)(\omega - \omega_-)}
  = \frac{X}{\omega - \omega_+} + \frac{Y}{\omega - \omega_-}
\end{equation}
with unknown constants $X$ and $Y$.
Multiplying through by $(\omega - \omega_+)(\omega - \omega_-)$ and solving gives
\begin{align*}
  X =& \frac{1}{\omega_+ - \omega_-} =   \frac{1}{2 \omega_0'} \\
  Y =& \frac{1}{\omega_- - \omega_+} = - \frac{1}{2 \omega_0'}
\end{align*}
so
\begin{equation}
  \phi(\omega) = - \frac{J(\omega)}{2 \omega_0'}
  \left[ \frac{1}{\omega - \omega_+} - \frac{1}{\omega - \omega_-} \right] \, .
\end{equation}
If the drive is a narrow band signal near a central frequency $\Omega$, we can write it as
\begin{equation}
  J(t)
  = I(t) \cos(\Omega t) - Q(t) \sin(\Omega t)
  = \Re \left[ z(t) \exp \left( i \Omega t \right) \right] \, ,
\end{equation}
where $z(t) \equiv I(t) + i Q(t)$.
In this case we have
\begin{align*}
  \tilde{J}(\omega)
  &= \frac{1}{2} \left(
      \tilde{I}(\omega - \Omega) + i \tilde{Q}(\omega - \Omega)
    + \tilde{I}(\omega + \Omega) - i \tilde{Q}(\omega + \Omega)
  \right) \\
  &= \frac{1}{2} \left(
      \tilde{z}(\omega - \Omega) + \tilde{z}(-\omega - \Omega)^*
  \right)
\end{align*}
where we've used the fact that $\tilde{I}(\omega)^* = \tilde{I}(-\omega)$ and $\tilde{Q}(\omega)^* = \tilde{Q}(-\omega)$ because $I(t)$ and $Q(t)$ are real.
Combining results, we have
\begin{align*}
  \tilde{\phi}(\omega)
  =& - \frac{1}{4 \omega_0'}
  \left( \tilde{z}(\omega - \Omega) + \tilde{z}(-\omega - \Omega)^* \right)
  \left(
    \frac{1}{\omega - \omega_+} - \frac{1}{\omega - \omega_-}
  \right) \\
  =& - \frac{1}{4 \omega_0'}
  \left(
      \frac{\tilde{z}(\omega - \Omega)}{\omega - \omega_+}
    - \frac{\tilde{z}(-\omega - \Omega)^*}{\omega - \omega_-}
    \underbrace{
      - \frac{\tilde{z}(\omega - \Omega)}{\omega - \omega_-}
      + \frac{\tilde{z}(-\omega - \Omega)^*}{\omega - \omega_+}
    }_\text{small}
  \right)
  \, .
\end{align*}
The second two terms in parentheses are small because e.g. $\tilde{z}(\omega - \Omega)$ is large when $\omega$ is near $+\Omega$, but $1 / (\omega - \omega_-)$ is large for $\omega$ near $-\Omega$.
Keeping only the larger terms, we get
\begin{equation}
  \tilde{\phi}(\omega) =
  - \frac{1}{4 \omega_0'}
  \left(
      \frac{\tilde{z}(\omega - \Omega)}{\omega - \omega_+}
    - \frac{\tilde{z}(-\omega - \Omega)^*}{\omega - \omega_-}
  \right) \, .
\end{equation}
Note that $\tilde{\phi}(-\omega) = \tilde{\phi}(\omega)^*$ because $\phi(t)$ is real, so we can write it as a Fourier transform over only positive frequencies:
\begin{align*}
  \phi(t)
  &= 2 \Re \int_0^\infty \frac{d\omega}{2\pi} \, e^{i \omega t} \tilde{\phi}(\omega) \\
  &= 2 \Re \int_0^\infty \frac{d\omega}{2\pi} \, e^{i \omega t} \left( \frac{-1}{4 \omega_0'} \right)
    \left(
      \frac{\tilde{z}(\omega - \Omega)}{\omega - \omega_+}
      - \underbrace{\frac{\tilde{z}(-\omega - \Omega)^*}{\omega - \omega_-}}_\text{small}
    \right) \\
  (\text{define } \omega \equiv \omega_f + \delta \omega)
  &= \Re
  \left(
    e^{i \omega_f t}
    \int_{-\omega_f}^\infty \frac{d \delta \omega}{2\pi} \,
    e^{i \delta \omega t}
    \left( \frac{-1}{2 \omega_0'} \right)
    \frac{\tilde{z}(\omega_f - \Omega + \delta \omega)}{\delta \omega - \Delta - i \beta}
  \right)
\end{align*}
where $\omega_f$ is the ``frame frequency'' typically chosen to be close to $\Omega$, and $\Delta \equiv \omega_0' - \omega_f$.
If $\tilde{z}$ is band limited to $[-B, B]$ where $B \ll \omega_0'$, then we can rewrite the limits of the integration as $\pm \infty$:
\begin{equation}
  \phi(t) = \Re
  \left(
    e^{i \omega_f t}
    \underbrace{
      \int_{-\infty}^\infty \frac{d \delta \omega}{2\pi} \,
      e^{i \delta \omega t}
      \left( \frac{-1}{2 \omega_0'} \right)
      \frac{\tilde{z}(\omega_f - \Omega + \delta \omega)}{\delta \omega - \Delta - i \beta}
    }_{\alpha(t)}
  \right)
  \label{eq:equation_of_motion_phasor_integral}
\end{equation}
The underbraced part $\alpha(t)$ is the part of the oscillator's motion that comes on top of a free oscillation at the frame frequency $\omega_f$.
Equation~(\ref{eq:equation_of_motion_phasor_integral}) is in the phasor form discussed above, but now the phasor $\alpha(t)$ is time dependent.
By inspection, the integral representation of $\alpha$ given in Eq.~(\ref{eq:equation_of_motion_phasor_integral} corresponds with the first order differential equation (remember $Q = \omega_0 / 2 \beta = \omega_0 / \kappa$)
\begin{equation}
  \dot{\alpha}(t) = \left(\frac{-i}{2 \omega_0'}\right) z(t) e^{i (\omega_f - \Omega) t} + (i \Delta - \kappa/2) \alpha(t)
  \, .
  \label{eq:equation_of_motion_phasor}  % export
\end{equation}

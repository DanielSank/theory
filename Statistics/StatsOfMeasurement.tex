%% LyX 1.6.6 created this file.  For more info, see http://www.lyx.org/.
%% Do not edit unless you really know what you are doing.
\documentclass[english]{article}
\usepackage[T1]{fontenc}
\usepackage[latin9]{inputenc}
\usepackage{amstext}
\usepackage{esint}
\usepackage{babel}

\begin{document}

\title{Statistics of Measurement}

\maketitle
We perform an experiment in which we measure a random variable $X$
which can take values $[0,1]$. On the $i^{\textrm{th}}$ experiment
the probability distribution for $X_{i}$ is\[
W(X_{i})=p_{i}\delta(X_{i}-1)+(1-p_{i})\delta(X_{i})\]
ie, the random variable has a probability $p$ of attaining the value
one, and a probability of $1-p$ of attaining the value zero.


\section*{What are the statistical properties of $X_{i}$?}

The mean of $X_{i}$ is\[
\langle X_{i}\rangle=\int W(X_{i})X_{i}dX_{i}=p_{i}\]
as we expect. The dispersion is\[
\langle(\Delta X_{i})^{2}\rangle=\langle(X_{i}-\langle X_{i}\rangle)^{2}\rangle=\langle X_{i}^{2}\rangle-\langle X_{i}\rangle^{2}\]
It's clear that the random variable $X^{2}$ has exactly the same
realization as $X$, so we have $\langle X_{i}^{2}\rangle=\langle X_{i}\rangle=p_{i}$.
This is also clear by use of the defining formula for averages:\[
\langle X_{i}^{2}\rangle=\int W(X_{i})X_{i}^{2}dX_{i}=p_{i}\]
Then we have\[
\langle(\Delta X_{i})^{2}\rangle=\langle X_{i}^{2}\rangle-\langle X_{i}\rangle^{2}=p_{i}-p_{i}^{2}=p_{i}(1-p_{i})\]
This is our first result.


\section*{Constant $p$}

Consider a case where $p$ is constant accross all experiments. Imagine
you take a bunch of data, ie. you have a string of zeros and ones.
From this string, how do you figure out what $p$ is? Use Bayes's
theorem, which says \[
W(A|B)=\frac{W(B|A)W(A)}{W(B)}\]
In our case we take $A$ to be the event that $p$ has a certain value,
and we take $B$ to be the event in which your measured string of
zeros and ones takes a particular value. Then we can write this as\[
W(p|\textrm{data})=\frac{W(\textrm{data}|p)W(p)}{W(\textrm{data})}\]
This is useful because $W(\textrm{data}|p)$ is easy to compute. Let
$M$ be the number of experiments, and $n$
\end{document}

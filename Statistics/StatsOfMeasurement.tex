\documentclass{article}
\usepackage{amstext}

\begin{document}

\title{Statistics of Measurement}

\maketitle
We perform an experiment in which we measure a random variable $X$ which can take values $[0,1]$.
On the $i^{\textrm{th}}$ experiment, the probability distribution for $X_{i}$ is
\begin{equation}
W(X_i) = p_i\ \delta(X_i - 1) + (1 - p_i) \delta(X_i) \, ,
\end{equation}
i.e. the random variable has a probability $p$ of attaining the value one, and a probability of $1-p$ of attaining the value zero.


\section*{What are the statistical properties of $X_i$?}

The mean of $X_i$ is
\begin{equation}
\langle X_i \rangle = \int W(X_i) X_i dX_i = p_{i}
\end{equation}
as we expect.
The dispersion is
\begin{equation}
\langle(\Delta X_i)^2 \rangle = \langle(X_i - \langle X_i \rangle)^2 \rangle = \langle X_i^2 \rangle - \langle X_i \rangle^2 \, .
\end{equation}
It's clear that the random variable $X^2$ has exactly the same realization as $X$, so we have $\langle X_i^{2} \rangle = \langle X_i \rangle = p_i$.
This is also clear by use of the defining formula for averages:
\begin{equation}
\langle X_i^2 \rangle = \int W(X_i) X_i^2 dX_i = p_i \, .
\end{equation}
Then we have
\begin{equation}
\langle(\Delta X_i)^2 \rangle = \langle X_i^2 \rangle - \langle X_i \rangle^2 = p_i - p_i^2 = p_i(1 - p_i) \, .
\end{equation}
This is our first result.


\section*{Constant $p$}

Consider a case where $p$ is constant accross all experiments.
Imagine you take a bunch of data, ie. you have a string of zeros and ones.
From this string, how do you figure out what $p$ is?
Use Bayes's theorem, which says
\begin{equation}
W(A|B)=\frac{W(B|A)W(A)}{W(B)} \, .
\end{equation}
In our case we take $A$ to be the event that $p$ has a certain value, and we take $B$ to be the event in which your measured string of zeros and ones takes a particular value.
Then we can write this as
\begin{equation}
W(p|\textrm{data})=\frac{W(\textrm{data}|p)W(p)}{W(\textrm{data})}
\end{equation}
This is useful because $W(\textrm{data}|p)$ is easy to compute. Let
$M$ be the number of experiments, and $n$
\end{document}

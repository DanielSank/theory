%% LyX 1.6.6 created this file.  For more info, see http://www.lyx.org/.
%% Do not edit unless you really know what you are doing.
\documentclass[12pt,a4paper,twoside,english]{revtex4}
\usepackage[T1]{fontenc}
\usepackage[latin1]{inputenc}
\usepackage{amsmath}
\usepackage{amssymb}
\usepackage{esint}

\makeatletter
%%%%%%%%%%%%%%%%%%%%%%%%%%%%%% Textclass specific LaTeX commands.
\@ifundefined{textcolor}{}
{%
 \definecolor{BLACK}{gray}{0}
 \definecolor{WHITE}{gray}{1}
 \definecolor{RED}{rgb}{1,0,0}
 \definecolor{GREEN}{rgb}{0,1,0}
 \definecolor{BLUE}{rgb}{0,0,1}
 \definecolor{CYAN}{cmyk}{1,0,0,0}
 \definecolor{MAGENTA}{cmyk}{0,1,0,0}
 \definecolor{YELLOW}{cmyk}{0,0,1,0}
 }

\makeatother

\usepackage{babel}

\begin{document}

\title{Statistics}


\author{Daniel Sank}

\maketitle

\section{Fundamental Information}

This chapter is devoted to laying out the basic useful facts about
probability distributions and random variables. 


\subsection{Change of Variables in Probability Distributions}

Consider a random variable $X$ with probability distribution $p_{X}(x)$.
Remember that $p_{X}(x)$ is the probability that the random variable
$X$ takes the value $x$. Now define a new random variable $Y$ which
is related to $X$ by $y=f(x)$. What is $p_{Y}(y)$? Let's ask, \textquotedbl{}what
is the probability that $Y$ takes a value in the set $S$?\textquotedbl{}
Clearly an answer to this question is

\begin{eqnarray*}
P(Y\in S) & = & P(X\in f^{-1}(S))\\
\int_{S}p_{Y}(y) & = & \int_{f^{-1}(S)}p_{X}(x)\end{eqnarray*}


Using the change of variables formula on the right hand side gives
\[
P(Y\in S)=\int_{S}p_{Y}(y)=\int_{f^{-1}(S)}p_{X}(x)=\int_{y\in S}p_{X}(f^{-1}(y))|\det[Df^{-1}](y)|\]
 Therefore we have \[
p_{Y}(y)=p_{X}(f^{-1}(y))~|\det[Df^{-1}](y)|\]
 Remember that if $f^{-1}$ is not single valued you have to make
sure to keep track of all the places that $X$ attains a value such
that $f(x)\in S$.


\subsubsection{Application in Programming}

Suppose you have a computer that can generate random numbers $X$
in the interval $[0,1]$ with a uniform distribution. How do you make
a routine that spits out numbers $Y$ with a specified distribution
$p_{Y}(y)$? We can use the above formula to solve this problem. Formally
the problem is \[
p_{Y}(y)=p_{X}(f^{-1}(y))~\det|[Df^{-1}](y)|\]
 given $p_{Y}(y)$ find the function $f$.

Define \[
x=f^{-1}(y)=\int_{-\infty}^{y}p_{Y}(y')~dy'\]
 Then $f^{-1}:R\rightarrow[0,1]$ because $p_{Y}$ is a normalized
probability distribution, so $x\in[0,1]$. In principle we can solve
for $y=f(x)$. Then \[
p_{Y}(y)=p_{X}(x=f^{-1}(y))|\det[Df^{-1}](y)|=1*p_{Y}(y)\]
 as it should be. So, the function $f$ that you want is the inverse
of $\int_{-\infty}^{y}p_{Y}(y')~dy'$.


\subsection{Some more basic stuff}


\section{Common Distributions}


\subsection{Uniform distribution}

Consider a random variable $X$ distributed uniformly over $[a,b]$.
The distribution is\[
p_{X}(x)=\left\{ \begin{array}{c}
\frac{1}{b-a}\qquad\textrm{for }a<x<b\\
0\qquad\textrm{otherwise}\end{array}\right.\]
The variance is\[
\sigma^{2}=\frac{(b-a)^{2}}{12}\]

\end{document}

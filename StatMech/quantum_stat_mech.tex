%% LyX 1.6.6 created this file.  For more info, see http://www.lyx.org/.
%% Do not edit unless you really know what you are doing.
\documentclass[english,aps,manuscript]{revtex4}
\usepackage[T1]{fontenc}
\usepackage[latin9]{inputenc}
\usepackage{amstext}

\makeatletter
%%%%%%%%%%%%%%%%%%%%%%%%%%%%%% Textclass specific LaTeX commands.
\@ifundefined{textcolor}{}
{%
 \definecolor{BLACK}{gray}{0}
 \definecolor{WHITE}{gray}{1}
 \definecolor{RED}{rgb}{1,0,0}
 \definecolor{GREEN}{rgb}{0,1,0}
 \definecolor{BLUE}{rgb}{0,0,1}
 \definecolor{CYAN}{cmyk}{1,0,0,0}
 \definecolor{MAGENTA}{cmyk}{0,1,0,0}
 \definecolor{YELLOW}{cmyk}{0,0,1,0}
 }

\makeatother

\usepackage{babel}

\begin{document}

\title{Thermodynamic Green's Functions}

\maketitle

\section{Fundamentals}


\subsection{Definitions}

\[
G(\alpha,\alpha')=\frac{1}{i}\langle T\left(\psi(\alpha)\psi^{\dagger}(\alpha')\right)\rangle\]


\[
G_{2}(\alpha\beta,\alpha'\beta')=\frac{1}{i^{2}}\langle T\left(\psi(\alpha)\psi(\beta)\psi^{\dagger}(\beta')\psi^{\dagger}(\alpha')\right)\rangle\]
Here $T$ is the time order operator that puts everything so that
the earliest times are on the right. It introduces a minus for each
elementry permutation when dealing with fermions. For example,\begin{eqnarray*}
T\left(\psi(\alpha)\psi^{\dagger}(\alpha')\right) & = & \psi(\alpha)\psi^{\dagger}(\alpha')\quad\textrm{for}\; t_{\alpha}>t_{\alpha'}\\
 &  & \pm\psi^{\dagger}(\alpha')\psi(\alpha)\quad\textrm{for}\; t_{\alpha}<t_{\alpha'}\end{eqnarray*}
We also define correlation functions\begin{eqnarray*}
G^{>}(\alpha,\alpha') & = & \frac{1}{i}\langle\psi(\alpha)\psi^{\dagger}(\alpha')\rangle\\
G^{<}(\alpha,\alpha') & =\pm & \frac{1}{i}\langle\psi^{\dagger}(\alpha')\psi(\alpha)\rangle\end{eqnarray*}
The notation indicates that $G^{>}=G$ in the case that $t_{\alpha}>t_{\alpha'}$
and that $G^{<}=G$ in the case that $t_{\alpha}<t_{\alpha'}$.


\subsection{Boundary Condition}

\[
G^{<}(\alpha,\alpha')|_{t_{\alpha}=0}=\pm e^{\beta\mu}G^{>}(\alpha,\alpha')|_{t_{\alpha}=-i\beta}\]
This is proved by algebra on the definitions,\begin{eqnarray*}
G^{<}(\alpha,\alpha')|_{t_{\alpha}=0} & = & \pm\frac{1}{i}\langle\psi^{\dagger}(\alpha')\psi(\alpha)\rangle\\
 & = & \pm\frac{1}{i}\frac{\textrm{Tr}\left[e^{-\beta(H-\mu N)}\psi^{\dagger}(r_{\alpha'},t_{\alpha'})\psi(r_{\alpha},0)\right]}{\textrm{Tr}\left[e^{-\beta(H-\mu N)}\right]}\\
\textrm{cyclicity of the trace} & = & \pm\frac{1}{i}\frac{\textrm{Tr}\left[e^{-\beta(H-\mu N)}\left\{ e^{\beta(H-\mu N)}\psi(r_{\alpha},0)e^{-\beta(H-\mu N)}\psi^{\dagger}(r_{\alpha'},t_{\alpha'})\right\} \right]}{\textrm{Tr}\left[e^{-\beta(H-\mu N)}\right]}\\
 & = & \pm\frac{1}{i}\langle e^{\beta(H-\mu N)}\psi(r_{\alpha},0)e^{-\beta(H-\mu N)}\psi^{\dagger}(r_{\alpha'},t_{\alpha'})\rangle\end{eqnarray*}
Assuming that $N$ and $H$ commute we can use\[
e^{-\beta\mu N}\psi(r_{\alpha},0)e^{\beta\mu N}=e^{\beta\mu}\psi(r_{\alpha},0)\]
and\[
e^{\beta H}\psi(r_{\alpha},0)e^{-\beta H}=\psi(r_{\alpha},-i\beta)\]
to get\begin{eqnarray*}
G^{<}(\alpha,\alpha')|_{t_{\alpha}=0} & = & \pm\frac{1}{i}e^{\beta\mu}\langle\psi(r_{\alpha},-i\beta)\psi^{\dagger}(r_{\alpha'},t_{\alpha'})\rangle\\
G^{<}(\alpha,\alpha')|_{t_{\alpha}=0} & = & \pm e^{\beta\mu}G^{>}(\alpha,\alpha')|_{t_{\alpha}=-i\beta}\end{eqnarray*}

\end{document}

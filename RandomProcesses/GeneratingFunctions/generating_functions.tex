\documentclass{article}

% General physics constructs
\newcommand{\bra}[1]{\langle #1 |}
\newcommand{\ket}[1]{| #1 \rangle }
\newcommand{\braket}[2]{\langle #1|#2\rangle}
\newcommand{\bbraket}[3]{ \langle #1 | #2 | #3 \rangle }
\newcommand{\boltzmann}{k_b}

% Common math
\newcommand{\norm}[1]{\left \lvert #1 \right \rvert}
\newcommand{\abs}[1]{\left \lvert #1 \right \rvert}  % These two are redundant. Consider removing one.

\newcommand{\avg}[1]{\left \langle #1 \right \rangle}  % Should get rid of this, as "average" isn't specific.
\newcommand{\angavg}[1]{\left \langle #1 \right \rangle}

\newcommand{\VS}{\textit{\textbf{V}}}
\newcommand{\Tr}{\textrm{Tr}}
\renewcommand{\Re}{\textrm{Re}}
\renewcommand{\Im}{\textrm{Im}}
\newcommand{\basis}[1]{\{\ket{#1}\}}

% Quantum
\newcommand{\nboseeinstein}{n_\text{BE}}
\newcommand{\gammaup}{\Gamma_\uparrow}
\newcommand{\gammadown}{\Gamma_\downarrow}
\newcommand{\gammaupdown}{\Gamma_{\uparrow \downarrow}}
\newcommand{\gammaemission}{\Gamma_\text{loss}}
\newcommand{\qualityfactoremission}{Q_{d,\text{loss}}}

% Qubits
\newcommand{\omegaqubit}{\omega_{10}}

% Circuits
\newcommand{\impedance}{Z_0}
\newcommand{\resistorsource}{R_s}
\newcommand{\vsource}{V_s}
\newcommand{\vsourcerms}{V_{s,\text{rms}}}
\newcommand{\vloadrms}{V_{l,\text{rms}}}

% Signals and noise
\newcommand{\psdsingle}{S_\text{ss}}
\newcommand{\psddouble}{S_\text{ds}}
\newcommand{\noiseavailable}{S_{p,a}^e}
\newcommand{\spectralengineer}{S^e}
\newcommand{\spectralsymmetric}{S^\text{symm}}
\newcommand{\spectralattenuator}{\spectralengineer_{\poweravailable, \text{att.}}}

% Microwaves
\newcommand{\vright}{V_+}
\newcommand{\vleft}{V_-}
\newcommand{\iright}{I_+}
\newcommand{\ileft}{I_-}
\newcommand{\poweravailable}{P_a}

% Figures. Example usage:
% \quickfig{\columnwidth}{my_image}{This is the caption}{fig:my_fig}
\DeclareRobustCommand{\quickfig}[4]{
\begin{figure}
\begin{centering}
\includegraphics[width=#1]{#2}
\par\end{centering}
\caption{#3}
\label{#4}
\end{figure}
}

\DeclareRobustCommand{\quickwidefig}[4]{
\begin{figure*}[h]
\begin{centering}
\includegraphics[width=#1]{#2}
\par\end{centering}
\caption{#3}
\label{#4}
\end{figure*}
}

\DeclareRobustCommand{\quickfigcentered}[4]{
  \begin{figure}
  \makebox[\textwidth][c]{\includegraphics[width=#1]{#2}}
  \caption{#3}
  \label{#4}
  \end{figure}
}

%Packages
\usepackage{amsmath}
\usepackage{amstext}
\usepackage{amssymb}
\usepackage{appendix}
\usepackage{coseoul}
\usepackage{graphicx}
\usepackage{import}
\usepackage{lscape}
\usepackage{modular}

\usepackage[pdfpagemode=UseNone,pdfstartview=FitH,colorlinks=true,linkcolor=blue,citecolor=blue,urlcolor=blue]{hyperref}
\usepackage[all]{hypcap}



\title{Generating functions for the first passage problem}
\author{Daniel Sank}


\begin{document}
\maketitle


\section{Introduction}

The problem is to compute ``first passage'' times for a process.
By first passage time we mean the average time taken before a process arrives at a particular state \emph{for the first time}.
This is a hard problem, but we will reduce it to the problem of computing the unconstrained probability for a process to be in a particular state after a certain time, which is much easier.
For example, with a Markov process, the probability distribution after $n$ steps is just given by the $n^{\text{th}}$ power of the process matrix.
First passage times are more difficult because it's not immediately clear how to mathematically account for the fact that the process ends once it arrives at the target state.


\section{Constrained and unconstrained probabilities}

Denote by $p_{ji}(n)$ the probability that a process goes from state $\ket{i}$ to state $\ket{j}$ after $n$ steps.
Here we are talking about unconstrained probabilities, ie. the processes never ends but keeps progressing freely at each step.
Denote by $f_{ji}$ the probability that, starting from state $\ket{i}$, a process winds up in state $\ket{j}$ \emph{for the first time} after $n$ steps.
The mean first passage time is $\sum_{n}nf_{ji}(n)$.
The trouble is that we don't know how to compute $f_{ji}$.
The key to our calculation is that we will relate $f_{ji}$ to $p_{ji}$.

We can break up any trajectory which starts at $\ket{i}$ and ends at $\ket{j}$ into two parts: the process makes its first arrival at $\ket{j}$, and then wanders around and eventually comes back to $\ket{j}$.
Mathematically this is expressed as
\begin{equation}
p_{ji} = \sum_{k=1}^n f_{ji}(k)p_{jj}(n-k) \quad (j\neq i) \, .
\end{equation}
In English, the probability of starting at $\ket{i}$ and then winding up at $\ket{j}$ after $n$ steps is equal to the sum of the probability that you go from $\ket{i}$ to $\ket{j}$ for the first time in $k$ steps, and then wander around for $n-k$ steps, eventually finishing at $\ket{j}$, summed over all possible intermediate path lengths $k$.

The obvious problem with this formula is that it involves the thing we want to know $f_{ji}(n)$, on the right hand side.
How do we fix this? Note that the sum is a convolution.
This suggests that we can decouple the convolved quantities via some kind of Fourier transform.
In fact we use the so called ``generating functions'' or ``z-transform,''
\begin{equation}
P_{ji}(z)=\sum_{n=0}^{\infty}z^{n}p_{ji}(n)\qquad F_{ji}(z)=\sum_{n=1}^{\infty}z^{n}f_{ji}(n) \, .
\end{equation}
The mean first passage time is just
\begin{equation}
\langle n \rangle \equiv \sum_n n f_{ji}(n) = \left. \frac{d}{dz}\right|_{z=1}F_{ji}(z)
\end{equation}
so if we can compute $F_{ji}(z)$ we will have a solution to our problem.
Note that higher derivatives of $F$ provide higher moments of the first passage problem.
This is why $F$ is called a ``generating function''.

To relate $P_{ji}(z)$ and $F_{ji}(z)$ we insert the convolution into the definition of $P_{ji}(z)$
\begin{align*}
P_{ji}(z)
& = \sum_{n=0}^{\infty}z^{n}\sum_{k=1}^{n}f_{ji}(k)p_{jj}(n-k)\\
& = \sum_{n=0}^{\infty}\sum_{k=1}^{n}z^{n-k}p_{jj}(n-k)\, z^{k}f_{ji}(k)\\
& = \sum_{n=0}^{\infty}z^{n}p_{jj}(n)\sum_{k=1}^{\infty}z^{k}f_{ji}(k)\\
& = P_{jj}(z)F_{ji}(z)\\
\textrm{therefore}\quad F_{ji}(z) & = P_{ji}(z)/P_{jj}(z) \, .
\end{align*}
Therefore, the mean first passage time is
\begin{equation}
\langle n \rangle = \lim_{z \rightarrow 1} \frac{d}{dz} \frac{P_{ji}(z)}{P_{jj}(z)} \, .
\end{equation}
This is a truly remarkable result: we have now expressed the solution to the first passage problem entirely in terms of the unconstrained process.
For the simple case of a Markov process the generating functions $P$ are easy to compute because they are geometric sums of $p$.

\end{document}

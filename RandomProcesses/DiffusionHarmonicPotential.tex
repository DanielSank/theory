%% LyX 1.6.6 created this file.  For more info, see http://www.lyx.org/.
%% Do not edit unless you really know what you are doing.
\documentclass[english,aps,manuscript]{revtex4}
\usepackage[T1]{fontenc}
\usepackage[latin9]{inputenc}
\usepackage{esint}

\makeatletter
%%%%%%%%%%%%%%%%%%%%%%%%%%%%%% Textclass specific LaTeX commands.
\@ifundefined{textcolor}{}
{%
 \definecolor{BLACK}{gray}{0}
 \definecolor{WHITE}{gray}{1}
 \definecolor{RED}{rgb}{1,0,0}
 \definecolor{GREEN}{rgb}{0,1,0}
 \definecolor{BLUE}{rgb}{0,0,1}
 \definecolor{CYAN}{cmyk}{1,0,0,0}
 \definecolor{MAGENTA}{cmyk}{0,1,0,0}
 \definecolor{YELLOW}{cmyk}{0,0,1,0}
 }

\makeatother

\usepackage{babel}

\begin{document}

\title{Random Diffusion in a Harmonic Potential}


\author{Daniel Sank}


\date{8 December 2009}


\affiliation{University of California Santa Barbara}

\maketitle

\section{Statement of the Problem}

Newton's equation of motion for a particle in a viscous medium subject
to random driving is\[
m\ddot{x}=-kx-2m\gamma\dot{x}+m\xi(t)\]
where $m\xi(t)$ is the random driving force and $m\gamma$ is the
friction coefficient. We would like to study the stochastic properties
of this system, for example the probability distribution for the position
$x$ and perhaps it's correlation function.


\section{Solution by Fourier Transform}

Rearranging we have\[
\ddot{x}+2\gamma\dot{x}+\omega_{0}^{2}x=\xi(t)\]
We define Fourier transforms\begin{eqnarray*}
x(t) & = & \int\frac{d\omega}{2\pi}\tilde{x}(\omega)e^{i\omega t}\\
\tilde{x}(\omega) & = & \int dt\, x(t)e^{-i\omega t}\end{eqnarray*}
Then transforming the equation of motion gives\[
(-\omega^{2}+2i\gamma\omega+\omega_{0}^{2})\tilde{x}=\tilde{\xi}(\omega)\]
\[
\tilde{x}(\omega)=\frac{-\tilde{\xi}(\omega)}{\omega^{2}-2i\gamma\omega-\omega_{0}^{2}}\]
To find the correllation function of $x$ we can use the Weiner-Kintchein
relation\[
\int e^{i\Omega\tau}S_{p}(\Omega)\frac{d\Omega}{2\pi}=\int dt\, f(t)f(t+\tau)\]
where $S_{p}$ is the Physicist's power spectrum, $S_{p}(\omega)=|\tilde{x}(\omega)|^{2}$.
We are therefore lead to calculate $|\tilde{x}(\omega)|^{2}$,\begin{eqnarray*}
|\tilde{x}(\omega)|^{2} & = & \frac{|\tilde{\xi}(\omega)|^{2}}{(\omega^{2}-\omega_{0}^{2})^{2}+(2\gamma\omega)^{2}}\\
\int dt\, x(t)x(t+\tau) & = & \int\frac{d\omega}{2\pi}\frac{|\tilde{\xi}(\omega)|^{2}\, e^{i\omega\tau}}{(\omega^{2}-\omega_{0}^{2})^{2}+(2\gamma\omega)^{2}}\end{eqnarray*}

\end{document}

\levelstay{Autocorrelation}

\leveldown{Problem}

According to the random step model of Brownian motion, the particle position is, after $n$ random steps, given by
\begin{equation*}
  X(n) = \sum_{i=1}^n X_i
\end{equation*}
where the $X_i$ are independent displacements with $\avg{X_i}=0$ and $\avg{X_i^2} = \Delta x^2$ for all $i$.
Of course, after $m$ random steps (with $m \leq n$), the particle position is $X(m)$.
In general, $X(n)$ and $X(m)$ are different random variables.
\begin{itemize}
  \item[a.] Find $\text{cov}(X(n), X(m))$.
  \item[b.] Find $\text{cor}(X(m), X(m))$.
  \item[c.] Show that $X(n)$ and $X(m)$ become completely uncorrelated as $m/n \to 0$ and completely correlated as $m/n \to 1$.
    The quantity $\text{cov}(X(n), X(m))$ is sometimes referred to as an \textit{autocovariance} and $\text{cor}(X(n), X(m))$ as an \textit{autocorrelation} because they compare the same process variable at different times.
\end{itemize}

\leveldown{Solution}

For $n > m$, we can write $X(n) = X(m) + \sum_{i=m+1}^n X_i$.
Therefore
\begin{align*}
  a. \qquad \qquad
  \text{cov}(X(n), X(m))
  &\equiv \avg{X(n) X(m)} - \underbrace{\avg{X(n)} \avg{X(m)}}_0 \\
  &= \avg{X(m)^2 + X(m) \sum_{i=m+1}^n X_i} \\
  &= \avg{X(m)^2} + \sum_{j=1}^m \sum_{i=m+1}^n \avg{X_i X_j} \\
  &= \avg{X(m)^2} + \sum_{j=1}^m \sum_{i=m+1}^n \underbrace{\avg{X_i} \avg{X_j}}_0 \\
  &= m \Delta x^2 \, .
\end{align*}
As $\text{var}(X(m)) = m \Delta x^2$ and $\text{var}(X(n)) = n \Delta x^2$, we have
\begin{align*}
  b. \qquad \qquad
  \text{corr}(X(m)X(m))
  &= \frac{\text{cov}(X(m), X(n))}{\sqrt{\text{var}(X(m)) \text{var}(X(n))}} \\
  &= \frac{m\Delta x^2}{\sqrt{nm \Delta x^4}} \\
  &= \sqrt{\frac{m}{n}} \\
  c. \qquad \qquad \text{Trivial result of }b \, .
\end{align*}

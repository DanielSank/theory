\levelstay{Terminal Speed}

\leveldown{Problem}

One way to determine the viscous drag parameter $\gamma$ is to apply a steady force $F$ to a particle of mass $M$ and measure its mean terminal speed $v_d$.

\begin{itemize}
  \item Express $\gamma$ in terms of $F$, $M$ and $v_d$.
  \item Use the fluctuation-dissipation theorem to express the fluctuation parameter $\beta^2$ in terms of $kT$, $F$, $M$, and $v_d$, where $T$ is the temperature.
\end{itemize}

\levelstay{Solution}

The drag parameter $\gamma$ is defined by $\dot{v} = - \gamma v$.
Newton's law says that $\text{Force} = \dot{p} = M \dot{v}$, so the drag force is $- M \gamma v$.
Assuming the drag force to be in equilibrium with a driving force $F$, we get
\begin{align}
  F =& M \gamma v_d \\
  \text{or} \qquad \gamma =& F / (M v_d) \, .
\end{align}

The fluctuation-dissipation theorem says
\begin{displaymath}
  \frac{M \beta^2}{4 \gamma} = \frac{kT}{2}
\end{displaymath}
so
\begin{displaymath}
  \beta^2 = \frac{2 F k T}{M^2 v_d} \, .
\end{displaymath}

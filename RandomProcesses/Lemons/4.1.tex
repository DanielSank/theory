\levelstay{Single-Slit Diffraction}

\leveldown{Problem}

According to the probability interpretation of light, formulated by Max Born in 1962, light intensity at a point is proportional to the probability that a photon exists at that point.
\begin{itemize}
  \item[a.] What is the probability density $p(x)$ that a single photon passes through a narrow slit and arrives at a position $x$ on a screen parallel to and at a distance $d$ beyond the barrier?
    Each angle of forward propagation $\theta$ is the uniform random variable $U(0, \pi/2)$.
    See Figure 4.5.
    Hint: Each differential range of realizations $(\theta + d\theta, \theta)$ maps into a differential range of realizations $(x+dx, x)$ in such a way that $p(\theta)d\theta = p(x)dx$, where the relationship between $\theta$ and $x$ is clear from the geometry.
  \item[b.] The light intensity produced by diffraction through a single, narrow slit, as found in alnmost any introductory physics text, is proportional to
    \begin{displaymath}
      \frac{1}{r^2} \frac{\sin^2 \left[ (\pi a / \lambda) \sin \theta \right]}{\sin^2 \theta}
    \end{displaymath}
    where $r$ is the distance from the center of the slit to an arbitrary place on the screen, $a$ is the slit width, and $\lambda$ the light wavelength.
    Show that for slits so narrow that $\pi a / \lambda \ll 1$, the above light intensity is proportional to the photon probability density arrived at in part a.
\end{itemize}

\levelstay{Solution}

The hint is an informal statement about the change of variables theorem applied to probability.
To understand what's going on clearly, note that the probability that an event happens in the shaded region in $X$ is
\begin{displaymath}
  \int P_x(x) dx = \int P_x(f^{-1}(y)) \text{det}(Df^{-1})(y) dy \, .
\end{displaymath}
Therefore, the probability density for the variable $y$ is defined by the equation
\begin{displaymath}
  P_y(y) = P_x(f^{-1}(y)) \text{det}(Df^{-1})(y) \, .
\end{displaymath}

\quickfig{0.6\columnwidth}
{{4.1_maps}.pdf}
{Given a probability density $P_x$ and a map $f$ to another variable $Y$, we have a new probability density $P_y = P_x \circ f^{-1}$.}
{fig:4.1_maps}

In our problem, $\theta$ plays the role of $X$ and $x$ plays the role of $Y$ (I'm sorry, this is stupidly confusing and we should change the variable names).
Given the equation $x = d \tan(\theta)$, we see that $\tan$ plays the role of $f$.
Therefore,
\begin{equation*}
  P_x(x) = \frac{1}{\pi} \frac{d}{dx} \tan^{-1}(x/d)
\end{equation*}
where the $1/\pi$ is the normalization of the uniform variable $\theta$, which goes from $-\pi/2$ to $\pi/2$.
Using $(d/dx)\tan^{-1}(x) = 1/(1 + x^2)$, we find
\begin{equation*}
  \qquad P_x(x) = \frac{1}{\pi} \frac{1}{d \left( 1 + (x/d)^2 \right)}
\end{equation*}
which solves part a.

For part b, use $\sin(x) \approx x$ for $x \ll 1$.
Then the expression given in the book becomes
\begin{align*}
  \frac{1}{r^2} \frac{((\pi a / \lambda) \sin \theta)^2}{\sin^2 \theta}
  &= \frac{1}{r^2} (\pi a / \lambda)^2 \\
  &= \left( \frac{\pi a}{\lambda} \right)^2 \frac{1}{d^2 + x^2} \, .
\end{align*}
This expression has the same dependence on $x$ as does the result from part a.

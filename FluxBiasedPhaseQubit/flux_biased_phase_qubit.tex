%% LyX 1.6.6 created this file.  For more info, see http://www.lyx.org/.
%% Do not edit unless you really know what you are doing.
\documentclass[twocolumn,english,aps,prl]{revtex4}
\usepackage[T1]{fontenc}
\usepackage[latin9]{inputenc}
\usepackage{amsmath}
\usepackage{amssymb}
\usepackage{esint}
\usepackage{graphicx}

\makeatletter
%%%%%%%%%%%%%%%%%%%%%%%%%%%%%% Textclass specific LaTeX commands.
\@ifundefined{textcolor}{}
{%
 \definecolor{BLACK}{gray}{0}
 \definecolor{WHITE}{gray}{1}
 \definecolor{RED}{rgb}{1,0,0}
 \definecolor{GREEN}{rgb}{0,1,0}
 \definecolor{BLUE}{rgb}{0,0,1}
 \definecolor{CYAN}{cmyk}{1,0,0,0}
 \definecolor{MAGENTA}{cmyk}{0,1,0,0}
 \definecolor{YELLOW}{cmyk}{0,0,1,0}
 }

%%%%%%%%%%%%%%%%%%%%%%%%%%%%%% User specified LaTeX commands.

\makeatother

\usepackage{babel}

\begin{document}

\title{Flux Biased Phase Qubit}


\author{Daniel Sank}


\affiliation{Department of Physics, University of California, Santa Barbara, California
93106, USA}


\date{\today}
\begin{abstract}
I calculate the bias dependence of relevant qubit parameters, like resonance frequency. I also show the link between the quantum results and a classical circuit model.
\end{abstract}
\maketitle

\section{Reduction to qubic well}

The Hamiltonian for the phase qubit is \begin{eqnarray}
H &=& \underbrace{\frac{Q^{2}}{2C}}_{\mathrm{Kinetic}}+\underbrace{E_{J}\left[\frac{\left(\delta-\varphi\right)^{2}}{2\lambda}-\cos\delta\right]}_{\textrm{Potential }U(\delta)} \label{eq:hamiltonian} \\
\left[ \delta,Q \right] &=& i2e \label{eq:commutator}
\end{eqnarray}
where $\delta$ is the phase accross the junction, $\varphi\equiv2\pi\Phi_{\mathrm{bias}}/\Phi_{0}$
is the dimensionless bias flux bias, $E_{J}\equiv\Phi_{0}I_{0}/2\pi$
is the junction energy scale, and $\lambda\equiv L/L_{J_{0}}$ is
the ratio of the shunt inductance to the junction inductance, where
$L_{J_{0}}$ is the zero bias inductance of the junction. Note that
$\delta$ and $Q$ aren't exactly cononically conjugate because their
commutator is $[\delta,Q]=2ei$. At zero flux bias there is a potential
minimum for $\delta=0$. As $\varphi$ is increased, this minimum
moves up in energy and becomes more shallow. For a critical value
of the flux bias, $\varphi_{c}$ the well disappears. It is near this
condition that the well is sufficiently shallow and nonlinear to serve
as a qubit. To find the critical flux we observe that the critical
condition is reached when the extrema collide with the inflection
point between them. The equation for the extrema is\begin{eqnarray}
\frac{dU}{d\delta} & = & 0\nonumber \\
\frac{\delta-\varphi}{\lambda} & = & -\sin\delta\label{eq:extrema}\end{eqnarray}
and the equation for the inflection points is\begin{eqnarray}
\frac{d^{2}U}{dU^{2}} & = & 0\nonumber \\
\frac{1}{\lambda} & = & -\cos\delta_{c}\label{eq:criticalDelta}\\
\delta_{c} & = & \frac{\pi}{2}+\sin^{-1}\left(1/\lambda\right)+(2\pi n)\nonumber \end{eqnarray}
Since $\sin^{-1}\left(1/\lambda\right)$ is a small number for $\lambda\sim4.6$,
the critical phase $\delta_{c}$ is slightly larger than $\pi/2$.
Note that these equations imply\begin{align*}
\sin\delta_{c} & =\sqrt{1-\lambda^{-2}}\\
\cos\delta_{c} & =-1/\lambda\end{align*}
which are important in going between the mathematical parameter $\delta_{c}$
and the physically controllable parameter $\lambda$.

We focus attention on the inflection point near $\delta=0$ by considering
only $n=0$ in equation (\ref{eq:criticalDelta}). We find the critical
flux by requiring the extrema to collide with the inflection point
by substituting $\delta_{c}$ into (\ref{eq:extrema}). Solving for
the flux,\begin{eqnarray}
\delta_{c}-\varphi_{c} & = & -\lambda\sin(\delta_{c})\nonumber \\
\varphi_{c} & = & \delta_{c}-\tan\delta_{c}\label{eq:criticalFlux}\\
\textrm{or}\qquad\varphi_{c} & = & \frac{\pi}{2}+\sin^{-1}\left(1/\lambda\right)+\lambda\sqrt{1-\lambda^{-2}}\nonumber \end{eqnarray}
Since $\lambda=4.6$ for normal phase qubits, we approximate $\sqrt{1-\lambda^{-2}}\approx1$
and $\sin^{-1}(1/\lambda)\approx1$ to get

\begin{eqnarray*}
\varphi_{c} & \approx & \frac{\pi}{2}+\lambda\\
\Phi_{\mathrm{c}} & \approx & \Phi_{0}\left(\frac{1}{4}+\frac{\lambda}{2\pi}\right)\approx0.98\Phi_{0}\end{eqnarray*}
This condition tells us how much flux we need to be able to couple
into the qubit circuit in order to acheive a useable qubit well. Note
that the number of $\Phi_{0}$'s that we need increased linearly with
the shunt inductance $L$.

Now we expand the potential $U(\delta)$ in (\ref{eq:hamiltonian})
near $\delta_{c}$ by defining $\delta=\delta_{c}+y$,\begin{eqnarray*}
U(\delta) & =E_{J} & \left[\frac{\left(\delta_{c}+y-\varphi\right)^{2}}{2\lambda}-\cos\left(\delta_{c}+y\right)\right]\end{eqnarray*}
Expanding for small $y$ and dropping terms that don't depend on $y$
results in\begin{displaymath}
U(\delta)=E_{J}\sin\delta_{c}\left[\epsilon^2 \frac{y}{2}-\frac{y^{3}}{6}\right]\end{displaymath}
where\begin{align}
\epsilon &= \sqrt{1-\delta_{c} \cot\delta_{c}}\sqrt{2(1-\varphi/\varphi_{c})}\label{eq:epsilonDef}\\
\textrm{or}\qquad\epsilon &= \sqrt{\frac{\varphi_{c}}{\lambda\sqrt{1-\lambda^{-2}}}}\sqrt{2(1-\varphi/\varphi_{c})}\nonumber \end{align}
This agrees with Tsuyoshi's (21) and (22). This potential curve has a local minimum at $y=-\epsilon$. Expanding
around that point by writing $\delta=y+\epsilon$ gives\begin{equation}
U(\delta)=E_{J}\sin\delta_{c}\left[\epsilon\frac{\delta^{2}}{2}-\frac{\delta^{3}}{6}\right]\label{eq:potential}\end{equation}
For reference, similar analysis for the current biased junction results
in
\begin{displaymath}
U_{I}(\delta)=E_{J}\left[\epsilon_{I} \frac{\delta^{2}}{2}-\frac{\delta^{3}}{6}\right]\quad\epsilon_{I}\equiv\sqrt{2(1-I/I_{0})}\end{displaymath}
The flux biased case differs from the current biased case in two ways.
First, the prefactor $E_{J}$ from the current biased case goes to
$E_{J}\sin\delta_{c}$. This is effectively a change in the critical
current, but the change is small since $\delta_{c}$ is close to $\pi/2$.
This agrees with John's writeup {}``Mapping of Flux-Bias to Current-Bias
Circuits in a Phase Qubit.'' The other change is a modification of
the $\epsilon$ parameter by a the square root factor in the flux
biased case. This factor is is really close to unity, but will result
in a small modification of the plasma frequency as we'll see later.

In what follows we will use the parameter $\epsilon$ to do computation,
but for actually plotting curves it's better to use a scaled flux
bias $x\equiv\varphi/\varphi_{c}$ to remove the dependence of the
critical bias flux on the qubit loop inductance. Inserting this definition
into (\ref{eq:epsilonDef}), using the definition of $\varphi_{c}$,
and packaging the $\delta_{c}$ dependent prefactor into a scaling
function gives \begin{eqnarray*}
\epsilon & = & \sqrt{1-\delta_{c}\cot\delta_{c}}\sqrt{2}\left(1-x\right){}^{1/2}\\
\epsilon & = & f(\delta_{c})\sqrt{2}\left(1-x\right)^{1/2}\end{eqnarray*}
Note that $f$ can be expressed in terms of either $\delta_{c}$
or $\lambda$ \begin{align}
f(\delta_{c}) & =\sqrt{1-\delta_{c}\cot\delta_{c}}\\
f(\lambda) & =\sqrt{1+\frac{\pi/2+\sin^{-1}(1/\lambda)}{\lambda\sqrt{1-\lambda^{-2}}}}\nonumber \end{align}



\section{Qubit parameters}

Now we derive equations for the dependence of qubit parameters like
the number of states, nonlinearity and so forth. We will write the
important relations in terms of $\epsilon$ and also in terms of $x$.
The expressions with $\epsilon$ are useful for finding scaling relationships
between the computed quantities, while the relationships using $x$
are more useful for generating plots that match experimental data,
since $x$ is proportional to the bias flux.

The first quantity we want is the plasma frequency $f_{p}$. My Quantum
Harmonic Oscillator Cheat Sheet says that for a system with Hamiltonian \begin{displaymath}
H=\frac{1}{2}\alpha x^{2}+\frac{1}{2}\beta y^{2}\qquad[x,y]=i\gamma \end{displaymath}
the plasma frequency is determined by \begin{displaymath}
\hbar \omega_{p}=\gamma\sqrt{\alpha\beta} \end{displaymath}
Comparison to (\ref{eq:potential}) neglecting the cubit term and (\ref{eq:commutator}) gives $\alpha=E_{J}\sin\left(\delta_{c}\right)\epsilon$,
$\beta=1/C$, and $\gamma=2e$. Plugging into the formula for $f_{p}$
results in\begin{eqnarray}
f_{p} & = & f_{0}\sqrt{\sin\delta_{c}}\epsilon^{1/2}\nonumber \\
f_{p} & = & f_{0}\sqrt{\sin\delta_{c}}\sqrt{f(\delta_{c})}2^{1/4}(1-x)^{1/4}\label{eq:plasmaFrequency}\end{eqnarray}
where $f_{0}\equiv\left(2\pi\sqrt{L_{J_{0}}C}\right)^{-1}$ is a characteristic frequency of the device which is $\sim13\mathrm{GHz}$ for
a $2\mu\mathrm{A}$ junction and 1pF shunt capacitor.

Next we compute the barrier height. From Figure ?? this is clearly
seen to be $\Delta U=U(2\epsilon)$. Plugging into (\ref{eq:potential})
gives\begin{equation}
\Delta U=\frac{2}{3}E_{J}\sin\left(\delta_{c}\right)\epsilon^{3}\label{eq:barrierHeight}\end{equation}
which agrees with Tsuyoshi's result.

The number of states in the well is given by $N=\Delta U/hf_{p}$.
Plugging in expressions from (\ref{eq:plasmaFrequency}) and (\ref{eq:barrierHeight})
gives\begin{eqnarray}
N & = & \frac{2}{3}\frac{F_{0}}{f_{0}}\sqrt{\sin\delta_{c}}\epsilon^{5/2}\nonumber \\
 & = & \frac{2}{3}\frac{F_{0}}{f_{0}}\sqrt{\sin\delta_{c}}2^{5/4}f(\delta_{c})^{5/2}(1-x)^{5/4}\label{eq:eqNumberOfStates}\end{eqnarray}
where $F_{0}\equiv I_{0}/2\pi(2e)$ is a characteristic frequency
of the junction, near 1000GHz for a $2\mu\mathrm{A}$ junction.

The first transition frequency is found by perturbation theory to
be (see Tsuyoshi's writeup)\begin{equation}
f_{10}=f_{p}\left(1-\frac{5}{36}\frac{1}{N}\right)\label{eq:transitionFrequency}\end{equation}
and the nonlinearity is\[
f_{\mathrm{nl}}=f_{p}\frac{5}{36}\frac{1}{N}\]
An important quantity is the fractional nonlinearity,\begin{eqnarray*}
\frac{f_{\mathrm{nl}}}{f_{p}} & = & \left(\frac{36}{5}N-1\right)^{-1}\\
 & = & \left(\frac{24}{5}\frac{F_{0}}{f_{0}}\sqrt{\sin\left(\delta_{c}\right)}2^{5/4}f(\delta_{c})^{5/2}(1-x)^{5/2}-1\right)^{-1}\end{eqnarray*}
which, interestingly, improves as $\lambda$ is increased! This is
good for the possibility of high inductance qubits.


\section{Comparison to classical circuit model}

While the transition frequency and nonlinearity depend on the cubic
part of the potential and are therefore quantum in nature, the plasma
frequency is a purely classical quantitiy. As such, we should be able
to derive an expression for it from simple circuit modelling. In order
to do this, we have to figure out how much inductance the junction
has as a function of bias flux. Consider the circuit diagram in Figure
??. As flux is coupled into the circuit, the circulating current will
increase according to $\dot{\Phi}=L(I)\dot{I}$ where $L(I)$ is the
total circuit inductance. This inductance is just the series inductance
of the loop inductance $L$ and the junction inductance $L_{J_{0}}/\sqrt{1-(I/I_{0})^{2}}$.
Inserting these expressions and integrating gives\begin{eqnarray}
\Phi & = & \int_{0}^{I}\left(L+\frac{1}{\sqrt{1-(I/I_{0})^{2}}}\right)dI\\
2\pi\frac{\Phi}{\Phi_{0}} & = & \lambda\frac{I}{I_{0}}+\sin^{-1}\left(I/I_{0}\right)\\
\varphi & = & \lambda\sin\delta+\delta\label{eq:deltaVsPhi}\end{eqnarray}
which is exactly equivalent to (\ref{eq:extrema}). If we want to
compare to the quantumly derived expression for the plasma frequency,
we have to get $\delta$ in terms of the bias parameter $x$. Writing
$\varphi=x\varphi_{c}$ and using the definition (\ref{eq:criticalFlux})
of $\varphi_{c}$ we get\begin{eqnarray}
x\left(\delta_{c}-\tan\delta_{c}\right) & = & \lambda\sin\delta+\delta\\
x & = & \frac{\delta-\tan\delta\left(\cos\delta/\cos\delta_{c}\right)}{\delta_{c}-\tan\delta_{c}}\label{eq:xOfDelta}\end{eqnarray}
This is a relation between the flux bias and the phase accross the junction,
with $\delta_{c}$ (or $\lambda$) as a parameter. It can't be analytically
inverted to obtain $\delta$ as a function of $x$ even if we were
to approximate the denominator for $\delta_{c}$ very close to $\pi/2$
(large $\lambda$). It can, however, be numerically inverted in order
to obtain curves for the junction inductance vs. bias. For our present
purposes we don't need to invert because we can just insert equation
(\ref{eq:xOfDelta}) into our expression for $f_{p}$ and compare
with a classical model. The classical model is just that the circuit
is a parallel LC, whose total inductance $L_{\mathrm{T}}$ is the
parallel combination of the shunt inductance and the junction inductance,\begin{eqnarray*}
L_{\mathrm{T}} & = & \frac{L_{J}L}{L_{J}+L}\\
L_{\mathrm{T}} & = & L_{J_{0}}\frac{\lambda\sec\delta}{\lambda+\sec\delta}\end{eqnarray*}
The classical frequency $\left(2\pi\sqrt{L_{\mathrm{T}}C}\right)^{-1}$
then turns out to be\begin{eqnarray}
f_{p,\mathrm{classical}} & = & f_{0}\sqrt{\cos\delta-\cos\delta_{c}}\label{eq:plasmaFrequencyClassical}\end{eqnarray}
One could in principle show that this expression is equivalent to
(\ref{eq:plasmaFrequency}) for $\delta$ near $\delta_{c}$ but it's
easier to just plot $f_{p,\mathrm{classical}}$ and $f_{p}$ vs. $\delta$ (or $x$)
for various values of either $\delta_{c}$ or $\lambda$, and you
can see that they match up. This is done in Figure \ref{Fig:quantumClassicalCompare}.

\begin{figure}
\begin{centering}
\includegraphics[scale=0.18]{quantumClassicalCompare.eps}
\par\end{centering}

\caption{Plasma frequencies shown for the quantum (red) and classical (blue) models. Here $\lambda=4.6$.}
\label{Fig:quantumClassicalCompare}
\end{figure}

\section{Junction phase vs. operating frequency}
We can use our formula linking $x$ and $\delta$ to obtain curves for the junction phase versus
qubit operating frequency. This is done simply by numerically solving for $\delta$ as a function of $x$
and then plotting both $\delta(x)$ and $f_{10}(x)$ parametrically on the same axes. This is done in
Figure \ref{fig:deltaVsFreq}. 

\begin{figure}
\begin{centering}
\includegraphics[scale=0.18]{deltaVsFreq.eps}
\par\end{centering}

\caption{$\delta$ vs $f_{10}$. Assuming $\lambda=4.6$}
\label{fig:deltaVsFreq}
\end{figure}

It may be useful to know the inductance of the junction during qubit operation. The inductance is given as $L_J=L_{J_{0}}\sec(\delta)$. We plot the scaling factor $\sec(\delta)$ versus qubit frequency in Figure \ref{fig:inductanceFactor}.

\begin{figure}
\begin{centering}
\includegraphics[scale=0.18]{inductanceFactorVsFrequency.eps}
\par\end{centering}

\caption{$\sec(\delta)$ vs $f_{10}$. Assuming $\lambda=4.6$, $f_0=12.6$GHz, and $F_0=993$GHz.}
\label{fig:inductanceFactor}
\end{figure}


\end{document}

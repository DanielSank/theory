\levelstay{Ramsey}

In this section, we study the time dependence of the $p_\ket{0}$ for the sequence
\begin{equation}
  X_{\pi/2} \longrightarrow \text{Wait time }t \longrightarrow X_{-\pi/2}
\end{equation}
which is known as a ``Ramsey sequence''.

\leveldown{General noise spectrum}

The phase accumulated due to noise during the wait period is
\begin{align}
\phi(t)
  &= \int_0^t \left( \omega_0 + \delta \omega (t') \right) \, dt' \nonumber \\
  &= \omega_0 t + \int_0^t \delta \omega (t') \, dt' \nonumber \\
  &= \omega_0 t + \frac{\partial \omega}{\partial \lambda} \int_0^t \delta \lambda(t') \, dt' \label{eq:sure_and_random}
\end{align}
where $\lambda$ is the physical quantity that is actually causing the qubit frequency noise, ie. $\lambda$ could be a magnetic flux controlling the frequency of an electron transition in an atom.
There are two terms: the sure term depending on $\omega_0$ and the random term depending on $\delta \lambda$.
If $\lambda(t)$ is Gaussian process then $\phi(t)$ is Gaussian distributed, roughly because the distribution of the sum of random variables is usually Gaussian distributed.\footnote{This is called the ``central limit theorem''.}
Note that when we say that $\phi$ is ``distributed'' in here we're implicitly referring to an ensemble of experiments.
In other words, if we were to measure $\phi(t)$ many times and make a histogram, that histrogram would have a Gaussian shape.
For the rest of this discussion, we focus only on the random term because in practical applications the pulse phases are adjusted so that we are effectively working in a rotating frame where $\omega_0 = 0$.

Denote as $p_0(\phi)$ the probability, as a function of $\phi$, to measure $\ket{0}$ at the end of the Ramsey sequence.
If $\phi = 0$, then the first and second rotations cancel to send the qubit to $\ket{0}$, so $p_0(0) = 1$.
On the other hand, if $\phi = \pm \pi$, then the two rotations combine to send the qubit to $\ket{1}$, so $p_0(\pm \pi) = 0$.
For other values of $\phi$, the probability must oscillate sinusoidally, so
\begin{equation}
  p_0 (\phi) = \frac{1}{2}\left(1 + \cos(\phi) \right) \, .
\end{equation}
For each certain wait time $t$, we repeat the experiment many times, measuring the qubit each time and computing the probability $p_0$.
Therefore, we sample the probability distribution of $\phi$, so
\begin{equation}
  p_0(t)
  = \langle\frac{1}{2}\left(1+\cos(\phi)\right)\rangle
  = \frac{1}{2}+\frac{1}{2}\langle\cos(\phi)\rangle
\end{equation}
where $\langle \cdot \rangle$ denotes an average of the ensemble of experiments.
Denote the probability distribution of $\phi$ by $\Phi(\phi)$.
Then we have
\begin{align}
  \langle\cos\left(\phi\right)\rangle
  & \equiv \int_{-\infty}^{\infty}\cos(\phi)\Phi(\phi)\, d\phi \nonumber \\
  & = \int_{-\infty}^{\infty}\exp(i\phi)\Phi(\phi)\, d\phi\qquad\textrm{because }\Phi\textrm{ is even} \nonumber \\
  & = \langle\exp(i\phi)\rangle \, .
\end{align}

To proceed we need to develop a mathematical identity concerning Gaussian distributions.
Consider a variable $x$ with Gaussian distribution.
Averages of function of $x$ are given by
\begin{equation}
  \langle f(x)\rangle
  = \frac{\int f(x)\exp(-\frac{1}{2}ax^{2})\, dx}{\int\exp(-\frac{1}{2}ax^{2})\, dx} \, .
\end{equation}
It is straightforward to show that $\langle x^{2}\rangle = 1 / a$, and that $\langle\exp(Ax)\rangle = \exp(A^2 / 2 a)$.
Therefore,
\begin{equation}
  \langle \exp(Ax)\rangle
  = \exp\left(\frac{1}{2}A^{2}\frac{1}{a}\right)
  = \exp \left( \frac{1}{2}\langle(Ax)^{2}\rangle \right) \, .
\end{equation}

Coming now back to the main calculation, we have
\begin{equation}
  \langle \cos\left(\phi\right) \rangle
  = \langle \exp(i\phi) \rangle
  = \exp\left( -\frac{1}{2} \langle \phi^{2} \rangle \right)
\end{equation}
and
\begin{equation}
  p_0(t) = \frac{1}{2} + \frac{1}{2} \exp \left( - \frac{1}{2} \angavg{\phi^2} \right) \, .
\end{equation}
Finally, we find $\angavg{\phi^2}$ using Eq.~(\ref{eq:sure_and_random}) as
\begin{align}
  \langle \phi^2 \rangle
  &= \left( \frac{\partial \omega}{\partial\lambda} \right)^2 \left\langle \int_{0}^{t}\lambda(t')\, dt' \, \int_0^t \lambda(t'') \, dt'' \right\rangle \nonumber \\
  &= \left( \frac{\partial\omega} {\partial\lambda} \right)^2 \int_0^t dt' \int_0^t dt'' \langle\lambda(t') \lambda(t'') \rangle \nonumber \\
  &= \left(\frac{\partial\omega} {\partial\lambda}\right)^2 \int_0^t dt' \int_0^t dt'' \int_0^{\infty} S_{\lambda}(f) \cos \left( 2\pi f(t''-t') \right) \, df \nonumber \\
  &= \left(\frac{\partial\omega}{\partial\lambda} \right)^2 \int_0^{\infty} S_{\lambda}(f) \frac{\sin\left(\pi ft\right)^{2}}{\left(\pi f\right)^2} \, df \nonumber \\
  &= \left(\frac{\partial\omega}{\partial\lambda} \right)^2 t^2 \int_0^{\infty} S_{\lambda}(f) \, \text{sinc} \left(\pi f t \right)^2 \, df
\end{align}
where $\text{sinc}(x) = \sin(x) / x$.
Putting it all together we get
\begin{equation}
  p_0(t) = \frac{1}{2} + \frac{1}{2} \exp \left(-\frac{1}{2} \left(\frac{\partial\omega_{01}}{\partial\lambda}\right)^2 t^2 \int_0^{\infty} S_\lambda (f) \, \text{sinc}(\pi f t)^2\, df \right)
\end{equation}
or writing $z = f t$
\begin{equation}
  p_0(t) = \frac{1}{2} + \frac{1}{2} \exp \left( -\frac{1}{2} \left( \frac{\partial\omega_{01}}{\partial\lambda} \right)^2 t \int_0^\infty S_{\lambda} \left( z/t \right) \, \text{sinc}(\pi z)^2 \, dz \right) \, .
\end{equation}


\levelstay{White noise spectrum}

Consider the case $S_{\lambda}(f) = S_{\lambda}$ over a frequency range $0$ to $f_\text{max}$.
In thise case,
\begin{equation}
  p_0(t) = \frac{1}{2} + \frac{1}{2}\exp \left( -\frac{1}{2}\left( \frac{d\omega_{10}}{d\lambda} \right)^2 t S_{\lambda} \int_0^{f_\text{max}t} \left( \frac{\sin(\pi z)} {\pi z} \right)^2 dz \right) \, .
\end{equation}
The integral can be approximated as
\begin{equation}
  \int_0^{f_\text{max}t} \left( \frac{\sin(\pi z)}{\pi z} \right)^2 dz \approx \frac{1}{2+(f_{\text{max}}t)^{-1}} \, .
\end{equation}
The approximation has a maximum error of 20\% occuring near $f_{\text{max}}t = 1$. 
Finally, we have for white noise in the Ramsey sequence
\begin{equation}
  p_0(t) \approx \frac{1}{2} + \frac{1}{2} \exp \left( -\frac{1}{2} \left( \frac{d\omega_{01}}{d\lambda} \right)^2 t S_{\lambda} \frac{1}{2 + (f_{\text{max}}t)^{-1}} \right) \, .
\end{equation}


\levelstay{1/f noise spectrum}

Now consider the case where $S_{\lambda}(f)=S^{*}/f$.
The integral we have to do is then
\begin{equation}
  \int_{z_\text{min}}^{z_\text{max}}S^{*}\frac{t}{z}\text{sinc}^2 \left(\pi z \right) dz
  = t S^*_lambda \times I
\end{equation}
where $I$ is the integral over frequency,
\begin{equation}
  I = \int_{z_{\text{min}}}^{z_{\text{max}}} \text{sinc}(\pi z)^{2}\frac{dz}{z} \, .
\end{equation}
Making the substitution $x=\log z$ we get
\begin{equation}
  I = \int_{x_\text{min}}^{x_\text{max}}\text{sinc}^2(\pi e^{x})\, dx \, .
\end{equation}
We assume that $x_\text{min}$ is a large negative number because $z_\text{min}$ typically much less than 1.
For example, the $1/f$ noise spectrum may continue below $1\,\text{mHz}$, while $t$ ranges from $0$ to perhaps $1\,\text{ms}$.
Changing variables again $x \rightarrow -x$ we get
\begin{align}
  I & = \int_{-x_\text{max}}^{-x_\text{min}}\text{sinc}^2 \left(\pi e^{-x}\right) \, dx\\
  & = \underbrace{\int_{-x_\text{max}}^0\text{sinc}^{2}\left(\pi e^{-x}\right)\, dx}_{I_1} + \underbrace{\int_0^{-x_{min}}\text{sinc}^2 \left(\pi e^{-x}\right) \, dx}_{I_2} \, .
\end{align}
Let's look at $I_1$.
What value do we take for $x_\text{max}$?
We work in the limit that the upper cutoff frequency of the $1/f$ spectrum is high compared to the times in the experiment, i.e. we take $x_\text{max}=\infty$.
In thise case we have
\begin{equation}
  I_1 = \int_{-\infty}^{0}\textrm{sinc}^{2}\left(e^{-x}\right)\, dx = -\text{Ci}(2\pi)=0.0226
\end{equation}
where $\text{Ci}(x) = -\int_{x}^{\infty}\cos(x')/x'\, dx'$.

The second integral is
\begin{align*}
  I_2 = \int_0^{N}\text{sinc}^2 \left(\pi e^{-x}\right)\, dx
  & = -\text{Ci}(2\pi e^{-x})|_0^N \\
  &+ \frac{1}{4\pi^{2}}e^{2x}\left[1-\cos\left(2\pi e^{-x}\right)\right]|_{0}^{N} \\
  &+ \frac{1}{2\pi}e^{x}\sin\left(2\pi e^{-x}\right)|_{0}^{N}
\end{align*}
where $N = -x_\text{min} = -\log(f_\text{min} t) \gg 1$.
We have to look at each of these three terms for $x=0$ and as $x=N\rightarrow\infty$.

\begin{itemize}

  \item $-\text{Ci}(2\pi e^{-x})$: For $x=0$ we have just $-\text{Ci}(2\pi)$.
  For $x \rightarrow \infty$, we have $-\left[\gamma+\log\left[2\pi e^{-N}\right]+O^{2}(e^{-N})\right]$ where $\gamma$ is Euler's constant.

  \item $(1/4\pi^2) e^{2x}\left(1-\cos(2\pi e^{-x})\right)$: For $x=0$ we get just 0.
    For $x \rightarrow \infty$ we get
    \begin{equation*}
      \frac{1}{4 \pi^2} e^{2x} \left[ 1 - \left( 1 - \frac{1}{2} 4 \pi^2 e^{-2x} + O^4 (e^{-x}) \right) \right] \approx \frac{1}{2} \, .
    \end{equation*}

  \item $\sin \left(2 \pi e^{-x}\right) / (2\pi)$: For $x=0$ this terms goes to 0.
    For $x \rightarrow \infty$ we have
    \begin{equation*}
      \frac{1}{2 \pi} \left(2 \pi e^{-x} + O^3(e^{-x}) \right) = 1 + O^3(e^{-x}) \approx 1 \, .
    \end{equation*}
\end{itemize}
Putting this all together we get for the second integral,
\begin{equation}
  I_2 = \int_0^{-x_\text{min}}\text{sinc}^2 \left( \pi e^{-x} \right)\, dx
  = \frac{3}{2} -\gamma -\log \left(2\pi \right) - \log\left( e^{-N} \right) + \text{Ci}\left(2\pi\right) \, .
\end{equation}
Adding $I_1 + I_2$, the $\text{Ci}(2\pi)$ terms cancel and we're left with
\begin{align}
  I
  & = \frac{3}{2}-\gamma - \log\left(2\pi\right) + N \nonumber \\
  \text{substitute } N = - \log\left(f_\text{min}t\right) \qquad
  & = \frac{3}{2} - \gamma - \log\left(2\pi\right) - \log\left( f_\text{min}t \right) \nonumber \\
  & = \log\left(\frac{\exp((3/2) - \gamma) / 2\pi}{f_\text{min} t} \right) \nonumber \\
  & = \log\left(\frac{0.4005}{f_\text{min}t}\right)
\end{align}
This matches the result obtained numerically in Ref.~\cite{Martinis:bias_noise:2003}.
Putting everything together the Ramsey decay envelope is
\begin{equation}
  p_0(t)
  = \frac{1}{2} + \frac{1}{2} \exp \left( -\frac{1}{2} \left( \frac{\partial\omega}{\partial\lambda} \right)^2 t^2 S^*_\lambda \ln\left(\frac{0.4005}{f_{\text{min}} t} \right) \right) \, .
\end{equation}
The log factor is in the neighborhood of 24 for values of $t$ encountered in real experiments.
Defining the decay constant by the equation
\begin{equation}
  p_0(t) \propto \exp \left[ - \left( t / T_{1/f,\text{Ramsey}} \right)^2 \right]
\end{equation}
we find
\begin{equation}
  T_{1/f,\text{Ramsey}}
  = \left( \frac{\partial\omega}{\partial\lambda} \right)^{-1} \frac{1}{\sqrt{12 S^*_\lambda}}
  \, .
\end{equation}

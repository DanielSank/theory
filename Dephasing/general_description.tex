\levelstay{Introduction}

The sequence for a Ramsey experiment is
\begin{equation}
X_{\pi/2}\rightarrow\textrm{wait for time }t\rightarrow X_{-\pi/2} \, .
\end{equation}
The first pulse rotates the qubit from $\ket{0}$ to a point on the equator of the Bloch sphere.
The qubit state then idles on the equator for time $t$.
In the absence of noise the state is unchanged during this time, so the final pulse rotates the state back to $\ket{0}$.
When there is noise the final state is not exactly $\ket{0}$.
It is shown elsewhere that energy relaxation effects ($T_1$) and frequency fluctuation effects $(T_2)$ can be considered separately because their effects on the decay envelopes of various experiments factor apart; for example the decay of a Ramsey fringe for a qubit under the influence of energy decay with time constant $T_1$ and phase decay with time constant $T_\phi$ is $\exp(-t(2T_1 + T_\phi)$.
Therefore, we consider only the effect of dephasing, ie. no energy relaxation.

During the part of the sequence in which the qubit idles at the equator of the Bloch sphere, the $\ket{0}\rightarrow\ket{1}$ transition frequency, and therefore the precession frequency, fluctuates and causes the qubit state to diffuse (make a random walk)
on the equator.
Therefore, the second rotation results in a qubit state different from $|0\rangle$ and so the measured probability of finding $\ket{0}$ is not 1.
The probability is what we can measure, and the decrease away from 1 is what we call the decay of the Ramsey curve.

If the qubit state is completely diffused on the equator then the expected value of the zero state probability after the second pulse is $\frac{1}{2}$.
Therefore, we expect our calculations to yield a curve that goes from 1 to $1/2$ as the wait time goes from zero to infinity.

\leveldown{Note on conventions}

In this document, spectral densities, e.g. $S(\omega)$, are single sided spectral densities.
This is the thing you read off of a spectrum analyzer, and it is defined for positive frequencies only.
The single sided spectral density $S$ satisfies the Wiener-Khinchin theorem
\begin{equation}
  \langle f(t) f(0) \rangle = \int_0^\infty \frac{d\omega}{2\pi} S_f(\omega) \cos(\omega t)
\end{equation}
and conversely
\begin{equation}
  S(\omega) = 2 \int_{-\infty}^\infty dt \, \langle f(0) f(t) \rangle \cos(\omega t) \, .
\end{equation}


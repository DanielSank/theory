\levelstay{Spin Echo}

\leveldown{General noise spectrum}

The spin echo sequence works similarly to the Ramsey except that the
qubit experiences a $\pi$-pulse halfway through.
This means that any frequency detuning essentially switches sign halfway through the sequence.
Therefore, the total phase noise experienced during the sequence is
\begin{equation}
  \phi = \frac{\partial\omega}{\partial\lambda} \left( \int_0^{t/2} \delta \lambda(t_1) dt_1 -\int_{t/2}^t \delta \lambda(t_2) dt_2 \right) \, .
\end{equation}
Using the same arguments as for the Ramsey sequence we can write down the measured probability at the end of the sequence
\begin{equation}
  p_0(t) = \frac{1}{2}+\frac{1}{2}\exp\left[-\frac{1}{2}\langle\phi^{2}\rangle \right] \, .
\end{equation}
The computation of $\langle\phi^2 \rangle_\phi$ proceeds similarly to the Ramsey case:
\begin{align}
  \langle\phi^{2}\rangle
  &= \left( \frac{\partial\omega}{\partial\lambda} \right)^2 \left\langle \left( \int_0^{t/2} \lambda(t_1) dt_1 - \int_{t/2}^t \lambda(t_2) dt_2 \right) \times \right. \nonumber \\
  & \quad \left. \left( \int_0^{t/2} \lambda(t_3) dt_{3} - \int_{t/2}^t \lambda(t_4) dt_4 \right) \right \rangle \nonumber \\
  & = \left( \frac{\partial \omega_{10}}{\partial\lambda} \right)^2
  \int_{f_{\textrm{min}}}^{\infty} df S_{\lambda}(f)
  \left[
  \int_0^{t/2} \int_0^{t/2} dt_1 \, dt_3 \cos[2\pi f (t_1 - t_3)] \right. \nonumber \\
  & \quad + \int_{t/2}^t \int_{t/2}^t dt_{2} \, dt_{4} \cos[2\pi f (t_2-t_4)] \nonumber \\
  & \quad - 2 \left. \int_{0}^{t/2}\int_{t/2}^{t} dt'dt'' \cos[2\pi f (t'-t'')] \right] \nonumber \\
  & = \left( \frac{\partial\omega}{\partial\lambda} \right)^2 2 \int_{f_{\text{min}}}^\infty S_\lambda (f) \left[ \left(\frac{\sin\left(\pi ft/2\right)}{\left(\pi f\right)}\right)^2 \left( 1 - \cos \left( \pi f t \right) \right) \right]df
  \label{eq:spinEchoRawIntegral} \\
  & = \left( \frac{\partial\omega}{\partial\lambda}\right)^2 t \int_{t f_{\text{min}}/2}^{\infty} S_\lambda \left( \frac{2 z}{t} \right) \left[ \left( \frac{\sin\left(\pi z\right)}{\left(\pi z\right)} \right)^2 \left(1 - \cos\left(2 \pi z \right) \right) \right] dz \, .
\end{align}
Note that Eq.~(\ref{eq:spinEchoRawIntegral}) matches Eq.~(35b) in Ref.~\cite{Martinis:bias_noise:2003}.
Plugging this into the formula for $p(t)$ yields
\begin{align}
  p_0(t)
  &= \frac{1}{2} + \frac{1}{2} \exp \left( - \frac{1}{2} \left( \frac{\partial\omega_{10}}{\partial\lambda} \right)^{2} t \int_{t f_{\text{min}}/2}^{\infty} S_\lambda \left( \frac{2z}{t} \right) \left[ \frac{\sin\left(\pi z\right)^{2}}{\left(\pi z\right)^{2}} \left(1 - \cos\left(2\pi z\right)\right) \right] dz \right) \, .
\end{align}


\levelstay{1/f noise spectrum}

In the case of 1/f noise where $S(f)=S^{*}/f$ the integral becomes
\begin{equation}
  S^* \frac{t}{2} \underbrace{\int_{t f_{\textrm{min}}/2}^{\infty} \left[ \frac{\sin\left(\pi z\right)^2}{\left(\pi z\right)^2} \left(1 - \cos\left(2\pi z\right)\right) \right] \frac{dz}{z}}_I
  = \frac{1}{2} t S^{*} \times I \, .
\end{equation}
Making the change of variables $x=\ln z$ we get
\begin{equation}
  I = \int_{\ln\left(tf_{\textrm{min}}/2\right)}^{\ln\left(tf_{\textrm{max}}/2\right)}\left(\frac{\sin\left(\pi e^{x}\right)}{\left(\pi e^{x}\right)}\right)^{2}\left(1-\cos\left(2\pi e^{x}\right)\right)dx \, .
\end{equation}
We now break the integral into two pieces
\begin{align}
I &=
\underbrace{\int_{\ln\left(tf_{\textrm{min}}/2\right)}^{0}\left(\frac{\sin\left(\pi e^{x}\right)}{\left(\pi e^{x}\right)}\right)^{2}\left(1-\cos\left(2\pi e^{x}\right)\right)dx}_{I_{1}} \nonumber \\
&+ \underbrace{\int_{0}^{\ln\left(tf_{\textrm{max}}/2\right)}\left(\frac{\sin\left(\pi e^{x}\right)}{\left(\pi e^{x}\right)}\right)^{2}\left(1-\cos\left(2\pi e^{x}\right)\right)dx}_{I_{2}} \, .
\end{align}
The second term $I_{2}$ is very small even if the upper limit of integration is extended to infinity.
Numerically we find $I_{2}\approx0.032$.
The first term requires only slightly more attention.
Taking the experiment times $t$ to be in the range $10\,\text{ns}$ to $1\,\text{ms}$ and the total time of the experiment to be on the order of minutes or an hour\footnote{This is the total time over which all points in the experiment are averaged}, we find $\ln(t f_{\text{min}} / 2) < -18$.
Inspection of a plot of the integrand shows that this is sufficiently far to the left of zero that the integral is completely insensitive to the value of this lower limit.
We therefore numerically integrate in $\textit{Mathematica}$ from $-10$ to with result $I_1 \approx 1.35$.
The spin-echo decay curve for 1/f noise is therefore
\begin{equation}
  p_0(t) = \frac{1}{2} + \frac{1}{2} \exp \left(-\frac{1.38}{4} \left( \frac{\partial\omega}{\partial\lambda} \right)^2 t^2 S^* \right) \, .
\end{equation}
So we have a Gaussian envelope, as in the case of the Ramsey sequence with $1/f$ noise, but now with a different decay constant.
Defining the decay constant by the equation
\begin{equation}
  p_0(t) \propto \exp \left[ - \left( t / T_{1/f,\text{echo}} \right)^2 \right]
\end{equation}
we find
\begin{equation}
  T_{1/f,\textrm{echo}}
  = \left( \frac{\partial\omega}{\partial\lambda} \right)^{-1} \frac{2}{\sqrt{1.35 \, S^*}}
  \, .
\end{equation}
The echo time is therefore calculated to be approximately $1.72/0.289=6$ times larger than the Ramsey decay times.


\levelstay{Infinite dimensions}

Consider the differential equation
\begin{equation}
  \frac{d}{dt} u(t) = - \gamma u(t) + J(t)
  \, .
\end{equation}
We're going to solve this using the idea of Green vectors, although now that we're handling functions we'll call what we're doine the ``method of Green functions''.
Remember that a function can be thought of as a representation of a vector in a specific basis.
Given a function $f:\mathbb{R} \rightarrow \mathbb{R}$, the numbers $f(x)$ can be thought of as the components of a vector $\ket{f}$ in a basis made up of vectors whose components, in their own basis, are zero everywhere except for at one $x'$.
That sounds a lot like delta functions.
Think of it like this:
\begin{align*}
  \ket{f} &= \ket{f} \\
  &= \sum_{x'} \ket{x'}\braket{x'}{f} \\
  &= \sum_{x'} \ket{x'} f(x')
  \, .
\end{align*}
Of course, this sum is really an integral,
\begin{equation}
  \ket{f} = \int dx' \ket{x'} f(x')
\end{equation}
and taking components
\begin{equation}
  \braket{x}{f} = \int dx' \braket{x}{x'} f(x')
\end{equation}
we see that $\braket{x}{x'}$ must be equal to $\delta(x - x')$.

Returning to the differential equation, we can now appreciate that $u(t)$ and $J(t)$ are components of vectors $\ket{u}$ and $\ket{J}$, i.e. the differential equation can be written as
\begin{equation*}
  \bbraket{t}{\frac{d}{dt}}{u} = - \gamma \braket{t}{u} + \braket{t}{J}
\end{equation*}
and peeling off the $\bra{t}$ from the left of each term, we're left with
\begin{equation}
  \left( \frac{d}{dt} + \gamma \right) \ket{u} = \ket{J}
\end{equation}
which is the same equation we studied in the finite dimensional case where now our linear transformation is\footnote{It's easy to see that this is indeed a linear transformation because the derivative is linear, i.e. $(d/dt)(f + g) = df/dt + dg/dt$.}
\begin{equation}
  T = \frac{d}{dt} + \gamma \, .
\end{equation}
Therefore, we can write down the solution using Eq.~(\ref{eq:three_step_process_c}):
\begin{equation}
  u(t) \equiv \braket{t}{u} = \int dt' J(t')
  \underbrace{\sum_j \frac{\braket{t}{F_j}\braket{F_j}{t'}}{t_j}}_{\braket{t}{G_{t'}}}
  \, .
  \label{eq:green_function_solution_mixed}
\end{equation}
The notation for the Green vector $\ket{G_{t'}}$ means ``the Green vector when the source is a delta function at time $t$''.
We've dropped the basis label because it's obvious that a source ``at time $t$'' is in the time basis.
Speaking of bases, we wrote the sum over $k$ as an integral because we know that the $E$ basis is the set of delta functions in time, but we haven't said yet what the $F$ basis is.
As discussed above, the $F$ basis is that which diagonalizes $T$.
In our present case, $T$ involves a time derivative, so it's reasonable to try exponential functions, as they are eigenfunctions of the derivative:
\begin{equation*}
  \frac{d}{dt} \exp(i \omega t) = (i \omega) \exp(i\omega t)
  \, .
\end{equation*}
Now of course, here we're expressing our $F$ basis vectors \emph{in the E basis} because we described them by giving their components.
We can infer, then, that an $F$ basis vector $\ket{\omega}$ should satisfy $\braket{t}{\omega} = \exp(i \omega t)$.
Taking that inference seriously, we convert the sum to an integral:\footnote{The factor of $2\pi$ in the integration measure corresponds to a choice about Fourier transform normalization. This is a somewhat subtle point that we're not getting into now.}
\begin{equation}
  u(t) = \int dt' J(t') \int \frac{d \omega}{2\pi} \frac{e^{i \omega (t - t')}}{t_\omega}
  \label{eq:green_function_solution_eigenvalues}
\end{equation}
where we relabled $t_j^F \rightarrow t_\omega$ for reasons that should be obvious.
So, what is $t_\omega$?
It supposed to be the eigenvalue of $T$ acting on basis vector $\ket{\omega}$, which we can evaluate in the $T$ basis (a.k.a the time basis):
\begin{align*}
  T \ket{\omega}
  &= \left( \frac{d}{dt} + \gamma \right) e^{i \omega t} \\
  &= ( i \omega + \gamma ) e^{i \omega t} \\
  \rightarrow t_\omega &= i\omega + \gamma
  \, .
\end{align*}
Therefore, our Green function is
\begin{equation*}
  \braket{t}{G_{t'}} = \int \frac{d \omega}{2\pi} \frac{e^{i \omega (t - t')}}{i \omega + \gamma}
\end{equation*}
and
\begin{equation}
  u(t) = \int dt' J(t') \int \frac{d\omega}{2\pi} \frac{e^{i \omega (t - t')}}{i \omega + \gamma}
\end{equation}
The integral over $\omega$ can be done easily by contour integration, with result
\begin{equation}
  \braket{t}{G_{t'}} = \left\{
    \begin{array}{rl}
      e^{-\gamma (t - t')} & \text{for } t > t' \\
      0 & \text{otherwise}
    \end{array}
  \right.
  \, .
\end{equation}
Therefore, the solution to our differential equation is the convolution
\begin{equation}
  u(t) = \int dt' J(t') G_{t'}(t) = \int_{-\infty}^t dt' J(t') e^{-\gamma (t - t')}
  \, .
\end{equation}
What does this solution mean intuitively?
It says that the value of $x$ at time $t$ is built up of contributions from all times $t' < t$.
For each $t'$, we weight the source value $J(t')$ by the Green's function $G_{t'}(t)$, which tells us how much the source at $t'$ influences the result at time $t$.

\leveldown{Discussion}
Equation (\ref{eq:green_function_solution_eigenvalues}) is an integral representation of the solutio to \emph{any} time invariant linear differential equation with a source term.
Other such equations include the wave equation and the diffusion equation.
So now you know how to solve two of the most important equations in physics, and find their Green functions.
The eigenvalue denominators are found just by converting all derivatives to $(i \omega)$.
Let's practice on the diffusion equation, which also gives us a chance to see how this works with \emph{partial} differential equations.

\levelstay{Partial differential equation}

The diffusion equation is
\begin{equation}
  \frac{\partial u}{\partial t} - \diffusionconst \frac{\partial^2 u}{\partial x^2} = J(x, t)
\end{equation}
or in a basis-independent form,
\begin{equation}
  \left( D_t - \diffusionconst D_x^2 \right) \ket{u} = \ket{J}
  \, .
\end{equation}
We use Eq.~(\ref{eq:green_function_solution_mixed}) to find the solution.
First, the Green's function is
\begin{align}
  \braket{t, x}{G_{t', x'}}
  &= \int \frac{dk}{2\pi}\frac{d\omega}{2\pi} \frac{\braket{t, x}{\omega, k} \braket{\omega, k}{t', x'}}{\text{eigenvalue}(\omega, k)} \nonumber \\
  &= \int \frac{dk}{2\pi}\frac{d\omega}{2\pi} \frac{e^{i\omega (t-t')} e^{i k(x - x')}}{\text{eigenvalue}(\omega, k)}
  \, .
\end{align}
The eigenvalue of our linear transformation $(D_t - \diffusionconst D_x^2)$ is, by reasoning explained above, equal to $(i \omega + \diffusionconst k^2)$, so the Green function is
\begin{align}
  \braket{t, x}{G_{t', x'}}
  &= -i \int \frac{dk}{2\pi}\frac{d\omega}{2\pi} \frac{e^{i\omega (t-t')} e^{i k(x - x')}}{\omega - i \alpha k^2} \nonumber \\
  &= -i \int \frac{dk}{2\pi} e^{i k(x - x')}
  \int \frac{d\omega}{2\pi} \frac{e^{i\omega (t - t')}}{\omega - i \alpha k^2} \nonumber
  \, .
\end{align}
The integral over $\omega$ can be done via contour integration.
There's a pole at $\omega = i \alpha k^2$ which we pick up if we close the contour in the upper half plane, which is appropriate when $t > t'$.
When $t < t'$ we close the contour in the lower half plane where there are no poles and the integral is zero.
This makes sense, because the solution at time $t$ should be influenced only by sources at earlier times.
Doing the integral we get
\begin{align}
  \braket{t, x}{G_{t', x'}}
  &= -i \int \frac{dk}{2\pi} e^{i k(x - x') - \diffusionconst k^2 (t - t')} \nonumber \\
  &= \frac{\Theta(t-t')}{\sqrt{4 \pi \diffusionconst (t - t')}} \exp \left(
    - \frac{(x - x')^2}{4 \diffusionconst (t - t')}
  \right)
  \, . \nonumber
\end{align}
This is the famous spreading Gaussian characteristic of diffusion.
The width of the distribution evolves as $\sigma(t) = \sqrt{2 \diffusionconst} \sqrt{t - t'}$.
Plugging this function into Eq.~(\ref{eq:green_function_solution_mixed}) gives the solution
\begin{align}
  u(x, t)
  & = \int_{-\infty}^\infty dx' \int_{-\infty}^t dt' \nonumber \\
  &\frac{J(x', t')}{\sqrt{4 \pi \diffusionconst (t - t')}} \exp
    \left( - \frac{(x - x')^2}{4 \diffusionconst (t - t')} \right)
\end{align}
which says that we have a spreading Gaussian centered at each point in space-time that the source $J$ acts.

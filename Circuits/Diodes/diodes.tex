\documentclass{article}
%Packages
\usepackage{amsmath}
\usepackage{amstext}
\usepackage{amssymb}
\usepackage{appendix}
\usepackage{coseoul}
\usepackage{enumerate}
\usepackage{graphicx}
\usepackage{import}
\usepackage{lscape}
\usepackage{modular}

\usepackage[pdfpagemode=UseNone,pdfstartview=FitH,colorlinks=true,linkcolor=blue,citecolor=blue,urlcolor=blue]{hyperref}
\usepackage[all]{hypcap}


% General physics constructs
\newcommand{\bra}[1]{\langle #1 |}
\newcommand{\ket}[1]{| #1 \rangle }
\newcommand{\braket}[2]{\langle #1|#2\rangle}
\newcommand{\bbraket}[3]{ \langle #1 | #2 | #3 \rangle }
\newcommand{\norm}[1]{\| #1\|}
\newcommand{\avg}[1]{\left \langle #1 \right \rangle}
\newcommand{\angavg}[1]{\left \langle #1 \right \rangle}
\newcommand{\abs}[1]{\left \lvert #1 \right \rvert}
\newcommand{\VS}{\textit{\textbf{V}}}
\newcommand{\Tr}{\textrm{Tr}}
\renewcommand{\Re}{\textrm{Re}}
\renewcommand{\Im}{\textrm{Im}}
\newcommand{\basis}[1]{\{\ket{#1}\}}

\newcommand{\omegaqubit}{\omega_{10}}

% Figures. Example usage:
% \quickfig{\columnwidth}{my_image}{This is the caption}{fig:my_fig}
\DeclareRobustCommand{\quickfig}[4]{
\begin{figure}
\begin{centering}
\includegraphics[width=#1]{#2}
\par\end{centering}
\caption{#3}
\label{#4}
\end{figure}
}

\DeclareRobustCommand{\quickwidefig}[4]{
\begin{figure*}[h]
\begin{centering}
\includegraphics[width=#1]{#2}
\par\end{centering}
\caption{#3}
\label{#4}
\end{figure*}
}


\title{Diodes}
\author{Daniel Sank}
\date{24 January 2011}

\begin{document}

\maketitle

The IV equation for a diode is
\begin{equation}
I_{d}=I_{0}(e^{V_{d}/V_{0}}-1)
\end{equation}
The input diode on the Vishay IL66/ILD66/ILQ66 photocouplers have values $I_{0}=5.1$ nA and $V_{0}=0.076$ V, giving a current of 10 mA for 1.1 V forward voltage drop.
From the IV equation, we can compute the diode's current dependent resistance,
\begin{align}
  dI_{d}/dV_{d}
  &= \frac{I_{0}}{V_{0}}e^{V_{d}/V_{0}} \nonumber \\
  R_d \equiv dV_{d}/dI_{d} &= \frac{V_0}{I_0} e^{-V_d / V_0} \nonumber \\
  &= R_{d,0} e^{-V_d / V_0} = R_{d,0} \frac{1}{I_d / I_0 + 1}
\end{align}
where $R_{d,0} = V_0 / I_0$ is the diode resistance at zero current.
$R_{d,0}$ is typically large; for the Vishay diodes mentioned above it is on the order of $10 \, \text{M}\Omega$.
This is why diodes can be thought of as current limiters.
As their voltage drop passes $V_0$, the differential resistance drops exponentially, meaning that increasing voltage would lead to exponentially rising current.
As this happens, something else in the circuit will limit the current.
As a specific example, consider a diode in series with a resistor and a voltage $V$ applied across both elements.
The currents through the resistor and diode are
\begin{equation}
  I_{d} = I_{0}(e^{V_d / V_0} - 1) \qquad I_R = (V - V_d) / R \, .
\end{equation}
Since the circuit is series, these currents must be the same, so
\begin{equation}
  \frac{V - V_d}{R} = I_0 (e^{V_d / V_0} - 1)
  \, .
  \label{eq:equal_currents}
\end{equation}
What are the diode current and voltage as a function of source voltage for a given resistor?
As a function of $V_d$ the left side of Eq.~(\ref{eq:equal_currents}) defines a line with negative slope and y intercept of $V/R$.
The right side defines an exponential curve with y-intercept of zero.
These curves obviously cross for some value of $V_{d}$, and we can see that due to the exponential curve's large slope, the value of $V_{d}$ for which the curves cross is insensitive to $V$ once $V$ is above a certain value.
This is why diodes can be though of having a fixed forward voltage drop once they are ``on''.

Let's take a quantitative look.
Look at $dV_d / dV$ by differentiating Eq.~(\ref{eq:equal_currents}) with respect to $V$
\begin{align}
  \frac{1}{R} \left( 1 - \frac{dV_d}{dV} \right) &= \frac{I_0}{V_0} e^{V_d / V_{0}} \frac{dV_d}{dV} \nonumber \\
  \frac{d V_d}{dV}
  &= \left( \frac{R}{R_{d,0}} e^{V_d / V_0} + 1 \right)^{-1} \nonumber \\
  &= \frac{R_{d,0}}{R \, e^{V_d / V_0} + R_{d,0}} \nonumber
  \, .
\end{align}
Except for the exponential, this would be the usual voltage division equation for the voltage drop across a resistor $R_{d,0}$ in series with a resistor $R$.
At zero bias $(V_d = 0)$, we do in fact have the usual voltage division equation.
Becuase $R_{d,0} \gg R$, most of the voltage is dropped across the resistor.
However, once $V_d$ significantly exceeds $V_0$, $dV_d / dV$ tends to zero.
In other words, as we crank up $V$, we see an initial voltage rise on the diode until the diode voltage exceeds $V_0$ at which point the diode voltage is mostly constant.

\end{document}

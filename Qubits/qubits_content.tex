\subimportlevel{./}{parallel_LC}{0}
\subimportlevel{./}{driving.tex}{0}
\subimportlevel{./}{coupling.tex}{0}
\subimportlevel{./}{rotating_frame}{0}
\subimportlevel{./}{asymmetric_drive}{0}
\subimportlevel{./}{decoherence.tex}{0}
\subimportlevel{./}{contributors.tex}{0}

%\subsubsection{Do LO phases matter?}

%We can now answer the question of whether we need to keep LO oscillator phases stable across multiple repetitions of a multiple qubit experiment. Consider an experiment in which we have to capacitively coupled qubits and an adjustable coupling $g(t)$. We put the qubits on resonance and rotate each qubit's frame at  qubit frequency. As previously explained in this doubly rotating frame the intrinsic qubit Hamiltonians are zero and the coupling Hamiltonian is $g(t)\left( \sigma^+ \sigma^- + \sigma^- \sigma^+ \right)$. We now subject the system to one of the following control sequences \begin{eqnarray}
%X_{\pi/2}^{(1)} \rightarrow & \textrm{turn g on for a swap} & \rightarrow X_{\pi/2}^{(2)} \rightarrow \textrm{measure qubit 2} \nonumber \\
%X_{-\pi/2}^{(1)} \rightarrow & \textrm{turn g on for a swap} & \rightarrow X_{\pi/2}^{(2)} \rightarrow \textrm{Measure qubit 2} \nonumber \end{eqnarray}
%In the first case we will measure $\ket{1}$ but in the second we will measure $\ket{0}$.  As explained in the section on driving in the rotating frame, the difference in the leading $X$ pulses is just a difference in the phase of the drive voltage. This phase could come from an intentional choice of our I and Q signals, or from a change in the LO phase. Therefore, in general LO phase drift will mess up experiments. This could be verified by the pulse sequence we've considered here, measuring ramsey fringes on qubit 2 as a function of the drive phase on qubit 1.

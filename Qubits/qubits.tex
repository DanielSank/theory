\documentclass[twocolumn]{article}

% General physics constructs
\newcommand{\bra}[1]{\langle #1 |}
\newcommand{\ket}[1]{| #1 \rangle }
\newcommand{\braket}[2]{\langle #1|#2\rangle}
\newcommand{\bbraket}[3]{ \langle #1 | #2 | #3 \rangle }
\newcommand{\norm}[1]{\| #1\|}
\newcommand{\avg}[1]{\left \langle #1 \right \rangle}
\newcommand{\angavg}[1]{\left \langle #1 \right \rangle}
\newcommand{\abs}[1]{\left \lvert #1 \right \rvert}
\newcommand{\VS}{\textit{\textbf{V}}}
\newcommand{\Tr}{\textrm{Tr}}
\renewcommand{\Re}{\textrm{Re}}
\renewcommand{\Im}{\textrm{Im}}
\newcommand{\basis}[1]{\{\ket{#1}\}}

\newcommand{\omegaqubit}{\omega_{10}}

% Figures. Example usage:
% \quickfig{\columnwidth}{my_image}{This is the caption}{fig:my_fig}
\DeclareRobustCommand{\quickfig}[4]{
\begin{figure}
\begin{centering}
\includegraphics[width=#1]{#2}
\par\end{centering}
\caption{#3}
\label{#4}
\end{figure}
}

\DeclareRobustCommand{\quickwidefig}[4]{
\begin{figure*}[h]
\begin{centering}
\includegraphics[width=#1]{#2}
\par\end{centering}
\caption{#3}
\label{#4}
\end{figure*}
}

%Packages
\usepackage{amsmath}
\usepackage{amstext}
\usepackage{amssymb}
\usepackage{appendix}
\usepackage{coseoul}
\usepackage{enumerate}
\usepackage{graphicx}
\usepackage{import}
\usepackage{lscape}
\usepackage{modular}

\usepackage[pdfpagemode=UseNone,pdfstartview=FitH,colorlinks=true,linkcolor=blue,citecolor=blue,urlcolor=blue]{hyperref}
\usepackage[all]{hypcap}


\newcommand{\citeinternaltype}{Ref.\,}
\newcommand{\Citeinternaltype}{Reference}
\newcommand{\citeinternalref}[1]{\cite{Sank:#1}}


\newcommand{\Seng}{S^e}

\title{Superconducting Qubits 101}
\author{Daniel Sank \\
\small Google Quantum AI \\
\small Formerly Department of Physics, UCSB}
\date{19 August 2012}

%\hypersetup{draft}
\begin{document}

\maketitle
\tableofcontents

\section{Introduction - statement of purpose}

Working with superconducting qubits is a lot of fun.
The design process offers a single focal point through which one can learn and apply quantum mechanics, electromagnetism, statistical physics, circuit theory, and several aspects of practical engineering.
The intended application of the qubit is often described mathematically.
For example, superconducting qubits used for quantum annealing are described in terms of various forms of the Ising model, while those used for gate-based computation may be described in terms of their capacity for single- and two-qubit gates.
In all cases, to work with superconducting qubits, one must understand how the mathematical representations of control pulses and qubit-qubit coupling arise from the qubits' physical properties like inductance and capacitance.
For example, one must know how to relate the coupling strength $g$ between two capacitively coupled qubits to those qubits' circuit parameters and the capacitance of the coupling capacitor.

Uri Vool has written an excellent paper on the quantization of arbitrary lumped-element electrical circuits, including a discussion of dissipative elements \cite{Vool:quantumCircuits}, and that work must be considered prerequisite reading for this one.\footnote{Vool's paper is a spiritual rewrite of an earlier \mbox{\href{http://qulab.eng.yale.edu/documents/reprints/Houches_fluctuations.pdf}{set of course notes}} for the Les Houches school, written Michel Devoret. Among other things, the rewrite corrects several unfortunate typos from the original.}
In other words, Reference \cite{Vool:quantumCircuits} tells you how to write the Hamiltonian for a given electrical circuit.
The next step toward a practical understanding of qubit control and coupling is to apply the Hamiltonian technique to the most common qubit circuits, examine how controls and coupling work, and arrive at useful design formulae.
One can imagine a sort of ``superconducting qubit builder's and user's guide'' which provides a list of useful and important design formulae, as well as enough of a demonstration of where those formulae come from such that the designer can break new ground on their own, knowing well what practical considerations must be taken into account.
This document intends to be such a guide.

\subsection{Extensibility}

This document is not, nor can it even be, comprehensive.
However, it does offer a route to ensure that important missing elements are added as time goes on and that the quality is always increasing: this document's source code lives in a \href{https://github.com/danielsank/theory}{publicly accessible repository} on GitHub.
Typos, identification of unclear sections, and requests for additional information reported on the \href{https://github.com/danielsank/theory/issues}{issue tracker} will lead to improvements in the document.
In this way, readers become authors, shared knowledge grows, and we all learn what need to know faster than we could before.

A central concept in signal processing is that a signal can be expressed as a linear superposition of sinusoids.\footnote{More generally, functions can be thought of as vectors and expressed as linear superpositions of basis vectors, where the basis is chosen to suit the problem at hand. In problems with translation invariance, sinusoids are convenient because they are eigenvectors of translation. For example, defining $E: \reals \rightarrow \complexes$ by the equation $E(t) = \exp(i \omega t)$, we have $E(t + \delta t) = \exp(i \omega \delta t) E(t)$.}
The usual Fourier transform expresses a function $f: \reals \rightarrow \complexes$ as a continuous superposition i.e. integral
\begin{equation}
  f(t) = \int \frac{d \omega}{2\pi} \tilde{f}(\omega) e^{i \omega t} \, .
\end{equation}
The utility and properties of the Fourier transform should be at least somewhat familiar to the audience of this book; if not, see ??.
However, the Fourier transform pertains to function defined over a domain of infinite extent and with infinite resolution.
In real life applications with e.g. experimental data, discrete time digital devices such as digital-to-analog converters (DAC) and analog-to-digital (ADC) converters, or in crystals with discrete unit cells, we have either finite domains, finite resolution, or both.
What we imagine to be less familar to readers here, is how to handle the subtleties introduced in those cases.

As suggested above, there are four major Fourier transforms, each characterized by the extent of its domain and the resolution on that domain.
These transforms are listed in Table~\ref{tab:four_transforms}.
\begin{table}
  \begin{center}
    \begin{tabular}{|r|c|c|c|c|}
      \hline
      \textbf{Name}                   & \textbf{Extent}  & \textbf{Resolution}  & \textbf{Transformed extent}   & \textbf{Transformed resolution} \\
      \hline
      \hline
      Fourier transform               & infinite                & continuous                  & infinite                & continuous \\
      \hline
      Discrete time Fourier transform & infinite                & discrete                    & finite                  & continuous \\
      \hline
      Fourier series                  & finite                  & continuous                  & infinite                & discrete \\
      \hline
      Discrete Fourier transform      & finite                  & discrete                    & finite                  & discrete \\
      \hline
    \end{tabular}
    \caption{The four Fourier transforms}
    \label{tab:four_transforms}
  \end{center}
\end{table}
Let's go through the cases in table one at time.
To clarify the language, we talk about functions (i.e. signals) of time whose transforms are functions of frequency.
However these considerations apply to functions defined over space, or anything else.
As illustrated above, the Fourier transform (FT) maps a continous function over the full real line to another continous function over the full real line.
That is, given an infinite extent of time with infinite time resolution, we have a transform defined over infinite extent of frequency with infinite frequency resolution.

\leveldown{Discrete time Fourier transform (DTFT)}
Suppose that our function is defined over the integers, i.e. the time samples run infinitely far into the past and future, but with discrete instead of continous resolution $\delta t$.
Intuitively, this function cannot have frequency components with frequencies beyond $1/\delta t$, so we should expect the transformed function to be defined only within a finite range $[0, 1/\delta t]$.
This is the case with the discrete time Fourier transform (DTFT), which maps a function $f: \integers \rightarrow \complexes$ defined on the integers to a function $F: [0, 1] \rightarrow \complexes$ defined on a finite but continuous interval:
\begin{equation}
  f_n = \int_0^1 d\nu \, F (\nu) e^{i 2 \pi \nu n}
  \qquad
  F(\nu) = \sum_{n=-\infty}^\infty f_n e^{-i 2\pi \nu n}
  \, .
\end{equation}
Compared to the Fourier transform, the DTFT has limited frequency extent because the original signal is not known with enough time resolution to resolve frequencies beyond $1/\delta t$.

\levelstay{Fourier series (FS)}
Suppose that our function is defined over a continous but finite interval $[0, T]$, i.e. the time runs over a limited range but with infinite resolution.
Intuitively, this function cannot have frequency components distinguished more finely than $1/T$, so we should expect the transformed function to be defined over a discrete set of frequencies separated by $\delta f = 1 / T$.
This is the case with the Fourier series (FS), which maps a function $f: [0, 1] \rightarrow \complexes$ defined over a finite continous interval to a function $F: \integers \rightarrow \complexes$ defined over the integers:
\begin{equation}
  f(t) = \sum_{k=-\infty}^\infty F_k e^{i 2 \pi t k}
  \qquad
  F_k = \int dt \, f(t) e^{-i 2 \pi t k}
  \, .
\end{equation}
Compared to the Fourier transform, the FS has limited frequency resolution because the original signal is not known for long enough time to resolve the difference between two frequencies closer together than $\delta f = 1 / T$.
Notice that the DTFT and FS are opposites.

\levelstay{Discrete Fourier transform (DFT)}
Now suppose that our function is defined over a finite set of discrete samples $\{0, \delta t, 2 \delta t,\ldots, T = (N-1) \delta t\}$.
In this case, we have the limitations of both the DTFT and the FS, i.e. we expect our transformed function to be defined only up to a maximum frequency $1 / \delta t$ and only with frequency resolution $\delta f = 1 / T$.
This is the case with the discrete Fourier transform (DFT), which maps a function $f: \{0, 1,\ldots N-1\} \rightarrow \complexes$ to a function $\tilde f: \{0, 1,\ldots N-1\}$:
\begin{equation}
  f_n = \sum_{k=0}^{N-1} \tilde{f}_k e^{i 2 \pi n k}
  \qquad
  \tilde{f}_k = \sum_{n=0}^{N-1} f_n e^{-i 2 \pi n k}
  \, .
\end{equation}
Compared to the Fourier transform, the DFT has both limited freuqency range and limited frequency resolution.

\levelstay{Outline}

In the remainder of this chapter, we investigate the consequences of the finite and infinite time resolution and extent of these transforms.


\subimportlevel{./}{driving.tex}{0}
\subimportlevel{./}{coupling.tex}{0}
\subimportlevel{./}{rotating_frame}{0}
\subimportlevel{./}{decoherence.tex}{0}

%\subsubsection{Do LO phases matter?}

%We can now answer the question of whether we need to keep LO oscillator phases stable across multiple repetitions of a multiple qubit experiment. Consider an experiment in which we have to capacitively coupled qubits and an adjustable coupling $g(t)$. We put the qubits on resonance and rotate each qubit's frame at  qubit frequency. As previously explained in this doubly rotating frame the intrinsic qubit Hamiltonians are zero and the coupling Hamiltonian is $g(t)\left( \sigma^+ \sigma^- + \sigma^- \sigma^+ \right)$. We now subject the system to one of the following control sequences \begin{eqnarray}
%X_{\pi/2}^{(1)} \rightarrow & \textrm{turn g on for a swap} & \rightarrow X_{\pi/2}^{(2)} \rightarrow \textrm{measure qubit 2} \nonumber \\
%X_{-\pi/2}^{(1)} \rightarrow & \textrm{turn g on for a swap} & \rightarrow X_{\pi/2}^{(2)} \rightarrow \textrm{Measure qubit 2} \nonumber \end{eqnarray}
%In the first case we will measure $\ket{1}$ but in the second we will measure $\ket{0}$.  As explained in the section on driving in the rotating frame, the difference in the leading $X$ pulses is just a difference in the phase of the drive voltage. This phase could come from an intentional choice of our I and Q signals, or from a change in the LO phase. Therefore, in general LO phase drift will mess up experiments. This could be verified by the pulse sequence we've considered here, measuring ramsey fringes on qubit 2 as a function of the drive phase on qubit 1.


\bibliographystyle{plain}
\bibliography{../Bibliography/references_main,../Bibliography/references_local}

\end{document}

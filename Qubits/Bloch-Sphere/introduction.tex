% bloch_sphere_introduction
\levelstay{Introduction}

Superconducting resonant circuits are important elements for quantum computation.
When the circuits are cooled to temperatures such that the ambient thermal energy $k_b T$ is significantly smaller (approximately $5\times$) than the circuit's quantum energy, i.e.
\begin{equation}
  k_b T \ll \hbar \omega \label{eq:block_sphere_introduction:condition}
\end{equation}
the circuit's energy levels quantize and the circuit can be used as a quantum bit (qubit).
Qubits made in this way are called ``superconducting quits''.

Circuits with sufficiently high frequency and low temperature such as to satisfy condition (\ref{eq:block_sphere_introduction:condition}) are said to be in the ``quantum limit''.
Condition (\ref{eq:block_sphere_introduction:condition}) can in fact be realized in dilution refrigerators which can cool to about $10\,\text{mK}$.
Noting that $\hbar / k_b = 48\,\text{mK} / (2 \pi) \text{GHz}$, we see that a circuit in a dilution refrigerator will be in the quantum limit if that circuit has a resonance frequency of
\begin{equation}
  \omega / 2 \pi > 5 \times \frac{k_b}{2\pi \hbar} \, 10\,\text{mK} \approx 1 \, \text{GHz}
  \, .
\end{equation}
Therefore, superconducting qubits are simply microwave resonators.

It follows that the logical state of the qubit is manipulated by microwave pulses applied to the circuit.
This technical note explains how the qubit state depends on the applied microwaves.

However, we're going to do this a bit differently.
The entire discussion will be classical.

\levelstay{Decoherence}

Consider a two level system with Hamiltonian
\begin{equation*}
  H_S / \hbar = - \frac{\Omega}{2} \sigma_z
\end{equation*}
and an interaction with some other operator of the form
\begin{equation*}
  V / \hbar = g(F \otimes \sigma_x) \, .
\end{equation*}
It turns out (see Annealer Handbook appendix) that
\begin{equation*}
  \Gamma_\uparrow = g^2 S_F(- \Omega) \qquad \Gamma_\downarrow = g^2 S_F(\Omega)
\end{equation*}
where $S_F(\Omega)$ is the spectral density of $F$ defined as
\begin{equation*}
  S_F(\Omega) \equiv \int_{-\infty}^\infty dt \avg{F(t) F(0)} e^{i \Omega t} \, .
\end{equation*}

\leveldown{Driving}

From Sank thesis appendix, the drive Hamiltonian in the lab frame is
\begin{equation*}
  H_d = h_d f(t) \sigma
\end{equation*}
with
\begin{center}
  \begin{tabular}{r | c | c}
    & \textbf{Charge} & \textbf{Flux} \\
    \hline
    $\sigma$ & $\sigma_y$ & $\sigma_x$ \\
    \hline
    $f(t)$ & $V_d(t)/V_d$ & $I_d(t) / I_d$ \\
    \hline
    $h_d$ & $Q_\text{zpf} V_d(C_d/C_q)$ & $\Phi_\text{zpf} I_d (M_d/L_q)$
  \end{tabular}
\end{center}
For a resonant drive with envelope $w(t)$, i.e. $f(t) = w(t) \cos(\omega t + \phi)$, we find in the rotating frame
\begin{equation*}
  H_d \, (\text{rotating frame}) = h_d \frac{w(t)}{2} \left( \cos(\phi) \sigma_x + \sin(\phi) \sigma_y \right) \, .
\end{equation*}
The angle $\phi$ is a phase reference for the drive signal and we can shoose it to be whatever want.
For convenience, we choose $\phi=0$ for the flux case and $\phi=\pi/2$ for the charge case, allowing us to write the unitary operator driving the qubit as
\begin{align*}
  U(t)
  &= \exp \left( -\frac{i}{\hbar} \int_0^t H_d(t') \, dt' \right) \\
  &= \exp \left( -\frac{i}{\hbar} \frac{h_d}{2} \sigma \int_0^t w(t') dt' \right)
\end{align*}
where $\sigma$ stands for $\sigma_x$ for the flux case and $\sigma_y$ for the charge case.
A $\pi$-pulse occurs when the argument of the exponent is equal to $\pm i \sigma \pi / 2$, i.e. when
\begin{equation*}
  \frac{1}{2} \frac{h_d}{\hbar} \int_0^t w(t') \, dt' = \pi / 2
\end{equation*}
or assuming $w(t)$ is a constant $w$,
\begin{equation}
  \frac{1}{\pi} \frac{h_d}{\hbar} w \Delta t = 1 \, .
\end{equation}
It turns out that
\begin{align*}
  \text{(charge)} \quad \frac{h_d}{\hbar} &=
  \frac{1}{\sqrt{2}} \left(\frac{C_d}{C_q} \right) \sqrt{\frac{(R_K / 8\pi)}{Z}} \frac{V_d}{\Phi_0/2\pi} \\
  \text{(flux)} \quad \frac{\hbar}{h_d} &=
  \frac{1}{\sqrt{2}} \left( \frac{M_d}{L_q} \right) \sqrt{\frac{Z}{(R_K/8\pi)}} \frac{I_d}{2e}
\end{align*}
where $Z$ is the qubit impedance and $R_K \equiv h/e^2 = 25,812 \, \Omega$, giving $R_K/8\pi = 1,027 \, \Omega$.
Combining results and assuming that $w(t)$ is a constant over a time interval $\Delta t$,  we get a pi pulse when
\begin{align*}
  \text{(charge)} \, \frac{1}{\pi\sqrt{2}} \left( \frac{C_d}{C_q} \right) \sqrt{\frac{(R_K/8\pi)}{Z}} \frac{V_d w \Delta t}{\Phi_0/2\pi} &= 1 \\
  \text{(flux)} \, \frac{1}{\pi\sqrt{2}} \left( \frac{M_d}{L_q} \right) \sqrt{\frac{Z}{(R_K/8\pi)}} \frac{I_d w \Delta t}{2e} &= 1 \, .
\end{align*}
Note that $V_d w$ is just the amplitude of the sinusoidal drive voltage at the qubit's drive capacitor, and similarly for $I_d w(t)$.

\levelstay{Decay}

For a capacitively coupled qubit in the harmonic limit, the loaded $Q$ from the drive line is (see Daniel's loaded mode writeup)
\begin{align*}
  Q_c (\text{capacitive}) &= \left( \frac{C_q}{C_d} \right)^2 \frac{Z_q}{R_e} \\
  Q_c (\text{inductive}) &= \left( \frac{L_q}{M_d} \right)^2 \frac{R_e}{Z_q}
\end{align*}
where $R_e$ is the external line resistance, usually $50 \, \Omega$.

Eliminating the capacitances and inductances in favor of quality factor, we find
\begin{align*}
  \text{(charge)} \quad
    \frac{1}{\pi\sqrt{2}} \sqrt{\frac{1}{Q_c}}
    \sqrt{\frac{(R_K/8\pi)}{R_e}} \frac{V_d w \Delta t}{\Phi_0/2\pi} &= 1 \\
  \text{(flux)} \quad
    \frac{1}{\pi\sqrt{2}} \sqrt{\frac{1}{Q_c}}
    \sqrt{\frac{R_e}{(R_K/8\pi)}} \frac{I_d w \Delta t}{2e} &= 1 \, .
\end{align*}
These equations link the loaded quality factor of the (harmonic) qubit to the external ($50 \, \Omega$) resistance, the pi-pulse length, and the driving amplitude.
Note that although $\hbar$ is present via $R_K$, the qubit parameters are absent!
Therefore, this equation is quantum, but does not involve the internal structure of the qubit in any way.

In the Annealer-Handbook, it is shown that given the driving Hamiltonian
\begin{displaymath}
  H_d = h_d f(t) \sigma_{x,y} \, ,
\end{displaymath}
the up/down transition rate due to drive noise is
\begin{equation*}
  \Gamma = \left( \frac{h_d}{\hbar} \right)^2 S_f(\omega_q) \, .
\end{equation*}
Plugging in the values for $h_d/\hbar$ for each case gives
\begin{align*}
  \text{(charge)} \, \Gamma =&
    \frac{1}{2} \left( \frac{C_d}{C_q} \right)^2
    \left( \frac{R_K/8\pi}{Z_q} \right) \frac{S_{V_d}(\omega_q)}{(\Phi_0/2\pi)^2} \\
  \text{(flux)} \, \Gamma =&
    \frac{1}{2} \left( \frac{M}{L_q} \right)^2
    \left( \frac{Z_q}{R_K/8\pi} \right) \frac{S_{I_d}(\omega_q)}{(2e)^2} \, .
\end{align*}

\levelstay{Summary}

The results of this section are summarized as follows
\begin{table*}
  \centering
  \begin{tabular}{|r|c|c|c|}
    \hline
    & General & Charge & Flux \\
    \hline \hline
    $h_d/\hbar$
      &
      & $\frac{1}{\sqrt{2}} \xi z^{-1/2} \frac{V_d}{\Phi_0/2\pi}$
      & $\frac{1}{\sqrt{2}} \xi z^{1/2} \frac{I_d}{2e}$
      \\
    \hline
    $Q_c$
      &
      & $\xi^{-2} \frac{Z_q}{R_e}$
      & $\xi^{-2} \frac{R_e}{Z_q}$
      \\
    \hline
    $\pi$-pulse
      & $\frac{1}{\pi} \frac{h_d}{\hbar} w \Delta t = 1$
      & $\frac{1}{\pi\sqrt{2}} \xi z^{-1/2} \frac{V_d w \Delta t}{\Phi_0/2\pi} = 1$
      & $\frac{1}{\pi\sqrt{2}} \xi z^{1/2} \frac{I_d w \Delta t}{2e} = 1$
      \\
      &
      & $\frac{1}{\pi\sqrt{2}} \sqrt{\frac{1}{Q_c}} r_e^{-1/2} \frac{V_d w \Delta t}{\Phi_0/2\pi} = 1$
      & $\frac{1}{\pi\sqrt{2}} \sqrt{\frac{1}{Q_c}} r_e^{1/2} \frac{I_d w \Delta t}{2e} = 1$
      \\
    \hline
    $\Gamma_{\uparrow\downarrow}$
      & $(h_d/\hbar)^2 S$
      & $\frac{1}{2} \xi^2 z^{-1} \frac{S_{V_d}(\omega_q)}{(\Phi_0/2\pi)^2}$
      & $\frac{1}{2} \xi^2 z \frac{S_{I_d}(\omega_q)}{(2e)^2}$
      \\
    \hline
  \end{tabular}
  \caption{caption}
\end{table*}
where $z \equiv Z_q/(R_K/8\pi)$, $r_e \equiv R_e / (R_K/8\pi)$, and
\begin{displaymath}
  \xi = \frac{C_d}{C_q} \quad \text{or} \quad \frac{M_d}{L_q} \, .
\end{displaymath}

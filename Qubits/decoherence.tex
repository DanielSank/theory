\levelstay{Decoherence and control in the harmonic limit}

\leveldown{Introduction}

As mentioned previously, for both capacitive and inductive driving, we can write the driving Hamiltonian in the lab frame as
\begin{equation*}
  H_d = h_d f(t) \sigma
\end{equation*}
with
\begin{center}
  \begin{tabular}{|r|c|c|}
    \hline
    & \textbf{Charge} & \textbf{Flux} \\
    \hline \hline
    $\sigma$ & $\sigma_y$ & $\sigma_x$ \\
    \hline
    $f(t)$ & $V_d(t)/V_d$ & $I_d(t) / I_d$ \\
    \hline
    $h_d$ & $Q_\text{zpf} V_d(C_d/C_q)$ & $\Phi_\text{zpf} I_d (M_d/L_q)$ \\
    \hline
  \end{tabular}
\end{center}
Using the forms of $Q_\text{zpf}$ and $\Phi_\text{zpf}$ given above, we can rewrite $h_d$ as
\begin{align*}
  \text{(charge)} \quad \frac{h_d}{\hbar} &=
  \frac{1}{\sqrt{2}} \left(\frac{C_d}{C_q} \right) \sqrt{\frac{(R_K / 8\pi)}{Z_q}} \frac{V_d}{\Phi_0/2\pi} \\
  \text{(flux)} \quad \frac{h_d}{\hbar} &=
  \frac{1}{\sqrt{2}} \left( \frac{M_d}{L_q} \right) \sqrt{\frac{Z_q}{(R_K/8\pi)}} \frac{I_d}{2e}
\end{align*}
where $Z_q$ is the qubit impedance and $R_K \equiv h/e^2 = 25,812 \, \Omega$, giving $R_K/8\pi = 1,027 \, \Omega$.

\levelstay{Pi pulse length}

For a resonant drive with envelope $f(t) = \cos(\omega t + \phi)$, we found in the rotating frame
\begin{equation*}
  H_d \, (\text{rotating frame}) = \frac{h_d}{2} \left( \cos(\phi) \sigma_x + \sin(\phi) \sigma_y \right) \, .
\end{equation*}
The angle $\phi$ is a phase reference for the drive signal and we can choose it to be whatever want.
For convenience, we choose $\phi=0$ for the flux case and $\phi=\pi/2$ for the charge case, allowing us to write the unitary operator driving the qubit as
\begin{align*}
  U(\Delta t)
  &= \exp \left( -\frac{i}{\hbar} \int_0^{\Delta t} H_d(t') \, dt' \right) \\
  &= \exp \left( -\frac{i}{\hbar} \frac{h_d}{2} \sigma \Delta t \right)
\end{align*}
where $\sigma$ stands for $\sigma_x$ for the flux case and $\sigma_y$ for the charge case.
A $\pi$-pulse occurs when the argument of the exponent is equal to $\pm i \sigma \pi / 2$, i.e. when
\begin{equation*}
  \frac{1}{2} \frac{h_d}{\hbar} \Delta t = \pi / 2 \, .
\end{equation*}
Using the forms of $h_d/\hbar$ given above, we get the condition for a pi pulse as
\begin{align*}
  \text{(charge)} \, \frac{1}{\pi\sqrt{2}} \left( \frac{C_d}{C_q} \right) \sqrt{\frac{(R_K/8\pi)}{Z_q}} \frac{V_d \Delta t}{\Phi_0/2\pi} &= 1 \\
  \text{(flux)} \, \frac{1}{\pi\sqrt{2}} \left( \frac{M_d}{L_q} \right) \sqrt{\frac{Z_q}{(R_K/8\pi)}} \frac{I_d \Delta t}{2e} &= 1 \, .
\end{align*}
Note that $V_d$ is just the amplitude of the sinusoidal drive voltage at the qubit's drive capacitor, and similarly for $I_d$.

\levelstay{Decay from spontaneous emission}

In the case of a nearly harmonic qubit like the transmon, the effect of dissipation from the drive line can be computed from classical electronics arguments.
The quality factor $Q_d$ induced by the driving circuit is
\begin{align*}
  Q_d (\text{capacitive}) &= \left( \frac{C_q}{C_d} \right)^2 \frac{Z_q}{R_e} \\
  Q_d (\text{inductive}) &= \left( \frac{L_q}{M_d} \right)^2 \frac{R_e}{Z_q}
\end{align*}
where $R_e$ is the external line resistance, usually $50 \, \Omega$.
Using these formulae, we can eliminate the capacitances and inductances from the formulae for the pi pulse length, arriving at
\begin{align*}
  \text{(charge)} \quad
    \frac{1}{\pi\sqrt{2}} \sqrt{\frac{1}{Q_d}}
    \sqrt{\frac{(R_K/8\pi)}{R_e}} \frac{V_d \Delta t}{\Phi_0/2\pi} &= 1 \\
  \text{(flux)} \quad
    \frac{1}{\pi\sqrt{2}} \sqrt{\frac{1}{Q_d}}
    \sqrt{\frac{R_e}{(R_K/8\pi)}} \frac{I_d \Delta t}{2e} &= 1 \, .
\end{align*}
These equations link the loaded quality factor of the (harmonic) qubit to the external ($50 \, \Omega$) resistance, the pi-pulse length, and the driving amplitude.
Note that although $\hbar$ is present via $R_K$, the qubit parameters are absent!
Therefore, this equation is quantum, but does not involve the internal structure of the qubit in any way.

\levelstay{Up/down transition rates from noise}

If the qubit with Hamiltonian
\begin{equation*}
  H_q / \hbar = - \frac{\Omega}{2} \sigma_z \, ,
\end{equation*}
interacts with some other system through an interaction term
\begin{equation*}
  V / \hbar = g(F \otimes \sigma_x) \, .
\end{equation*}
then the qubit experiences up and down transition rates due to noise in the $F$ operator with rate (REFERENCE)
\begin{equation*}
  \Gamma_\uparrow = g^2 S_F(- \Omega) \qquad \Gamma_\downarrow = g^2 S_F(\Omega)
\end{equation*}
where $S_F(\Omega)$ is the (two sided) spectral density of $F$ defined as
\begin{equation*}
  S_F(\Omega) \equiv \int_{-\infty}^\infty dt \avg{F(t) F(0)} e^{i \Omega t} \, .
\end{equation*}
In the case that $F$ is classical, e.g. the driving voltage or current used for controlling the qubit, it is common to work with the ``engineer's'' spectral density defined as $\Seng_F \equiv 2 S_F$, defined only for positive frequencies.
$\Seng$ is the quantity one measures with e.g. a spectrum analyzer.

Given the driving Hamiltonian
\begin{displaymath}
  H_d = h_d f(t) \sigma_{x,y} \, ,
\end{displaymath}
the up/down transition rate due to drive noise is therefore
\begin{equation*}
  \Gamma = \left( \frac{h_d}{\hbar} \right)^2 S_f(\omega_q) = \frac{1}{2} \left( \frac{h_d}{\hbar} \right)^2 \Seng_f(\omega_q) \, .
\end{equation*}
Plugging in the values for $h_d/\hbar$ for each case gives
\begin{align*}
  \text{(charge)} \, \Gamma =&
    \frac{1}{4} \left( \frac{C_d}{C_q} \right)^2
    \left( \frac{R_K/8\pi}{Z_q} \right) \frac{\Seng_{V_d}(\omega_q)}{(\Phi_0/2\pi)^2} \\
  \text{(flux)} \, \Gamma =&
    \frac{1}{4} \left( \frac{M_d}{L_q} \right)^2
    \left( \frac{Z_q}{R_K/8\pi} \right) \frac{\Seng_{I_d}(\omega_q)}{(2e)^2} \, .
\end{align*}
Eliminating the coupling capacitance and inductance in favor of $Q_d$ gives
\begin{align}
  \text{(charge)} \quad \Gamma
  =& \frac{1}{4} \frac{1}{Q_d}
  \left( \frac{R_K/8\pi}{R_e}\right) \frac{\Seng_{V_d}(\omega_q)}{(\Phi_0/2\pi)^2} \\
  \text{(flux)} \quad \Gamma
  =& \frac{1}{4} \frac{1}{Q_d} \left( \frac{R_e}{R_K/8\pi} \right) \frac{\Seng_{I_d}(\omega_q)}{(2e)^2} \, .
\end{align}

\levelstay{Universal formulae - Power}

Using the $Q$ induced by the driving circuit as a universal parameter and referring to the available signal and noise power from the drive circuit, we can write our design formulae in a remarkably concise way.

Consider the $\Gamma$ formula in the charge coupled case.
Plugging in the formula for the loaded $Q$ due to the drive circuit, we find
\begin{align*}
  \Gamma
  & = \frac{1}{4} \frac{1}{Q_d} \frac{R_K / 8\pi}{R_e} \frac{\Seng_{V_d}}{(\Phi_0 / 2\pi)^2} \\
  & = \frac{1}{4} \frac{1}{Q_d \hbar R_e} \Seng_{V_d} \, .
\end{align*}
Noting that $\Seng_{V_d}$ is the spectral density of voltage at the drive capacitor and that the drive capacitor is effectively an open circuit, we can write $\Seng_{V_d} = 4 \Seng_a R_e$ where $\Seng_a$ is the available noise power spectral density from the drive circuit.
This gives
\begin{equation}
  \boxed{\Gamma = \frac{\Seng_a}{Q_d \hbar} \, .}
\end{equation}
This wonderfully simple result applies unchanged to the inductive drive case.

Through similar reasoning, we can write the pi pulse time in terms of the signal power $P_a$ available from the drive circuit as
\begin{equation}
  \boxed{\Delta t = \frac{\pi}{2} \sqrt{ \frac{Q_d \hbar}{P_a}}\, .}
\end{equation}
It is tempting to combine these equations into a single design rule $\Gamma \Delta t = (\pi/2) \sqrt{Q_d \hbar / P_a}$, but this is highly misleading as it fails to capture the fact that $\Gamma$ and $\Delta t$ scale differently with $Q_d$.
In other words, for a given value of $Q_d$, we \emph{cannot} independently fix \emph{either} $\Delta t$ or $\Gamma$.

An example application of these formulae, we plot in Figure \ref{fig:gamma_and_pulse_length} the $\Gamma$ and pi pulse lengths for a nearly harmonic circuit as a function of the attenuation between the driver and the qubit.

\quickfig{\columnwidth}
{gamma_and_pulse_length.pdf}
{$\Gamma$ and pi pulse length $\Delta t$ for a nearly harmonic qubit.
We use $Q_d = 40 \times 10^6$, corresponding to $T_1=1\,\text{ms}$ for a qubit with resonance frequency 6\,GHz.
We assume a noise spectral density of $8\,\text{nV}/\sqrt{\text{Hz}}$ and a maximum signal power corresponding to $1\,\text{V}_\text{pp}$.}
{fig:gamma_and_pulse_length}

\levelstay{Summary}

The results of this section are summarized in the table.
\begin{table*}
  \centering
  \begin{tabular}{|r|c|c|c|}
    \hline
    & General & Charge & Flux \\
    \hline \hline
    $h_d/\hbar$
      &
      & $(1/\sqrt{2}) \xi z^{-1/2} D$
      & $(1/\sqrt{2}) \xi z^{1/2} D$
      \\
    \hline
    $Q_d$
      &
      & $\xi^{-2} (Z_q/R_e)$
      & $\xi^{-2} (R_e/Z_q)$
      \\
    \hline
    $\Delta t$
      & $(\pi/2) \sqrt{Q_d \hbar / P_a}$
      & $(1/\pi\sqrt{2}) \xi z^{-1/2} D \Delta t \geq 1$
      & $(1/\pi\sqrt{2}) \xi z^{1/2}  D \Delta t \geq 1$
      \\
      &
      & $(1/\pi\sqrt{2}) \sqrt{1/Q_d} r_e^{-1/2} D \Delta t \geq 1$
      & $(1/\pi\sqrt{2}) \sqrt{1/Q_d} r_e^{1/2}  D \Delta t \geq 1$
      \\
    \hline
    $\Gamma$
      & $\Seng_a / Q_d \hbar$
      & $(1/4) \xi^2 z^{-1} \Seng_D$
      & $(1/4) \xi^2 z      \Seng_D$
      \\
      &
      & $(1/4) r_e^{-1} \Seng_D / Q_d$
      & $(1/4) r_e      \Seng_D / Q_d$
      \\
    \hline
  \end{tabular}
  \caption{Summary of results for a qubit driven by charge or flux. In this table, $z\equiv Z_q/(R_K/8\pi)$ and $r_e \equiv R_e / (R_K/8\pi)$. We denote by $\xi$ the dimensionless coupling, i.e. $\xi \equiv C_d/C_q$ for charge and $\xi \equiv M_d/L_q$ for flux. Similarly, $D$ denotes a drive \emph{amplitude} with dimensions of 1/time, i.e. $D \equiv V_d/(\Phi_0 / 2\pi)$ for charge and $I_d/2e$ for flux. $\Seng_D$ indicates drive spectral density at the qubit, and also has dimensions of 1/time. For the ``General'' column, $\Seng_a$ and $P_a$ denote \emph{available} noise power spectral density and signal power ($\Seng_a$ has dimensions of energy and $P_a$ has dimensions of power). Note that $D$ and $\Seng_D$ are not available quantities; they correspond to the current/voltage at the drive capacitor or inductor.}
\end{table*}
